
Ce voyage intellectuel s’est souvent révélé éprouvant, mais il a pu être mené à terme grâce à l’aboutissement de multiples contributions, accompagnements, soutiens, tolérance, patience, compassion et aussi grâce à l’amour des professeurs, collègues, membres de ma famille, proches, amis et des institutions qui, depuis quatre années, ont toujours été là pour m'aider à faire avancer cette embarcation du savoir. Par ces remerciements, je souhaite leur exprimer ma gratitude infinie.

Tout d’abord, je dédie ce mémoire à mon père, parti trop tôt, qui a toujours souhaité que je poursuive la voie de l’histoire. Ce n’est que bien plus tard, lorsque j’ai posé le pied en France, que j’ai entamé ce parcours. Et ma mère, qui est toujours fière, bien qu'elle ne comprenne pas vraiment mes études.

Je repense aux origines de ce parcours, là où tout a commencé, à mes professeurs et collègues de l’Université Lumière Lyon 2. Je suis reconnaissante à Aurélien MONTEL, qui a éclairé mon chemin, m’a inspirée et m'a ouvert grand les portes de l’histoire, en particulier celles de l’histoire religieuse, me permettant ainsi de nourrir l’idée de poursuivre ce master. J’adresse également mes remerciements à Frédéric ROUSTAN pour sa bienveillance et sa patience. Il a été le trait d’union qui m’a guidée vers mon directeur de mémoire actuel.

Je n'aurai pas pu démarrer ce mémoire, ni le mener à bien, sans l’encadrement de mon tuteur, Pascal BOURDEAUX. Aussi, je lui adresse ma plus sincère gratitude, car il a été le lien qui m’a permis cette rencontre éclairante avec Nguyễn Kim Muôn durant un après-midi de printemps 2021. Il m’a offert une expérience riche et authentique à la croisée de la recherche historique et religieuse, et consacré un temps précieux pour orienter mes recherches et m’accompagner dans mes études. Son érudition, son expérience approfondie dans la recherche et ses encouragements opportuns ont été non seulement une immense source d’inspiration, mais aussi une boussole qui m'ont aidées à surmonter les difficultés et à mener ce parcours à son terme.

Je souhaite également exprimer ma reconnaissance aux professeurs du Master Études asiatiques de l’EPHE, ainsi qu’aux institutions partenaires telles que l’EHESS et l’EFEO, pour la richesse des savoirs, des méthodes de recherche et des perspectives qu’ils m'ont enseignées. Leurs enseignements et échanges m’ont offert non seulement des connaissances précieuses, mais aussi un cadre intellectuel stimulant qui a largement nourri et élargi ma recherche. Je tiens aussi à remercier l'assistant social du service Vie étudiante de l'EPHE et le Welcome Desk de l'Université PSL pour leur aide et précieux conseils tout au long de mes études.

Dans la quête et la collecte des indices laissés par ce personnage iconique, je tiens à remercier chaleureusement l'équipe d'enseignement de la formation à l’enquête d’archives de l’EHESS, et plus particulièrement Vanessa CARU, ma tutrice de stage qui m’a guidée lors de ma première venue aux ANOM et a partagé avec moi ses compétences et techniques de recherche documentaire. J’exprime aussi ma très grande reconnaissance envers Paul SORRENTINO, toujours attentif et serviable, membre du jury qui aura pris le temps de lire et d’évaluer ce mémoire.

Je remercie énormément l'équipe pédagogique du master Technologies numériques appliquées à l'Histoire à l’ENC-PSL (2024), qui m’a permis de maîtriser de nouvelles théories et méthodes pratiques de recherche. Grâce à eux, j'ai pu me donner les moyens techniques dont j’avais besoin pour finaliser ce mémoire, et je maîtrise maintenant de nouveaux outils informatiques qui constituent les clés de ma future carrière professionnelle.

Au cours de ce travail, j’ai reçu un appui précieux de la part de plusieurs institutions d’archives et bibliothèques. 
Je remercie :
\begin{itemize}
    \item la Bibliothèque nationale de France (BnF)
    \item la Bibliothèque nationale du Vietnam
    \item la Bibliothèque de l’Université Aix-Marseille
    \item le Centre des Archives nationales II (ANV2) à Hô Chi Minh-Ville
    \item les Archives nationales d'outre-mer
    \item les Archives historiques du Crédit Agricole SA
\end{itemize}

Je souhaite exprimer tout particulièrement ma profonde gratitude aux moines et fidèles bouddhistes des temples Hùng Long Tự et Long Vân Tự, en particulier à \textit{thầy} Thích Thiện Thông, \textit{thầy} Thích Bửu Đăng et \textit{thầy} Thích Thiện Trí, pour leur accueil chaleureux, la mise à disposition de documents précieux et le partage de souvenirs vivants sur le Vénérable Nguyễn Kim Muôn. Je les remercie également pour les délicieux repas végétariens qu'ils m'ont offerts, et qui m'ont aidé à avoir une vision plus complète, approfondie et humaine de leur vie et de leur héritage.

Ce travail aurait été encore plus difficile sans de généreux soutiens financiers et des bourses d’études accordées par diverses institutions. Je remercie donc le Programme Science des religions de l’EPHE pour les deux bourses de mobilité du Programme Science des religions pour la recherche au Vietnam et à Aix-en-Provence obtenues en 2021 et 2022. Je remercie également la Direction des Relations Internationales de l’EPHE-PSL pour la bourse de mobilité Île-de-France destinée à mes recherches au Vietnam, ainsi que l’École française d’Extrême-Orient (EFEO) et son antenne de Hô Chi Minh-Ville pour leur soutien institutionnel et financier à mon travail de terrain au Vietnam en 2024.

J'ai également une pensée pour les institutions et organismes qui m'ont permis de développer mes compétences professionnelles au travers de missions et de stages rémunérés, et qui m'ont permis d'élargir mon champ de recherche afin de créer des conditions favorables à l'accomplissement de mes recherches.\\
Je tiens à remercier :
\begin{itemize}
    \item {Les Archives du Sénat}
    \item {Le Laboratoire IRDIS au Campus CNRS de Villejuif}
    \item {La Maison des Sciences de l'Homme Mondes - Université Paris Nanterre}
    \item {L'Institut historique allemand à Paris (IHA)}
\end{itemize}
Vos espaces de travail m'ont offert un cadre stimulant et propice à la finalisation de ce mémoire.

Je souhaite également remercier mes amis de recherche, des écoles et de la vie, avec qui nous avons échangé et partagé de précieuses expériences et conversations, celles et ceux qui m’ont soutenue lors des séjours de recherches et les amitiés sincères qui se sont révélées dans les jours sombres avec des soupirs et se sont épanouies dans cet univers de savoir : \textit{cô} Phương Ngọc, \textit{chị} Dung, \textit{chị} Khải Đơn, Guilhem, Sunny, Édouard, Maxime, Anna, Nedjima, Linh, Ánh, Thanh,  Trang, An, Quỳnh, Nhật, Hưng, Vinh, \textit{thầy} Hương Hải, Pablo mais aussi tous ceux dont le nom ne figure pas ici, et une pétale pour Vy, qui a toujours été là depuis le début. Vos présence ont toutes marqué mon chemin et contribué à cette réussite.

Enfin, je remercie celui qui m’a accompagnée tout au long du chemin, m’offrant motivation, énergie positive, joie et encouragements pour achever ce mémoire. Grâce à l’amour, à la patience, à la disponibilité pour me relire et corriger minutieusement ce travail, Thomas qui m'accompagne à chaque pas de ma vie. Je veux aussi mentionner la douce présence de Moon, mon chat tricolore, qui a su apaiser les moments de solitude lors de l’écriture de ce mémoire.
