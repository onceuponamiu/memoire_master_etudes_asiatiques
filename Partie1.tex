\setcounter{chapter}{0} 
\chapter{Vie séculière et entrée dans la vie religieuse}
\section{Sa vie avant la religion}

Nguyễn Kim Muôn, dit Giai Minh\footnote{Giai Minh \textit{thoàn sư} comme Giai Minh \textit{thiền sư} : maître zen et aussi Giai Minh \textit{cư sĩ}: pratiquant laïc}, dit \textit{Quan tự đạo nhơn}\footnote{l’homme du Dao qui contemple},  naît en 1892 au sein d'une famille aisée\footnote{Selon Maître Minh Út cependant, selon Phillipe PEYCAM, Nguyễn Kim Đính, frère de Nguyễn Kim Muôn qui est né dans une famille modeste, mentionné dans son livre The birth of Vietnamese political journalism: Saigon, 1916-1930} de Bình Hòa, dans la province de Gia Định, qui est aujourd'hui l'arrondissement Bình Thạnh à Ho-Chi-Minh-Ville. Son père s'appelle Nguyễn Hữu Đạo et sa mère Trần Thị Tấn.\footnote{Note du Gouvernement aux administrateurs de province 12 septembre 1935} 

Il a un frère aîné nommé Nguyễn Kim Đính, qui deviendra le Rédacteur en Chef du journal Le Courrier Indochinois [traduction officielle de \textit{Đông Pháp thời báo}] de 1923 à 1927. Ce dernier travaille d'abord comme simple commis dans la fonction publique et entre dans le monde du journalisme en 1913, en tant que directeur de rédaction, avant de devenir rapidement propriétaire de sa propre entreprise. Ses nombreuses expériences professionnelles dans diverses revues font de lui une des figures les plus influentes de la presse vietnamienne des années 1920. Il dirige par exemple des journaux édités en français, comme l'Écho Annamite et La Tribune Indochinoise. Le 4 avril 1923, Nguyễn Kim Đính reçoit l'autorisation de lancer \textit{Đông Pháp thời báo} ; ce processus étant officiellement approuvé, cela fait de lui le premier « sujet français indigène » à détenir légalement la propriété d'un journal en \textit{quốc ngữ}. Le journal est publié par l'imprimerie de sa femme, Thạnh Thị Mậu, il s'agit d'un journal de quatre pages, imprimé à trois mille exemplaires par numéro et publié trois fois par semaine \footnote{Philippe PEYCAM, The birth of Vietnamese political journalism: Saigon, 1916-1930, New York, Columbia University press, 2012.}. Sa femme est décrite dans la presse de l'époque comme une commerçante de l’ancienne génération qui a gagné de l'expérience en achetant des produits locaux pour les revendre dans les marchés des cantons et des villages. Forte de ses compétences en commerce, Thạnh Thị Mậu joue un rôle crucial dans l'entreprise de son mari, gérant les finances et les opérations quotidiennes de l'imprimerie, qui porte d'ailleurs son nom. C'est d'ailleurs elle qui sauvera l'entreprise familiale de la faillite en mobilisant ses propres ressources financières pour en éponger les dettes.
Ce n'est qu'à partir de l’année 1940 que la gestion de l’Imprimerie est cédée à un certain Nguyễn Kim Ky.
Selon la thèse de CAO Vy, malgré le peu d'informations sur ce personnage, il s'agit sans doute d’un membre de la famille de Nguyễn Kim Đính et Thanh Thị Mậu ; Nguyễn Kim Đính et Nguyễn Kim Ky partageant le même nom de famille :\\ « Nguyễn Kim ». 
L'Imprimerie Thạnh Thị Mậu est active de 1927 à 1945 et c'est d'ailleurs elle qui imprime les tout premiers ouvrages de Nguyễn Kim Muôn en 1927.\footnote{Vy CAO, Histoire de l’imprimerie, du livre et de l’édition vietnamienne en Cochinchine : Traitement et analyse du fonds Indochinois (1890-1945), thèse de l'université Aix-Marseille, 2025, p.190}\\
\noindent\rule{0.35\linewidth}{0.6pt}

\clearpage
Pour en revenir à Nguyễn Kim Muôn, il se marie à Nguyễn Thị Hương, dite Madame Nguyễn Kim Muôn, surnommée Xuân Hương, Giai Minh \textit{đạo cô}, ou bien encore Giai Hương\footnote{Nguyễn Kim Muôn,\textit{Đại đạo truyền chơn} [Grande voie transmettant la vérité], Đức Lưu Phương, Saïgon, 1930, p.2}. Cette dernière est membre de la Société théosophique de France, et partagera sa vie jusqu'en 1933, date de son décès prématuré à la pagode Long Vân, des suites de problèmes de santé\footnote{Nguyễn Kim Muôn déclare dans l'interview du journal \textit{Phụ nữ \textit{Tân Văn}} : <<\textit{Vợ tôi cũng đi tu, mới quá cố hồi tháng trước}; Ma femme aussi était religieuse. Elle est décédée le mois dernier >> [notre traduction].\\ Huấn Minh, \textit{Phật giáo chấn hưng?}; Renouveau du bouddhisme [notre traduction], \textit{Phụ nữ \textit{Tân Văn}},  La Gazette des femmes [traduction issue de la thèse CAO Vy], numéro 200, le 18 mai 1933}.  Ils auront un fils appellé Nguyễn Kim Hài\footnote{Le fils de Nguyễn Kim Muôn est mentionné dans le livre Tao Tö King qui est copié et imprimé par Nguyễn Kim Muôn lui-même, et dont il rédige la préface} dit Minh Hài\footnote{Lão Tseu, Đạo đức Kinh; traduit par Huy Hồng Đăng, Bảo Tồn, Saigon, 1933, p.55}

\begin{figure}[H] \centering \includegraphics[width=0.5\textwidth]{chua/nkm.jpg} \caption{Portrait de Nguyễn Kim Muôn au temple Hùng Long. Source: \copyright{} NGUYỄN Lê Thủy Tiên, 20/08/2022} \label{fig:bonzesses} \end{figure}

Nguyễn Kim Muôn achève ses études secondaires au sein du système scolaire français à Saïgon et obtient le Certificat Diplômé\footnote{Selon un document interne trouvé au temple Long Vân, \textit{Tiểu sử Sư Muôn}}, soit l'équivalent du Baccalauréat à l'époque. Après l'obtention de ce diplôme, il est recruté pour travailler dans le domaine maritime – un métier qui lui permet de découvrir le monde extérieur, notamment les pays européens et le Japon. En 1917, à l'âge de 25 ans, il décide de poursuivre ses études en France, à l’École Supérieure de Commerce de Paris. Après avoir terminé son cursus, il rentre au Vietnam en 1923 et occupe un poste de secrétaire à la Banque de l’Indochine à Saïgon.

\section{Son entrée dans la vie religieuse}

\subsection{Sa découverte du bouddhisme \textit{Tịnh độ} et de la Théosophie}

Pendant qu’il travaille à la Banque de l’Indochine, Nguyễn Kim Muôn manifeste très tôt un esprit empreint de moralité et se tourne vers la voie spirituelle. En 1927, il commence à approfondir les enseignements du bouddhisme, en particulier la pratique de la Terre Pure (\textit{Tịnh độ}). Conscient de la mission spirituelle qui lui a été confiée, il fonde la Terre pure du non-agir \textit{Tịnh độ vô vi}, une forme de pratique combinant foi traditionnelle et éthique dans la vie quotidienne. Le Gouvernement général de la Cochinchine autorise cette pratique ; aussi, Nguyễn Kim Muôn encourage-t-il activement les gens à s'engager sur cette voie et à en suivre les pratiques spirituelles : la récitation du nom du Bouddha, le régime végétarien, l'action de faire le bien et d'éviter le mal.

Pendant ses années d’activité professionnelle, Nguyễn Kim Muôn observe un régime alimentaire particulier : il ne se nourrit exclusivement que de fruits givrés. Ce mode de vie ascétique et rigoureux reflète sa détermination spirituelle et son engagement profond dans la voie de la vertu. C’est également à cette époque qu'il prend pour nom religieux Giai Minh.

Par la suite, il va utiliser le mot "Minh", issu de son propre nom de Dharma, pour conférer un nom religieux aux disciples masculins qu’il ordonnera, selon un schéma tel que : Minh A, Minh B, etc.

Pendant ses congés, Nguyễn Kim Muôn parcourt les six provinces du sud du Vietnam (Cochinchine), pour y prêcher la voie bouddhique, propager les enseignements spirituels et guider les pratiquants sur le chemin de l’Éveil. Il commence également à accueillir ses premiers disciples, y compris des pratiquants âgés et respectés dans la communauté bouddhiste, comme Minh Huyền et Minh Thành – ce dernier deviendra plus tard Vice-Patriarche (\textit{Phó tăng thống}) de l’Église bouddhique \textit{Lục hòa tăng} en 1972. 

Il rédige une œuvre intitulée \textit{Tịnh độ hữu vi,} considérée comme le premier texte systématisant les rites de culte et de récitation, ainsi que les méthodes de pratique associées à cette voie. À l'époque, cet ouvrage se répand rapidement dans les six provinces de la Cochinchine.

Nguyễn Kim Muôn entretient une relation étroite avec M. Raymond Grégoire, Ingénieur des Distilleries d’Indochine, et représentant à Saïgon de la Société Théosophique de Paris. Leur affinité intellectuelle et spirituelle les lie fortement, et ils échangent ensemble sur de nombreux sujets religieux et philosophiques. En 1929, ils organisent une tournée de conférences sur le bouddhisme et la théosophie :

\textbf{À Saïgon :}
\begin{itemize}

   \item Au temple Tam Tông Miếu, le 21 juillet 1929\footnote{Lettre confidentielle du 25 septembre 1929 au Gouverneur de la Cochinchine}

    \item À la pagode Võ Đế, le 16 novembre 1929\footnote{Lettre confidentielle du 25 septembre 1929 au Gouverneur de la Cochinchine} qui se trouve être l'appartement de Lương Khắc Ninh\footnote{Lương Khắc Ninh (1862-1943), dont le nom de plume était Dũ Thúc, était une figure active dans de nombreux domaines culturels à Saïgon de 1900 aux années 1930, il est membre du Conseil Privé de la Cochinchine.}.
\end{itemize}

\textbf{En province :}
\begin{itemize}
    \item À la pagode de Phạm Hữu Đức, le 15 juillet 1929, à Baria, qui se situe aujourd'hui dans la province de Bà Rịa-Vũng Tàu\footnote{Lettre confidentielle 15 juin 1929 au Gouverneur de la Cochinchine}.

    \item Chez Phạm Đăng Xuân, le 12 septembre 1929, dans la ville de Tân Niên Tây, provinde de Gò Công, qui correspond aujourd'hui à la province de Tiền Giang; ainsi qu'à la pagode Phước Long Tự, à Cai Lậy, dans la ville de Mỹ Tho, Tiền Giang\footnote{Rapport de la province de Gocong du 13 septembre 1929 au Gouverneur de la Cochinchine, signé par Đào Văn Đính}.

    \item À la pagode Cao Hải Nhật, le 15 septembre 1929, à Cai Lậy, qui correspond aujourd'hui à la ville de Mỹ Tho, province de Tiền Giang\footnote{Note postale du délégué Cailay à l'Administrateur de la ville de Mỹ Tho, dans la province de Gocong, le 17 septembre 1929, signée par Phước}.

    \item À la pagode Phước Long Tự, le 6 octobre 1929, au village Thắng Nhi (Cap Saint-Jacques), qui correspond aujourd'hui à la ville de Vũng Tàu \footnote{Lettre à l'Administrateur, chef de la province du Cap Saint Jaques le 24 septembre 1929}.

    \item Chez Giáp Văn Tý, le 27 octobre 1929, dans la commune de Tương Bình Hiệp, à Thủ Dầu Một, qui correspond aujourd'hui à la commune de Chánh Hiệp, ville de Thủ Dầu Một, province de Bình Dương.\footnote{Lettre confidentielle du 28 octobre 1929 à l'Administrateur, chef de la province de Thủ Dầu Một, signée par Lê Minh Canh}. 
    
    \item Dans le village de Phước Vĩnh Tây, le 5 janvier 1930\footnote{Lettre confidentielle du 9 janvier 1930 à l'Administrateur, chef de la province de Cangiôc (Tiền Giang), signée par Tân}. 
\end{itemize}
\clearpage
Ces interventions sont bien accueillies par le public et ne suscitent aucune controverse. À cette occasion, Nguyễn Kim Muôn compose même un poème en hommage à leur amitié :

\begin{quote}
\textit{Bác Vật là ông này\\
Vì đạo đổi qua đây\\
Nhỏ lớn không có vợ\\
Bình sanh ăn trái cây\\
Cùng Giai Minh kết bạn\\
Thờ Chân – Lý làm thầy\\
Sống cuộc đời thanh đạm\\
Nay hưu trí về Tây}.
\end{quote}

\begin{quote}
<< Voici le Docteur savant,\\
Venu ici pour sa dévotion.\\
Jamais marié depuis l’enfance,\\
Il vit simplement de fruits.\\
Amitié sincère avec Giai Minh,\\
Vérités spirituelles, leur lumière.\\
Vie simple, sans attachement terrestre,\\
Aujourd’hui retraité, il rentre vers l’Ouest. >>
\end{quote}[notre traduction]

En outre, Nguyễn Kim Muôn et M. Raymond ont en fait tous deux l'intention de quitter leur emploi pour partir étudier au Tibet, et développer leurs enseignements spirituels. Ainsi, le 1\textsuperscript{er} février 1930, Nguyễn Kim Muôn démissionne officiellement de la Banque de l’Indochine, dans l’espoir de se rendre au Tibet pour poursuivre son chemin spirituel. Toutefois, des difficultés financières les forcent à mettre fin à leur projet. En 1930, M. Raymond rentre en France, tandis que Nguyễn Kim Muôn se retire en ermite, accompagné de quatre disciples (deux hommes, deux femmes), sur l’île de Thủ Chu, aujourd'hui située dans la ville Phú Quốc, dans la commune de Thổ Châu, au sein de la province de Kiên Giang. Le journal Trung Lập du 10 décembre 1930 publie un article :

« On dit que M. Nguyễn Kim Muôn, autrefois employé dans une banque où il percevait un salaire mensuel de trois cents piastres, a pourtant tout quitté pour entrer dans les ordres et rechercher la voie spirituelle. Il vit actuellement seul sur un îlot près de Phú Quốc. D’après les disciples de M. Nguyễn Kim Muôn, seule son école représenterait la véritable voie, conformément à l’enseignement du Bouddha Śākyamuni. »
(\textit{Thấy nói ông Nguyễn Kim Muôn hồi trước làm việc ở một ngân hàng kia, lương mỗi tháng đến ba trăm đồng, thế mà bỏ đi, xuất gia cầu đạo, hiện nay ở một mình nơi một hòn cù lao gần Phú Quốc. Cứ như lời đồ đệ ông Nguyễn Kim Muôn thì duy có phái ông Muôn là chánh đạo, theo đúng Phật Thích Ca mà thôi}.)

Ou encore, selon les propos d’un de ses disciples rapportés dans ce même article :

« Nous sommes un groupe de pratiquants bouddhistes appartenant à l’école de Huệ Năng, animés d’une volonté pure de cultiver notre esprit pour atteindre l’Éveil. Sous la direction de notre maître, M. Nguyễn Kim Muôn, dont le nom religieux est Giai Minh thoàn sư, nous nous appliquons à suivre fidèlement la voie tracée par le Bouddha Śākyamuni. C’est pourquoi notre maître s’est retiré de la vie séculière (\textit{li gia}) et s’est isolé (\textit{nhập thất}) sur une île au large ; dans quelques années, lorsqu’il aura atteint la réalisation spirituelle, il reviendra pour enseigner (\textit{hoằng hóa}), guider les êtres et raviver le bouddhisme (\textit{chấn hưng Phật giáo}). » (\textit{Chúng tôi là một bọn tu Phật theo phe Huệ Năng, một lòng thanh tịnh tu cho thành Phật, theo Minh sư của chúng tôi là ông Nguyễn Kim Muôn, pháp danh là Giai Minh thoàn sư, thầy của chúng tôi dốc lòng tu theo như đức Thích Ca, nên đã li gia và nhập thất ngoài hải đảo, vài năm nữa đây đắc thành sẽ về mà hoằng hóa chúng sanh và chấn hưng Phật giáo.})\footnote{Lê Văn Kỷ, \textit{Một phái tu Phật xuất hiện ra sau khi phê bình bản dịch Sự tích Phật Thích ca [Une école bouddhique est apparue après avoir critiqué la traduction de La Vie du Bouddha Śākyamuni] [notre traduction]}, Trung Lập, Saïgon, 1930, n°6319, p.5}

Bien qu’ayant prévu des vivres pour six mois, la situation se complique quand une violente tempête s'abat sur eux, détruisant leurs abris et gâtant leurs provisions. Un disciple est donc envoyé en mer sur un radeau pour chercher de l'aide. Ce dernier est alors repêché par un bateau japonais, et grâce au soutien de la diaspora vietnamienne à Singapour, arrive à collecter des fonds qui permettront le retour du Maître à Saïgon\footnote{Selon un document interne trouvé au temple Long Vân, \textit{Tiểu sử Sư Muôn}}.

Pendant ce temps, Nguyễn Kim Muôn et les trois disciples restés sur l’île vivent dans une extrême précarité, jusqu'à la venue des secours, subsistant seulement avec du riz mélangé à des cœurs de palmier et des herbes sauvages. Le bonze compose d'ailleurs à cette période plusieurs poèmes relatant cette vie d’ascèse difficile, dont voici un exemple :

\begin{quote}
    \textit{Một cái túi, một con dao \\
Thầy đi trước đệ tử theo sau \\
Lớp tầm đủng đỉnh làm dưa muối, \\
Lớp kiếm nấm mèo nấu cháo rau. \\
Hẩm hút cũng là xong bữa vậy, \\
Xóp ve chưa để đói ngày nào. \\
Chắc còn oan nghiệt theo gia báo, \\
Phải trả cho rồi biết nói sao?}
\end{quote}

\begin{quote}
<< Un sac, un couteau, \\
Le maître marche devant, le disciple suit derrière. \\
Les uns, tranquillement, préparent des légumes au sel, \\
Les autres cherchent des champignons noirs pour cuire une soupe aux herbes. \\
Même dans la frugalité, le repas est accompli, \\
Jamais nous n’avons laissé la faim frapper un seul jour. \\
Sans doute reste-t-il des dettes karmiques à solder, \\
Il faut les payer jusqu’au bout, que dire d’autre ? >>
\end{quote}[notre traduction]

Malgré ce premier échec, il entreprend un second séjour à Thủ Chu, cette fois avec plus d’une dizaine de disciples sous la direction de Minh Thành. Malheureusement, leur bateau se retrouve une fois encore pris dans une tempête et, avec un gouvernail cassé, dérive pendant des jours, laissant tous les passagers à bout de force. Nguyễn Kim Muôn prie alors avec ferveur pour que la tempête cesse et qu’une légère pluie vienne les sauver, ce qui se produit, assez miraculeusement, leur permettant ainsi de recueillir de l’eau potable. À l'issue de ce miracle, des passagers témoins de la scène, qui étaient jusqu'ici non religieux, choisissent alors de le rejoindre et de se faire ordonner disciples.

Le bateau finit par s’échouer à Réam, au Cambodge. Le groupe est aussitôt arrêté par les autorités françaises, car soupçonné de piraterie. Nguyễn Kim Muôn tente de s’expliquer en français, mais les autorités restent inflexibles. Nous n'avons que peu de détails sur cet événement, mais nous savons que, par un heureux hasard, un ancien collègue français de Nguyễn Kim Muôn, de passage à Réam, le reconnaît alors sur-le-champ et intervient en sa faveur, ce qui conduit à leur libération. 

Ils repartent donc ensuite vers Rạch Giá. Informé de l’existence de l’île Hòn Thơm près de Phú Quốc, propice à la pratique religieuse, Nguyễn Kim Muôn s’y installe brièvement. Mais les autorités coloniales françaises refusent qu’un groupe vietnamien s’établisse sur une île sans contrôle officiel. Nguyễn Kim Muôn doit alors quitter Hòn Thơm pour Phú Quốc à la fin de l’année 1930. Sur les conseils d’habitants locaux, il trouve refuge à Suối Đá, où il installe son ermitage et fonde une communauté religieuse.

Peu après, de nombreux disciples le rejoignent à Phú Quốc. La communauté compte déjà plus de cinquante pratiquants, chacun vivant dans sa propre cellule en toute simplicité. Les hommes s’occupent de l’aménagement du terrain et de la construction, tandis que les femmes récoltent les matériaux pour les toitures. Ensemble, ils posent les bases d’un centre bouddhique durable sur l’île.

Après une période de prédication à Phú Quốc, Nguyễn Kim Muôn et ses disciples construisent plus de vingt \textit{kuti} (cellules individuelles d'habitation) ainsi qu’un grand temple en bois, bambou et feuilles, formant un lieu de pratique spirituelle d’une grande solennité. Toutefois, lorsqu’il demande l’autorisation de célébrer une cérémonie d’inauguration, les autorités coloniales françaises lui reprochent de ne jamais avoir demandé la permission de construire un temple. Nguyễn Kim Muôn explique alors habilement qu’il s’agit simplement d’un ermitage destiné à la pratique religieuse, et non d’un temple au sens strict du terme. Cependant, il est dénoncé par des sources anonymes à l'administration locale, qui envisage alors de fermer son établissement définitivement.

En pleine tourmente, Nguyễn Kim Muôn reçoit une convocation du district. Pensant qu’il s’agit d’une sanction, il est surpris d’apprendre qu’il a en réalité obtenu un permis officiel de construction pour son temple, ainsi qu’un diplôme de bonze de Haut Moine, signé par le \textit{Tham biện} de la province de Hà Tiên, c'est à dire un haut fonctionnaire de la période coloniale française. C'est d'ailleurs à partir de ce moment là que Nguyễn Kim Muôn signe systématiquement ses communications publiques et livres publiés avec la mention française << Bonze diplômé >>. Ce n’est que bien plus tard qu’il apprendra que ce geste venait en réalité d’un ancien camarade d’études, le \textit{đốc phủ tấn} (gouverneur de district), qui, en découvrant le dossier de Nguyễn Kim Muôn, à Hà Tiên, avait discrètement préparé les documents officiels et les avaient fait signer par les autorités sans en informer Nguyễn Kim Muôn.

Toutefois, en janvier 1931, il choisit finalement de s’installer sur l’île de Phú Quốc, où il achète une cocoteraie pour 1 500 000 piastres et obtient une concession provisoire de 10 hectares. Avec une trentaine de disciples, hommes et femmes, il y développe une communauté spirituelle vouée à la pratique religieuse.

L’organisation qu’il fonde prend le nom de \textit{Đạo phật thích ca} (la religion bouddhique de Shakyamuni), avec pour symbole une croix gammée. Animé d’une sincérité et d’un zèle profonds, Nguyễn Kim Muôn espère établir en ce lieu une pagode idéale pour pratiquer le chemin de l’éveil selon l’esprit du bouddhisme. Il prône l’abolition de l’instinct sexuel parmi ses disciples, aspirant à une vie de pureté et de détachement, fidèle à l’idéal de libération qu’il poursuit avec ferveur.

Cependant, en 1932, en raison de difficultés financières, Nguyễn Kim Muôn est contraint de quitter Phú Quốc et retourne s’installer à Saïgon. Il y loue un local rue Hamelin et fonde un centre bouddhique où il imprime et vend des ouvrages religieux afin de subvenir aux besoins de ses disciples restés à Phú Quốc.

En 1933, un fidèle offre à Nguyễn Kim Muôn un petit sanctuaire dédié à la Miếu Bà Chúa Xứ\footnote{La miếu \textit{Bà chúa xứ} est un temple vietnamien, est dédiée au culte de Bà Chúa Xứ, une divinité très vénérée.}, situé dans les rizières en périphérie de Gia Định (ancienne route de Hàng Xanh). Il le rénove et lui donne le nom de Long Vân Tự (autorisation délivrée le 22 mai 1933). Dès lors, la doctrine bouddhique qu’il enseigne connaît un développement fulgurant. Des dizaines de disciples viennent alors se faire ordonner, et des milliers de pratiquants laïcs choisissent de se convertir et de suivre son enseignement.

Nguyễn Kim Muôn fait alors évoluer sa méthode de pratique spirituelle, passant de la voie de la pratique méritoire (\textit{Tu phước hữu vi}), c'est à dire composée de rituels, prosternations et prières, à une voie de sagesse méditative (\textit{Tu huệ vô vi}) centrée sur la méditation profonde et le retour à soi, sans dépendance à une autorité divine. Il encourage ses disciples à vivre de manière autonome : habits simples, travail artisanal comme la fabrication de tofu fermenté et de plats végétariens, vendus deux fois par mois lors des jours sacrés (le 1\textsuperscript{er} et le 15 du mois lunaire).

Concernant l’enseignement, Nguyễn Kim Muôn donne deux sessions quotidiennes (à l’aube et au crépuscule) durant lesquelles il prêche, enseigne les textes bouddhiques et guide la pratique méditative. Il est également réputé pour son talent poétique, capable de composer des vers spontanément, notamment dans la forme classique du \textit{bát cú liên hoàn} (poèmes en huit vers à rimes croisées). Il laisse d’ailleurs derrière lui des milliers de poèmes spirituels, véritables manuels de pratique méditative, précieusement conservés par ses disciples qui les considèrent comme un patrimoine doctrinal, rarement diffusé à l’extérieur.

Enfin, Nguyễn Kim Muôn maîtrise la dactylographie à un niveau remarquable : tout en conversant avec ses visiteurs, il est capable de taper à la machine de longs textes doctrinaux sans avoir besoin de rédiger de brouillon au préalable.


\subsection{\textit{Hội phước thiện nhà phật} denommée l'Œuvre de Charité (traduire par Association de bienfaisance bouddhique)}

\subsubsection{Présentation et contexte}

En 1933, Nguyễn Kim Muôn entame les démarches pour la création de l'association \textit{Hội phước thiện nhà phật} dénommée en français Œuvre de Charité. Il faut près de deux ans pour que l'administration française lui réponde, en lui octroyant l'autorisation (Autorisation numéro 1356 du 22 Mars 1935 de M. Michel Pagès le Gouverneur de la Cochinchine). Le premier congrès inaugural a lieu le 2 juin 1935, pendant lequel Nguyễn Kim Muôn expose l'esprit et les objectifs de l'Association.

L'œuvre de charité dirigée par Nguyễn Kim Muôn trouve son inspiration profonde dans ses interactions en France avec la Mission de Ramakrishna, une communauté bouddhiste du XX\textsuperscript{e} siècle\footnote{\textsc{BOURDEAUX}, Pascal, \textit{Bouddhisme Hòa Hảo, d’un monde l’autre : religion et révolution au Sud Viêt Nam (1935-1955)}, Paris, Les Indes savantes, « Vietnamica », 2022, p.94}. Cette organisation, fondée par Swami Vivekananda en 1897, se consacre à diffuser les enseignements de Ramakrishna qui prônent les valeurs universelles d'amour et d'harmonie religieuse. Son engagement dans des projets éducatifs, culturels, médicaux et de secours à travers le monde a profondément touché Nguyễn Kim Muôn, qui en a modelé sa propre vision de la charité, en adhérant au principe que le service à l'humanité est un service à Bouddha. Dans le journal \textit{Cùng Bạn}, du 16 Septembre 1933, à Long Vân, l'article intitulé « Autour de l'inauguration de la pagode Long Van » (\textit{Chung quanh vụ Khánh thành chùa Long vân}) contient également une photo du Bodhisattva Ramakrishna, dont l'œuvre de charité de Nguyễn Kim Muôn s'inspire\footnote{Đạo Nhân, \textit{Chung quanh vụ Khánh thành chùa Long vân} [Autour de l'inauguration de la pagode Long Van] [notre traduction], \textit{Cùng Bạn}, 18 septembre 1935}.

\begin{figure}[H] \centering \includegraphics[width=0.6\textwidth]{chua/cungban.jpg} \caption{Article du Journal \textit{Cùng Bạn}, 1933} \label{fig:bonzesses} \end{figure}

Ramakrishna Paramahamsa, dont les enseignements sous-tendent la mission, est un mystique indien du XIX\textsuperscript{e} siècle qui encourage la compréhension et l'acceptation entre les diverses traditions religieuses. Sa philosophie est que toutes les voies spirituelles mènent à un même but ultime et que servir les autres équivaut à servir le Dharma lui-même. Cette philosophie trouve un écho particulier en France, où les centres de la Mission de Ramakrishna servent non seulement à promouvoir les idées de leur fondateur mais aussi à encourager des initiatives de bien-être communautaire et interreligieux, reflétant ainsi une continuité avec les idéaux de la Renaissance du Bengale qui mêle spiritualité et modernité.

Inspiré par ce modèle, Nguyễn Kim Muôn établit son organisation bouddhiste pour apporter un soutien non seulement matériel mais aussi spirituel aux personnes défavorisées. La mission se distingue par sa critique du retrait du monde, souvent associée à la pratique spirituelle traditionnelle. Selon Nguyễn Kim Muôn, se retirer pour méditer seul est une démarche égoïste si elle n'est pas accompagnée d'un engagement actif envers le bien-être d'autrui. Cette perspective est renforcée par l'enseignement bouddhiste du \textit{thiện căn phước đức}, qui affirme que la véritable pratique spirituelle se doit d'être manifestée par des actes concrets de bienfaisance.

L'organisation mise en place par Nguyễn Kim Muôn vise à soutenir les individus dans le besoin à travers des distributions régulières de nourriture, de médicaments et d'assistance financière. Ces distributions, qui ont lieu lors de la nouvelle lune et de la pleine lune, montrent un engagement à utiliser les ressources de l'organisation de manière efficace et ajustée à ses capacités, tout en respectant des principes de compassion et d'aide mutuelle.

L'approche de Nguyễn Kim Muôn illustre une application pratique de l'idéalisme spirituel, transformant les principes de la Mission de Ramakrishna en un plan d'action concret qui non seulement répond aux besoins immédiats mais sert également de catalyseur pour un changement social plus large. Par ces efforts, Nguyễn Kim Muôn rend non seulement hommage à l'esprit de compassion et d'universalité prôné par Ramakrishna, mais adapte également ses enseignements à la réalité contemporaine, offrant un exemple frappant de la façon dont la spiritualité peut influencer positivement la société moderne.

\subsubsection{Structure organisationnelle détaillée de l'Œuvre de Charité} 

Composition du Comité d'Organisation\footnote{Nguyễn Kim Muôn, Statut, impr. Bảo Tồn, Saigon, 1935, p.2}
 :

\begin{itemize}
    \item Président : Nguyễn Kim Muôn, chargé de représenter l'association auprès des autorités, surveiller l'application des statuts, convoquer et présider les réunions du conseil ainsi que les assemblées générales, mais aussi gérer la correspondance. Le président a également le pouvoir de décision finale en cas d'égalité de vote au sein du conseil.
    \item Vice-présidents : Bà Nguyễn Thị Ngàn et Ông Huỳnh Văn Hổ assistent le président et prennent ses fonctions en son absence, avec priorité donnée au plus âgé.
    \item Secrétaire : Le Hòa thượng Minh Huyền est responsable de la rédaction des procès-verbaux du conseil d'administration et des assemblées générales, de la tenue des registres officiels, et de la correspondance avec les membres et le public.
    \item Secrétaire adjoint : Ni cô Diệu Tịnh assiste le secrétaire principal dans toutes ses tâches et le remplace en cas d'absence.
    \item Trésorier : Le Hòa thượng Minh Thành gère toutes les transactions financières de l'association, y compris la collecte des cotisations et le paiement des dépenses. Il est aussi chargé de la tenue rigoureuse des comptes et présente un bilan financier à chaque réunion du conseil et lors des assemblées générales.
    \item Trésorier adjoint : Minh Viển soutient le trésorier dans ses fonctions et assure l'intérim en son absence.
    \item Commissaires aux comptes : Un groupe incluant M. Nguyễn Văn Lành, Mme. Diệu Tạo, Nguyễn Văn Trọng, et Lê Văn Lực, chargé de superviser la gestion financière et d'assurer la régularité des comptes présentés.
    \item Conseillers juridiques : Composé de M. Phạm Văn Cang, M. Huỳnh Công Thanh, M. Võ Hiển Vinh, et Mme. Phủ Quí Trần Thị Thiệt, ce groupe conseille l'association sur les aspects légaux et assure la conformité avec les lois locales.
\end{itemize}

Chaque membre du conseil d'administration détient des responsabilités spécifiques qui facilitent la gestion quotidienne et stratégique de l'association. Les vice-présidents, par exemple, non seulement soutiennent le président, mais garantissent aussi la continuité de la gestion en son absence, assurant ainsi une stabilité organisationnelle. Le secrétaire, en tenant rigoureusement les registres des procès-verbaux, assure la transparence et le suivi des décisions prises. Le rôle des commissaires aux comptes est crucial pour maintenir l'intégrité financière, tandis que les conseillers juridiques garantissent que l'organisation reste en conformité avec les réglementations en vigueur.

L'architecture organisationnelle de l'association reflète un équilibre entre gouvernance, opérations et conformité, crucial pour toute œuvre de charité. La présence de multiples vice-présidents et la distinction entre les rôles de trésorier et trésorier-adjoint montre une attention particulière à la continuité opérationnelle et à la gestion des risques. Cette structure favorise non seulement un contrôle rigoureux des finances mais aussi une distribution claire des tâches pour améliorer l'efficacité et la réactivité de l'organisation face aux défis quotidiens.

La séparation des rôles et la spécialisation au sein du conseil permettent de répondre aux exigences administratives, financières, et légales de manière professionnelle. L'implication active des différents commissaires aux comptes et des conseillers juridiques souligne l'engagement de l'association envers la transparence et l'éthique, des qualités essentielles pour maintenir la confiance des donateurs, des bénéficiaires, et des autorités réglementaires. Cette approche stratégique assure non seulement la conformité et une gestion efficace de l'organisation, mais crée également un environnement propice à une croissance durable et à un impact social étendu.

\subsubsection{Statuts}

Le gouverneur de Cochinchine a approuvé les statuts de l'Association Bouddhiste qui comprend en totalité 7 chapitres et 35 articles, rédigés à partir du 24 septembre 1933. \footnote{Nguyễn Kim Muôn, Statut, impr. Bảo Tồn, Saigon, 1935, p.4}

L'article 1 spécifie le nom et le lieu du siège de l'Association, situé à la pagode Long Vân Tự, qui sert de centre pour organiser les activités caritatives et les réunions. 

Les articles 2 et 3 définissent le but de l'Association, qui est d'assister les personnes en difficulté telles que les handicapés, les pauvres, les malades et les chômeurs, tant sur le plan matériel que spirituel. L'Association organise des actions de secours deux fois par mois, les jours de nouvelle lune et de pleine lune, en distribuant de la nourriture, des médicaments et de l'argent.

L'article 4 interdit aux membres de discuter de politique nationale ou de participer à des activités qui ne correspondent pas aux objectifs de l'Association. 

L'article 5 classe les membres en deux catégories : les membres d'honneur, qui ont significativement soutenu l'Association ; et les membres actifs, qui respectent les statuts de l'Association, indépendamment de leur sexe ou de leur nationalité. 

Les articles 6 à 8 traitent des procédures d'adhésion, de départ, ou d'expulsion de l'Association. Les membres doivent soumettre une demande écrite au président de l'Association, et peuvent être exclus de la liste des membres après avoir été avertis et n'ayant pas payé leur cotisation pendant six mois, ou pour violation des règles internes et des lois, causant des troubles au sein de l'Association.

Les articles 9 à 16 définissent la structure du conseil d'administration, composé d'un président, de deux vice-présidents, d'un secrétaire et d'un secrétaire adjoint, d'un trésorier et d'un trésorier adjoint, de quatre contrôleurs et de quatre conseillers communautaires. Le conseil d'administration est élu lors de l'assemblée générale annuelle pour un mandat de trois ans et peut être reconduit. Les autorités coloniales françaises ont le droit de réviser et de demander l'exclusion de tout membre du conseil présentant des comportements anormaux.

Les élections du conseil d'administration se déroulent par vote secret, et les membres travaillent bénévolement. L'Association rembourse les frais de déplacement des membres qui utilisent des transports publics pour assister aux réunions. Tous les six mois, l'Association doit envoyer les procès-verbaux des réunions, la liste des membres, et les registres financiers au préfet de la province de Gia Định pour examen.

Les articles 29 à 31 traitent des finances de l'Association, incluant les cotisations des membres et les dons collectés lors des activités. Le président a l'autorité de décider des dépenses inférieures à 25 \textit{đồng}, les montants plus élevés nécessitant l'approbation du conseil d'administration. Le trésorier peut conserver en caisse un montant ne dépassant pas 150 \textit{đồng}, le reste devant être déposé à la Banque de l'Indochine à Saïgon.

Enfin, les articles 32 à 35 stipulent que tous les membres doivent respecter les termes des statuts. Les infractions sont sanctionnées selon la gravité des actes. Toute modification des statuts doit être approuvée par les autorités coloniales. En cas de dissolution de l'Association, ses biens sont transférés à l'Association Médicale de Gia Định.

On peut voir que les statuts de l'Association illustrent une structure organisationnelle rigoureuse et formalisée, ce qui est typique des organisations à cette époque en Cochinchine sous l'administration française. Cette structuration reflète non seulement un besoin de discipline et d'ordre mais aussi une tentative de se conformer aux exigences administratives imposées par les autorités coloniales. En interdisant aux membres de discuter de politique nationale ou de participer à des activités hors des objectifs de l'association (article 4), les statuts cherchent à limiter les risques de dissidence ou d'interférence politique, ce qui montre une prudence envers le contexte colonial sensible. De plus, la distinction entre les membres d'honneur et les membres actifs (article 5) permet à l'association de valoriser les contributions significatives tout en ouvrant la porte à une participation plus large et inclusive, indépendamment de l'origine des membres. Cela peut être interprété comme un effort pour renforcer le soutien communautaire tout en reconnaissant formellement les efforts des soutiens importants. D’ailleurs, les procédures d'adhésion et de discipline (articles 6-8) soulignent l'importance de la conformité et de la responsabilité parmi les membres. L'exigence de soumettre une demande écrite pour adhérer ou quitter l'association ainsi que les mécanismes pour gérer les manquements aux paiements ou les comportements perturbateurs assurent que l'association maintient un contrôle strict sur sa composition et ses finances. Les articles relatifs à la gestion financière (articles 29-31) révèlent aussi une préoccupation pour la transparence et la bonne gestion des fonds. En limitant le montant que le trésorier peut conserver en caisse et en exigeant que les montants plus importants soient déposés dans une banque, les statuts visent à prévenir les abus financiers et à garantir une comptabilité claire. Enfin, un des articles sur la gestion administrative inclus l'obligation pour les membres d'envoyer un rapport semestriel au préfet, ce qui indique un haut niveau de surveillance et de responsabilité, non seulement envers les membres mais aussi envers les autorités coloniales, assurant ainsi que l'association opère de manière transparente et conforme aux attentes réglementaires. Cette structuration et ces règles rigides sont probablement des mécanismes de prévention contre la corruption ou le détournement de fonds, des préoccupations courantes dans les organisations caritatives opérant dans des contextes complexes et potentiellement instables.

\subsubsection{Activités}

L'Association organise ses activités caritatives les jours de nouvelle lune et de pleine lune chaque mois, en présence d'employés du gouvernement pour garantir la transparence des opérations. Un secrétaire rédige un compte rendu de chaque session afin d'assurer une documentation claire et précise. Les actions de l'association se divisent en deux aspects principaux : matériel et spirituel. Sur le plan matériel, l'association fournit une aide directe sous forme de riz, de sel, de médicaments, notamment la Tubéradine\footnote{Le médicament Tubéradine servait pour le traitement de la toux, et plus particulièrement dans le cas de la tuberculose} dont la pagode Long Vân Tự était d'ailleurs le fournisseur et grossiste exclusif dans la région\footnote{Nguyễn Kim Muôn, Liên hoa đạo tập, [Le lotus de Ramakrisna], Bảo Tồn, Saïgon, 1934}, ou bien d'argent, visant à soulager immédiatement les difficultés quotidiennes des personnes démunies. Sur le plan spirituel, l'accent est mis sur la discipline morale pour promouvoir la pureté et la dignité.

Les membres de l'association sont encouragés à contribuer régulièrement, et l'association accepte également les dons de personnes non-membres. Les principaux bénéficiaires des efforts de l'association sont : 
\begin{itemize}
    \item Les personnes défavorisées qui cherchent activement à corriger leurs erreurs et à cultiver leur moralité, récompensant ceux qui, à travers des épreuves, réalisent leurs manquements et cherchent à s'améliorer par la pratique de la méditation, du végétarisme, et par le respect de préceptes religieux tels que la récitation de prières et l'étude des écritures.
    \item Les repentis qui, bien que faisant face à des difficultés personnelles, abandonnent leurs mauvaises habitudes telles que la mastication du bétel, le tabagisme, l'alcool, les jeux d'argent et la débauche.
\end{itemize} 

Sur le plan spirituel, Nguyễn Kim Muôn organise des sessions de prêche sur le bouddhisme et les principes éthiques de la vie, soulignant que la générosité matérielle ne devrait pas être une solution de long terme, afin d'éviter que les bénéficiaires ne deviennent dépendants et réticents à l'effort personnel. L'objectif est d'éduquer et de motiver les personnes à être actives et autonomes.

L'intégration par l'Association d'une aide matérielle directe (riz, sel, médicaments, argent) avec un soutien spirituel offre un modèle holistique de charité qui répond non seulement aux besoins immédiats, mais favorise également un développement personnel durable. Cette approche complète peut aider à briser le cycle de la pauvreté en adressant à la fois les symptômes et les causes profondes des difficultés. De plus, la présence de fonctionnaires lors des distributions de denrées auprès des bénéficiaires, ainsi que le maintien d'une documentation rigoureuse, illustrent un engagement de transparence et de responsabilité, essentiels pour maintenir la confiance des donateurs et des participants. En effet, ces fonctionnaires participent uniquement en tant qu'observateurs et non pas en tant qu'acteurs des activités associatives, et s'assurent que les opérations en cours sont bien en accord avec la réglementation locale, ce qui augmente leur légitimité et leur efficacité. En mettant l'accent sur la correction des comportements et l'amélioration personnelle, l'association s'efforce de créer un impact durable, plutôt que de simplement fournir une aide temporaire aux nécessiteux.

\section{L'affaire de 1935, controverses et scandales dans la presse} 
\subsection{Présentation détaillée de l'affaire}

Avant que l’affaire n’éclate en 1935, la journaliste Mộng Hoa\footnote{Mộng Hoa est une journaliste du journal quotidien \textit{Saïgon}. À ce jour, nous ne disposons pas de d'avantage d'information à son propos.} avait déjà écrit un article dans le quotidien \textit{Saïgon}, daté du 6 avril 1934, et intitulé \textit{Một khoản đời của ông Nguyễn Kim Muôn}, que l'on peut traduire par « Un chapitre de la vie de M. Nguyễn Kim Muôn ». Selon elle, Nguyễn Kim Muôn utiliserait le bouddhisme pour s'enrichir personnellement, et ce dernier ne se serait pas rendu à Phú Quốc pour méditer, mais pour échapper à ses créanciers. Selon ses dires, il aurait même extorqué d'importantes sommes d'argent aux habitants locaux en abusant de leur trop grande confiance envers la religion, faisant passer son enrichissement personnel pour une collecte d'offrandes.

C'est à la suite de la transmission n°1649, datée du 1\textsuperscript{er} mars et de la lettre n°1984 du 27 février 1934, que les autorités provinciales lancent une enquête approfondie sur Nguyễn Kim Muôn. Ces communications internes de l'administration font référence à un article intitulé \textit{Kẻ làm chồng, làm cha có tội}, que le bureau des traductions officiel de la Cochinchine traduit par « Les vrais coupables, ce sont les parents, les maris ». Cet article, paru dans le journal hebdomadaire \textit{Tân Thời} du 21 février 1934, dénonce les agissements répréhensibles du bonze chef Nguyễn Kim Muôn, responsable de la pagode Long Vân Tự, située dans le village de Bình Hòa Xã, à la limite de Thanh Mỹ An. Entre autre, l'article relate des témoignages qui sous-entendent des relations charnelles entre Nguyễn Kim Muôn et plusieurs de ses plus jeunes bonzesses. Ces communications internes nous confirment que l'administration coloniale établit un lien entre ces dénonciations et les activités de l'Association de bienfaisance bouddhique \textit{Hội phước thiện nhà phật}.

Voici un résumé des témoignages présents dans cet article, selon les communications internes de l'administration coloniale :
Ancien comptable à la Banque de l’Indochine, Nguyễn Kim Muôn dirige cette pagode depuis plus de trois ans. Il y a fait ériger plusieurs petites habitations autour du sanctuaire principal, pour héberger les bonzes et les bonzesses de sa communauté. Il occupe quant à lui une maisonnette isolée à l’arrière du complexe.

Or, de nombreux témoignages font état d’un comportement pour le moins troublant. En effet, le bonze aurait pour habitude de faire venir, de jour comme de nuit, des bonzesses dans sa maisonnette. Ces dernières y seraient enfermées avec lui pendant de longues heures, chacune à leur tour. Ce qui s’y passe reste un mystère, mais suscite de nombreuses interrogations.

Un autre témoignage rapporte que la nuit, vêtu d'une simple culotte, Nguyễn Kim Muôn aurait été vu en train de circuler autour de la pagode pour rendre visite aux bonzesses. De plus, lors des prières nocturnes, deux d’entre elles se tiennent systématiquement à ses côtés. L'article mentionne qu'il disposerait également d’un appareil photographique personnel, et que les bonzesses présentes au sein de la pagode sont majoritairement des jeunes filles originaires de Vĩnh Long (villages de Long Châu, Long An) et de la province de Sa Đéc. Elles sont placées sous la surveillance de Phạm Thị Dành, veuve d’un interprète de la gendarmerie à Vĩnh Long, connue dans la pagode sous le nom religieux de Diêu Tạo. Cette dernière est spécialement chargée de préparer les repas pour Nguyễn Kim Muôn et s’assoit à ses côtés pendant qu'il mange.

Même si les faits dénoncés dans la presse ne peuvent pas être attestés par des preuves irréfutables, ils paraissent néanmoins hautement vraisemblables. En effet, la rumeur publique évoque déjà des relations intimes entre le bonze et certaines bonzesses, notamment une jeune femme corpulente prénommée Thị Y, la nièce de Lê Văn Diên, un ancien \textit{Chánh lục bộ}\footnote{Chánh lục bộ : poste de fonctionnaire dans l’appareil administratif des dynasties féodales vietnamiennes, notamment sous les dynasties Lê et Nguyễn.} de Cần Giờ. Elle aurait été aperçue lui rasant la moustache dans sa chambre.

La lettre précise que l’année précédente, Nguyễn Kim Muôn avait déjà sollicité la création d’une société de bienfaisance nommée \textit{Hội phước thiện nhà phật}, mais que sa demande, enregistrée sous la lettre n°622 du 27 avril 1934, avait fait l’objet d’un avis défavorable en raison d’informations peu flatteuses concernant la pagode et son responsable.

Malgré ce précédent, l’association avait finalement été autorisée par l’arrêté n°1356 en date du 22 mars 1935, avec son siège social établi directement à la pagode Long Vân Tự. Ainsi, en raison des éléments préoccupants entourant ses activités, les autorités chargent le commissaire de police de Gia Định, ainsi que le délégué administratif de Gò Vấp, d’assurer une surveillance discrète mais constante de l’association, et de rendre compte régulièrement de son fonctionnement.

Enfin, la lettre précise que compte tenu de la moralité jugée douteuse de Nguyễn Kim Muôn, il est recommandé d’envisager sérieusement le retrait de l’autorisation récemment accordée. L’affaire arrive à un point critique avec le premier article du journaliste Bút Sơn\footnote{Bút Sơn était un pseudonyme utilisé pour signer ses articles de presse, tandis que, pour ses peintures, il signait généralement BS. Son véritable nom était Lê Minh Đức, né en 1914 à Saïgon et décédé en 1941 à Huế. Il était le benjamin d’une fratrie de quatre enfants, tous journalistes. Son père était originaire de Tân Định (Saïgon) et sa mère du village de Phong Thạnh, dans la province de Bạc Liêu. Orphelin de mère dès son jeune âge et vivant loin de son père, Bút Sơn fut élevé et pris en charge par ses frères et sœurs, en particulier par son frère aîné, le journaliste Lê Trung Nghĩa.}, journaliste et illustrateur pour le journal \textit{Tân Văn} à Saïgon\footnote{PHẠM, Công Luận, \textit{Biếm họa trên báo chí Sài Gòn trước 1975} [Caricatures dans la presse de Saïgon avant 1975], NXB Thế Giới, TP. Hồ Chí Minh, 2024,p97.}, publié le 29 juin 1935, sous la forme d’un récit d’enquête que nous allons récapituler ici :

Il y raconte sa visite à la pagode Long Vân Tự, un temple bien connu à l'époque, dans la région de Saïgon. Pourtant, la réputation de ce temple ne vient pas de sa sainteté, ni de la rigueur de sa pratique religieuse, mais plutôt des rumeurs étranges et persistantes qui l'entourent ; certains allant même jusqu’à le surnommer, en plaisantant à moitié, l’endroit où le roi des démons se cache sous l’apparence du Bouddha.

Le chef du temple est un personnage marquant : Nguyễn Kim Muôn, appelé Giai Minh dans la vie religieuse. Il est connu pour son talent oratoire exceptionnel, une éloquence presque hypnotique grâce à laquelle il attire des dizaines, voire des centaines d’adeptes, dont une grande majorité de jeunes femmes, souvent séduisantes.

Dès l’entrée dans la pagode, Bút Sơn ressent un  malaise, quelque chose de profondément anormal. La première pièce est un vaste salon, relié de chaque côté à deux petites chambres couvertes de rideaux sombres. Sur ces tentures, on peut y lire de nombreuses citations morales, brodées ou imprimées… en français. Cette mise en scène donne une impression étrange : celle d’une certaine piété affichée mais également celle d’un décor théâtral destiné à masquer quelque chose de plus obscur.
\clearpage
\begin{figure}[H] \centering \includegraphics[width=0.6\textwidth]{chua/butson.jpg} \caption{Portrait de Bút Sơn. Source:PHẠM, Công Luận, \textit{Biếm họa trên báo chí Sài Gòn trước 1975} [Caricatures dans la presse de Saïgon avant 1975], NXB Thế Giới, TP. Hồ Chí Minh, 2024,p.94} \label{fig:bonzesses} \end{figure}


En avançant, il découvre une grande salle à manger, et la scène qui s’y déroule est très troublante : plus de quarante convives y sont réunis, dont seulement sept hommes. Les jeunes nonnes sont quant à elles bien habillées, souriantes, et plaisantent gaiement. Il s'agit là d'une atmosphère qui rappelle bien plus une réception mondaine qu’un repas spirituel dans un monastère bouddhique.

Bút Sơn remarque alors que la table du chef spirituel est traitée avec un soin particulier, et qu'un verre de bière bien frais est déjà servi à côté de son bol de riz. Là où l’on s’attend à l’austérité d’un moine, on y trouve un homme savourant une bière en plein jour, sous les yeux de ses fidèles, sans que personne n’y voit rien à redire.

Après le repas, le journaliste est conduit à l’arrière, où se trouve un petit pavillon au centre d’un jardin, servant de salle d’enseignement religieux. Il n’y a ni autel, ni statue de Bouddha, ni encens, juste une grande estrade de bois. À la question de cette absence d’icônes, le moine guide répond avec un sourire : 

« Nous avons atteint un niveau de conscience tel que les formes extérieures sont devenues inutiles. Nous n’avons plus besoin de statues ni de sutras. »

Ce pavillon se trouve entre deux zones : à gauche, la résidence privée du maître ; à droite, les cellules des nonnes, deux par chambre. Lorsque Bút Sơn demande à les visiter, il se heurte à un refus immédiat. Tous les murs sont couverts de lourdes tentures sombres, et personne n’est autorisé à entrer. Selon lui, la pagode semble autant conçue pour prier que pour s'y cacher.

Un détail attire particulièrement son attention : un fil de fer long d’environ vingt mètres qui relie le pavillon à la résidence du chef religieux. Le guide lui explique qu’il s’agit d’une sonnette utilisée pour prévenir le bonze lorsqu'un visiteur souhaite s'entretenir avec lui. Bút Sơn se demande alors si ce dispositif n’a pas déjà pu être utilisé pour avertir Nguyễn Kim Muôn de l'arrivée imminente d'inspecteurs, expliquant ainsi les échecs répétés des enquêtes menées par les autorités.

Pendant toute la visite, le guide adopte une attitude de prudence extrême : aucune photographie n'est autorisée, avec un accès limité à certaines pièces seulement, et de vagues réponses mesurées. Selon le journaliste, il est évident que son guide a été entraîné à dissimuler et contrôler l’information.

Le reportage se termine sur une observation amère mais éloquente : la résidence du chef est contiguë aux chambres des nonnes, seulement séparées par le pavillon. En revanche, les dortoirs des autres bonzes se trouvaient eux bien plus loin. Et Bút Sơn de poser une question ironique : Cette disposition si discrète, si bien pensée, ne permet-elle pas au maître de profiter, en toute quiétude, des nuits de pleine lune en belle compagnie ?

Ce premier article fut un choc. Sans accuser formellement Nguyễn Kim Muôn, Bút Sơn, par le biais d’observations minutieuses, de détails troublants et d’un style chargé d’ironie, pousse le lecteur à douter.
S’agit-il vraiment d’un temple, ou d’un masque religieux cachant désir, argent et pouvoir ?\\
\noindent\rule{0.35\linewidth}{0.6pt}

\clearpage
Le deuxième article, publié le 6 juillet 1935 sous le titre \textit{Núp sau lưng Phật} (Derrière Bouddha), poursuit l’enquête avec un témoignage inédit.

\begin{figure}[H]
    \centering
    \includegraphics[width=0.5\textwidth]{images/nupsaulungphat.jpeg}
    \caption{Article du journal \textit{Tân Văn} - source : Gallica}
\end{figure}

Cette fois, Bút Sơn adopte un ton plus intimiste. Il raconte sa rencontre avec une vieille nonne, âgée d’environ cinquante ans, vivant avec sa fille de quinze ans dans une maison isolée, près de la pagode.

Au début, elle refuse de parler, manifestement hantée par la peur. Mais au bout d’un long silence, elle se confie, non par colère, mais comme si elle déposait un fardeau qu’elle avait porté depuis trop longtemps.

Selon elle, le maître de la pagode, Sư Muôn (Nguyễn Kim Muôn), est un orateur redoutable et son charme est tel qu’il parvient à séduire sans effort de nouveaux adeptes. Mais, ajoute-t-elle, ce chef religieux ne traite pas tout le monde de la même manière. Il privilégie les riches, ceux qui peuvent apporter une contribution financière, plutôt que ceux animés d’une vraie foi.

Alors qu’il prêche le renoncement et la vie simple, Sư Muôn vit dans un confort extravagant : il dispose d'un coffre rempli d’argent, d'une automobile Talbot brillante avec laquelle il effectue des promenades quotidiennes, d'oiseaux d’ornement, et boit des bières à volonté. Elle se souvient aussi d'un mendiant venu faire l’aumône qu'il aurait repoussé sans pitié, et chassé sans un mot de compassion.

« Un tel homme est-il encore digne d’être à la tête d’un ordre religieux ? » murmure-t-elle, les yeux perdus au loin.

Elle marque une pause, et poursuit à voix basse. Selon Bút Sơn, ce qu’elle dit alors semble à peine croyable, même pour elle : \\
Sư Muôn n’est pas insensible à la beauté féminine. Il s'entoure, pour ses soins particuliers, des plus jeunes et jolies nonnes, les chargeant de lui raser la tête, de laver ses vêtements, et de faire son lit. Parmi elles, deux sortent du lot : Diệu Y, 18 ans, et Diệu Hồng, 21 ans. Diệu Hồng, en particulier, est souvent convoquée dans sa chambre pour lui apporter du café, et y reste pendant des heures. 

Un jour, la vieille nonne Diệu Huê est entrée à l’improviste et a surpris une scène compromettante. Elle s’est indignée, mais en réponse, Sư Muôn l'a frappée violemment avec un balai, en l’insultant grossièrement. De plus, elle affirme l’avoir vue, à de nombreuses reprises, plaisanter et toucher les nonnes de manière déplacée, allant même jusqu'à leur taper sur les fesses, sans aucune retenue ni honte.

Mais le pire reste à venir car, selon elle, il est d’usage à Long Vân Tự que les moines et les nonnes se baignent ensemble dans un petit ruisseau devant la pagode, et ce entièrement nus. Une coutume qu’elle qualifie de « révoltante », à la limite de l’orgie spirituelle.
« Hommes et femmes, tous à nu, comme s’ils participaient à une fête de la débauche. Est-ce cela, la voie de la purification ? » demande-t-elle d’une voix tremblante.
Puis elle conclut, dans un souffle :
« Viennent-ils ici pour laver les impuretés du monde, ou pour s’enfoncer plus profondément dans la boue du désir ? »

Puis, comme nous le relate le journaliste, elle se retire ensuite, sans haine ni larmes, juste une fatigue infinie, celle de quelqu’un qui a trop vu, trop su, et qui ne croit plus aux apparences.

Par la suite, dans le numéro du \textit{Tân Văn} daté du 27 juillet 1935, en réponse à la série d’enquêtes qui avait suscité une grande agitation dans les milieux bouddhistes et au-delà, Nguyễn Kim Muôn publie un article intitulé \textit{Tu trước mặt người} (Pratiquer la religion devant les hommes), sous forme de lettre ouverte adressée à l’opinion publique et au journaliste Bút Sơn.


Il y explique que son silence jusqu’ici n’était pas dû à l’ignorance des calomnies proférées contre lui, mais au fait qu’il était trop absorbé par sa pratique religieuse et par la formation de ses disciples. Cependant, ces derniers temps, les articles publiés dans \textit{Tân Văn}, notamment ceux signés par Bút Sơn, étaient devenus si virulents et personnels qu’ils l’avaient contraint à sortir du silence monastique pour s’exprimer clairement et publiquement.

Tout d'abord, Nguyễn Kim Muôn commence par rejeter vigoureusement l'accusation de Bút Sơn qui l'a qualifié de bête féroce. À ses yeux, il s’agit d’une insulte cruelle et injustifiée. Il affirme que même un criminel ne mérite pas un tel qualificatif, et souligne qu’il n’a jamais séduit, manipulé, ni nui à qui que ce soit. Il dénonce une attaque dégradante contre sa dignité personnelle et spirituelle.

\begin{figure}[H]
    \centering
    \includegraphics[width=0.5\textwidth]
    {images/tutruocmatnguoi.jpeg}
    \caption{Article du journal \textit{Tân Văn} - source : Gallica}
\end{figure}

Sur la question du fonds de secours pour les pauvres, il dément catégoriquement toute accusation de détournement et déclare aussi être le fondateur même de ce fonds, auquel chaque membre contribue par une modeste cotisation mensuelle, suffisante pour assurer uniquement le fonctionnement de la pagode. La comptabilité, selon lui, est tenue de manière transparente, vérifiée régulièrement et auditée par un comité de surveillance intègre, composé d’hommes sans tache ni reproche.

En ce qui concerne le fameux coffre-fort présenté comme regorgeant de billets, Nguyễn Kim Muôn reconnaît son existence, mais précise qu’il s’agit là d’un coffre en étain, propriété personnelle de la nonne Diệu Thanh. Un tel coffre, d’après lui, ne peut en aucun cas contenir d’importantes sommes d’argent. Il se dit stupéfait qu’un journaliste réputé comme Bút Sơn ait pu publier des affirmations non vérifiées, fondées sur de simples rumeurs. Il affirme également que lui-même et ses disciples vivent de leur propre labeur et non grâce aux offrandes ou aux donations. Ils produisaient du \textit{tương} (pâte de soja), du \textit{chao} (tofu fermenté), et vendent parfois des ouvrages religieux dont il est lui-même l'auteur. Ces modestes ventes leur permettent de subvenir à leurs besoins et de payer les impôts, car, souligne-t-il, ils ne sont exempts d’aucune charge publique. Bien que leur vie soit simple, elle reste honnête et indépendante.

À propos de certains détails comme le fait de conduire une voiture ou de boire de la bière, il répond qu’il s’agit là de libertés personnelles, sans conséquences néfastes pour autrui, et ne méritant donc pas d’être jugées sévèrement.

Sur l’absence de statues de Bouddha dans sa pagode, Nguyễn Kim Muôn ne nie rien. Il explique que lui et ses disciples suivent une voie religieuse dépouillée de formes extérieures, libérée des rituels, des symboles et même des divinités. Il déclare clairement qu’ils ne vénéraient ni Bouddha, ni Dieu, ni esprits, ni anges. Leur pratique consiste uniquement à se connaître soi-même, à ne faire aucune distinction de classe ou de race, et à se libérer par leurs propres efforts des souffrances de la vie terrestre.

Concernant les bains mixtes entre bonzes et nonnes, il reconnaît cette pratique, mais affirme qu’il s’agit là d’une activité physique saine, équivalente à n’importe quel sport, comme le football ou le tennis. Pour lui, les moines et les nonnes ont aussi besoin de cultiver leur corps, et tant que ces bains se font dans un esprit pur, il n’y a rien là de répréhensible.

Il admet que sa manière de pratiquer peut désorienter les bouddhistes traditionnels, mais insiste sur le fait que la voie religieuse doit évoluer. Lui et ses adeptes ont abandonné les conventions et les rituels qu’ils jugeaient obsolètes, pour suivre ce qu’il appelle la voie naturelle : une spiritualité lumineuse, sans artifice, fondée sur l’autodiscipline et l’auto-libération. Nguyễn Kim Muôn termine ainsi son texte par une affirmation forte et directe :

\begin{quote}
« Je ne me cache derrière aucune statue de Bouddha, car il n’y en a pas là où je médite. »
\end{quote}

Il finit en invitant ceux souhaitant démasquer des faux religieux, à se rendre à Baria ou à Châu Đốc, des lieux où, selon lui, subsistent encore des formes superstitieuses déguisées en religion. L’article se conclut par sa signature en tant que Président de l’Association bouddhique d’assistance sociale, comme pour souligner qu’il n’a rien à cacher, qu’il ne fuit rien, et qu’il assume pleinement et publiquement la voie qu’il a choisie.

Par la suite, Bút Sơn poursuit  son article dans le journal \textit{Tân Văn} ; cette fois, il affirme sans détour que Nguyễn Kim Muôn n’est qu’un faux dévot : un imposteur qui se cache derrière la religion pour mieux exploiter ses semblables, et un débauché incapable de maîtriser ses passions charnelles.

Selon lui, si Nguyễn Kim Muôn souhaite mener une vie de débauche, soit ; mais encore faudrait-il qu’il ôte d’abord sa robe de bonze et cesse de se cacher derrière une fausse image d’ermite, ajoute-t-il avec amertume. Ensuite, Bút Sơn tourne en dérision la déclaration de Nguyễn Kim Muôn selon laquelle il pratiquerait la continence. En réalité, souligne-t-il, le bonze entretiendrait des relations sexuelles avec des bonzesses. Fait encore plus troublant, une rumeur persistante lui prête même le pouvoir d’empêcher toute conception lors de ces relations.

De plus, le journaliste accuse Nguyễn Kim Muôn d’administrer aux nouvelles converties une soi-disant eau sainte, qui contiendrait, en vérité, une substance aphrodisiaque dissoute.
Ce breuvage aurait pour effet, selon l’auteur, d’altérer le discernement de ces jeunes femmes, au point que, sous l’influence des désirs, une sainte ne saurait résister aux avances du chef bonze. Par ailleurs, le journal signale qu’il existe, derrière la pagode, une chambre discrète, ainsi qu’une barque amarrée au bord de la rivière qui longe le temple. Ces éléments suggèrent fortement l’existence d’un lieu réservé à ses pratiques sexuelles.

Mais l'accusation la plus grave repose sur le témoignage d’un certain H.D., une connaissance de Nguyễn Kim Muôn, à qui ce dernier se serait confié dans l’intention de le convertir. Offusqué par ces révélations, H.D. explique avoir rejeté sa proposition. Puis, peu après cet événement, en feuilletant un des livre religieux appartenant au bonze, il affirme avoir découvert une photographie d’une bonzesse dans une position extrêmement indécente.
Pour ne pas heurter son lectorat, le magazine \textit{Tân Văn} décide bien entendu de ne pas publier le cliché, mais l'histoire ne s'arrête pas là. En effet, le journaliste révèle que Nguyễn Kim Muôn va jusqu’à photographier les femmes avec qui il a des relations, alors qu’elles sont encore nues. Selon lui, lorsqu’il a satisfait ses désirs sexuels, Sư Muôn pousserait l’excentricité jusqu’à photographier celles avec qui il est en relation, dans sa tenue d’Ève. L’article se conclut alors par une accusation cinglante : derrière les apparences d’un bonze respectable, Sư Muôn aurait en réalité l’âme d’un satyre. Par cette publication, le journal \textit{Tân Văn} appelle clairement ses lecteurs à dénoncer et à boycotter non seulement Nguyễn Kim Muôn, mais aussi la pagode Long Vân.

\clearpage
Le 2 septembre 1935, Nguyễn Kim Muôn, supérieur de la pagode Long Vân, adresse un rapport officiel à l’Administrateur, chef de la province de Gia Định, afin de signaler des événements graves survenus la veille dans l’enceinte de la pagode.

Dès l’ouverture de sa lettre, il affirme être la cible d’une campagne de presse prolongée. Cette médiatisation provoque un afflux massif de curieux à la pagode Long Vân Tự, générant des troubles importants à l’ordre public, au point qu’il redoute de ne pouvoir contenir la situation sans l’appui des forces de police.

Selon son témoignage, le 1\textsuperscript{er} septembre 1935, un journaliste annamite nommé Bút Sơn, représentant du journal \textit{Tân Văn}, s’est rendu à la pagode accompagné de plusieurs amis et collaborateurs, muni d’un appareil photographique. Ce groupe aurait empêché l’ouverture de la séance du Conseil d’administration de l’Œuvre de charité dirigée par Nguyễn Kim Muôn. Pire encore, ils auraient poussé la foule à envahir la salle où se trouvait le bureau du comité.

Bien que le Délégué administratif de la région de Gò Vấp ait demandé à plusieurs reprises à la foule de se calmer, ces appels sont restés lettre morte. La réunion a donc dû être reportée, suite à un accord entre le délégué et le comité de l’Œuvre.

Une fois le Délégué de Gò Vấp et le Commissaire de police de Gia Định repartis, la situation a empiré. Un autre groupe de personnes a envahi l’intérieur de la pagode et a violemment provoqué Nguyễn Kim Muôn. Pour éviter toute altercation physique, il a appelé ses sympathisants au calme et à la retenue.

Nguyễn Kim Muôn désigne nommément deux personnes comme responsables directs de cette agitation :

1. Cao Chánh – Celui-ci a pris la parole en vietnamien devant la foule, en évoquant le droit de réunion, la liberté de parole et de conscience, dans le but de contraindre Nguyễn Kim Muôn à répondre publiquement aux accusations de la presse, qu’il considère comme calomnieuses.

2. Thầy giáo Hạnh – Enseignant dans une école primaire située rue Richard à Saïgon, il a proclamé devant l’assemblée, en français : « La masse est au-dessus du Gouvernement, vous devez parler et non vous retrancher derrière la loi. »
   (<<\textit{Quần chúng đứng trên chính quyền, ông phải lên tiếng thay vì trốn sau pháp luật.}>>)

Selon Nguyễn Kim Muôn, ces déclarations, malveillantes par nature, ont exacerbé la foule, entraînant un désordre accru. Ces faits se sont déroulés en présence de deux Français, dont il promet de communiquer les noms en temps voulu.

Nguyễn Kim Muôn insiste sur le fait que l’Œuvre de charité qu’il dirige est une institution privée, légalement autorisée par le gouvernement français. Face aux risques de violence et de diffamation, il demande l’intervention, le soutien et la protection des autorités. Sur le plan juridique, il indique avoir mandaté un avocat afin d’intenter une action en justice pour diffamation, avec l’objectif de défendre son honneur face à ce qu’il considère être une campagne de presse malveillante.

En conclusion, Nguyễn Kim Muôn déclare officiellement qu’à compter du 2 septembre 1935, toute visite à la pagode Long Vân Môn est interdite, sans exception – une mesure préventive visant à garantir la sécurité et le bon ordre dans l’enceinte religieuse. Il termine sa lettre avec des formules respectueuses, en soulignant sa fidélité aux autorités et son espoir d’être soutenu.



Le 5 septembre 1935, Nguyễn Văn Thiết, directeur du journal \textit{Tân Văn}, adresse lui aussi une lettre aux autorités de Gia Định pour demander l’autorisation d’organiser une conférence publique à la pagode Long Vân. Il souhaite poser directement à Nguyễn Kim Muôn des questions sur les accusations portées contre lui dans la presse et recueillir ses explications de vive voix.

Mais le 7 septembre, Nguyễn Kim Muôn quitte soudainement la pagode pour se rendre à Baria. Le lendemain, 8 septembre, Nguyễn Văn Thiết transmet aux autorités provinciales l’ensemble des articles publiés à son sujet, accompagnés de deux photographies de bonzesses nues, dans le but de renforcer sa plainte contre lui.

Le matin du 9 septembre 1935, l’administrateur de la province de Gia Định se rend personnellement à la pagode Long Vân Tự pour y mener une visite d’inspection. Voici le résumé de son inspection : 

Cette pagode, située en périphérie de la ville, est dirigée par un certain Nguyễn Kim Muôn, dont la réputation suscite depuis quelque temps de sérieuses réserves de la part des autorités locales. Dès son arrivée, l’administrateur est frappé par l’étrangeté du lieu. L’endroit se présente extérieurement comme une pagode, mais l’atmosphère qui y règne semble bien éloignée du dépouillement et de la sérénité attendus dans un sanctuaire bouddhique. À l’intérieur, Nguyễn Kim Muôn le reçoit avec un air à la fois courtois et nerveux. Lorsqu’il est interrogé sur sa formation religieuse, l’homme ne peut présenter aucun document attestant d’un cursus traditionnel : ni diplôme, ni certificat de noviciat, ni trace d’une quelconque ordination. Il avoue alors, sans détour, avoir étudié la morale bouddhique par lui-même, et explique qu’il l’applique selon ses propres interprétations spirituelles, sans attachement strict à une école ou à une lignée doctrinale.

La visite se poursuit dans les différents bâtiments annexes. L’administrateur découvre qu’autour de la pagode vivent plusieurs femmes, qui se disent bonzesses. Pourtant, elles conservent toutes leurs cheveux - un fait inhabituel, voire contraire à la discipline monastique. Certaines sont encore très jeunes, à peine sorties de l’adolescence. Leur apparence juvénile, leur démarche vive, leurs regards curieux contrastent fortement avec la sobriété et l’humilité attendues d’une vie religieuse. Cette présence féminine constante, en si grand nombre, jette le doute sur la rigueur de la vie ascétique censée être pratiquée en ces lieux.

L’espace personnel de Nguyễn Kim Muôn intrigue particulièrement. Il s’est fait construire, à l’écart des autres bâtiments, une cellule en dur à deux étages. Le rez-de-chaussée lui sert de bureau. À l’étage, une petite chambre à coucher, modeste mais bien isolée. Cette cellule n’est pas reliée à la pagode principale, et semble avoir été pensée pour lui offrir un isolement complet, à l’abri des regards. Les femmes et les enfants, eux, vivent dans des dépendances séparées, proprement tenues, bien ordonnées, comme une maison de famille plus qu’un couvent.

Quant au bâtiment religieux lui-même, il est extrêmement sommaire. On y trouve un seul autel, sans statue, sans encens ni offrandes visibles. Devant cet autel, une large estrade en bois accueille les fidèles lors des cérémonies. Plus surprenant encore : une petite estrade surélevée est réservée à Nguyễn Kim Muôn, tournée face au public, lui permettant d’occuper une position dominante pendant les offices. Le dispositif, bien loin des codes liturgiques habituels, renforce l’impression d’un culte de la personnalité, centré sur la figure du chef plutôt que sur la doctrine du Bouddha.

Dans la cellule privée du maître des lieux, un objet retient l’attention de l’administrateur : un ouvrage intitulé << La Femme >>, du docteur Galtier-Boissière. Le ton du livre, ainsi que ses sujets traitant de psychologie féminine et des relations sociales, sont jugés peu compatibles avec une vie spirituelle rigoureuse. Lorsqu’il comprend que l’administrateur a remarqué le livre, Nguyễn Kim Muôn tente visiblement de le faire disparaître. Quelques instants après le départ de la délégation, le livre est recherché, mais ce dernier a été brûlé à la hâte. Seuls des fragments calcinés sont retrouvés sur place, que l’administrateur fait placer sous pli séparé afin de les transmettre à ses supérieurs comme pièce à conviction.

Interrogé sur la propriété du lieu, Nguyễn Kim Muôn affirme être le propriétaire légal du terrain ainsi que de l’ensemble des bâtiments. Il produit également une carte-diplôme de bonze, délivrée à Hà Tiên. Toutefois, cette carte n’est assortie d’aucun certificat religieux en bonne et due forme. Elle semble n’être qu’une formalité administrative, sans fondement spirituel réel. Ce vide institutionnel inquiète l’administrateur, qui voit là un usage opportuniste du statut religieux à des fins qui semblent personnelles.

Face à la gravité de la situation et à l’accumulation de soupçons, Nguyễn Kim Muôn finit par proposer deux options : soit il démissionne de son poste de président de l’Œuvre de Charité, auquel cas la pagode ne servira plus de siège à l’association ; soit il convoque une assemblée générale pour demander la dissolution pure et simple de l’Œuvre.

Dans sa conclusion, l’administrateur exprime de sérieuses réserves : pour lui, Nguyễn Kim Muôn ne peut pas être considéré comme un religieux légitime du culte bouddhique, car il ne dispose d’aucune reconnaissance officielle et son mode de vie contredit les principes fondamentaux de cette tradition. Il estime que l’Œuvre de Charité pourrait bien n’être qu’un paravent servant des intérêts privés. Il recommande ainsi, dans son rapport, que l’autorisation administrative accordée à cette Œuvre soit retirée, ce qui entraînerait sa dissolution immédiate. Il demande également qu’une enquête soit ouverte sur la validité du statut de bonze revendiqué par Nguyễn Kim Muôn, afin que, si nécessaire, sa carte-diplôme soit annulée. La pagode Long Vân Tự, conclut-il, ne saurait être considérée comme une véritable pagode. Elle n’est, dans sa structure, son usage et son esprit, rien d’autre qu’une maison particulière aménagée autour d’une figure centrale autoproclamée, et n’a ni la forme ni la vocation d’un lieu de culte authentique.

Le 10 septembre, un nouveau rapport surgit, rapporté que le bonze Minh Huyền, accusé d’avoir abusé d’une bonzesse, il a été expulsé de la pagode par Nguyễn Kim Muôn. Il nourrirait une certaine rancune envers lui et chercherait désormais à se venger.

Face à l’agitation grandissante, le chef local des services de police du Gouverneur de la Cochinchine écrit à l’administrateur de la province de Gia Định, le 12 septembre 1935. Il indique que l’enquête en cours n’a pas permis de prouver que le comportement de Nguyễn Kim Muôn soit à l’origine du scandale. En conséquence, les faits reprochés ne justifient pas une intervention directe des autorités, qui risquerait au contraire de porter atteinte au prestige du pouvoir colonial.

Selon lui, les campagnes de presse sont avant tout orchestrées par des adversaires de Nguyễn Kim Muôn, dont la conduite morale n’est pas plus exemplaire que la sienne. Il ajoute que vérifier la validité des éléments fournis par le journal exigerait la convocation d’un grand nombre de témoins et qu'une telle procédure serait complexe et sensible. En effet, elle pourrait provoquer de nouvelles tensions, d’autant que certains documents du dossier ont été dérobés à leur propriétaire et touchent à leur vie privée.

Dans ces conditions, il propose de recommander aux plaignants de saisir la juridiction compétente en matière religieuse. Toutefois, puisqu’un certain nombre de faits répréhensibles semblent avérés, il envisage deux mesures administratives : d’une part, la révocation de l’autorisation n°1356, délivrée le 22 mars 1935 à l’association fondée par Nguyễn Kim Muôn ; d’autre part, la fermeture de la pagode Long Vân Tự.

Il joint à sa lettre plusieurs pièces recueillies par l’administrateur de Gia Định : des informations personnelles sur Nguyễn Kim Muôn, un extrait du rapport d’un agent envoyé sur place pour observer les activités des bonzes et bonzesses, ainsi qu’un résumé traduit des lettres envoyées par Nguyễn Kim Muôn à Nguyễn Văn Thiết.

Le 30 septembre 1935, le chef des services de police du Gouverneur de la Cochinchine adresse un nouveau courrier à l’administrateur. Il estime que l’agitation médiatique autour de Nguyễn Kim Muôn est en train de retomber, et qu'il convient de ne pas raviver les troubles par une action officielle. Les intentions réelles des détracteurs de Nguyễn Kim Muôn restent floues et ne permettent pas aux autorités de prendre clairement position.\footnote{Lettre confidentielle signée par Michel Pagès, gouverneur de la Cochinchine, à l'administrateur, chef de province de Gia Định, le 30 septembre 1935}

Il reconnaît toutefois que Nguyễn Kim Muôn semble avoir détourné l’association caritative à des fins personnelles, et que les autorités ne sauraient être tenues pour responsables de l’usage qu’il en fait. Il demande donc à l’administrateur de convoquer Nguyễn Kim Muôn dès son retour à la pagode, afin de lui demander de supprimer les noms des autorités françaises (le Gouverneur, l’Administrateur, le Chef de province, etc.) de la liste des membres d’honneur de son association, qu’il aurait ajoutés sans autorisation. En parallèle, il souhaite que l’administrateur vérifie si les statuts de l’association \textit{Hội phước thiện nhà phật} sont bien en conformité avec la loi, tant en ce qui concerne son mode d’administration que ses objectifs. Enfin, conformément à une recommandation antérieure, il confirme qu’une enquête doit être lancée sur la validité des diplômes de Nguyễn Kim Muôn, notamment les conditions dans lesquelles lui a été délivrée sa carte de bonze, afin de déterminer si elle est légitime. Il conclut en affirmant que la fin définitive des polémiques est une condition \textit{sine qua non} à la prise de sanctions par les autorités envers Nguyễn Kim Muôn sans qu'elles soient soupçonnées d'avoir un quelconque parti pris ou de défendre les intérêts d'une influence extérieure.

Dans \textit{Tiểu sử Sư Muôn}, la biographie de Maitre Muôn, document interne de la pagode Long Vân, le bonze Minh Út relate cette période tumultueuse qui entoure son maître. Il explique que, à l’époque, le journal \textit{Tân Văn}, jusque-là peu populaire, et ne vendant pas plus d’une centaine d’exemplaires par jour, a connu un succès fulgurant dès la publication d'articles à charge contre Nguyễn Kim Muôn. Pendant près de trois mois, les ventes explosèrent, preuve, selon lui, de la curiosité du public pour cette affaire. Il poursuit en expliquant que les visiteurs affluaient en nombre au temple Long Vân Tự. Intriguée par les accusations, cette foule transformait ce lieu de culte en une véritable foire. Il raconte que Minh Thành, un disciple, a demandé un jour à un jeune novice de compter les visiteurs à l’entrée de la pagode. Du matin au soir, ce dernier en aurait comptabilisé jusqu’à 1 400, même sous une pluie battante. Les fidèles laissant énormément de traces de boue derrière eux avec leurs sandales, les disciples ont littéralement pu ramasser un panier rempli de boue à la fin de la journée, en nettoyant les carreaux de terre cuite de la pagode.

\begin{figure}[H]
    \centering
    \includegraphics[width=0.8\textwidth]{chua/bao.jpg}
    \caption{Article du journal Le populaire à propos de l'affaire de 1935. Source: Journal Le Populaire du 11 septembre 1935}
\end{figure}

Devant ce tumulte, certains disciples laïcs, voyant leur maître injustement attaqué, expriment le souhait de riposter en publiant des articles en réponse, voire de poursuivre les journalistes en justice. Mais Nguyễn Kim Muôn refuse. Il explique que toute personne entrant dans une pagode devient un observateur social. Ainsi, il affirme que depuis le début des troubles, des dizaines de milliers de visiteurs sont venus au temple, et que cela suffit à prouver la vérité ou le mensonge. Nul besoin donc, selon lui, de se justifier, de répondre ni même de porter plainte, car cela entacherait la patience et la dignité d’un religieux. Il accepte ces épreuves comme les fruits d’un karma antérieur et affirme qu’il les assumera entièrement. Quant au journaliste Bút Sơn, aussi malveillant soit-il, Nguyễn Kim Muôn croit qu’un jour, sa conscience finira par le rattraper et qu'il regrettera ses actions inhumaines et injustes.

Devant l’ampleur de l’affaire, le \textit{Tham biện} (Administrateur) de la province de Gia Định rédige un rapport au Gouverneur général pour demander le retrait de l’autorisation de l’association \textit{Hội phước thiện nhà phật}. Cependant, le Gouverneur rétorque que, sauf infraction avérée au règlement de l’association, il n’y a aucune base légale pour la retirer. Finalement, les autorités locales se contenteront de poster des soldats à l’entrée de la pagode pour en assurer la sécurité. Le tumulte s'apaise enfin, après cent jours de tourments, et le maître et ses disciples peuvent retrouver enfin leur sérénité et une vie monastique paisible.

Enfin, d'un point de vue objectif, Minh Út souligne qu’autrefois, Nguyễn Kim Muôn travaillait à la Banque de l’Indochine avec un salaire de près de 400 piastres par mois, une somme considérable bien supérieure à celle d’un chef de district de l’époque, et qu'il aurait très bien pu mener une vie de luxe et de plaisir en suivant cette voie. Il a pourtant choisi d’abandonner honneurs et richesses pour se consacrer entièrement à la voie bouddhique et à l’enseignement des fidèles. Ainsi, selon Minh Út, il est absurde de croire aux accusations de déchéance morale colportées par Bút Sơn.
\\
\noindent\rule{0.35\linewidth}{0.6pt}
\clearpage
Durant cette période difficile, Nguyễn Kim Muôn compose d'ailleurs un poème exprimant son acceptation du karma et sa foi inébranlable dans la vertu:

\begin{verse}
\textit{Cho hay quả nghiệp Phật cho nhồi \\
Mắc mớ chi người xúm nổi sôi \\
Muôn kẻ a dua đào gốc rễ, \\
Một mình ngay thẵng vững cây chồi, \\
Chánh ngôn nhược phản màng chi vội \\
Giả nghịch cùng chơn phải vậy rồi. \\
Qua khỏi trăm ngày êm tợ ngủ, \\
Từ đây dứt hết việc lôi thôi.
}
\end{verse}

\begin{quote}
\og Qu’il soit clair que c’est le karma que le Bouddha m’a assigné. \\
Pourquoi donc tant de gens s'agitent autour de moi ? \\
Des milliers d’imitateurs cherchent à m’arracher jusqu’à la racine, \\
Mais seul, droit, je garde la pousse droite et ferme. \\
Même si la vérité semble niée, pourquoi se précipiter à répondre ? \\
Le faux s’oppose toujours au vrai, ainsi en est-il. \\
Après cent jours, tout redevient paisible comme le sommeil, \\
À partir de maintenant, plus aucun tumulte. \fg{}
\end{quote} [notre traduction]


Après cent jours de tumulte, le maître et ses disciples retrouvent enfin le chemin de la pratique spirituelle, dans la sérénité et la paix, comme il se doit.

\subsection{Analyse de l'affaire}
\subsubsection{Présentation des sources de l'affaire}
Le dossier relatif à l’affaire Nguyễn Kim Muôn repose sur un ensemble de sources variées et complémentaires, qui permettent de retracer le déroulement des événements, d’en analyser la portée sociale et religieuse, et de comprendre les stratégies de discours adoptées par les différents acteurs. Il se compose principalement de trois types de documents, aujourd’hui conservés dans plusieurs institutions.

Tout d’abord, l’essentiel des documents officiels se trouve au Centre des Archives nationales II du Vietnam, au sein du fonds << Gouvernement Cochinchine >>, dans le dossier coté 26031, intitulé « Dossier relatif à la campagne de presse contre le bonze Nguyen Kim Muon et à la propagande théosophique, années 1935–1936 ». Il contient des correspondances confidentielles échangées entre la police coloniale, l’administrateur de la province de Gia Định et le gouverneur général de l’Indochine. Ces documents administratifs comprennent des rapports d’inspection, des télégrammes, des comptes rendus de réunion, ainsi que des pièces annexes telles que des extraits de journaux, des lettres de plainte ou des notes manuscrites. Ils offrent un regard institutionnel sur le scandale, en montrant comment les autorités perçoivent l’affaire, y réagissent, hésitent à intervenir ou tentent de la contenir.

Le deuxième ensemble fondamental de sources provient de la presse vietnamienne de l’époque, en particulier du journal \textit{Tân Văn}, qui joue un rôle déterminant dans la mise en lumière de l’affaire. Une série d’articles d’enquête, signés par le journaliste et illustrateur Bút Sơn, a été publiée de manière continue dans les numéros 53 à 58, soit sur une période de plus de deux mois, entre la fin du mois de juin et le milieu du mois de septembre 1935. À cette époque, \textit{Tân Văn} est un hebdomadaire, ce qui signifie que ces publications ont entretenu la pression médiatique semaine après semaine. Les articles sont riches en descriptions, observations, entretiens, et mettent en scène avec ironie et précision les pratiques controversées de Nguyễn Kim Muôn. Ils sont encore disponibles aujourd’hui dans les collections numériques de la Bibliothèque nationale de France (Gallica) et la Bibliothèque nationale du Vietnam, ce qui en facilite l’accès et l’analyse.

En parallèle de ces récits journalistiques, le dossier comprend aussi une lettre ouverte de défense rédigée par Nguyễn Kim Muôn lui-même, publiée dans \textit{Tân Văn} sous le titre \textit{Tu trước mặt người} (Pratiquer la religion devant les hommes), dans laquelle il répond de manière détaillée aux accusations portées contre lui. Ce texte constitue une source essentielle pour comprendre comment Nguyễn Kim Muôn se présente en tant que moine, sa vision personnelle du bouddhisme, et son interprétation de la spiritualité << naturelle >>, en rupture avec l’orthodoxie rituelle. Enfin, l’affaire est aussi évoquée, sous un autre angle, dans la biographie de Nguyễn Kim Muôn écrite par son disciple, le bonze Minh Út. Ce texte, de nature hagiographique, relate les événements de 1935 comme une épreuve injuste, inscrite dans le karma du maître, et met en avant sa sérénité face à la diffamation. Il témoigne de la manière dont l’affaire a été intégrée dans la mémoire religieuse interne du groupe, transformée en moment de persécution spirituelle.

En dehors de ces sources principales, il est possible qu’il existe d’autres documents encore inexplorés, notamment des photographies compromettantes mentionnées à plusieurs reprises dans la presse, des archives judiciaires si une plainte a bien été déposée, ou encore des articles publiés dans d’autres journaux contemporains, qui auraient pu réagir à la polémique, la soutenir ou la critiquer. L’exploitation de ces sources secondaires permettrait d’élargir le champ de l’analyse et de mieux saisir la réception publique de l’affaire à l’échelle de la société coloniale.

\subsubsection{L’origine de cette affaire }
L’affaire Nguyễn Kim Muôn trouve son origine dans un faisceau de tensions accumulées autour de la figure de ce bonze non conventionnel, dont la trajectoire personnelle et les pratiques religieuses déroutent autant qu’elles intriguent. Avant même que l’affaire n’éclate publiquement en 1935, des rumeurs circulent déjà dans les milieux bouddhistes et dans la presse. Un premier article signé par la journaliste Mộng Hoa paraît dès avril 1934 dans le journal \textit{Saïgon}. Elle y accuse Nguyễn Kim Muôn d’avoir profité de la crédulité populaire pour s’enrichir, après avoir quitté ses fonctions à la Banque de l’Indochine et fui ses créanciers pour s’installer à Phú Quốc. À partir de ce moment, l’attention portée à ce personnage ne cesse de croître.

Le déclenchement officiel de l’affaire intervient au début de l’année 1935, à la suite d’un article publié dans l’hebdomadaire \textit{Tân Thời} du 21 février, intitulé \textit{Kẻ làm chồng, làm cha có tội} (Les vrais coupables, ce sont les parents et les maris ). Cet article, bien que ne nommant pas directement Nguyễn Kim Muôn, est perçu par les autorités comme une attaque indirecte contre lui, en raison de son contenu jugé subversif. L’administration de Gia Định, alertée, commence alors une enquête officielle, appuyée par une série de rapports de police, télégrammes et échanges internes.

Mais ce sont bien les articles de \textit{Tân Văn}, à partir de fin juin 1935, qui vont véritablement donner à l’affaire une autre dimension. Le journaliste Bút Sơn, dans un style à la fois littéraire, satirique et très documenté, y décrit ses visites à la pagode Long Vân Tự, les scènes troublantes qu’il y observe, les témoignages qu’il recueille, et les doutes qu’il soulève sur l’intégrité morale du chef religieux. Ces articles, publiés de manière hebdomadaire dans les numéros 47 à 58 entre fin juin et mi-septembre 1935, construisent peu à peu une image publique de Nguyễn Kim Muôn comme celle d'un imposteur, d'un manipulateur, et de la figure d’un bouddhisme dévoyé.

Ce qui rend l’affaire particulièrement sensible, c’est << la dissociation manifeste entre l’apparence religieuse du personnage >> - un bonze entouré de disciples, prêchant la paix et l’austérité - et << les pratiques qu’on lui prête >> : promiscuité avec de jeunes bonzesses, rituels hétérodoxes, bains mixtes, fortune personnelle, vie confortable, voire photographie de femmes dénudées, autrement dit de la pornographie. La tension entre cette image sacrée et ces comportements, perçus comme immoraux, alimente la colère de l’opinion, d’autant plus que Nguyễn Kim Muôn attire autour de lui un cercle de fidèles grandissant, souvent composé de jeunes femmes.

En somme, l’origine de l’affaire réside à la fois dans une réalité religieuse diversifiée, dans un contexte social en pleine mutation, et dans l’essor d’un journalisme d’enquête vietnamien qui, pour la première fois, joue un rôle d’arbitre moral et de dénonciateur public. C’est cette conjonction entre l’inquiétude des autorités, l’indignation populaire et le regard critique de la presse qui fait de l’affaire Nguyễn Kim Muôn un événement aussi retentissant.


\subsubsection{Des pratiques bouddhiques hétérodoxes qui choquent}
L’affaire Nguyễn Kim Muôn en 1935 dépasse largement les limites d’un simple scandale moral ou sexuel : en effet, elle dévoile une série de pratiques religieuses choquantes pour la société coloniale et constitue en même temps l’expression d’une transformation silencieuse du paysage spirituel du Viêt Nam. Ce qui se déroule au temple Long Vân Tự, sous le regard scrutateur de la presse et des autorités, représente un défi à l’autorité religieuse, aux normes traditionnelles du bouddhisme, ainsi qu’à la relation entre la foi et les formes de pratique.

La première chose qui surprend les visiteurs est l’absence totale de rituels classiques : pas de statues de Bouddha, pas d’encens, pas de récitations de sutras. Interrogé à ce sujet, Nguyễn Kim Muôn répond : « La forme est inutile lorsque l’esprit a atteint un certain niveau de pureté. » Il défend une voie dite << naturelle >>, affranchie des symboles, visant une libération intérieure sans médiation ni rituels extérieurs. Une telle conception bouleverse profondément un contexte bouddhique encore fortement marqué par le ritualisme. Le rejet explicite des symboles, des divinités, voire même du Bouddha, équivaut à une rupture du sacré, perçue comme une dérive, voire une hérésie. Cette attitude ne dérange pas seulement les fidèles laïcs, mais irrite aussi les moines traditionnels, qui y voient une prétention à s’élever au-dessus de la Loi et de la discipline bouddhique. Par ailleurs, l’identité personnelle de Nguyễn Kim Muôn demeure trouble : ancien employé de la Banque de l’Indochine, il affirme avoir appris le bouddhisme en autodidacte. Au cours de l’enquête, il est incapable de présenter un certificat d’ordination ou un diplôme de prise de vœux monastiques. Sa carte de moine, délivrée à Hà Tiên, semble davantage administrative que religieuse. Ce manque de légitimité officielle suscite la méfiance des autorités et fait naître une question : s’agit-il d’un véritable religieux ou d’un imposteur ? Le temple, décrit comme un bien personnel, ressemble plus à une communauté fermée autour d’un chef charismatique qu’à un monastère reconnu.

Ce qui choque le plus reste la forte présence de jeunes nonnes dans l’enceinte du temple. Elles sont nombreuses, souvent très jeunes, gardent leurs cheveux longs, vivent dans des chambres proches de celle du maître et accomplissent pour lui des tâches particulièrement intimes : cuisiner, lui raser la tête, laver ses vêtements, lui masser les pieds. Certaines, comme Diệu Y ou Diệu Hồng, sont fréquemment appelées dans sa chambre privée et y restent des heures. Ce qui scandalise ici, ce n’est pas seulement la mixité de genre dans un espace monastique, mais le déséquilibre de pouvoir : un homme, chef spirituel, dirige une communauté de femmes entièrement soumises, silencieuses, et probablement fascinées. Le modèle d’une communauté égalitaire est remplacé par une structure quasi patriarcale, où le maître domine émotionnellement, spirituellement, et peut-être même sexuellement.

La presse s’est rapidement emparée de détails encore plus choquants. Selon les témoignages recueillis par le journaliste Bút Sơn, les moines et les nonnes se baignaient nus ensemble dans une rivière, au nom de la << pureté naturelle >>. Nguyễn Kim Muôn justifie cette pratique comme une << activité physique saine >>, comparable au sport, mais dans la pratique, même dans le sport, la séparation des sexes est de mise. Il nie toute connotation sexuelle, affirmant que << l’esprit purifie le corps >>. Mais ce raisonnement ne convainc ni les autorités ni l’opinion publique. L’image d’un moine entouré de femmes nues en pleine nature suffit à déclencher l’indignation de la société. Au lieu de symboliser la liberté, ce corps exposé devient le point de départ d’un scandale.

Pire encore, le directeur de \textit{Tân Văn} affirme l’existence de photographies de nonnes nues, prises par Nguyễn Kim Muôn lui-même, possiblement après des relations sexuelles. Bien que ces images n’aient jamais été publiées, elles deviennent un symbole de la figure du moine lubrique. Face à de telles vagues de critiques, Nguyễn Kim Muôn prend la parole pour se justifier. Selon lui, si les associations bouddhiques et les revues spirituelles appellent aujourd’hui à la réforme et à la << renaissance >> du bouddhisme, c’est précisément parce que les générations précédentes n’ont pas su recevoir l’enseignement du Bouddha de manière authentique. Elles se sont contentées d’imiter sans comprendre, reproduisant des formes vides de sens. C’est cette imitation aveugle, selon lui, qui a mené le bouddhisme vietnamien à la décadence. Il affirme que : si dès le départ les religieux avaient conservé l’esprit véritable de l’enseignement bouddhique, il n’aurait pas été nécessaire de réformer, de créer des associations, ni d’éditer des revues.

Nguyễn Kim Muôn se présente comme un homme moderne, conscient des limites du bouddhisme traditionnel. Il prétend vouloir << renverser les opinions erronées >> et corriger les déviations intellectuelles des bonzes de son temps. Pour lui, la préservation du vrai bouddhisme exige une transformation intérieure profonde des religieux : ils doivent abandonner le réflexe d’imitation et développer leur propre compréhension. Il appelle à l’intégration des acquis modernes et à l’exploration de nouvelles méthodes de pratique. Dans son ouvrage \textit{Nguồn gốc đạo phật} (Les origines du bouddhisme), il écrit :

« J’ai construit un temple sans objets de culte, car je ne suis pas superstitieux. Je respecte toutes les religions, car je ne suis pas sectaire. Je suis indépendant, je ne dépends d’aucune organisation, je ne me laisse dominer par personne, car je ne suis soumis à aucune sphère. Je médite, je m’isole, mais je ne me cache pas dans la montagne pour fuir le monde, car je ne suis pas égoïste. J’imprime, j’écris, je diffuse des livres spirituels, j’enseigne la vertu, j’exhorte le peuple à vivre selon le Dharma : c’est cela, la compassion. J’ai fondé l’Association de la Vertu pour venir en aide aux pauvres : c’est cela, la charité. Ma voie repose sur un seul mot : << Non. >> Non à un temple personnel, non à une appartenance fixe, non à l’égoïsme, non au sectarisme, non à la matérialité. >> [notre traduction].

Mais les rapports de la police de Gia Định et les descriptions des journalistes dépeignent une vie quotidienne en contradiction avec l’idéal d’ascèse bouddhique. On parle de repas accompagnés de bière fraîche, de réceptions animées, d’une voiture Talbot, d’oiseaux exotiques en cage et d’un coffre-fort personnel. Nguyễn Kim Muôn explique que ses revenus proviennent de la vente de tofu fermenté, de livres religieux et d’offrandes volontaires. Pourtant, dans une société où le moine est associé à la sobriété, à la simplicité et au végétarisme, ces éléments sont perçus comme les signes d’un détournement spirituel : un usage du religieux pour masquer des ambitions de pouvoir, d’argent et de plaisirs mondains.

Les articles de presse accusent Nguyễn Kim Muôn d’avoir exploité le bouddhisme à des fins lucratives, en organisant des rituels ésotériques, des cérémonies mystiques et spectaculaires. Le journaliste Thuận Phong mène une enquête approfondie, interrogeant plusieurs figures religieuses et intellectuelles de l’époque pour recueillir leur avis sur les pratiques spirituelles du maître Kim Muôn et sur l’Association philanthropique bouddhique qu’il dirige.

Par exemple, le moine Từ Phong (de la pagode Giác Hải) déclare : « Si l’on pratique en dehors de toute religion, cela relève de la voie démoniaque, et non de la voie bouddhique. Pour être clair, une telle pratique est de l’ordre de la superstition et de l’hérésie. Dans la tradition bouddhique, on doit toujours respecter les préceptes, suivre la conduite de Bouddha et ses enseignements. On ne peut prétendre pratiquer sans appartenir à une religion. Hors de la voie religieuse, il est impossible d’échapper au cycle de la naissance et de la mort. »\footnote{Thuận Phong,\textit{Thuận Phong, Nhà tu chân chính đối với sư Muôn,} Les vraies  moines envers bonze Muôn [notre traduction], journal \textit{Tân Văn}, numéro 56, p.7 }

Phạm Công Tắc\footnote{Phạm Công Tắc (1890–1959) était un leader vietnamien dans l'établissement et le développement de la religion caodaïsme, fondée en 1926.}, pour sa part, estime qu’un vrai religieux doit être prêt à se sacrifier entièrement pour défendre son idéal et sa doctrine. Concernant la situation actuelle de Nguyễn Kim Muôn, il voit deux options possibles : soit il se réfugie dans le silence en restant fidèle à la doctrine du Bouddha, soit il prend la parole publiquement pour diffuser ses idées et prouver qu’il croit sincèrement à sa propre voie. Il conclut que si Nguyễn Kim Muôn reste ambigu et évasif, il risque de prouver lui-même à l’opinion publique qu’il n’est qu’un imposteur se servant de la religion pour tromper les autres. \footnote{Huỳnh Hoài Lạc, Nhà tu chân chính đối với sư Muôn, Les vraies  moines envers bonze Muôn [notre traduction], journal \textit{Tân Văn}, numéro 56, p.6}

Le moine Thiện Chiếu\footnote{Maître Thiện Chiếu (1898-1974), ou Thích Thiện Chiếu, était un moine bouddhiste, un patriote anti-français, un membre de la Ligue de la jeunesse révolutionnaire, un membre du Parti communiste indochinois et un écrivain sous le nom de plume Xich Lien.}, quant à lui, affirme que : « Aujourd’hui, toutes les institutions bouddhiques portent déjà des traces de corruption. Les agissements de Kim Muôn doivent être abordés car l’opinion publique en parle déjà. J’ai déjà dénoncé cette doctrine dès ses débuts, car je savais qu’elle était erronée. Aujourd’hui, elle ne fait que récolter les conséquences de ses propres dérives. Kim Muôn a utilisé la superstition pour manipuler le peuple ; il est donc naturel que sa doctrine soit rejetée par la société. Le bouddhisme est une religion sans dieu, et ce que fait Nguyễn Kim Muôn, en flattant la crédulité populaire, est contraire à ses fondements. Lorsque son stratagème sera découvert, il devra disparaître. »\footnote{Huỳnh Hoài Lạc, Nhà tu chân chính đối với sư Muôn, Les vraies  moines envers bonze Muôn [notre traduction], journal \textit{Tân Văn}, numéro 56, p.6}

Un groupe de bonzes et de fidèles bouddhistes du Centre Việt Nam (Annam) publie également une déclaration en sept points :
\begin{itemize}
    \item Nous ne reconnaissons pas Nguyễn Kim Muôn comme un de nos coreligionnaires.
    \item Ses actions ne relèvent ni du Mahāyāna, ni du Theravāda.
    \item L’eau parfumée qu’il appelle << eau douce de la sagesse >> n’est pas la véritable eau de Prajñāpāramitā du bouddhisme.
    \item Il n’a jamais été reconnu comme maître ou supérieur par la communauté monastique.
    \item Il pratique selon une voie démoniaque que les anciens maîtres du bouddhisme ont toujours condamnée.
    \item Son comportement est en contradiction totale avec la doctrine et les préceptes du Bouddha.
    \item Il ignore complètement le sens de la théorie du salut des âmes en détresse.
\end{itemize}

Le lettré laïc Hoài Liên (Nguyễn Văn Tố) ajoute : « Un religieux qui ne parvient pas à atteindre l’éveil suprême, mais retombe dans les catégories des auditeurs, des sages solitaires, des religions extérieures, des dieux ou des démons, c’est qu’il confond l’illusion avec la vérité. Une telle personne ne pourra jamais atteindre l’Éveil, même après des vies entières de pratique. De plus, se proclamer maître suprême sans agir en accord avec les principes fondamentaux du bouddhisme de Śākyamuni est une absurdité. Et mettre la statue de Bouddha dehors, refuser de l’honorer dans le temple, n’est-ce pas une aberration ? » \footnote{Nguyễn Văn Tố, \textit{Nói chuyện đạo với Sư Muôn}, Discussion de la Voie avec Maitre Muôn, [notre traduction], journal \textit{Tân Văn}, numéro 55, p.16}

Face à une telle controverse, les autorités coloniales se retrouvent désemparées. Elles hésitent à intervenir ouvertement, de peur d’attiser les tensions religieuses ou d’être accusées de parti pris. L’ambiguïté du statut de Nguyễn Kim Muôn les pousse à la prudence : il n’enfreint pas la loi, mais il viole toutes les normes implicites. L’affaire devient alors un cas-limite, révélateur des difficultés rencontrées par le régime colonial pour encadrer des formes religieuses nouvelles, mouvantes et non institutionnalisées. Le débat dépasse la question du contrôle religieux : il touche à la définition même de ce qu’est un « vrai moine », une « vraie religion ».

\subsubsection{Les conséquences}

 L’affaire Nguyễn Kim Muôn provoque une série de conséquences pour sa carrière religieuse tant sur le plan religieux que social et administratif. À court terme, l’écho médiatique des articles publiés dans \textit{Tân Văn} déclenche une agitation populaire d’une ampleur inhabituelle autour de la pagode Long Vân Tự, transformant ce lieu de culte en un véritable théâtre des controverses. Selon l'écrit du bonze Minh Út, disciple de Nguyễn Kim Muôn, il y aurait eu jusqu’à 1 400 visiteurs en une seule journée, certains venant par curiosité, d’autres pour juger, d’autres encore pour soutenir. La pagode devient ainsi un espace hybride, à la fois religieux, médiatique et social.

Face à cet afflux incontrôlé, les autorités coloniales de la province de Gia Định sont contraintes d’intervenir. L’enquête administrative conclut à de nombreuses irrégularités : absence de formation religieuse certifiée, pratiques hétérodoxes, organisation opaque de l’association caritative, et confusion entre intérêt personnel et mission religieuse. L’administrateur recommande le retrait de l’autorisation officielle accordée à le \textit{Hội phước thiện nhà phật}, ainsi que la fermeture de la pagode.

Pour Nguyễn Kim Muôn et ses disciples, les conséquences sont à la fois publiques et personnelles. D’un côté, leur réputation est sérieusement entachée. Le bonze devient dans l’imaginaire collectif une figure de scandale, mêlant manipulation, luxe, transgression sexuelle et discours mystique. D’un autre côté, cette crise marque un tournant silencieux dans sa trajectoire spirituelle. Après 1935, Nguyễn Kim Muôn cesse totalement d’écrire et de publier des ouvrages religieux. Il quitte temporairement la pagode Long Vân Tự pour aller à Baria, aujourd'hui est province Bà Rịa Vũng Tàu comme pour se retirer de la scène publique et apaiser les tensions. Ce silence éditorial prolongé contraste fortement avec son activité précédente, marquée par la diffusion de nombreux textes à visée doctrinale et spirituelle.
Jusqu'à présent, les bouddhistes semblent ignorer cet événement. Les moines actuels de la pagode Long Vân ont pris le relais, mais ils n'ont guère parlé de Nguyễn Kim Muôn aux bouddhistes, ni des scandales de 1935.

\clearpage
\section{Sa vie après l'affaire de 1935 (1935-1946)}

Les informations concernant Nguyễn Kim Muôn après 1935 proviennent principalement d’une biographie rédigée par l’un de ses disciples. Selon ce document, après les événements survenus en 1935, Nguyễn Kim Muôn est invité par Nguyễn Văn Sâm, rédacteur en chef du journal Đuốc Nhà Nam, à collaborer à la diffusion de l’enseignement bouddhique. Ce journal consacre à Nguyễn Kim Muôn un supplément intitulé \textit{Phụ trương đuốc chơn lý} (Supplément de la Torche de la Vérité). Toutefois, malgré des recherches approfondies, aucune trace de ce supplément n’a été retrouvée dans les numéros encore conservés du journal.

La mission de propagation du Dharma du << Vénérable >> au temple Long Vân s’achève en 1938, lorsqu’il transmet l’enseignement à son disciple Minh Thành, de son vrai nom Lê Vân Điền. Il naît en 1893 dans le village de Phước Vĩnh Tây, province de Long An. En 1927, il entre officiellement dans les ordres, reçoit l’enseignement du maître Giai Minh et se voit conférer le nom religieux de Minh Thành. Il est l’un des disciples les plus âgés du Vénérable Nguyễn Kim Muôn. Après son ordination, il s’installe à Rạch Giá avec le maître Giai Minh et dix-huit autres moines pour y pratiquer la voie.

Entre 1938 et 1944, Nguyễn Kim Muôn retourne avec quelques disciples à Phú Quốc pour restaurer un ancien temple tombé en ruine. Durant cette période, il pratique une ascèse rigoureuse, « torse nu, travaillant la terre et cultivant des légumes ». En 1942, il réussit à économiser suffisamment pour acheter des tuiles et refaire le toit du temple, jusque-là recouvert de chaume.

En 1944, il construit une plateforme appelée \textit{kỳ thọ} en hauteur, pour guider ses disciples dans la méditation profonde, selon les quatre moments sacrés de la journée : \textit{Tý} (minuit), \textit{Ngọ} (midi), \textit{Mẹo} (l’aube), \textit{Dậu} (le crépuscule). En 1945, après la reddition du Japon, Nguyễn Kim Muôn est élu chef du mouvement de la Jeunesse Pionnière, puis nommé Commissaire aux Affaires sociales au sein du Comité populaire local.

En 1946, lors du débarquement des troupes françaises à Phú Quốc, le Comité populaire est dissous et plusieurs membres se réfugient dans la forêt pour résister. Le temple de Nguyễn Kim Muôn est incendié par les Français, et il est lui-même arrêté et transféré vers Rạch Giá. En route, un groupe de résistants profite de l’inattention des soldats pour les attaquer, s’emparer du bateau et conduire le Vénérable à Hòn Sơn Rái (aujourd’hui Hòn Sơn, commune de Lại Sơn, district de Kiên Hải, province de Kiên Giang). Cependant, il y est accusé à tort d’être un « lettré occidental collaborant avec les Français » et est exécuté.\footnote{Minh Út, \textit{Tiểu sử Sư Muôn}, document intérieur de la pagode Long Vân}

Lors d’un échange avec le moine Thích Thiện Trí\footnote{Citation de notre interview avec \textit{Thầy} Thích Thiện Trí qui est Secrétaire administratif et général de la pagode Long Vân le 22 août 2022}, chargé de la communication du temple Long Vân, ce dernier précise que la mort de Nguyễn Kim Muôn ne repose sur aucune source officielle ; il s’agirait surtout de rumeurs. Certains pensent qu’il a été arrêté puis se noya lors de son transfert, mais au sein du temple, on évoque une cause « sensible », potentiellement liée au Việt Minh\footnote{Le Viêt Minh est l'abréviation de \textit{Việt Nam Độc lập Đồng minh Hội}, une alliance politique et une force armée fondées par Hồ Chí Minh et le Parti communiste indochinois le 19 mai 1941.}. Le jour où Nguyễn Kim Muôn quitte le temple est le 8\textsuperscript{ème} jour du 10\textsuperscript{ème} mois lunaire, date qui est depuis choisie par les deux temples comme jour de commémoration annuelle \textit{lễ húy kỵ}\footnote{Dans la culture traditionnelle vietnamienne, la \textit{lễ húy kỵ}, aussi connue sous les noms de \textit{húy nhật} ou \textit{kỵ nhật}, est la journée commémorative pour honorer les défunts.

Le terme \textit{húy} signifie << éviter >> ou << s'abstenir de prononcer >>, en référence au respect du nom des disparus. Le mot \textit{kỵ} a une signification similaire d'évitement ou d'abstinence. Ainsi, le terme \textit{húy kỵ} désigne un jour où l'on s'abstient de toute festivité joyeuse pour se consacrer au souvenir et au respect de la personne décédée.

Cette cérémonie est d'une grande importance, car elle exprime la piété filiale et la gratitude des descendants envers leurs ancêtres. Ce jour-là, la famille ou les disciples se rassemblent pour préparer un repas spécial en offrande au défunt. C'est également une occasion de se réunir, d'évoquer des souvenirs et de préserver les traditions familiales.} de Nguyễn Kim Muôn.