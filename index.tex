\setcounter{chapter}{1}
\section*{Liste d'écrits de NGUYỄN Kim Muôn au dépôt légal à la BnF}

\begin{itemize}
\item Nguyễn Kim Muôn, \textit{Tịnh Độ Tông} [Bouddhisme de la Terre Pure], Thạnh Mậu, Saigon, 1927.
\item Nguyễn Kim Muôn, \textit{Phật giáo khuyên tu} [Morale bouddhique], Xưa Nay, Saigon, 1928.
\item Nguyễn Kim Muôn, \textit{Tịnh độ vô vi} [Prières bouddhiques], Đức Lưu Phương, Saigon, 1928.
\item Nguyễn Kim Muôn, \textit{Thờ trời tu Phật} [Prières bouddhiques], Xưa Nay, Saigon, 1929.
\item Nguyễn Kim Muôn, \textit{Kim cang kinh chơn giải} [Sur le bouddhisme], Đức Lưu Phương, Saigon, 1929.
\item Nguyễn Kim Muôn, \textit{Chấn hưng Phật giáo} [Propagation bouddhique], Đức Lưu Phương, Saigon, 1929.
\item Nguyễn Kim Muôn, \textit{Đạo có một} [Bouddhisme], Đức Lưu Phương, Saigon, 1929.
\item Nguyễn Kim Muôn, \textit{Đạo Phật Thích Ca. Thuyết pháp} [Dissertation sur le bouddhisme], Đức Lưu Phương, Saigon, 1929.
\item Nguyễn Kim Muôn, \textit{Phật giáo khuyên tu} [Prières bouddhiques], Xưa Nay, Saigon, 1929.
\item Nguyễn Kim Muôn, \textit{Phật giáo vệ sinh} [Prescriptions bouddhiques], Đức Lưu Phương, Saigon, 1929.
\item Nguyễn Kim Muôn, \textit{Đạo Phật Thích Ca} [La religion de Thích Ca : bouddhisme], Đức Lưu Phương, Saigon, 1929.
\item Nguyễn Kim Muôn, \textit{Huệ cảnh Tây phương. Đạt Ma bửu quyển} [Histoire du prince Đạt Ma devenu bouddhiste], Đức Lưu Phương, Saigon, 1930.
\item Nguyễn Kim Muôn, \textit{Đại đạo truyền chơn} [Propagation du bouddhisme], Đức Lưu Phương, Saigon, 1930.
\item Nguyễn Kim Muôn, \textit{Thiên cơ trực chỉ} [Bouddhisme ésotérique], Đức Lưu Phương, Saigon, 1930.
\item Nguyễn Kim Muôn, \textit{Đạo Phật Thích Ca. Phật giáo khuyên tu} [Le bouddhisme. Exhortation à suivre le bouddhisme], Bùi Văn Nhân, Saigon, 1932.
\item Nguyễn Kim Muôn, \textit{Tịnh độ tông « Tịnh độ hưu vi »...} [Livre de prières bouddhiques], Xưa Nay, Saigon, 1932.
\item Nguyễn Kim Muôn, \textit{Đoạn dâm căng...} [Pour maîtriser le désir, bouddhisme], Đức Lưu Phương, Saigon, 1932.
\item Nguyễn Kim Muôn, \textit{Đeo theo chưng Phật} [Sur les traces de Bouddha], Đức Lưu Phương, Saigon, 1932.
\item Nguyễn Kim Muôn, \textit{Dục tâm. Tâm hư tắc thân ngưng} [Le désir (bouddhisme)], Đức Lưu Phương, Saigon, 1932.
\item Nguyễn Kim Muôn, \textit{Phật đạo. Giải về hai chữ Đạo Đức} [La doctrine du bouddhisme], Xưa Nay, Saigon, 1932.
\item Nguyễn Kim Muôn, \textit{Phật giáo khuyên tu} [Encouragement au bouddhisme], Huỳnh Kim Danh, Saigon, 1932.
\item Nguyễn Kim Muôn, \textit{Danh truyền đao tập} [Recueil de la transmission de la voie], Saigon, Bảo Tồn,1932 
\item Nguyễn Kim Muôn, \textit{Tại sao tôi tu Phật} [Pourquoi je suis bouddhiste ?], Đức Lưu Phương, Saigon, 1932.
\item Nguyễn Kim Muôn, \textit{Ai muốn tu? Phật giáo vấn đáp} [Qui veut être bonze ?], Đức Lưu Phương, Saigon, 1933.
\item Nguyễn Kim Muôn, \textit{Cao Đài chơn giải} [La doctrine caodaïste expliquée], Thanh Thị Mậu, Saigon, 1933.
\item Nguyễn Kim Muôn, \textit{Đạo Phật Thích Ca} [Bouddhisme ésotérique], Bảo Tồn, Saigon, 1933.
\item Nguyễn Kim Muôn, \textit{Khẩu khuyết} [L’éducation de la respiration], Bảo Tồn, Saigon, 1933.
\item Nguyễn Kim Muôn, \textit{Lục tự chơn giải} [Explication des six mots « Nam mô A Di Đà Phật »], Đức Lưu Phương, Saigon, 1933.
\item Nguyễn Kim Muôn, \textit{Phép công phu} [L’éducation du corps], Đức Lưu Phương, Saigon, 1933.
\item Nguyễn Kim Muôn, \textit{Tu thân} [L’éducation de soi-même], Xưa Nay, Saigon, 1933.
\item Nguyễn Kim Muôn, \textit{Đạo Phật Thích Ca. Lục tự chơn giải} [Explication des six mots], Bảo Tồn, Saigon, 1933.
\item Nguyễn Kim Muôn, \textit{Đạo Phật Thích Ca. Gốc đạo là Từ bi Bác ái} [Dogmes fondamentaux : charité], Bảo Tồn, Saigon, 1933.
\item Nguyễn Kim Muôn, \textit{Gương Huệ} [Miroir de la sagesse], Bảo Tồn, Saigon, 1933.
\item Nguyễn Kim Muôn, \textit{Một chữ thương} [La pitié (bouddhisme)], Bảo Tồn, Saigon, 1933.
\item Nguyễn Kim Muôn, \textit{Liên Hoa đạo tập} [Le Lotus de Râma-Krishna], Bảo Tồn, Saigon, 1934.
\item Nguyễn Kim Muôn, \textit{Đạo Phật Thích Ca} [Bouddhisme], Bảo Tồn, Saigon, 1934.
\item Nguyễn Kim Muôn, \textit{Công phu} [La doctrine du cœur], Xưa Nay, Saigon, 1935.
\item Nguyễn Kim Muôn, \textit{Đạo khả đạo} [Le véritable chemin de la religion], Bảo Tồn, Saigon, 1935.
\item Nguyễn Kim Muôn, \textit{Đời người giải thoát} [La vie libérée], Đức Lưu Phương, Saigon, 1935.
\item Nguyễn Kim Muôn, \textit{Phật giáo} [Bouddhisme], Bảo Tồn, Saigon, 1935.
\item Nguyễn Kim Muôn, \textit{Phép thanh tịnh} [La pureté], Bảo Tồn, Saigon, 1935.
\end{itemize}

\section*{Revues de l’époque coloniale
}
\begin{itemize}
  \item \textit{Cùng Bạn} (Avec toi [notre traduction])
  \item \textit{Dân Quyền}
  \item \textit{Điễn Tín}
  \item \textit{Đuốc Nhà Nam}
  \item \textit{Gia Định báo} (Journal de Gia Định [traduction officielle])
  \item \textit{Lục Tỉnh Tân Văn} (La Gazette des six provinces)
  \item  Le populaire
  \item \textit{Nam Kỳ kinh tế báo} (L’Information économique de Cochinchine [traduction de la BULAC])
  \item \textit{Phong Hóa}
  \item \textit{Phụ Nữ tân văn} (La Gazette des femmes [notre traduction])
  \item \textit{Sài Gòn}
  \item \textit{Tân Thời}
  \item \textit{Tân Văn}
  \item \textit{Tiếng Chuông Sớm}
  \item \textit{Trung lập}
\end{itemize}

\section*{Archives consultées}

\textbf{Archives Nationales d’Outre-Mer (ANOM), Aix-en-Provence, France
}
\begin{itemize}
  \item GGI-56351 : Note sur le bouddhisme
  \item GGI-65539 : Bouddhisme. Diverses sectes religieuses, 1928–1938
  \item GGI-65540 : Bouddhisme. Diverses sectes religieuses. Tonkin, Annam, Cambodge
  \item GGI-59768 : Position du feu de Dương Đông , Phú Quốc
  \item HCI-CD-176 : Partis religieux : catholiques, bouddhistes, caodaïstes, Hòa Hảo, partis nationalistes 
  \item RSC-227 : Cultes catholiques, bouddhistes, caodaïstes, musulmans 
  \item  Fonds de Marie de Saigon, Etat civil de la Marie de Saigon, les séries :  MS8, MS9, MS10, MS11 
  \item GGI 65552 : Notes. Correspondance, rapports sur le caodaisme en Cochinchine adressé au gouvemeur général.Rapports mensuels du résident de Tay Ninh, Notes confidentielles du service de la súreté sur le caodaïsme en Cochinchine.
  \item GGI 65556 : Notes confidentielles du service de la sûreté sur le caodaïsme.
\end{itemize}

\textbf{Centre national d’archives n°2 (ANV2), Hô Chi Minh ville, Viêt Nam
}
\begin{itemize}
  \item 26031 Dossier relatif à la campagne de presse contre le bonze Nguyen Kim Muon et à la propagande théosophique années 1935-1936
  \item 26029 Dossier relatif à la propagande théosophique, à la demande d'autorisation de constitution en Cochinchine d'une branche de la Société théosophique années 1929-1935
  \item 26085 Dossier relatif à la secte caodaiste de Pham Cong Tac à Tayninh années 1935-1938
  \item 26091 Dossier relatif aux activités de la Secte Caodaiste Tien Thien années 1934-1939
  \item 26039 Dossier relatif aux demandes d'ouverture de pagodes bouddhisques formulées par des bonzes ou des particuliers des provinces, de souscription pour la construction des temples, pagodes,… années 1935-1939
  \item 26045 Dossier relatif aux investigations faites dans les pagodes de Baria par M. Paris année 1941
  \item 26046 Dossier relatif à la construction, l'entretien des pagodes et diverses affaires du culte bouddhique années 1886-1942
  \item 26052 Dossier relatif à l'Association pour l'étude et la conservation du bouddhisme en Cochinchine années 1928-1943
  \item 26058 Dossier relatif à la création et aux activités des pagodes des provinces années 1930-1945. Tome 1
  \item 26059 Dossier relatif à la création et aux activités des pagodes des provinces années 1930-1945. Tome 2
  \item 19139 Dossier relatif à l'organisation de la Société théosophique de France, branche de Baclieu année 1935, 1941
\end{itemize}

\textbf{Archives historiques du Crédit Agricole SA}
\begin{itemize}
    \item Les rapports-bilans de l’agence de Saïgon (1926-1937)
    \item Les lettres-bilans à la Banque d'Indochine, l’agence de Saigon (1926-1937)
    \item Les rapports-bilans de l’agence de Hà Tiên (1926-1935)
    \item Les rapports-bilans de l’agence de Kiên Giang (1926-1935)
\end{itemize}



