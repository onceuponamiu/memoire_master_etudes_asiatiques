Ce mémoire s’est donné pour objectif de reconstituer, à travers une approche croisée entre histoire, anthropologie religieuse et étude de sources textuelles, le parcours singulier de Nguyễn Kim Muôn (1892-1946), moine bouddhiste vietnamien à la fois controversé et novateur. Cette figure longtemps oubliée de l’histoire religieuse du Viêt Nam colonial apparaît aujourd’hui comme un témoin précieux des tensions, aspirations et recompositions spirituelles d’une époque marquée par de profonds bouleversements.

En retraçant sa biographie, nous avons pu mieux cerner les conditions d’émergence de son engagement religieux. Formé en partie en France et sensibilisé aux courants théosophiques, Nguyễn Kim Muôn incarne un profil atypique dans le paysage monastique vietnamien : ni héritier d’une lignée traditionnelle, ni totalement extérieur aux institutions, il trace sa propre voie, fondée sur la rigueur morale, l’autodiscipline, la pratique méditative et un engagement concret dans la société par l’action caritative.

Ses écrits, diffusés largement grâce à l’imprimerie, constituent un volet central de son projet spirituel. En mettant l’accent sur la récitation du nom du Bouddha, le végétarisme, la suppression des désirs charnels et l’ascèse individuelle, Nguyễn Kim Muôn propose une réforme du bouddhisme centrée sur l’intériorité, la pureté de conduite et l’autonomie du pratiquant. Il critique ouvertement les dérives du clergé, le ritualisme vide de sens et les superstitions populaires. Cette approche, bien que rigoureuse, l’a néanmoins exposé à de nombreuses critiques et malentendus, notamment de la part de la presse.

En effet, l’<< affaire de 1935 >> constitue un tournant majeur dans la trajectoire du maître. Pris dans une spirale d’accusations morales, de soupçons de manipulation sexuelle, d’enrichissement personnel et de dérives sectaires, Nguyễn Kim Muôn devient un objet de fascination médiatique. L’analyse détaillée de cette polémique à travers les articles du journal Tân Văn met en lumière le rôle crucial de la presse coloniale dans la construction d’une image publique, souvent réductrice, de certaines figures religieuses marginales. Elle révèle également les clivages internes au sein du bouddhisme réformiste, tiraillé entre quête de légitimité et rejet des innovations trop radicales.

Toutefois, malgré ces controverses, Nguyễn Kim Muôn laisse derrière lui un héritage spirituel tangible. Les temples qu’il a fondés, notamment Long Vân Tự et Hùng Long Tự, continuent de perpétuer son enseignement. Sa mémoire est préservée dans les récits oraux, les pratiques commémoratives et les documents manuscrits encore conservés par ses disciples. Il a formé une génération de pratiquants, dont certains comme Thích Minh Thành sont devenus des figures reconnues du bouddhisme vietnamien d’après-guerre.

Ce mémoire a également permis de souligner l’intérêt d’une approche interdisciplinaire mêlant sources archivistiques, témoignages de terrain, outils numériques et lecture critique des textes. Le recours à des technologies comme l’OCR, la transcription automatique et le traitement algorithmique de données a facilité la constitution d’un corpus inédit, tout en posant de nouveaux défis en termes de fiabilité et de contextualisation. Cette démarche pourrait servir de modèle pour d’autres recherches sur des acteurs religieux méconnus ou marginalisés, dont la parole reste encore difficilement accessible.

Plus largement, l’étude de Nguyễn Kim Muôn invite à reconsidérer les modalités de transmission du religieux dans un contexte colonial. À l’écart des institutions canoniques, il construit une forme de religiosité « portative », adaptée à la société moderne, urbaine, mobile et souvent déracinée. Sa démarche préfigure, d’une certaine manière, des formes contemporaines de spiritualité individuelle, en rupture avec les structures hiérarchiques traditionnelles.

En ce sens, Nguyễn Kim Muôn peut être lu à la fois comme produit de son temps, celui d’un Viêt Nam en transition, entre traditions confucéennes, modernité coloniale et éveil national, et comme annonciateur de formes nouvelles de religiosité. Son insistance sur la transformation intérieure, la simplicité de vie, la pratique quotidienne et l’aide aux plus démunis résonne encore aujourd’hui, dans un monde en quête de repères éthiques et spirituels.

Ainsi, en redonnant une voix à ce personnage longtemps resté dans l’ombre, ce travail espère contribuer à une histoire plus inclusive, attentive aux périphéries du religieux et aux formes de dissidence constructive. Il reste encore beaucoup à faire : l’étude comparative avec d’autres figures du bouddhisme réformiste, l’analyse des correspondances coloniales, l’exploitation complète de ses écrits poétiques et doctrinaux, ou encore l’enquête ethnographique auprès des descendants spirituels de sa communauté.

Cette recherche, tout en restant modeste, ouvre donc la voie à une meilleure compréhension des multiples visages du bouddhisme vietnamien au XX\textsuperscript{e} siècle, et plus largement, des dynamiques religieuses à l’époque coloniale, entre innovation, contestation et quête de vérité intérieure.