Ce mémoire propose une étude monographique approfondie de la vie et de l’œuvre de Nguyễn Kim Muôn (1892–1946), moine bouddhiste moderniste du Sud Viêt Nam, également connu sous le nom religieux Giai Minh. Figure religieuse atypique, Nguyễn Kim Muôn se distingue par son engagement en faveur d’un bouddhisme réformé, centré sur la pratique personnelle, la moralité, et l’action sociale. Formé partiellement à l’étranger, il a su intégrer des influences intellectuelles variées, notamment la Théosophie et les mouvements de rénovation bouddhique chinois, dans un contexte colonial marqué par des transformations sociales, politiques et spirituelles profondes.

La première partie retrace sa biographie, depuis ses origines familiales et son parcours professionnel jusqu’à son entrée dans la vie religieuse, la fondation de communautés monastiques, et la création de l’Œuvre de Charité bouddhique. L’étude s’intéresse aussi à l’« affaire de 1935 », ainsi qu'aux controverses médiatiques et administratives qui ont marqué sa trajectoire.

La deuxième partie analyse ses écrits, comprenant plus de trente livrets et opuscules doctrinaux, afin de mieux comprendre sa conception du bouddhisme, ses méthodes de diffusion et sa vision de la réforme religieuse. Cette analyse met en évidence une pensée synthétique, mêlant pratiques traditionnelles, critique morale et orientation vers un engagement social actif.

La troisième partie explore la mémoire et l’héritage de Nguyễn Kim Muôn à travers les documents conservés dans les pagodes, les témoignages oraux recueillis sur le terrain, et les pratiques commémoratives contemporaines. Elle montre comment son influence, bien que marginalisée dans les récits historiques officiels, perdure au sein de réseaux restreints de fidèles dans le Sud Viêt Nam.

Sur le plan méthodologique, cette recherche combine enquête archivistique (France et Viêt Nam), collecte de sources imprimées issues du dépôt légal, observation ethnographique, entretiens oraux et exploitation d’outils numériques (OCR, traitement automatisé des données, transcription audio par IA). Ce croisement de méthodes a permis de reconstituer, à partir de sources fragmentaires, un portrait nuancé d’un acteur religieux singulier, révélateur des recompositions du bouddhisme vietnamien à l’époque coloniale tardive.

\textbf{Mots-clés :} Nguyễn Kim Muôn ; bouddhisme vietnamien ; réformes religieuses ; Cochinchine ; colonialisme français ; histoire religieuse ; modernisme bouddhique ; pagode ; pratique spirituelle ; action sociale.