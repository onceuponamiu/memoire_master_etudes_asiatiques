\section*{Annexe 1. Tableau des imprimeries vietnamiennes de Cochinchine analysées et présentes qui a travaillé avec Nguyễn Kim Muôn dans ce mémoire.}
\vspace{1cm}
{\small
\begin{longtable}{|p{3.5cm}|p{4cm}|p{2cm}|p{2cm}|p{1.5cm}|p{1.5cm}|}
\hline
{\centering\arraybackslash\textbf{Nom de l'imprimerie}} &
{\centering\arraybackslash\textbf{Adresse}} &
{\centering\arraybackslash\textbf{Ville}} &
{\centering\arraybackslash\textbf{Directeur}} &
{\centering\arraybackslash\textbf{Début d'activité}} &
{\centering\arraybackslash\textbf{Fin d'activité}} \\ \hline
\endhead
Bảo Tồn & 173-175 boulevard de la Somme & Sài Gòn & Lê Thị Hạnh & 1927 & 1947 \\ \hline
Bùi Văn Nhẫn & Rue George Clémenceau & Bến Tre & Bùi Văn Nhẫn & 1930 & 1937 \\ \hline
Đức Lưu Phương & 158 rue d'Espagne & Sài Gòn & Trương Văn Tuấn & 1923 & 1947 \\ \hline
Hồ Văn & 64 Quai de Vĩnh Phước & Sa Đéc & Hồ Văn Lang & 1924 & 1930 \\ \hline
Huỳnh Kim Danh & 477-479 Paul Blanchy & Sài Gòn & Huỳnh Kim Danh & 1910 & 1932 \\ \hline
Imprimerie de l'Ouest & 13 Boulevard Delanoue & Cần Thơ & Trần Đắt Nghĩa & 1917 & 1952 \\ \hline
Thạnh Thị Mậu & 5-7 rue de Reims & Sài Gòn & Thạnh Thị Mậu & 1927 & 1945 \\ \hline
Xưa Nay & 60-64 rue Bonard & Sài Gòn & Nguyễn Háo Vĩnh & 1925 & 1945 \\ \hline
\end{longtable}
}

\clearpage
\section*{Annexe 2. Liste des œuvres annoncées comme existantes par Nguyễn Kim Muôn que nous n’avons pas trouvées à la BnF.}
\label{annexe2_updated_table}

\begin{longtable}{|p{4.5cm}|p{6cm}|p{5cm}|}
\hline
\textbf{Titre du livre} & \textbf{Titre traduit en francais} & \textbf{Statut} \\ \hline
\endfirsthead
\hline
\textbf{Titre du livre} & \textbf{Titre traduit en francais} & \textbf{Statut} \\ \hline
\endhead
\hline
\endfoot
\hline
\endlastfoot
Chí pháp & Doctrine ultime & Absent du dépôt légal mais retrouvé dans les documents internes du temple Long Vân \\ \hline
Chơn lý & La verité & Absent du dépôt légal \\ \hline
Kim cang chí luận & Traité suprême du Vajra & Absent du dépôt légal \\ \hline
Kinh Bác dương (Kinh tụng) & Sūtra de Bác Dương (chant liturgique) & Absent du dépôt légal \\ \hline
Liên Hoa Đạo Tập & Recueil doctrinal du Lotus & Partiellement complet : fasc. 7 trouvé au dépot légal, fasc 1 à 6 manquants, et suivants si existants \\ \hline
Lục tự chơn giải & Explication authentique des six mots & Partiellement complet, fasc 1 à 3 trouvés au dépot légal, facs manquants : 4 à 8 \\ \hline
Một cái hiểm tượng & Un signe du danger & Absent du dépôt légal \\ \hline
Ngay thảo nhà Phật & Esquisses bouddhiques & Absent du dépôt légal \\ \hline
Phát minh & Création & Partiellement complet, fasc 10 et 20 trouvés dans documents internes de la pagode, fasc 1 à 9 et 11 à 19 potentiellement manquants \\ \hline
Sơn cư bá vịnh (có dạy làm thơ) & Cent poèmes d’un ermite des montagnes (enseigner par la poésie) & Absent du dépôt légal \\ \hline
Sơn cư chí luận & Traité suprême d’un ermite des montagnes & Absent du dépôt légal \\ \hline
Tâm pháp chỉ ngay & Doctrine de l’esprit montrée directement & Volé ou manquant à la Bnf \\ \hline
Tự lực & Autonomie & Absent du dépôt légal \\ \hline
\end{longtable}

\clearpage
\section*{Annexe 3. Généalogie de la famille de Nguyễn Kim Muôn}
\label{annexe3_genealogie}
\vspace{1cm}
\begin{figure}[h!]
    \centering
    \includegraphics[width=1\textwidth]{dl/tree.jpg}
    \caption{Source: Crée par Le Thuy Tien NGUYEN}
    \label{fig:genealogie_kimmuon}
\end{figure}
\vspace{0.5cm}

\clearpage
\section*{Annexe 4. Date de publication et tirages des œuvres de Nguyễn Kim Muôn}
\label{annexe4_table}
\vspace{1cm}
\begin{longtable}{|p{4.5cm}|p{3.5cm}|p{2cm}|p{2cm}|}
\hline
\textbf{Titre d'origine} & \textbf{Date de publication} & \textbf{Tirages} & \textbf{Nombre de pages} \\ \hline
\endfirsthead
\hline
\textbf{Titre d'origine} & \textbf{Date de publication} & \textbf{Tirages} & \textbf{Nombre de pages} \\ \hline
\endhead
\hline
\endfoot
\hline
\endlastfoot
Tịnh độ tông & 1927-1932 & 12500 & 83 \\ \hline
Phật giáo khuyến tu & 1928 / 1932 & 15500 & 32 \\ \hline
Thờ trời tu phật & 17 janvier 1929 / 4 décembre 1929 & 16000 / 1000 & 35 \\ \hline
Chấn Hưng Phật Giáo & 7 septembre 1929 & 1000 & 25 \\ \hline
Đạo có một & 21 octobre 1929 & 1000 & 35 \\ \hline
Huệ cảnh tây phang & 17 janvier 1930 & 3000 & 36 \\ \hline
Thích giáo chơn truyền & 22 janvier 1930 & 1000 & 46 \\ \hline
Đeo theo chưng phật & 6 février 1932 & 1000 & 92 \\ \hline
Phật Đạo & 13 juin 1932 & 1000 & 40 \\ \hline
Đoạn dâm căng & 28 novembre 1932 & 2000 & 46 \\ \hline
Dục Tâm & 5 décembre 1932 & 1000 & 28 \\ \hline
Tại sao tôi tu Phật? & 1932 & 1000 & 42 \\ \hline
Tu thân & 9 février 1933 & 1000 & 80 \\ \hline
Ai muốn tu? & 29 mars 1933 & 1000 & 24 \\ \hline
Khẩu khuyết & 28 avril 1933 & 1000 & 28 \\ \hline
Một chữ thương & 16 août 1933 & 1000 & 46 \\ \hline
Lục tự chơn giải & 26 août 1933 & 1000 & 40, 55, 42 (3 tomes) \\ \hline
Gương huệ & 5 septembre 1933 & 1000 & 40 \\ \hline
Cao đài chơn giải & 1933 & X & 48 \\ \hline
Phép công phu & 1933 & X & 70 \\ \hline
Phật giáo & 8 août 1935 & 1000 & 24 \\ \hline
Công phu & 29 août 1935 & 1000 & 11 \\ \hline
Phép thanh tịnh & 7 septembre 1935 & 1000 & 25 \\ \hline
Đạo khả đạo & 12 septembre 1935 & 1000 & 22 \\ \hline
Đời người giải thoát & 7 novembre 1935 & 1000 & 18 \\ \hline
\end{longtable}

\clearpage
\section*{Annexe 5. Liste des traductions du titre d’œuvres de Nguyễn Kim Muôn}
\label{annexe4_table_alpha_revisé}
\vspace{1cm}
\begin{longtable}{|p{3.5cm}|p{4cm}|p{4cm}|p{4cm}|}
\hline
\textbf{Titre original vietnamien} & \textbf{Titre traduit du dépôt légal de l’Indochine} & \textbf{Titre traduit selon notre traduction} & \textbf{Titre traduit selon Nguyễn Kim Muôn} \\ \hline
\endfirsthead
\hline
\textbf{Titre original vietnamien} & \textbf{Titre traduit du dépôt légal de l’Indochine} & \textbf{Titre traduit selon notre traduction} & \textbf{Titre traduit selon Nguyễn Kim Muôn} \\ \hline
\endhead
\hline
\endfoot
\hline
\endlastfoot
Ai muốn tu? & Qui veut être bonze (bouddhisme) ? & Qui veux être bouddhiste ? & \\ \hline
Cao đài chơn giải & La doctrine du caodaïsme expliqué et commentée & Explication authentique du caodaïsme & \\ \hline
Chấn Hưng Phật Giáo & Propagation bouddhique & Rénovation du bouddhisme & \\ \hline
Công phu & & & La doctrine du cœur \\ \hline
Đạo có một & Bouddisme & Il n'y a qu'un seul chemin & \\ \hline
Đạo khả đạo & La véritable chemin de la religion & La Voie qui peut être nommée & \\ \hline
Đạo phật thích ca & Le bouddisme ésotérique & Bouddhisme de Shakyamuni & \\ \hline
Danh truyền đạo tập & La Pratique de la religion & Recueil de la transmission de la voie & \\ \hline
Đeo theo chưng phật & Sur le trace de Bouddha & & Sur le trace de Bouddha \\ \hline
Đời người giải thoát & La vie libérée & & La vie libérée \\ \hline
Đoạn dâm căng & "Pour maîtriser le désir, bouddhisme" & Destruction de la source du désir & \\ \hline
Dục Tâm & Le désir ( bouddhisme) & & Le désir \\ \hline
Gương huệ & La prière huệ cảnh traduit et expliqué & Miroir de la sagesse & \\ \hline
Huệ cảnh tây phang & Histoire du prince Dạt-Ma qui se fait bouddhiste & & Le paradis de l'Ouest \\ \hline
Khẩu khuyết & L'éducation de respiration & Précepte oral & \\ \hline
Lục tự chơn giải & Explication de six mots: \textit{Nam mô a di đà phật} (salut au bouddha Amidah) & Explication authentique des six mots & \\ \hline
Một chữ thương & La pitié (bouddhisme) & Un mot : compassion & \\ \hline
Phép công phu & "L'éducation du corps, précepte à l'usage des bonzes" & Méthode de pratique & \\ \hline
Phép thanh tịnh & Boudhisme. La pureté. & & Le calme \\ \hline
Phật Đạo & La boudhisme du bouddhisme & & Le bouddhisme. Les explications de deux mots : religion et vertu. \\ \hline
Phật giáo & Le boudhisme & & Le bouddhisme \\ \hline
Phật giáo khuyến tu & Morale bouddhique & Bouddhisme de l’encouragement à la pratique & \\ \hline
Phật giáo vệ sanh & Prescriptions bouddhiques & L'Hygiène Bouddhiste & \\ \hline
Tại sao tôi tu Phật? & Pourquoi je suis boudhiste ? & & Pourquoi je suis boudhiste ? \\ \hline
Thờ trời tu phật & Pratiquer le bouddhisme & Vénérer le ciel et pratiquer le bouddhisme & \\ \hline
Thích giáo chơn truyền & "[La religion de Thích-Ca : bouddhisme" & Transmission authentique de l’enseignement bouddhique & \\ \hline
Tịnh độ tông & Bouddhism & Bouddhisme de la Terre Pure & \\ \hline
Tu thân & Éducation de soi-même. Bouddhisme & Se cultiver soi-même & \\ \hline
\end{longtable}

\section*{Annexe 6. Liste des notice des livres de Nguyễn Kim Muôn  réalisée à partir de la liste semestrielle des imprimés du dépôt légal.}
\label{annexe6_images}

\begin{figure}[H]
    \centering
    \includegraphics[width=1.3\textwidth]{dl/dl0.jpg}
\end{figure}

\begin{figure}[H]
    \centering
    \includegraphics[width=1.3\textwidth]{dl/dl1.jpg}
\end{figure}

\begin{figure}[H]
    \centering
    \includegraphics[width=1.3\textwidth]{dl/dl2.jpg}
\end{figure}

\begin{figure}[H]
    \centering
    \includegraphics[width=1.2\textwidth]{dl/dl3.jpg}
\end{figure}

\begin{figure}[H]
    \centering
    \includegraphics[width=1.2\textwidth]{dl/dl4.jpg}
\end{figure}

\begin{figure}[H]
    \centering
    \includegraphics[width=1.2\textwidth]{dl/dl5.jpg}
\end{figure}

\begin{figure}[H]
    \centering
    \includegraphics[width=1.2\textwidth]{dl/dl6.jpg}
\end{figure}

\section*{Annexe 7. Liste des livres de Nguyễn Kim Muôn réalisée à partir du site de la BnF.}
\begin{figure}[H]
    \centering
    \includegraphics[width=1.2\textwidth, angle=90]{dl/data.png}
    \caption{URL du catalogue de la BnF : \texttt{https://catalogue.bnf.fr/changerPage.do?motRecherche=Nguyen+kim+muon\&index=\&numNotice=\&listeAffinages=\&nbResultParPage=10\&afficheRegroup=false\&affinageActif=false\&pageEnCours=1\&nbPage=7\&trouveDansFiltre=NoticePUB\&trouverDans=\&triResultParPage=5\&typeNotice=\&critereRecherche=0}}
    \label{annexe7_image}
\end{figure}

\section*{Annexe 8. Listes des bonzes et des bonzesses décédées à la pagode Hùng Long}
\label{annexe8_hunglong}

\begin{figure}[H]
    \centering
    \includegraphics[width=0.8\textwidth]{dl/chua2.png}
    \caption{Liste des bonzesses décédées. Source: \copyright{} NGUYỄN Lê Thủy Tiên, 20/01/2024}
    \label{fig:bonzesses}
\end{figure}

\begin{figure}[H]
    \centering
    \includegraphics[width=0.8\textwidth]{dl/chua1.png}
    \caption{Liste des bonzes décédés. Source: \copyright{} NGUYỄN Lê Thuy Tiên, 20/01/2024}
    \label{fig:bonzes}
\end{figure}