\begin{itemize}
\item \textsc{BOURDEAUX}, Pascal, \textit{Bouddhisme Hòa Hảo, d’un monde l’autre : religion et révolution au Sud Viêt Nam (1935-1955)}, Paris, Les Indes savantes, « Vietnamica », 2022.
\item \textsc{BOURDEAUX}, Pascal, « Croyances populaires et rétorsion coloniale dans le delta du Mékong. Commentaires sur la découverte d’une secte religieuse au village Hὸa Hảo (mars-mai 1940) », \textit{Aséanie, Sciences humaines en Asie du Sud-Est}, vol. 16, n° 1, 2005, p. 109-142.
\item \textsc{BOURDEAUX}, Pascal, << Réflexions sur les écrits religieux à la lecture du dépôt légal de l’Indochine (1922-1944) >>, Le portail France-Vietnam, Éditions Kimé, 2021, p.51-73.
\item \textsc{BROCHEUX}, Pierre và HÉMERY, Daniel, \textit{Indochine la colonisation ambiguë 1858-1954} [Đông Dương, sự đô hộ mơ hồ 1858-1954], Nxb La Découverte, Paris, 2001.
\item \textsc{BÙI}, Trân Phượng, \textit{Viêt Nam 1918-1945, genre et modernité : émergence de nouvelles perceptions et expérimentations}, thèse de doctorat en histoire, sous la direction de Françoise Thébaud, Université Lumière, Lyon, 2008.
\item \textsc{CADIÈRE}, Léopold, \textit{Croyances et pratiques religieuses des Vietnamiens} [Các tín ngưỡng và tập tục tôn giáo của người Việt Nam], Ecole Française d’Extrême-Orient, Paris, 1992.
\item \textsc{CAO}, Văn Bền, \textit{Giai cấp công nhân Việt Nam thời kỳ 1936-1939} [Les classes ouvrières au Viêt Nam entre 1936-1939], Khoa học xã hội, Hà Nội, 1979.
\item \textsc{CHAMPION}, Françoise và HERVIEU-LEGER, Danièle, \textit{De l’émotion en religion. Renouveaux et traditions} [Về cảm xúc trong tôn giáo. Phục hưng và truyền thống], Le Centurion, Paris, 1990.
\item \textsc{CHESNAUX}, Jean (cb), \textit{Tradition et révolution au Vietnam} [Truyền thống và cách mạng ở Việt Nam], Anthropos, Paris, 1971.
\item \textsc{CLÉMENTIN-OJHA}, Catherine (cb), \textit{Le renouveau religieux en Asie} [Sự phục hưng tôn giáo ở châu Á], EFEO, Paris, 1997.
\item \textsc{DE}, Gantès, Gilles và Nguyễn Phương Ngọc (cb), \textit{Vietnam le moment moderniste} [Việt Nam, khoảnh khắc hiện đại], Nxb PUP, Aix-en-Provence, 2009.
\item \textsc{DE}, Hartingh, Bertrand, \textit{Les manifestations du renouveau religieux au Vietnam} [Các biểu hiện của sự phục hưng tôn giáo tại Việt Nam], EFEO, Paris, 1997, tr. 17-31.
\item \textsc{DƯƠNG}, Thanh Mừng, \textit{Phong trào chấn hưng Phật giáo miền Nam Việt Nam} [Le mouvement de rénovation bouddhique dans le Sud du Viêt Nam], NXB Đà Nẵng, Đà Nẵng, 2022.
\item \textsc{GIÁO HỘI LỤC HÒA TĂNG}, « Phổ cáo » [Proclamation], \textit{Phật học Tạp chí} [Revue d’études bouddhiques], số 1, trang bìa cuối, 1953.
\item \textsc{HUỲNH}, Minh, \textit{Gia Định xưa và nay} [Gia Định hier et aujourd’hui], Khai Trí, Sài Gòn, 1973.
\item \textsc{HUỲNH}, Tịnh Của, \textit{Đại Nam quốc âm tự vị} [Le Dictionnaire monolingue du Viêt Nam d’époque], Imprimerie Rey, Curiol Cie, Saigon, 1895–1896.
\item \textsc{HUỲNH}, Văn Tòng, \textit{Lịch sử báo chí Việt Nam} [Histoire de la presse vietnamienne], NXB Trí Đăng, Sài Gòn, 1973.
\item \textsc{JAMMES}, Jérémy, \textit{Le caodaïsme : rituels médiumniques, oracles et exégèses : approche ethnologique d’un mouvement religieux vietnamien et de ses réseaux}, thèse de doctorat en ethnologie, sous la direction de Bernard Formoso, Université Paris Nanterre, Paris, 2006.
\item \textsc{JAMMES}, Jérémy và \textsc{PALMER}, David A., « The Bible of the Great Cycle of Esotericism: From the Xiantiandao Tradition to a Cao Ðài Scripture in Colonial Vietnam », in \textsc{CLART}, Philip, \textsc{OWNBY}, David et \textsc{WANG}, Chi-tim (éds.), \textit{Text and Context in the Modern History of Chinese Religions}, Boston, Brill, 2020.
\item \textsc{LÊ}, Mạnh Thát (chủ biên), \textit{Từ điển bách khoa Phật giáo Việt Nam} [Dictionnaire encyclopédique du bouddhisme vietnamien], tập 1 và 2, Nxb Tổng hợp Thành phố Hồ Chí Minh, 2006.
\item \textsc{LÊ}, Ngọc Trụ, \textit{Mục-lục báo-chí Việt-ngữ 1865–1965: Ấn-hành nhân tuần-lễ kỷ-niệm 100 năm báo-chí Việt-ngữ} [Index de la presse vietnamienne (1865–1965) : publié à l’occasion de la semaine commémorative du centenaire de la presse vietnamienne], Tổng-bộ Văn-hóa Xã-hội, Hà Nội, 1966.
\item \textsc{LÊ}, Nicole Dominique, \textit{Les missions-étrangères et la pénétration française au Viêt-Nam}, Mouton, Paris, La Haye, Publications de l’Institut d’Études et Recherches Interethniques et Interculturelles (5 livres), 1975.
\item \textsc{MORLAT}, Patrice, \textit{La répression coloniale au Vietnam, 1908–1940}, L’Harmattan, Paris, 1990.
\item \textsc{NGUYỄN}, Đình Đầu, \textit{300 năm địa chính Sài Gòn–Thành phố Hồ Chí Minh} [300 ans de géographie administrative de Saïgon–Ho Chi Minh-Ville], viết chung, 1997.
\item \textsc{NGUYỄN}, Đình Đầu, \textit{Địa chí tỉnh Gia Định} [Monographie de la province de Gia Định], 1997.
\item \textsc{NGUYỄN}, Đình Đầu, \textit{Gia Định – Sài Gòn – TP HCM: Dặm dài lịch sử (1698–2020)} [Gia Định–Saïgon–Hô-Chi-Minh-Ville : longue histoire (1698–2020)], NXB Tổng hợp TP HCM, TP HCM, 2023.
\item \textsc{NGUYỄN}, Đình Tư, \textit{Đường phố nội thành Thành phố Hồ Chí Minh} [Les rues de Hô Chi Minh-Ville], Thành phố Hồ Chí Minh, Nhà xuất bản Tổng Hợp, 2020.
\item \textsc{NGUYỄN}, Lang (Thích Nhất Hạnh), \textit{Việt Nam Phật giáo sử luận} [Essai sur l'histoire du bouddhisme vietnamien], Nxb Lá Bối, 1973.
\item \textsc{NGUYỄN}, Mạnh Bổng, \textit{Muốn chấn hưng Phật giáo ngày nay nên làm thế nào cho hiệu quả} [Comment relancer efficacement le bouddhisme aujourd’hui ?], \textit{Đông Pháp Thời báo} [Revue Indochinoise], số 547, 23/2/1927.
\item \textsc{NGUYỄN}, Phương Ngọc, \textit{La Société d’enseignement mutuel du Tonkin (Hội Trí Tri, 1892-1946)} [Hội giáo dục tương trợ Bắc Kỳ], Nxb PUP, Aix-en-Provence, 2009, tr. 223-237.
\item \textsc{NGUYỄN}, Quốc Tuấn, \textit{Tôn giáo học và khảo cổ học tôn giáo ở Việt Nam} [Science des religions et archéologie religieuse au Viêt Nam], Nhà xuất bản Đại học Sư Phạm, Hà Nội, 2020.
\item \textsc{NGUYỄN}, Thế Anh, \textit{L’engagement politique du bouddhisme au Sud Vietnam dans les années soixante} [Sự dấn thân chính trị của Phật giáo ở miền Nam Việt Nam trong những năm 60], Nxb Indes Savantes, Paris, 2008, tr. 614-622.
\item \textsc{NGUYỄN}, Thế Anh, \textit{L’élite intellectuelle vietnamienne et le fait colonial dans les premières années du XX\textsuperscript{e} siècle} [Giới tinh hoa trí thức Việt Nam và thực tế thuộc địa trong những năm đầu thế kỷ XX], Nxb Indes Savantes, Paris, 2008, tr. 393-408.
\item \textsc{NGUYỄN}, Thế Anh, \textit{Le bouddhisme dans la pensée politique du Sud Vietnam} [Phật giáo trong tư tưởng chính trị ở miền Nam Việt Nam], Nxb EFEO, Paris, 2011, tr. 25-41.
\item \textsc{NGUYỄN}, Thế Anh, \textit{Monarchie et fait colonial au Viêt-Nam (1875-1925)} [Monarchie et fait colonial au Viêt-Nam (1875-1925)], L’Harmattan, Paris, 1992.
\item \textsc{NGUYỄN}, Văn Dũng, \textit{“Vấn đề cải cách và đổi mới của tôn giáo trong xã hội Phương Đông cận - hiện đại”} [Problème de la réforme et du renouveau de la religion dans les sociétés orientales modernes et contemporaines], Nghiên cứu tôn giáo, số 1, 2001, tr. 16-22; số 2, 2001, tr. 21-29.
\item \textsc{NGUYỄN}, Văn Tố, \textit{“Đồ thờ của ta”} [Nos objets de culte], Trí Tân, số 131, 24/2/1944.
\item \textsc{NGUYỄN}, Văn Xuân, \textit{Phong trào Duy Tân} [Le mouvement Duy Tân], Nxb Đà Nẵng, 1995.
\item \textsc{NINH}, Thị Sinh, \textit{Phong trào chấn hưng Phật giáo ở Bắc Kỳ, trường hợp Hội Phật giáo (1934-1945)} [Le mouvement de rénovation bouddhique au Tonkin, le cas de l’Association bouddhique (1934-1945)], Đại học Quốc gia Hà Nội, Hà Nội, 2020.
\item \textsc{PEYCAM}, Philippe M. F., \textit{The birth of Vietnamese political journalism}, New York, Columbia University Press, 2012.
\item \textsc{PHẠM}, Công Luận, \textit{Sài Gòn đẹp xưa} [Saïgon d'antan dans son élégance], NXB Lao Động, TP. Hồ Chí Minh, 2023.
\item \textsc{PHẠM}, Công Luận, \textit{Biếm họa trên báo chí Sài Gòn trước 1975,[Caricatures dans la presse de Saïgon avant 1975], NXB Thế Giới, TP. Hồ Chí Minh, 2024.}

\item \textsc{PHẠM}, Công Luận, \textit{Sài Gòn – phong vị báo Xuân xưa} [Saïgon — la saveur des anciens numéros spéciaux du Têt], NXB Văn Hóa – Văn Nghệ, TP. Hồ Chí Minh, 2018.
\item \textsc{PHẠM}, Công Luận, \textit{Tùy bút – Hồi ký – Giai thoại trên báo Xuân Sài Gòn xưa} [Essais — mémoires — anecdotes dans les anciens numéros spéciaux du Têt de Saïgon], NXB Văn Hóa – Văn Nghệ, TP. Hồ Chí Minh, 2020.
\item \textsc{PHẠM}, Đình Nhân, \textit{Những sự kiện lịch sử} [Les événements historiques], Nxb Văn hóa thông tin, Hà Nội, 1999.
\item \textsc{PHẠM}, Như Thơm (cb), \textit{Hồi ký Trần Huy Liệu} [Les mémoires de Trần Huy Liệu], Nxb Khoa học xã hội, Hà Nội, 1991.
\item \textsc{PHẠM}, Quỳnh, \textit{Phật giáo đại quan} [Vue d'ensemble sur le bouddhisme], Đông Kinh ấn quán, Hà Nội, 1931.
\item \textsc{PHAN}, Kế Bính, \textit{Việt Nam phong tục} [Coutumes vietnamiennes], Nxb Văn học, Hà Nội, 1999.
\item \textsc{SORRENTINO}, Paul, \textit{À l’épreuve de la possession. Chronique d’une innovation rituelle dans le Vietnam contemporain}, Nanterre, Société d’ethnologie, 2018.
\item \textsc{SƠN}, Nam, \textit{Đất Gia Định xưa – Bến Nghé xưa – Người Sài Gòn} [Anciennes terres de Gia Định – Bến Nghé – Le peuple de Sài Gòn], Trẻ, Ho-chi-minh ville, 1975.
\item \textsc{TẠ}, Chí Đại Trường, \textit{Người và đất Việt} [Les hommes et la terre du Viêt Nam], Nxb Văn nghệ, TP. Hồ Chí Minh, 2001.
\item \textsc{TẠ}, Thị Thúy (cb), \textit{Lịch sử Việt Nam từ 1919 đến 1930}, tập 8 [Histoire du Viêt Nam de 1919 à 1930, vol. 8], Nxb Khoa học xã hội, Hà Nội, 2013.
\item \textsc{TẠ}, Thị Thúy (cb), \textit{Lịch sử Việt Nam từ 1930 đến 1945}, tập 9 [Histoire du Viêt Nam de 1930 à 1945, vol. 9], Nxb Khoa học xã hội, Hà Nội, 2014.
\item \textsc{THÁI}, Hư (Thiểu Chiếu dịch), \textit{Vô thần luận} [L'athéisme], Nhà in Mỹ Khouan, Chợ Lớn, 1937.
\item \textsc{THÍCH}, Đổng Bổn, \textit{Tiểu sử danh tăng Việt Nam thế kỷ XX} [Les biographies des moines illustres vietnamiens du XX\textsuperscript{e} siècle], Nxb Tôn giáo, Hà Nội, 2002.
\item \textsc{THÍCH}, Hải Ân, \textit{Lịch sử Phật giáo xứ Huế} [Histoire du bouddhisme de la région de Hué], Nxb TP. Hồ Chí Minh, 2001.
\item \textsc{THÍCH}, Mật Thể, \textit{Việt Nam Phật giáo sử lược} [Histoire abrégée du bouddhisme vietnamien], Nxb Tân Việt, 1943.
\item \textsc{THÍCH}, Nhất Hạnh, \textit{Đạo Phật hiện đại hóa} [La modernisation du bouddhisme], Nxb Lá Bối, Saïgon, 1965.
\item \textsc{THÍCH}, Thanh Đạt, \textit{Báo chí Phật giáo với phong trào chấn hưng Phật giáo, KLTN Khoa Lịch sử, Đại học Tổng hợp Hà Nội} [La presse bouddhiste et le mouvement de la renaissance du bouddhisme, Mémoire de fin d'études du Département d'Histoire, Université de Hanoï], Hà Nội, 1994.
\item \textsc{THÍCH}, Thiện Ân, \textit{Phật giáo Việt Nam xưa và nay} [Le bouddhisme vietnamien d'hier et d'aujourd'hui], Nxb Đông Phương, Sài Gòn, 1965.
\item \textsc{THÍCH}, Thiện Hoa, \textit{50 năm (1920-1970) Chấn hưng Phật giáo Việt Nam hay là “ghi ơn tiền bối”} [50 ans (1920-1970) de renaissance du bouddhisme vietnamien ou “souvenir aux prédécesseurs”], Sài Gòn, 1970.
\item \textsc{THÍCH}, Trí Hải, \textit{Hồi ký thành lập Hội Phật giáo Việt Nam} [Mémoires sur la fondation de l'Association bouddhiste du Viêt Nam], Nxb Tôn giáo, Hà Nội, 2004.
\item \textsc{THÍCH}, Trung Hậu, \textit{Ca dao, tục ngữ Phật giáo Việt Nam} [Les chansons populaires et les proverbes bouddhistes du Viêt Nam], Nxb TP. Hồ Chí Minh, 2002.
\item \textsc{THIỀN}, Chiếu, \textit{Chân lý của tiểu thừa và đại thừa Phật giáo} [La vérité du Petit Véhicule et du Grand Véhicule du bouddhisme], Nhà in Mỹ Khouan, Chợ Lớn, 1937.
\item \textsc{THIỀN}, Chiếu, \textit{Phật giáo vấn đáp} [Questions-réponses sur le bouddhisme], Nhà in Đức-Lưu Phượng, Sài Gòn, 1929.
\item \textsc{THIỀN}, Chiếu, \textit{Tại sao tôi đã cảm ơn đạo Phật} [Pourquoi j'ai remercié le bouddhisme], Nam Cường thư xã, Mỹ Tho, 1937.
\item \textsc{THIỀU}, Chửu, \textit{Con đường học Phật ở thế kỷ thứ XX} [Le chemin de l'apprentissage du bouddhisme au XX\textsuperscript{e} siècle], Nxb Văn hóa Thông tin, Hà Nội, 2008.
\item \textsc{THIỀU}, Chửu, \textit{Phật giáo với nhân gian} [Le bouddhisme et le peuple], Nhà in Đức Tuệ, Hà Nội, 1936.
\item \textsc{THIỀU}, Chửu, \textit{Hán-Việt từ điển} [Dictionnaire sino-vietnamien], Nhà in Đức Tuệ, Hà Nội, 1942.
\item \textsc{THIỀU}, Chửu, \textit{Mấy phép tu hành thiết yếu của người tu tại gia} [Quelques pratiques essentielles pour les laïcs], Tăng trưởng học hội Phật giáo Bắc kỳ, Hà Nội, 1939.
\item \textsc{TOÀN}, Ánh, \textit{Phong tục thờ cúng trong gia đình Việt Nam} [Les coutumes de culte dans la famille vietnamienne], Nxb Đồng Nai, TP. Hồ Chí Minh, 1993.
\item \textsc{TOÀN}, Ánh, \textit{Tín ngưỡng Việt Nam}, 3 tập [Les croyances vietnamiennes, 3 vol.], Nxb TP. Hồ Chí Minh, 2000.
\item \textsc{TRẦN}, Đình Việt, \textit{Hội thảo khoa học 300 năm Phật giáo Gia Định- Sài Gòn- thành phố Hồ Chí Minh}, Nxb TP Hồ Chí Minh, 2002.
\item \textsc{TRẦN}, Hồng Liên, \textit{Đạo Phật trong cộng đồng người Việt ở Nam Bộ Việt Nam từ thế kỉ XVII đến 1975} [Le bouddhisme dans la communauté vietnamienne du Sud Viêt Nam du XVIIe siècle à 1975], Nxb Khoa học Xã hội, Thành phố Hồ Chí Minh, 1995.
\item \textsc{TRẦN}, Hồng Liên, \textit{Góp phần tìm hiểu Phật giáo Nam Bộ} [Contribution à la compréhension du bouddhisme du Sud], Nxb Khoa học xã hội, Hà Nội, 2004.
\item \textsc{TRẦN}, Văn Giáp, \textit{Le Bouddhisme en Annam des origines au XIIIᵉ siècle}, \textit{Bulletin de l’École française d’Extrême-Orient} (BEFEO), vol. 32, no. 1, 1932, p. 191–268, Hanoï : École française d’Extrême-Orient.
\item \textsc{TRẦN}, Văn Giáp, \textit{Lược truyện các tác gia Việt Nam} [Notices biographiques sur les auteurs vietnamiens], NXB Khoa Học Xã Hội, Hà Nội, 1962.
\item \textsc{TRẦN}, Trọng Kim, \textit{Phật giáo với cuộc nhân sinh} [Le bouddhisme et la vie humaine], Nxb Trung Bắc tân văn, Hà Nội, 1935.
\item \textsc{TRẦN}, Trọng Kim, \textit{Việt Nam sử lược} [Histoire abrégée du Viêt Nam], Nxb Vĩnh và Thanh, Hà Nội, 1925, 2 tập.
\item \textsc{TRƯỜNG ĐẠI HỌC KHOA HỌC XÃ HỘI VÀ NHÂN VĂN – ĐẠI HỌC QUỐC GIA TP. HỒ CHÍ MINH}, \textit{Chủ nghĩa hậu hiện đại và phong trào tôn giáo mới ở Việt Nam và thế giới} [Postmodernisme et nouveaux mouvements religieux au Viêt Nam et dans le monde], éd. scientifique Idref, Thành phố Hồ Chí Minh, Nhà xuất bản Đại học Quốc gia Thành phố Hồ Chí Minh, 2014.
\end{itemize}