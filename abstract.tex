This thesis presents an in-depth monographic study of the life and work of Nguyễn Kim Muôn (1892–1946), a modernist Buddhist monk from Southern Vietnam, also known by his religious name, Giai Minh. As an atypical religious figure, Nguyễn Kim Muôn is known for his commitment to develop a reformed Buddhism centered on personal practice, morality, and social action. Partially trained abroad, he successfully integrated various intellectual influences, notably Theosophy and Chinese Buddhist revival movements, within a colonial context marked by profound social, political, and spiritual transformations.

The first part presents his biography, from his family origins and professional career to his entry into religious life, the founding of monastic communities, and the creation of the Buddhist Charitable Works. The study also examines the "1935 affair", as well as the media and administrative controversies that marked his journey.

The second part analyzes his writings, including over thirty doctrinal booklets and pamphlets, to better understand his conception of Buddhism, his methods to spread his speech, and his vision of religious reform. This analysis highlights a synthetic thought, blending traditional practices, moral critique, and an orientation towards active social engagement.

The third part explores the memory and legacy of Nguyễn Kim Muôn through preserved documents found in pagodas, oral testimonies collected in the field, and contemporary commemorative practices. It shows how his influence, although marginalized in official historical narratives, endures within limited networks of followers in Southern Vietnam.

From a methodological perspective, this research combines archival investigation (in France and Vietnam), collection of printed sources from legal deposits, ethnographic observation, oral interviews, and the use of digital tools (OCR, automated data processing, AI audio transcription). This cross-referencing of methods made it possible to reconstruct, from fragmentary sources, a nuanced portrait of a singular religious figure, revealing the recompositions of Vietnamese Buddhism in the late colonial era.

\textbf{Keywords:} Nguyễn Kim Muôn; Vietnamese Buddhism; religious reforms; Cochinchina; French colonialism; religious history; Buddhist modernism; pagoda; spiritual practice; social action.