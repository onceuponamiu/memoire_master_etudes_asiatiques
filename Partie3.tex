\setcounter{chapter}{2}
\chapter{Mémoire et héritage de Nguyễn Kim Muôn} 
\section{Documentation découverte dans les pagodes} %(mettre l'accent sur le travail sur place : observation, entretiens, compilation de documents internes)

Cette troisième partie s’appuie sur une synthèse des enquêtes de terrain menées dans les temples liés à Nguyễn Kim Muôn à Saïgon, Phú Quốc et Đồng Nai, ainsi que sur des entretiens réalisés avec les moines et fidèles présents sur place, sur l’observation et la participation aux cérémonies commémoratives de Nguyễn Kim Muôn. La majorité des documents internes sont manuscrits ou photocopiés pour être conservés et transmis. Certains ont été dactylographiés par les temples, d'autres sont devenus inutilisables en raison de leur détérioration avancée.

Cette partie commence par la découverte de documents internes toujours conservés par les responsables actuels des deux temples. Au cours des enquêtes de terrain et des investigations dans les temples liés au maître zen Giai Minh – Nguyễn Kim Muôn, notamment au Long Vân Tự et au Hùng Long Tự, nous avons eu accès à des documents internes encore inédits. Ces œuvres sont principalement des manuscrits ou des copies, précieusement conservées et transmises au sein de la communauté bouddhiste. Elles n’ont jamais été publiées en raison de leur caractère interne et sensible, et sont considérées comme destinées uniquement aux disciples fidèles et engagés de Nguyễn Kim Muôn. Par ailleurs, certains documents pourraient être liés à des controverses passées autour du maître, ce qui rend leur publication encore plus délicate.

\subsection{Résumé des œuvre trouvées dans les pagodes }

\subsubsection{\textit{Phát Minh} (Création)}

Création présente un point de vue critique sur les méthodes de pratique traditionnelles, que l'auteur appelle \textit{Đạo đời xưa} (la Voie des temps anciens). Le point principal est que la pratique spirituelle, en se concentrant uniquement sur des formes extérieures comme la récitation de \textit{sūtras}, la prière et les rituels, affaiblit le << propre effort >> du pratiquant et ne l'aide pas à échapper au cycle de la réincarnation. Nguyễn Kim Muôn considère également le \textit{khối nghi} (le bloc de doute) comme un obstacle majeur au développement spirituel, dont le non-dépassement mène à des résultats négatifs.

\begin{figure}[H]
    \centering
    \includegraphics[width=0.6\textwidth]
    {chua/phatminh.jpg}
    \caption{Couverture du livre \textit{Phát minh}. Source: Pagode Long Vân}
\end{figure}


Par opposition, \textit{Phát minh} promeut une autre voie de pratique, appelée \textit{Đạo đời nay} (la Voie des temps modernes). Cette voie est décrite comme le \textit{Phi thường đạo} (la Voie extraordinaire), non limitée par des noms, des écritures ou des religions spécifiques. L'œuvre soutient que la vérité réside en chacun \textit{<< Phật tức tâm, tâm tức phật >>}. La pratique est comprise comme un processus d'effort personnel pour atteindre l'éveil. L'objectif de cette voie est la libération de toutes les contraintes du monde. L'auteur propose également des méthodes de pratique spécifiques pour s'améliorer et contrôler les habitudes négatives.

L'<< Écoute >> et la << Vue >> sont considérées comme les sources de l'aveuglement et de la réincarnation. L'auteur affirme que les sens externes sont illusoires et peuvent agiter l'esprit. Pour atteindre un état de paix intérieure, le pratiquant doit contrôler ces sens. Le doute est également perçu comme directement lié à l'<< Écoute >> et à la << Vue >>.

L'œuvre adopte une position critique envers les religions et les écoles spirituelles, estimant que les organisations << avec un nom et un âge >> sont des \textit{tà đạo} (voies hérétiques). Ce point de vue avance que la dépendance à la religion affaiblit la force intérieure de l'individu. L'œuvre conclut en insistant sur l'objectif de la pratique, qui est de retourner à l'<< origine >>. Nguyễn Kim Muôn propose un \textit{Phát nguyện mới} (Nouveau vœu) pour abandonner les pratiques anciennes et contrôler ses sens afin de revenir à la nature originelle de chacun.

\subsubsection{\textit{Chí pháp} (Enseignement suprême)}

\textit{Chí pháp} est un ouvrage d’environ 140 pages rédigé par Nguyễn Kim Muôn au mois de juin 1932 (soit le 10 mai de l’année \textit{nhâm thân}). Ce manuscrit est considéré comme un manuel ésotérique transmis dans le cercle familial, réservé uniquement à ceux qui ont renoncé à la vie mondaine. Son objectif principal est de transmettre le feu sacré (\textit{truyền hoả}) ou l’énergie du Cœur (\textit{Truyền tâm năng}), une méthode secrète de cultivation spirituelle, tout en mettant en garde contre les dangers potentiels pour ceux qui n'auraient pas les qualités requises mais tenteraient de la pratiquer. Le texte insiste sur le fait que seuls les individus dotés d’un karma favorable (\textit{nhân duyên}) et d’une volonté ferme (\textit{lòng kiên định}) peuvent suivre cette voie et en récolter les fruits.

\begin{figure}[H]
    \centering
    \includegraphics[width=0.6\textwidth]
    {chua/chiphap.jpg}
    \caption{Couverture du livre \textit{Chí pháp}. Source: Pagode Long Vân}
\end{figure}

Nguyễn Kim Muôn commence par citer une prophétie du Bouddha Shakyamuni \textit{Đức phật thích Ca}, annonçant l’ère de la décadence du Dharma, durant laquelle les moines oublieront les enseignements authentiques pour poursuivre la gloire et les intérêts matériels. Pourtant, le bouddhisme ne disparaîtra pas : des bodhisattvas (\textit{Bồ tát}) renaîtront pour le revitaliser. Le livre affirme que la vraie pratique spirituelle consiste à transformer la « nature humaine ordinaire » (\textit{tánh phàm}) – c’est-à-dire à surmonter les sept émotions et six désirs (\textit{thất tình và lục dục}) - pour s’élever au rang des sages et des bouddhas. Le « Dao » (\textit{Đạo}) n’est pas une simple récitation mécanique de mantras, mais repose sur la lumière intérieure (\textit{nội quang}) et le souffle intérieur (\textit{nội tức}), en commençant par la réforme de soi (\textit{tu thân}), c’est-à-dire la purification des six sens (\textit{sáu căn}) : les yeux, les oreilles, le nez, la langue, le corps et l’esprit.

\textit{Chí pháp} décrit de manière détaillée les différentes méthodes de cultivation, notamment la technique du « retour de la lumière vers l’intérieur » (\textit{hồi quang phản chiếu}), qui consiste à regarder en soi-même. L’auteur interprète la récitation \textit{Nam mô a di đà phật} non pas comme une simple prière, mais comme une « machine à faire naître un Bouddha » (\textit{cái máy làm phật}). Le livre mentionne également des pratiques spécifiques pour les hommes et les femmes. Pour les hommes, il existe la méthode du « blocage de la racine sexuelle » (\textit{đoạn dâm căn}), qui permet de maîtriser la libido par la volonté et la respiration, en ramenant l’essence sexuelle au cerveau. Pour les femmes, la technique dite du « trancher le dragon rouge » (\textit{trảm xích long}) sert à interrompre les menstruations et à raffiner le « bon sang » (\textit{huyết tốt}), facilitant ainsi leur avancement spirituel.

Nguyễn Kim Muôn critique vivement les pratiquants superficiels, qui se concentrent uniquement sur les apparences extérieures sans transformation intérieure. Il insiste sur la nécessité d’une dévotion absolue (\textit{lòng chí quyết}), d’un réel renoncement aux attachements familiaux et affectifs (\textit{ly gia cắc ái}), et de l’acceptation de l’austérité. Ce n’est qu’en atteignant une pureté complète (\textit{sự thanh tịnh hoàn toàn}) que le pratiquant peut cultiver l’« esprit solaire » (\textit{dương thần}), permettant de quitter le corps physique (\textit{thoát xác}) et de s’envoler vers l’Occident bouddhique (\textit{Tây phương}). Enfin, le livre exprime la tristesse et la désillusion de l’auteur face à ses disciples, qui manquent de persévérance et de courage pour rompre avec le monde, rendant ainsi ses enseignements précieux, vains et inutiles.

\clearpage
\subsubsection{Document interne du temple Hùng Long tự}

Un manuscrit a été recopié à la main par un disciple du temple. Il s'agit d'un document sur les méthodes de pratique spirituelle rédigé par Nguyễn Kim Muôn entre 1936 et 1941, comprenant des textes en prose et en poésie destinés à guider les disciples dans leur cheminement.

\begin{figure}[H]
    \centering
    \includegraphics[width=0.6\textwidth]
    {chua/sachhunglong.jpg}
    \caption{Première page des documents de la pagode Hùng Long. Source : Pagode de Hùng Long}
\end{figure}

Selon Nguyễn Kim Muôn, la pratique spirituelle permettrait à l’être humain d’éviter la maladie. Selon lui, les maladies proviennent du déséquilibre du « souffle vital et du sang » (\textit{khí huyết}), lui-même causé par les « sept émotions et six désirs » ainsi que par les soucis et les angoisses. La pratique assidue permet de réguler le souffle vital, concentrer l’esprit et l’énergie, et harmoniser le yin et le yang.\\
« Cultiver la nature » (\textit{tu tánh}) signifie cultiver la vertu, tandis que « cultiver la destinée » (\textit{tu mạng}) renvoie à la mise en œuvre concrète du chemin spirituel à travers les efforts personnels.

Nguyễn Kim Muôn insiste sur le fait que la pratique spirituelle doit être « pour soi-même ». Le Dharma ne se trouve pas ailleurs, mais réside en nous-mêmes : « Le Dharma est en soi ; en dehors de soi, il n’existe pas. » Le Ciel, le Bouddha, le Paradis comme l’Enfer sont en réalité dans notre propre esprit. Ainsi, le pratiquant doit revenir à l’intériorité, cultiver son propre cœur-esprit plutôt que chercher à l’extérieur, en se concentrant sur la transformation intérieure, sans rechercher les manifestations bruyantes ou extérieures.

Le manuscrit présente l’enseignement des « Quatre postures méditatives » (\textit{Tứ thoàn}), permettant de pratiquer la méditation en toute position : marcher, se tenir debout, s’allonger ou s’asseoir. Parmi celles-ci, la posture allongée (\textit{Ngọa thoàn}) est considérée comme la plus difficile, car pendant le sommeil, l’âme peut se détacher du corps et être perturbée par les « démons du sommeil ». Le « roi des démons » n’est pas une force extérieure, mais réside dans nos propres sens : les oreilles, les yeux, la bouche. Le pratiquant doit « fermer hermétiquement » ces trois portes pour atteindre la quiétude mentale, dompter les démons et les transformer en « protecteurs du Dharma ».

Le texte évoque aussi la notion du « Saint-Fœtus » (\textit{Thánh thai}), une pratique avancée. Il ne s’agit pas d’un fœtus au sens littéral, mais d’une union subtile entre l’« esprit » et le « souffle », née de la nature de Bouddha, lorsque l’adepte « réunit l’esprit et le souffle en un seul point ». Lorsque le souffle est pleinement accumulé, le « fœtus » est complet et peut émerger par le sommet du crâne (\textit{đảnh môn}), permettant à l’adepte de devenir un « enfant du Bouddha ». La « libération » signifie alors une liberté absolue de l’esprit et de la volonté, et non une fuite du monde suivie par un nouvel asservissement aux rituels comme les cultes ou les récitations.

En outre, à travers les poèmes « Lamentation personnelle » (\textit{Tự thán}) et « Le goût du Dharma » (\textit{Mùi đạo}), l’auteur exprime sa déception face aux disciples manquant de persévérance et promptement découragés. Il compare le chemin spirituel à une alternance d’amertume et de douceur : « L’amertume est le socle sur lequel on peut s’élever », tandis que « la douceur mène à l’abîme ». Nguyễn Kim Muôn exhorte ses disciples à garder le cœur ferme et à accepter les épreuves pour atteindre l’Éveil véritable.
\\
\noindent\rule{0.35\linewidth}{0.6pt}
\clearpage
\subsubsection{\textit{Tiểu sử sư Nguyễn Kim Muôn} - Biographie du Maître Nguyễn Kim Muôn}

Le document intitulé Biographie du Nguyễn Kim Muôn est un récit biographique rédigé par Minh Út, disciple du Maître Nguyễn Kim Muôn, et achevé le 29\textsuperscript{ème} jour du 12\textsuperscript{ème} mois lunaire de l’année Tân Hợi (1972).
Ce écrit retrace la vie et le parcours spirituel ardu du Maître Nguyễn Kim Muôn, également connu sous le nom religieux de Giai Minh.

\begin{figure}[H]
    \centering
    \includegraphics[width=0.6\textwidth]
    {chua/tieusu.jpg}
    \caption{Couverture du livre \textit{Tiểu sử sư Nguyễn Kim Muôn}. Source: Pagode Hùng Long}
\end{figure}

 Né en 1892 à Bình Hòa, Gia Định, il reçoit une éducation secondaire de style français et travaille comme comptable à la Banque de l’Indochine.

 Le contenu principal du texte met en lumière son cheminement spirituel, depuis les débuts de la diffusion de la méthode bouddhique appelée \textit{Pháp môn tịnh độ hữu vi} (La Voie de la Terre Pure Active), jusqu’aux épreuves intenses vécues lors de sa retraite spirituelle à Thủ Chu, suivies par la fondation du temple Hùng Long Tự sur l’île de Phú Quốc.

 Le document relate également les événements marquants de sa vie, comme la calomnie lancée contre lui par le journaliste Bút Sơn, et sa fin tragique avec son exécution sur l’île Hòn Sơn Rái en 1946.

 Avant sa mort, le Maître a laissé un message demandant à ses disciples de commémorer sa mémoire le 8\textsuperscript{ème} jour du 10\textsuperscript{ème} mois lunaire. Après sa disparition, son disciple Minh Út, accompagné de ses frères spirituels, reconstruit le temple et le renomme en pagode Hùng Long \textit{Hùng Long Tự}.

\section{Mémoire du fondateur depuis sa mort}

Outre les publications déposées au dépôt légal et les enseignements diffusés en interne, l’héritage et la mémoire de Nguyễn Kim Muôn sont aujourd’hui perpétués à travers deux grands temples : Hùng Long et Vân Long. 

\subsection{Pagode Hùng Long}
Le Tổ đình Hùng Long Tự, également connu sous le nom de Chùa Sư Muôn (Pagode du Maître Muôn), est située au numéro 3 – hameau Suối Đá – commune Dương Tơ – district de Phú Quốc – province de Kiên Giang. Le temple est fondé par Nguyễn Kim Muôn au début des années 1930. À cette époque, la pagode est construite de manière rudimentaire, avec un toit en feuilles sur un sol en terre battue, sous le nom initial de Hùng Nhĩ Am (1930–1932). Faute de moyens pour entretenir 50 disciples en formation, Nguyễn Kim Muôn retourne à Sài Gòn pour fonder un Phật viện (Institut bouddhique) et publier des livres afin de financer les études de ses disciples restés à Phú Quốc.

\begin{figure}[H]
    \centering
    \includegraphics[width=1\textwidth]
    {images/chronologie_hunglong.png}
    \caption{Chronologie de succession des bonzes chefs de la pagode Hùng Long }
\end{figure}
\clearpage
Après les accords de Genève de 1954, alors que la situation militaire se stabilise, Nguyễn Kim Muôn est déjà décédé. Son disciple, Sư Minh Thành (bonze Minh Thành), de son nom civil Lê Văn Điền, né en 1894 à Long An, deuxième trụ trì (bonze chef) du temple Long Vân Tự, délégue à Thầy Thích Minh Út (Maître Thích Minh Út), de son nom civil Lê Đình Phú, né en 1920, la direction du temple Hùng Long à Phú Quốc, et entreprend sa restauration.
Quelques temps plus tard, Thiền sư Minh Thành (Maître zen Minh Thành) est élevé au rang de Phó Tăng Thống (Vice Patriarche) de la Giáo hội Lục Hòa Tăng Việt Nam (Communauté des Moines bouddhistes du Vietnam). En 1974, Thiền sư Minh Thành décède à Long Vân Tự ; la même année, Thầy Thích Minh Út s’éteint au Hùng Long Tự.

\begin{figure}[H]
    \centering
    \includegraphics[width=0.8\textwidth]{chua/banthoto.jpg}
    \caption{L'autel des bonzes à la pagode Hùng Long. Source: \copyright{} NGUYỄN Lê Thủy Tiên, 20/08/2022}
    \label{fig:bonzesses}
\end{figure}
\clearpage
La personne qui succède à la direction du Hùng Long Tự est Ni sư Thích Nữ Diệu Thiện (bonze moniale Thích Nữ Diệu Thiện), nom civil Điện Thị Hoa, née en 1915 dans une famille paysanne à Nha Trang. Après son ordination au Long Vân Tự, elle suit Thầy Minh Út à Phú Quốc pour y poursuivre son apprentissage. Durant les deux guerres de résistance contre le colonialisme français et l’impérialisme américain, le Hùng Long Tự sert de base secrète, fournissant médicaments, vivres et servant de relais pour le \textit{Mặt trận giải phóng dân tộc} (Front de libération nationale) à Phú Quốc. Le 4 février 1988, Ni sư Thích Nữ Diệu Thiện reçoit l’Ordre de la Résistance de troisième classe décerné par le Conseil d’État. Elle s’éteint le 29 mars 2010, à l’âge de 95 ans.

\begin{figure}[H]
    \centering
    \includegraphics[width=0.8\textwidth]{chua/banthonisu.jpg}
    \caption{L'autel des nonnes à la pagode Hùng Long. Source: \copyright{} NGUYỄN Lê Thủy Tiên, 20/08/2022}
    \label{fig:bonzesses}
\end{figure}
\clearpage
Auparavant, en 1999, en raison de son âge avancé, Ni sư Thích Nữ Diệu Thiện avait désigné par testament spirituel un successeur dans la lignée bouddhiste, Thượng tọa Thích Thiện Thông (bonze supérieur Thích Thiện Thông), nom civil Phan Văn Thơm, actuellement trụ trì (bonze chef) du Long Tuyền Tự. Né en 1952 dans une famille ouvrière à Biên Hòa, il entre en religion au Long Vân Tự en 1962 et reçoit l’ordination de Tỳ kheo (\textit{bhikkhu}) en 1970 à la chùa Thanh Long de la ville de Biên Hòa. Après sa prise de fonctions, Thượng tọa Thích Thiện Thông lance un vaste projet de restauration du Tổ đình Hùng Long Tự, qui dure deux ans. 

\begin{figure}[H] \centering \includegraphics[width=0.8\textwidth]{chua/thienthong.jpg} \caption{Bonze chef de Hùng Long Tự Thích Thiện Thông. Source: \copyright{} NGUYỄN Lê Thủy Tiên, 20/08/2022} \label{fig:bonzesses} \end{figure}

\clearpage
La Chùa Hùng Long est construite selon une architecture populaire traditionnelle. De la porte principale à la pagode, on compte environ 800 mètres. Devant le \textit{Đại hùng bửu điện} (Grande salle du Trésor du Héros), se dresse la statue du \textit{Đương lai hạ sanh di lặc tôn phật} (Bouddha Maitreya du futur). Dans la cour, on trouve une statue du \textit{Đức quan âm bồ tát} (Bodhisattva Avalokiteshvara), un \textit{cột phướn} (mât de prière), à droite un rocher naturel appelé \textit{Ông Hổ} (Monsieur Tigre), à gauche \textit{Thanh Long} (Dragon bleu).


\begin{figure}[H]
    \centering
    \includegraphics[width=0.8\textwidth]{chua/chuasumuon1.jpg}
    \caption{Le portail du temple Hùng Long. Source: \copyright{} NGUYỄN Lê Thủy Tiên, 20/08/2022}
    \label{fig:bonzesses}
\end{figure}

À l’intérieur du \textit{Đại hùng bửu điện}, au centre, trônent les statues du \textit{Đức phật a di đà} (Bouddha Amitābha) et du \textit{Đức ta bà giáo chủ điều ngự bổn sư thích ca mâu ni phật} (Bouddha historique Shakyamuni, Maître de notre monde). À gauche et à droite se trouvent \textit{Đức quan âm} et \textit{Đức thế chí} (Mahasthamaprapta), à droite \textit{Địa tạng vương} (Kṣitigarbha), à gauche \textit{Đức quan âm bồ tát}, et à l’arrière, dans la salle des ancêtres (\textit{Hậu tổ}), sont vénérés Đức đạt ma tổ sư (Patriarche Bodhidharma), Giai Minh Thiền sư (Nguyễn Kim Muôn), Minh Thành Thiền sư, Minh Út Thiền sư, ainsi que les moines et nonnes ayant autrefois étudié dans ce temple.

Aujourd’hui, le Hùng Long Tự n’est pas seulement un lieu de culte pour les habitants locaux, mais également un site touristique réputé, attirant les visiteurs de tout le pays. Derrière le sanctuaire principal se trouve un ancien arbre géant nommé \textit{Kơ nia} (appelé \textit{cây Cầy} par les locaux), âgé de plusieurs centaines d’années, au pied duquel se trouve une statue du \textit{Đức bổn sư thích ca} (Bouddha Shakyamuni en méditation).

\clearpage
\begin{figure}[H]
    \centering
    \includegraphics[width=0.8\textwidth]{chua/sumuon2.jpg}
    \caption{Salle principale à la pagode Hùng Long. Source: \copyright{} NGUYỄN Lê Thủy Tiên, 20/08/2022}
    \label{fig:bonzesses}
\end{figure}


Selon les dernières volontés transmises, chaque année, les 7 et 8 du premier mois lunaire, le temple organise solennellement la \textit{lễ dâng hương bạch phật cầu quốc thái dân an} (cérémonie d’offrande d’encens pour la paix du pays et du peuple), ainsi que la \textit{đại lễ húy kỵ} (grande cérémonie commémorative) en mémoire de Nguyễn Kim Muôn. D’après Thượng tọa Thích Thiện Thông, environ 800 personnes se rendent chaque année à cette cérémonie. Le bonze Thiện Thông a déclaré : <<J'ai toujours étudié et pratiqué les méthodes de méditation, la vie monastique et j'admire l'érudition du maître fondateur Nguyễn Kim Muôn à travers ses recueils de poèmes.>>\footnote{Interview le 20 août 2022}

\clearpage
\subsection{Pagode Long Vân Tự}

La pagode Long Vân reçoit son autorisation par le gouvernement le 22 mai 1933 et célèbre son inauguration les 2, 3 et 4 septembre de la même année\footnote{Nguyễn Kim Muôn, Anonnce de célebration de inauguration, \textit{Cùng Bạn }, le 9 septembre 1933}. Elle constitue également un centre d’activités pour le bouddhisme Thích Ca. La pagode est construite sur un terrain offert en don pieux par le couple de propriétaires terriens Lê Văn Vang (1861-1920) et Nguyễn Thị Hiếu (1869-1939) . Dans l’enceinte initiale de la pagode se trouve une petite hutte de méditation appelée à l’origine \textit{Hòa nhơn tự}. Ce n’est que le 25 novembre 1963, conformément à une décision de la province de Gia Định, que la pagode est totalement renommée Long Vân Tự. Son nom est également associé au pont Long Vân (anciennement pont \textit{Rạch mới}), qui traverse le canal pour accéder à l’enceinte de la pagode. Pendant la période de la guerre franco-vietnamienne, l’adresse de la pagode Long Vân est située au n°59/1, hameau Bác Ái 8, commune de Bình Hòa, province de Gia Định. Aujourd’hui, cette adresse correspond au 125/72 Bùi Đình Túy, quartier 25, arrondissement de Bình Thạnh, Hô-Chi-Minh-Ville.



\begin{figure}[H]
    \centering
    \includegraphics[width=1\textwidth]
    {images/chrono_longvan.jpg}
    \caption{Chronologie de succession des bonzes chefs de la pagode Long Vân }
\end{figure}

Le disciple qui succède à la direction de la pagode Long Vân en tant que bonze chef est Thích Minh Thành, de son nom civil Lê Văn Điền – ce moine hérite directement de la fonction de bonze chef et contribue alors grandement à la restauration de la pagode à ses débuts. Il naît en 1893 au village de Phước Vĩnh Tây, canton de Phước Điền Hạ, district de Cần Giuộc, province de Long An (actuellement rattaché au district de Cần Giuộc, province de Long An). Issu d'une famille traditionnelle respectueuse des rites confucéens, son père est Lê Thanh Xuân et sa mère Ngô Thị Thính. Dès son jeune âge, il étudie à la fois les caractères chinois classiques (chữ Nho) et la langue nationale (chữ quốc ngữ), se distinguant par sa soif d'apprendre et un talent particulier pour la médecine traditionnelle orientale.

\begin{figure}[H]
    \centering
    \includegraphics[width=0.8\textwidth]{chua/giaiminh.jpg}
    \caption{Image de Nguyễn Kim Muôn et Thích Minh Thành. Source: \copyright{} NGUYỄN Lê Thủy Tiên, 01/11/2022}
\end{figure}

En 1914, à l’âge de 21 ans, il fait un pèlerinage au mont Chứa Chan (province de Đồng Nai), où il se convertit officiellement au Triple Joyau (\textit{Tam bảo}) et reçoit le nom religieux de Đông Hồng. En 1927, il entre en religion bouddhiste sous la tutelle de Nguyễn Kim Muôn, à Gia Dinh, qui lui donne le nom bouddhiste de Minh Thành. Après son ordination, il suit son maître, avec dix-huit condisciples, à Suối Đá (Rạch Giá - Hà Tiên), pour étudier et pratiquer au Hùng Nhĩ Am, un ermitage fondé par le maître sur un terrain acquis auprès de Madame Diệu Thâm. Plus tard, Minh Thành retourne à Gia Định lorsque Nguyễn Kim Muôn reçoit le terrain de Long Vân, offert par le couple Lê Văn Vang et Nguyễn Thị Hiếu. Il y commence alors son ministère religieux. 

En 1949, le bonze Minh Thành organise la première retraite d'été (\textit{an cư kiết hạ}) et ouvre une session de transmission des préceptes à la pagode Long Vân. En 1954, il entreprend officiellement la restauration et l'inauguration du nouveau bâtiment de la pagode. En 1955, il est solennellement invité à devenir Chef spirituel (Tăng trưởng) de la Communauté bouddhiste Lục hòa Tăng de la province de Gia Định, et Long Vân Tự est choisie comme siège officiel de cette communauté. Grâce à sa réputation, à sa sagesse et à sa vertu, il est élu en 1961 au poste de Président de l’Association centrale des fidèles bouddhistes Lục hòa, un poste équivalent aujourd’hui à celui de Président du Conseil d'administration bouddhiste provincial. Dans le cadre de ses activités de diffusion du Dharma à l’étranger, en 1964, le bonze accueille au Việt Nam le bonze Narada, un éminent moine venu de Ceylan (Sri Lanka). Ce dernier offre deux reliques sacrées (\textit{Xá-lợi}) à la Communauté Lục hòa Tăng – Phật tử, lesquelles sont aujourd’hui précieusement conservées à la Pagode ancestrale Giác Lâm (arrondissement de Tân Bình). Le bonze Thích Minh Thành décède le 28 juillet 1974 (le 10 du sixième mois lunaire de l’année Giáp Dần), à l’âge de 81 ans.

Son successeur direct est le bonze Thích Minh Nhuận, dont le \textit{pháp danh} (nom spirituel reçu lors de l’ordination) est Nhật Tư, nom civil Huỳnh Văn Tư, né en 1921 au village de Bình Hòa, tổng Phước Vinh Trung, province de Biên Hòa. Il entre dans la vie monastique en 1944, sous la guidance et l’ordination de son maître, le bonze Minh Thành. Résidant au Long Vân Tự (pagode Long Vân) depuis avant 1943, il occupe progressivement des fonctions importantes telles que Thủ tọa (moine principal du monastère), Thủ Bổn (trésorier) et Giáo thọ (enseignant monastique).

En 1969, il est officiellement nommé bonze chef du Long Vân Tự selon les instructions de son maître et par décision du Giáo Hội Lục Hòa Tăng (Congrégation des Moines en Harmonie, un précurseur régional de l’Église bouddhique unifiée). Après le décès de son maître en 1974, le bonze Minh Nhuận hérite pleinement de la lignée spirituelle, devenant le 41\textsuperscript{ème} patriarche de la \textit{chánh tông lâm tế} (école orthodoxe de la tradition Linji, ou Rinzai en japonais), branche transmise au Long Vân Tự.

Tout au long de sa vie, il s’efforce de restaurer et d’élargir les activités religieuses du monastère : développement du \textit{nông thiền} (agriculture monastique selon l’esprit zen), ouverture d’une classe de bienfaisance pour les enfants défavorisés, mise en place d’une \textit{phòng thuốc nam} (salle de médecine traditionnelle vietnamienne) gratuite, et nombreuses œuvres caritatives. Pendant la guerre, il apporte discrètement son aide aux cadres révolutionnaires, et reçoit en 1980, en reconnaissance de ses contributions, l’Insigne de la Ville de Hồ Chí Minh. En 1981, lors du Congrès de l’unification du bouddhisme vietnamien au temple Quán Sứ (Hanoï), il est élu au Hội đồng Chứng minh GHPGVN (Conseil de la Sagesse de l’Église bouddhique du Viêt Nam - GHPGVN).

À la fin de l’année 2006, sentant le poids de l’âge et de la maladie, le bonze Minh Nhuận convoque le moine Thích Bửu Đăng pour lui confier la gestion des affaires spirituelles de Long Vân Tự. Après une période de maladie, il s’éteint le 10 août 2014 (année \textit{Giáp Ngọ}), à l’âge de 94 ans, après 70 \textit{hạ lạp} (années de vie monastique depuis son ordination).

Le successeur et actuel bonze chef est Thích Bửu Đăng, pháp danh Bửu Đăng, pháp hiệu Quang Chiếu (nom spirituel complémentaire), nom civil Nguyễn Văn Viễn, né le 16 octobre 1973 à Củ Chi, Hồ Chí Minh-Ville. Il prend refuge le 16 mars 1990 au chùa Bửu Liên (pagode Bửu Liên, quartier 25, Bình Thạnh), sous la direction du regretté maître Thích Thanh Đức (1903–1997). Il obtint son diplôme d’études secondaires bouddhiques en 2000, suivi d’un diplôme supérieur en 2004, et reçoit l’ordination complète de Tỳ-kheo (\textit{bhiksu} – moine pleinement ordonné) en 2001.

\begin{figure}[H]
    \centering
    \includegraphics[width=0.8\textwidth]{chua/longvan1.jpg}
    \caption{L'autel des bonzes à la pagode Hùng Long. Source: \copyright{} NGUYỄN Lê Thủy Tiên, 10/07/2022}
\end{figure}

En 2006, il prend officiellement en charge la gestion du Long Vân Tự, conformément aux souhaits directs de son prédécesseur, le bonze Minh Nhuận. Depuis lors, il se consacre à la trùng hưng (restauration) du Tam Bảo (les Trois Joyaux : Bouddha, Dharma et Sangha) de la pagode, tout en réaménageant l’ensemble de l’architecture, des paysages et du \textit{chánh điện} (salle principale de culte), tout en poursuivant l’enseignement, l’accueil des novices et le rayonnement de l’institution.

Dans l’organisation ecclésiastique, le bonze Thích Bửu Đăng occupe plusieurs fonctions notables : Secrétaire général du Conseil bouddhique du district de Bình Thạnh, Vice-président du Conseil du district, membre permanent du Ban Nghi lễ (Comité des Rites) de Hô-Chi-Minh-Ville et membre du Comité des Rites national (mandat 2022–2027). Il est le 42\textsuperscript{ème} patriarche de la lignée Lâm Tế chánh tông au sein du Long Vân Tự, perpétuant et valorisant la tradition monastique héritée des patriarches fondateurs.

\clearpage
L’architecture actuelle de la pagode Long Vân suit le style traditionnel du bouddhisme Mahāyāna (\textit{Bắc tông}), avec un portail à trois entrées (\textit{cổng tam quan}) orné de motifs sculptés représentant des fleurs et des symboles traditionnels. Derrière ce portail s’étend une grande cour, au centre de laquelle se dresse une statue de la bodhisattva Avalokiteśvara (\textit{Quán thế âm bồ tát}).

Depuis l’entrée jusqu’au bâtiment principal (\textit{chánh điện}), la pagode Long Vân est entourée de nombreux arbres. La cour avant, spacieuse, est pavée de pierres et agrémentée de petits aménagements paysagers et de statues bouddhiques.
\begin{figure}[H]
    \centering
    \includegraphics[width=0.8\textwidth]{chua/chanhdien1.jpg}
    \caption{Salle principale de la pagode Long Vân. Source: \copyright{} NGUYỄN Lê Thủy Tiên, 10/07/2022}
\end{figure}
\clearpage
L’ensemble de la pagode présente une disposition symétrique. Les bâtiments secondaires, tels que la salle de conférence (\textit{giảng đường}) ou les tours funéraires (\textit{tháp thờ}), sont harmonieusement répartis autour du site. Au fil du temps, le temple a fait l’objet de plusieurs rénovations afin de préserver son charme ancien tout en répondant aux besoins spirituels des fidèles. Le portail à trois entrées, le bâtiment principal, la salle de conférence et les tours sont non seulement des lieux de culte, mais aussi des espaces où se tiennent des retraites, des enseignements religieux, favorisant ainsi la pratique spirituelle et la diffusion des enseignements du Bouddha.

\begin{figure}[H] \centering \includegraphics[width=0.4\textwidth]{chua/thap.jpg} \caption{Les tours funéraires (\textit{tháp thờ}). Source: \copyright{} NGUYỄN Lê Thủy Tiên,} 20/08/2022 \label{fig:bonzesses} \end{figure}


À l’intérieur du temple, le bâtiment principal abrite une grande statue du Bouddha Shakyamuni (\textit{Phật thích ca mâu ni}) trônant sur l’autel. D’autres autels sont consacrés aux bodhisattvas et aux maîtres spirituels (\textit{Tổ sư}), décorés de fleurs de lotus, de lampes à huile, de bougies et d’encensoirs.

Outre ces autels principaux, la pagode conserve également de nombreuses statues anciennes, notamment des statues en bronze et en bois représentant le Bouddha et des bodhisattvas. Plusieurs tours stupas (\textit{bảo tháp}) sont érigés dans la cour pour conserver les reliques (\textit{xá lợi}) des maîtres spirituels et des vénérables moines (\textit{Hòa thượng}) disparus. On y trouve également les tombes de deux bienfaiteurs, Lê Văn Vang et Nguyễn Thị Hiếu, qui avaient autrefois fait don du terrain pour la construction du temple. 

La pagode comprend aussi un restaurant végétarien et une boutique vendant des objets bouddhiques tels que des statues, des vêtements pour les fidèles, et des textes sacrés. Le Vénérable Thích Thiện Trí a partagé : « Depuis toujours, la pagode fonctionne de manière autonome, en vendant de la nourriture et des objets religieux pour subvenir à ses propres besoins. »
\clearpage
\section{Commémorations (description d’une cérémonie)}

Chaque année, la pagode Long Vân organise la cérémonie d’anniversaire de la disparition (\textit{húy kỵ}) du maître Nguyễn Kim Muôn le 8\textsuperscript{ème} jour du 10\textsuperscript{ème} mois lunaire, ainsi que les cérémonies commémoratives des autres vénérables. Ces cérémonies suivent fidèlement la tradition bouddhique. L’ensemble du rituel commence par la lecture solennelle du \textit{Kim chương} devant le Bouddha \textit{Bạch phật kim chương}. Il s’agit d’un acte d’ouverture solennel, se déroulant dans le grand hall principal, où encens, fleurs, fruits et instruments rituels sont soigneusement disposés. Le maître de cérémonie – généralement un supérieur hiérarchique ou un vénérable de l’école – allume l’encens et lit à haute voix le \textit{Kim chương}, c’est-à-dire une adresse solennelle aux Trois Joyaux, expliquant le motif de la cérémonie, la date, le lieu, et mentionnant le nom du maître défunt. Ce texte exprime les mérites spirituels et les actions vertueuses du vénérable maître, tout en invoquant la présence bienveillante des dix directions de Bouddhas, des bodhisattvas, des patriarches, des divinités protectrices et de l’esprit du défunt, pour qu’ils témoignent et bénissent la cérémonie. Toute l’assemblée monastique récite ensemble le nom du Bouddha, créant une atmosphère solennelle et pure, qui ouvre une journée de rituel empreinte de dignité religieuse.

\begin{figure}[H] \centering \includegraphics[width=0.4\textwidth]{chua/damgio.jpg} \caption{Cérémonie de commémoration du décès de Nguyễn Kim Muôn. Source: \copyright{} NGUYỄN Lê Thủy Tiên, 01/11/2022} \label{fig:bonzesses} \end{figure}


La cérémonie se poursuit avec le culte du midi \textit{cúng ngọ} et l’offrande à l’esprit du défunt \textit{cung tiến giác linh}, réalisés à l’heure exacte de midi – considérée comme l’instant le plus pur de la journée selon la tradition bouddhique. Le culte du midi consiste à offrir des éléments symboliques de pureté – encens, fleurs, eau, riz, fruits – aux dix directions de Bouddhas, dans un esprit de vénération et de prière pour la paix du monde et le bien-être de tous les êtres. Les textes récités sont généralement le chapitre \textit{Phổ môn}, le \textit{Sūtra Amitābha}, ou le \textit{Sūtra Shurangama}, selon la tradition de la pagode. 

Immédiatement après, a lieu l’offrande à l’esprit du défunt, organisée dans la salle des ancêtres ou près du \textit{stūpa} abritant la relique ou la tablette spirituelle du maître. Les disciples et les représentants de la Sangha offrent de l’encens, du thé, du riz, récitent des vers dédiés, et lisent un texte d’hommage - généralement écrit en vers - pour louer les vertus du maître et solliciter sa bienveillance afin qu’il accepte les offrandes.

Après les rituels d’offrande et d’hommage, l’assemblée se réunit dans la salle commémorative pour rendre hommage à la mémoire du maître défunt. C’est un moment solennel et chargé d’émotion, où disciples, étudiants et fidèles ayant reçu l’enseignement ou bénéficié de la compassion du maître expriment leur gratitude et leur respect. Un représentant, souvent un disciple principal, lit un texte de commémoration retraçant la biographie du maître, son parcours de pratique, ses actions méritoires dans la diffusion du Dharma, sa discipline exemplaire et sa grande compassion. Ce texte peut être à la fois sobre et émouvant, mais reflète toujours un profond esprit de reconnaissance.

\begin{figure}[H]
    \centering
    \includegraphics[width=0.9\textwidth]{chua/traitang.jpg}
    \caption{Cérémonie de commémoration du décès de Nguyễn Kim Muôn. Source: \copyright{} NGUYỄN Lê Thủy Tiên, 01/11/2022}
\end{figure}

La cérémonie se poursuit par l’offrande aux moines (\textit{Trai tăng}), considérée comme l’une des pratiques les plus vertueuses du bouddhisme \textit{Mahāyāna}. À ce moment, les disciples, les fidèles et la famille offrent aux moines présents des biens matériels : repas végétariens, vêtements, donations, dans un esprit de piété filiale, et prient pour que les mérites obtenus soient transférés à l’esprit du maître défunt. Avant de prendre leur repas, les moines écoutent un discours d’offrande solennel présenté par un représentant de la famille spirituelle, incluant un discours de vénération aux Trois Joyaux, un récapitulatif des enjeux de la cérémonie, et une invitation à accepter les offrandes avec compassion. Les moines procèdent ensuite à une brève récitation rituelle et prière de transfert de mérite, souhaitant le bien pour l’esprit du maître ainsi que pour tous les êtres. Dans la pratique bouddhique, cette pratique génère d’immenses mérites pour les donateurs et constitue une forme élevée de gratitude selon l’esprit du Dharma, les moines étant considérés comme un champ de mérite inépuisable.

\begin{figure}[H]
    \centering
    \includegraphics[width=0.5\textwidth]{chua/traitang2.jpg}
    \caption{Cérémonie de commémoration du décès de Nguyễn Kim Muôn. ('Source: \copyright{} NGUYỄN Lê Thủy Tiên, 01/11/2022}
\end{figure}

La journée de cérémonie se conclut avec le rituel de l’Offrande Mông Sơn (\textit{Tiểu thí mông sơn}), également appelé le rituel d’offrande aux esprits errants. Ce rituel profondément empreint de compassion vise à offrir nourriture et boisson aux âmes sans abri : les êtres affamés, les esprits solitaires et errants du royaume des préta (esprits affamés). Le Mông Sơn est généralement installé en plein air, dans un lieu propre et paisible. Sur l’autel sont disposés du riz blanc, de la soupe de riz, de l’eau pure, du sel, des fruits, des gâteaux, et des bâtons d’encens. Le maître de cérémonie, assisté de la Sangha, procède à la récitation du rituel d’offrande, comprenant la récitation du nom du Bouddha, le \textit{Dàbēi zhòu} (Grand Dharani de la Grande Compassion), les mantras de transformation de la nourriture et de l’eau, les vers d’offrande, et le geste symbolique de répandre le riz, le sel et l’eau. Ce rituel manifeste l’esprit de don inconditionnel, de compassion universelle, sans discrimination. À la fin, toute l’assemblée récite la prière de transfert de mérites, souhaitant aux esprits de trouver la paix et la délivrance, et priant pour que l’esprit du maître défunt poursuive son chemin d’éveil dans la lumière des dix directions de Bouddhas.
