\setcounter{chapter}{1} 
\chapter{Œuvres de Nguyễn Kim Muôn} 
\section{Écrits de Nguyễn Kim Muôn} 

Dans la période allant de 1927 à 1935, Nguyễn Kim Muôn publie un nombre considérable d'œuvres, 52 titre au total, comprenant deux livres écrits par sa femme Nguyễn Thị Hương : \textit{Phụ nữ tầm tu} (Les femmes à la recherche de la voie spirituelle) [notre traduction] en 1929, et\textit{ Sách nấu ăn chay} (Livre de cuisine végétarienne) [notre traduction] en 1929, ainsi que six textes bouddhiques qu’il a préfacés et fait imprimer : \textit{Tịnh độ vô vi} (Terre pure du non-agir) [notre traduction], \textit{Kim cang kinh chơn giải} (Explication authentique du Sūtra du Diamant) [notre traduction], \textit{Qui ngươn trực chỉ} (Indication directe de la source originelle) [notre traduction], \textit{Kim cang chú giải} (Commentaire du Sūtra du Diamant, traduit par Trương Văn Minh) [notre traduction], \textit{Luật tạo hóa} (La loi de la création, traduit par Mạch Quốc Thoại) [notre traduction] et \textit{Đạo đức kinh của Lão Tử} (Dao De Jing de Laozi, traduit par Huy Hồng Đăng) [notre traduction]. 

Il publie une série de 6 petits écrits \textit{Đạo phật thích ca} (Bouddhisme de de Śākyamuni) [notre traduction] en 1933 et 1934, et consigne les sermons prononcés au temple Long Vân. 31 titres écrits par Nguyễn Kim Muôn sont enregistrés durant cette période (en excluant les versions multiples d'un même ouvrage). Le contenu de ces publications se concentre principalement sur le bouddhisme et les doctrines associées, comme en témoignent des ouvrages tels que \textit{Tịnh độ tông}, \textit{Phật giáo khuyên tu} et \textit{Đạo phật thích ca}, qui seront étudiés plus en détails dans cette partie. 

L'ensemble de ces 31 œuvres de Nguyễn Kim Muôn, produites entre 1927 et 1935, est référencée au dépôt légal et est actuellement conservé à la Bibliothèque nationale de France (BnF). Selon le site officiel de la BnF, le dépôt légal est une obligation pour tout éditeur, imprimeur, producteur, importateur, de déposer chaque document qu’il édite, imprime, produit ou importe, auprès de l’organisme habilité à recevoir le dépôt en fonction de la nature du document. Cette obligation s’applique à tout document diffusé en nombre à un public s’étendant au-delà du cercle de famille.\footnote{source : Bibliothèque nationale de France : https://www.bnf.fr/fr/quest-ce-que-le-depot-legal}

Le dépôt légal indochinois est instaurée par l'arrêté du 31 janvier 1922 qui référence systématique tous les imprimés de cette époque, jusqu'en 1944. Il permet aujourd’hui d'avoir une vision exhaustive des publications locales.\footnote{Pascal Bourdeaux, << Réflexions sur les écrits religieux à la lecture du dépôt légal de l’Indochine (1922-1944) >>, Le portail France-Vietnam, Éditions Kimé, 2021, p.51-73}

Ces documents sont accessibles sous diverses formes, incluant des éditions papier, des microfiches, et des versions numérisées consultables via la plateforme publique Gallica de la Bibliothèque nationale de France. De plus, certaines d'entre elles sont uniquement disponibles en consultation sur Gallica Intra Muros, une ressource spécifique à l'accès interne au sein des institutions de la BnF.

\small
\begin{longtable}{|p{3.8cm}|p{3.8cm}|p{1.2cm}|p{2.8cm}|p{1.2cm}|}
\caption{Liste d'écrits de Nguyễn Kim Muôn}
\label{tab:nguyen-kim-muon-works} \\
\toprule
\textbf{Titre} & \textbf{Titre en français (traduit du dépôt légal de l'Indochine)} & \textbf{Année} & \textbf{Imprimerie} & \textbf{Page} \\
\midrule
\endfirsthead

\multicolumn{5}{c}{{\bfseries Bảng \thetable\ -- tiếp theo}} \\
\toprule
\textbf{Titre} & \textbf{Titre en français (traduit du dépôt légal de l'Indochine)} & \textbf{Année} & \textbf{Imprimerie} & \textbf{Pages} \\
\midrule
\endhead

\bottomrule
\endfoot

\bottomrule
\endlastfoot

Tịnh Độ Tông & Le Bréviaire Tịnh-Ðô Tông. Bouddhisme & 1927 & Thạnh Mậu & 93 \\
Phật giáo khuyên tu & Morale bouddhique & 1928 & Xưa Nay & 31 \\
Tịnh độ vô vi & Prières bouddhiques & 1928 & Đức Lưu Phương & 35 \\
Thơ trời tu phật & Prières bouddhiques & 1929 & Xưa Nay & 94 \\
Kim cang kinh chơn giải & Sur le bouddhisme & 1929 & Đức Lưu Phương & 117 \\
Chấn hưng phật giáo & Propagation bouddhique & 1929 & Đức Lưu Phương & 25 \\
Đạo có một & Bouddhisme & 1929 & Đức Lưu Phương & 35 \\
Đạo phật Thích ca. Thuyết pháp & Dissertation sur le bouddhisme & 1929 & Đức Lưu Phương & 22 \\
Phật giáo khuyên tu & Prières bouddhiques & 1929 & Xưa Nay & 31 \\
Phật giáo vệ sinh & Prescriptions bouddhiques & 1929 & Đức Lưu Phương & 22 \\
Đạo phật Thích Ca & La religion de Thích Ca : bouddhisme & 1929 & Đức Lưu Phương & 46 \\
Huệ cảnh tây phương. Đạt ma bửu quyện & Histoire du prince Đạt Ma qui se fait bouddhiste & 1930 & Đức Lưu Phương & 36 \\
Thơ trời tu phật & Le culte de l'Eternel et l'adoration du bouddha : bouddhisme & 1929 & Xưa Nay & 94 \\
Tịnh độ tông & Prières bouddhiques & 1929 & Xưa Nay & 82 \\
Đại đạo truyền chơn & Propagation du bouddhisme & 1930 & Đức Lưu Phương & 77 \\
Đạo phật Thích Ca. Thuyết pháp & Le bouddhisme. Ses théories & 1930 & Đức Lưu Phương & 22 \\
Thiên cơ trực chỉ & Le bouddhisme ésotérique & 1930 & Đức Lưu Phương & 62 \\
Đạo phật Thích Ca. Phật giáo khuyên’ tu & La religion de Thích Ca. Exhortation à suivre le bouddhisme & 1932 & Bùi văn Nhân & 31 \\
Tịnh độ tông « Tịnh độ hưu vi ». Quyển nay gồm rút cả « Tây qui trực chỉ » và « Lao nhơn đắc ngộ » & Livre de prières « Tịnh độ tông », « Tây qui trực chỉ » et « Lao nhơn đắc ngộ », bouddhisme & 1932 & Xưa Nay & 84 \\
Tịnh độ tông « Tịnh độ hưu vi ». Quyển nay gồm rút cả « Tây qui trực chỉ » và « Lao nhơn đắc ngộ » & Livre de prières « Tịnh độ tông », « Tây qui trực chỉ » et « Lao nhơn đắc ngộ », bouddhisme & 1932 & Thạnh mậu & 93 \\
Đoạn dâm căng & Pour maîtriser le désir, bouddhisme & 1932 & Đức Lưu Phương & 46 \\
Ai muốn tu? Phật giáo vấn đáp & Qui veut être bonze? (bouddhisme) & 1933 & Đức Lưu Phương & 24 \\
Cao đài chơn giải & La doctrine caodaïste expliquée et commentée & 1933 & Thanh thị Mậu & 48 \\
Đạo Phật Thích Ca & Le bouddhisme ésotérique & 1933 & Bảo Tồn & 43 \\
Đeo theo chưng Phật & Sur les traces de Bouddha & 1932 & Đức Lưu Phương & 92 \\
Dục tâm. Tâm hư tắc thân ngưng & Le désir (bouddhisme) & 1932 & Đức Lưu Phương & 28 \\
Khẩu khuyết & L’éducation de la respiration, préceptes à l’usage des bonzes & 1933 & Bảo Tồn & 28 \\
Lục tự chơn giải & Explication des six mots « Nam mô A di đà phật » (Salut au bouddha Amidah) & 1933 & Đức Lưu Phương & 40 \\
Phật đạo: Giải về hai chữ Đạo Đức & La doctrine du bouddhisme. Dissertation sur la religion et la vertu & 1932 & Xưa Nay & 40 \\
Phật giáo khuyến tu & Encouragement au bouddhisme & 1932 & Huỳnh Kim Danh & 28 \\
Phép công phu & L’éducation du corps, préceptes à l’usage des bonzes & 1933 & Đức Lưu Phương & 70 \\
Tại sao tôi tu Phật & Pourquoi je suis bouddhiste? & 1932 & Đức Lưu Phương & 42 \\
Tu thân... & L’éducation de soi-même (bouddhisme) & 1933 & Xưa Nay & 80 \\
Đạo Phật Thích Ca. Lục tự chơn giải & Explication des six mots : Nam mô A di đà Phật (salut au Bouddha Amidah) & 1933 & Bảo Tồn & 94 \\
Đạo Phật Thích ca. Gốc đạo phật là « Từ bi Bác ái » & Les dogmes fondamentaux du bouddhisme : charité et fraternité & 1933 & Bảo Tồn & 48 \\
Gương Huệ & La prière « Huê cảnh » traduite et expliquée & 1933 & Bảo Tồn & 37 \\
Một chữ thương & La pitié (bouddhisme) & 1933 & Bảo Tồn & 46 \\
Liên Hoa đạo tập & Le Lotus de Rama-Krishna & 1934 & Bảo Tồn & 49 \\
Đạo Phật Thích ca & Le bouddhisme & 1933 & Bảo Tồn & 16 \\
Đạo phật Thích ca & Le bouddhisme & 1934 & Bảo Tồn & 32 \\
Công phu & La doctrine du cœur & 1935 & Xưa Nay & 11 \\
Đạo khả đạo & Le véritable chemin de la religion & 1935 & Bảo Tồn & 22 \\
Đời người giải thoát & La vie libérée & 1935 & Đức Lưu Phương & 18 \\
Phật giáo & Le bouddhisme & 1935 & Bảo Tồn & 24 \\
Phép thanh tịnh & Bouddhisme. La pureté & 1935 & Bảo Tồn & 24 \\
\end{longtable}


Parmi les publications de Nguyễn Kim Muôn, \textit{Đạo phật thích ca} se distingue par l'existence de plusieurs numéros ou volumes. Plus précisément, l'édition publiée par Bảo Tồn en 1933 est désignée comme << facs 1 >> (probablement le premier fascicule ou fac-similé), comportant 16 pages. En 1934, ce même éditeur publie d'autres parties de cette œuvre, totalisant 32 pages et identifiées comme << facs 2,3,4,5,6 >>. Cette numérotation séquentielle indique une publication en plusieurs segments.

Concernant les rééditions, plusieurs œuvres de Nguyễn Kim Muôn sont concernées. Par exemple \textit{Tịnh độ tông} qui est publié initialement en 1927 par Thạnh Mậu avec 93 pages, puis réédité en 1929 par Xưa Nay avec 82 pages. En 1932, cette même œuvre fait l'objet d'une nouvelle publication auprès de plusieurs imprimeurs sous les titres \textit{Tịnh độ tông} ou \textit{Tịnh độ hưu vi}. 

Ce livre est le résumé de deux ouvrages : \textit{Tây qui trực chỉ} et \textit{Lao nhơn đắc ngộ}, imprimés à la fois par Xưa Nay (84 pages) et Thạnh mậu (93 pages). De manière similaire, l'imprimerie Xưa Nay publie \textit{Phật giáo khuyên tu} en 1928 et le réédite en 1929. L'œuvre \textit{Thơ trời tu phật} présente également deux versions enregistrées en 1929 par Xưa Nay, toutes deux de 94 pages. 
\textit{Đạo phật thích ca. Thuyết pháp} bénéficie de deux éditions en 1929 et 1930 par Đức Lưu Phương, qui édite \textit{Phép công phu} en 1933 et une version connexe intitulée \textit{Công phu} en 1935. Les titres \textit{Tu thân...} et \textit{Tại sao tôi tu phật} présentent une récurrence thématique en 1932 et 1933. \textit{Đạo phật thích ca} (avec ses variantes de titres) est réédité à plusieurs reprises entre 1929 et 1935. De plus, les publications de \textit{Lục tự chơn giải} en 1933, et \textit{Đạo phật thích ca. Lục tự chơn giải} en 1993, suggèrent plusieurs rééditions sur une période prolongée.

Au cours de sa période d'activité éditoriale, Nguyễn Kim Muôn collabore avec plusieurs maisons d'édition. Voici la liste des principaux imprimeurs ayant travaillé avec l'auteur entre 1927 et 1935 :

\begin{itemize}
    \item \textbf{Thạnh Mậu}, qui publie \textit{Tịnh độ tông} en 1927, et en fait une réimpression en 1932.
    \item \textbf{Xưa Nay}, qui édite plusieurs œuvres, notamment \textit{Phật giáo khuyên tu} en 1928, et en fait une réimpression en 1929, \textit{Thơ trời tu phật} en 1929, ainsi que deux rééditions de \textit{Tịnh độ tông} en 1929 et 1932, \textit{Phật đạo giải về hai chữ đạo đức} en 1932, \textit{Tu thân...} en 1933, et \textit{Công phu} en 1935.
    \item \textbf{Đức Lưu Phương}, qui imprime \textit{Tịnh độ vô vi} en 1928, \textit{Kim cang kinh chơn giải} en 1929, \textit{Chấn hưng phật giáo} en 1929, \textit{Đạo có một} en 1929, \textit{Đạo phật thích ca. Thuyết pháp} en 1929, avec une réédition en 1930, \textit{Phật giáo vệ sinh} en 1929, \textit{Đạo phật thích Ca} en 1929, \textit{Huệ cảnh tây phương. Đạt ma bửu quyện} en 1930, \textit{Đại đạo truyền chơn} en 1930, \textit{Thiên cơ trực chỉ} en 1930, \textit{Đoạn dâm căng, nam sát bạch hổ, nữ' trảm xích long} en 1932, \textit{Ai muốn tu? Phật giáo vấn đáp} en 1933, \textit{Đeo theo chưng Phật} en 1932, \textit{Dục tâm. Tâm hư tắc thân ngưng} en 1932, le premier tome de \textit{Lục tự chơn giải} en 1933, \textit{Phép công phu} en 1933, \textit{Tại sao tôi tu phật} en 1932, et \textit{Đời người giải thoát} en 1935.
    \item \textbf{Bùi văn Nhân}, qui publie \textit{Đạo phật thích Ca. Phật giáo khuyên tu} en 1932.
    \item \textbf{Huỳnh Kim Danh}, qui édite \textit{Phật giáo khuyến tu} en 1932.
    \item \textbf{Bảo Tồn}, qui réédite \textit{Đạo phật thích ca} en 1933, publie \textit{Cao đài chơn giải } en 1933, \textit{Khẩu khuyết} en 1933, édite le second et troisième tomes de \textit{Lục tự chơn giải} en 1933, ainsi que \textit{Gốc đạo phật là Từ bi Bác ái} en 1933, \textit{Long vân tự. Gia định. Gương huệ. Huê cảnh chơn giải} en 1933, \textit{Một chữ thương} en 1933, \textit{Liên hoa đạo tập} en 1934, les volumes de \textit{Đạo phật thích ca} en 1933 et 1934, \textit{Đạo khả đạo} en 1935, \textit{Phật giáo} en 1935, et \textit{Phép thanh tịnh} en 1935.
\end{itemize}
\noindent\rule{0.35\linewidth}{0.6pt}

\clearpage
\subsection{\textit{Tịnh độ tông} ; Bouddhisme [traduction du dépôt légal] ; Bouddhisme de la Terre pure [notre traduction] - 1927}

\textit{Tịnh độ tông}\footnote{Nguyễn Kim Muôn, \textit{Tịnh Độ Tông} [Bouddhisme de la Terre pure], Thạnh Mậu, Saigon, 1927} est un livre de 83 pages, imprimé à 9000 exemplaires par l'imprimerie Mậu Thị Thạnh à Saïgon en 1927. Il sera ensuite réédité à plusieurs reprises par l'imprimerie Xưa nay pour 1000 tirages le 6 avril 1929, 500 tirages le 29 juin 1929, 1000 tirages le 29 novembre 1929, et 1000 tirages le 15 mars 1932. Ces informations sont tirées des documents référencés et numérisés sur la plateforme \textit{Gallica} de la Bnf, qui liste les ouvrages du dépôt légal de l'Indochine. Les imprimeurs ont dans doute procédé à d'avantage de tirages, mais nous n'en avons pas trouvé de traces dans les ressources mises à disposition. 

\begin{figure}[H]
    \centering
    \includegraphics[width=0.4\textwidth]
    {images/tinhdotong.png}
    \caption{Couverture du livre Bouddhisme de la Terre pure - source : Gallica}
\end{figure}

Au début de sa carrière, Nguyễn Kim Muôn pratique la méthode bouddhiste de la Terre pure. La section Introduction de l'ouvrage commence par l’explication de l'auteur sur les raisons de la rédaction de ce livre. Il s’appuie sur des textes fondamentaux tels que \textit{Tây qui trực chỉ} et \textit{Lão nhơn đắc ngộ} du laïc Trần Phong Sắc\footnote{Trần Phong Sắc (1873 – 1928), de son vrai nom Trần Đình Diệm, connu sous le nom de plume Phong Sắc et le nom de courtoisie Đằng Huy, était un poète, écrivain et auteur de pièces de théâtre cải lương célèbre en Cochinchine au début du XX\textsuperscript{e} siècle.}, ainsi que \textit{Công quả cách}, afin de présenter de manière concise et accessible la pratique du bouddhisme selon la méthode de la Terre pure. Nguyễn Kim Muôn écrit qu'il a rassemblé et compilé toutes ces idées pour rédiger un guide sur la pratique de la Terre pure, permettant d’atteindre rapidement l’état de Bouddha, aussi simple qu’un geste de la main (page 5). Il souligne que cette méthode de pratique est simple et ne nécessite pas de rituels complexes, mais permet tout de même aux pratiquants d’atteindre rapidement le paradis occidental \textit{Cực lạc}. Avec ce livre, Nguyễn Kim Muôn espère ainsi aider les pratiquants en dégageant le chemin étroit pour les êtres afin qu’ils puissent emprunter la grande voie (page 5). Le lecteur est donc encouragé à apprendre la récitation du nom du Bouddha, à faire vœu de refuge\footnote{Prise de refuge : donner sa pleine confiance en les Trois Joyaux que sont le Bouddha (le Guide), le Dharma (le remède) et la Sangha (le Soutien) avec pour motivationm la volonté de mettre un terme à sa propre souffrance, à celle des autres êtres sensibles et ultimement, de réaliser la Bouddhéité.}, à garder un esprit sincère et à suivre une alimentation végétarienne, afin de faciliter l’accomplissement de sa pratique spirituelle.

La section Explication générale précise que la méthode de la Terre pure est une voie spirituelle spécifique enseignée par le Bouddha Shakyamuni pour permettre aux êtres de pratiquer plus facilement. Nguyễn Kim Muôn la décrit comme une méthode rapide, un enseignement simple comme un jeu d’enfant (page 7), transmise dans le \textit{Sūtra Mahāyāna} d’\textit{Amitābha}. Selon lui, cette pratique convient à tout le monde, quel que soit l’âge, la profession ou les conditions de vie, car cette discipline n’impose aucune contrainte et consiste simplement à suivre la voie de l’humanité et, à mi-chemin, la voie du Bouddha, ce qui est complet dans tous les cas (page 7).

L’objectif principal de cette pratique est d’atteindre la naissance en Terre pure, c'est-à-dire qu’après la mort, le pratiquant est accueilli par le Bouddha \textit{Amitābha} dans le monde occidental du \textit{Sukhāvatī}, où il renaît sous la forme d’une fleur de lotus et continue sa progression vers l’éveil (page 8).

Nguyễn Kim Muôn compare la pratique de la Terre pure à d’autres méthodes spirituelles comme l’alchimie interne ou la méditation profonde. Il affirme que la Terre pure est plus rapide et plus sûre (page 9), car selon lui, les autres méthodes sont comme naviguer sur une petite embarcation à travers un vaste océan (page 9). Cette voie permet aux pratiquants d’échapper au cycle des renaissances et d’atteindre l’illumination. Il insiste sur le fait que, qu'on pratique longtemps ou peu, qu’on accumule beaucoup ou peu de mérites, si l’on suit la méthode de la Terre pure, on devient Bouddha (page 8).

L’élément clé de cette pratique est la foi sincère et la persévérance. Nguyễn Kim Muôn conseille aux pratiquants de réciter \textit{Nam mô a di đà phật} au moins 300 000 fois avec un esprit pur afin de recevoir la bénédiction du Bouddha. Il affirme avec certitude que si l’on atteint 300 000 récitations, le Bouddha apparaîtra assurément pour donner son assurance (page 6). En parallèle, respecter les préceptes de base tels que l’alimentation végétarienne, le refus de tuer et la pureté mentale sont des conditions indispensables pour obtenir des résultats dans la pratique.

\textbf{La pratique quotidienne \textit{Nhựt khóa}}

Le \textit{Nhựt khóa} est un rituel de pratique quotidienne, effectué deux fois par jour, le matin et le soir. Il n’existe pas d’horaire fixe, chacun l’adapte à son emploi du temps. Avant de commencer, le pratiquant doit se laver le visage, les mains, se rincer la bouche, se concentrer et calmer son esprit (page 11). Ceux qui savent lire peuvent réciter des écrits bouddhiques à partir du livre de \textit{Nhựt khóa}, tandis que ceux qui ne savent pas peuvent quant à eux réciter \textit{Nam mô a di đà phật} jusqu’à atteindre 300 000 fois ou pratiquer la méthode des Dix Récitations (page 11).

Il affirme que maintenir une pratique quotidienne est essentiel pour stabiliser l’esprit et éviter les distractions de la vie courante. Nguyễn Kim Muôn insiste aussi sur le fait que, si l’on ne peut pas réciter à l’heure habituelle, il est possible de compenser en se tournant vers l’ouest et réciter dix fois le nom du Bouddha. Il recommande également l’usage d’un chapelet noir pour compter les récitations, car avec le temps, il devient sacré et efficace (page 12).

Il est possible de réciter intérieurement à tout moment : en position assise, debout ou même en travaillant. L’auteur nous précise qu'il est utile de réciter les écrits bouddhiques car ainsi, notre esprit est occupé à réciter, et ne se disperse pas vers d’autres préoccupations (page 12).

\textbf{Le culte et l’autel domestique}

Nguyễn Kim Muôn recommande d’installer l’autel domestique en direction de l’ouest, que ce soit à l’intérieur de la maison ou sous un auvent, tant que l’endroit est solennel. Il recommande également de le placer de telle manière à éviter que les femmes ne passent devant (page 13). L’autel peut être grand ou petit selon les moyens, mais doit contenir au minimum une image du Bouddha \textit{Amitābha}. Ceux qui en ont les moyens peuvent y ajouter des représentations de la Triple Gemme \textit{Tam bảo} ou des \textit{Bodhisattvas} comme \textit{Avalokiteśvara (Quan thế âm)} et \textit{Mahāsthāmaprāpta (Đại thế chí)}. L’eau d’offrande doit être de l’eau pure, sans thé ni alcool, car l’eau claire est appelée Eau de Pureté par le Bouddha (page 14).

Le pratiquant doit également préparer des objets tels qu’une lampe en cristal, un brûle-parfum et des fleurs de lotus ou d’autres fleurs pures. Avant d’effectuer les prosternations, il est impératif de se changer et de purifier son esprit, car se prosterner est une marque de respect envers le Bouddha (page 16). Lorsqu’on allume de l’encens, cela symbolise une offrande olfactive au Bouddha et l’expression de sa sincérité, car le Bouddha se nourrit uniquement de sincérité et de pureté d’intention (page 15).

En outre, Nguyễn Kim Muôn précise que l’idéal serait d’avoir un ermitage dédié à la pratique ou un autel éloigné des lieux de sommeil afin d’éviter toute distraction extérieure. Il indique clairement que l'autel doit être éloigné du lit, et que si l’on construit un ermitage, il doit être dans une pièce séparée (page 15).

Objets et offrandes 

Nguyễn Kim Muôn précise que les objets utilisés pour le culte doivent être soigneusement préparés et maintenus dans un état de propreté irréprochable. Il affirme que l’offrande au Bouddha est très simple, qu'elle ne nécessite pas de préparation compliquée, seulement de l’eau pure, des fleurs et des fruits (page 17).

Les éléments essentiels incluent :  

1. L’autel
- Il doit être orienté vers l’ouest.  
- Un petit autel peut contenir une image du Bouddha \textit{Amitābha}, tandis qu’un autel plus grand devrait inclure la Triple Gemme \textit{Tam Bảo}, à savoir le Bouddha \textit{Amitābha}, le \textit{Bodhisattva Avalokiteśvara (Quan thế âm)} et le \textit{Bodhisattva Mahāsthāmaprāpta (Đại thế chí).}
- L’eau d’offrande doit être de l’eau froide purifiée, appelée \textit{Tịnh thủy}. Elle doit être renouvelée quotidiennement avant toute offrande.

     2. Les offrandes   
- Seules certaines fleurs sont recommandées : le lotus, le lys ou le souci, car ce sont des fleurs pures appréciées par le Bouddha.  
- Les fruits doivent être lavés et essuyés avant d’être placés sur l’autel. Nguyễn Kim Muôn insiste sur le fait que, pour plus de pureté, les fruits doivent être soigneusement lavés et essuyés un à un (page 18). 

     3. Autres objets rituels   
- Un brûle-parfum pour y déposer les bâtons d’encens.  
- Une lampe en cristal (ou une lampe à huile ou des bougies), appelée \textit{Ngọn thái-cực-đăng} et accompagnée de quelques paires de bougies.  
- Un récipient spécial contenant de l’eau d’offrande, considérée comme ayant des vertus protectrices contre les mauvais esprits et les maladies (page 17).  

Le pratiquant doit rester pur et respectueux. Avant de faire brûler l’encens, il doit réciter le mantra \textit{Án lam}\footnote{Le mantra bouddhiste \textit{Án lam} (ou \textit{Om ram} en sanskrit) est une formule de purification.} afin de purifier l’offrande et exprimer sa dévotion au Bouddha (page 18). Nguyễn Kim Muôn rappelle que se prosterner est un signe de respect envers le Bouddha. Faire brûler de l’encens, c’est offrir son parfum au Bouddha pour qu’il en soit témoin (page 16).  

Il pointe aussi le fait que les offrandes doivent rester simples et empreintes de sincérité, et met aussi en garde sur le fait que les offrandes doivent rester simples, et que si l’on récite les écrits bouddhiques sans sincérité et avec un esprit dispersé, cela reste une simple formalité extérieure, difficilement reconnue par le Bouddha (page 18).  

Nguyễn Kim Muôn souligne également l’importance de rendre hommage aux divinités protectrices et aux ancêtres pour maintenir un équilibre entre pratique spirituelle et vie familiale. Parmi ces entités, il mentionne : 

\begin{itemize}
    \item \textit{Thổ địa} et \textit{Thổ thần} (Esprits de la terre et des Propriétés foncières) : Ce sont les esprits protecteurs des maisons et des terrains. Nguyễn Kim Muôn recommande d’allumer de l’encens et d’installer une tablette consacrée, le plus souvent ornée de caractères chinois (page 14). 
    \item \textit{Táo quân} (Esprit du Foyer – \textit{Đông trù tư mạng táo phủ thần quân}) : Il veille au bonheur familial et régit les affaires domestiques. L’auteur conseille d'en posséder au moins une statue, ou à défaut, d’accrocher une tablette votive (page 14).  
\end{itemize}

Ces autels secondaires peuvent d’ailleurs être placés sur l’autel principal du Bouddha pour faciliter les rituels, notamment lors des jours de célébration dédiés à chaque divinité. Nguyễn Kim Muôn précise que lors des jours d’anniversaire des divinités, il convient de leur faire une offrande sur l’autel du Bouddha, mais en mentionnant leurs noms dans les prières (page 14).  

L’ermitage \textit{Liêu}   

De plus, Nguyễn Kim Muôn recommande d’aménager un \textit{liêu} (un espace réservé ou un petit ermitage) pour les prières et la récitation quotidienne, en particulier pour ceux qui vivent dans des zones urbaines ou animées. Il explique que l’ermitage est un lieu plus paisible, à l’abri des regards indiscrets, notamment ceux des femmes et des enfants (page 15).  

Cet espace doit être situé dans un endroit calme, loin des chambres ou des lieux bruyants. Si possible, une petite section de l’ermitage peut être dédiée au culte des ancêtres ou aux esprits des défunts, leur permettant ainsi d’atteindre plus facilement la délivrance et la renaissance en Terre pure. Il insiste pour que l’ermitage soit un lieu idéal pour la prière et la récitation des écrits bouddhiques sans être dérangé (page 15).  

Il précise toutefois qu’il n’est pas nécessaire de construire un ermitage de grande taille. L’essentiel est de préserver un environnement pur et de bien l'orienter vers l’ouest. Il écrit que la taille importe peu, mais que le plus important est de respecter l’orientation vers l’ouest (page 15).  

Nguyễn Kim Muôn insiste aussi sur l’importance de l’offrande d’encens, qui symbolise la sincérité du pratiquant envers le Bouddha. Il explique que lorsqu’on allume de l’encens, on offre cinq parfums au Bouddha, pour que la fumée s’élève jusqu’à lui, symbolisant notre connexion avec le monde spirituel (page 16).\\
Les cinq  parfums spirituels \textit{Ngũ hương} sont :  

\begin{itemize}
    \item Le parfum des préceptes \textit{Giới hương} : Observer une conduite pure et respecter les préceptes bouddhiques.  
    \item Le parfum de la concentration \textit{Định tâm hương} : Maintenir un esprit calme et centré.  
    \item Le parfum de la sagesse \textit{Trí huệ hương} : Développer une compréhension éclairée et ne pas tomber dans l’ignorance.  
    \item Le parfum de la vision spirituelle \textit{Tri kiến hương} : Savoir que le Bouddha perçoit notre sincérité. 
    \item Le parfum de la délivrance \textit{Giải thoát hương} : Atteindre un état de paix intérieure et de liberté spirituelle (page 17).  
\end{itemize}

Selon lui, lorsqu’il fait brûler de l’encens, le pratiquant doit joindre les mains, s’incliner et réciter le mantra \textit{Án lam} plusieurs fois pour purifier son esprit et l’offrande. L’auteur insiste sur le fait que la sincérité dans cet acte est essentielle pour que le Bouddha en soit témoin.  

Le \textit{Sūtra} d’\textit{Amitābha}

Nguyễn Kim Muôn présente le \textit{Sūtra} d’\textit{Amitābha (Kinh a di đà)} comme un texte fondamental du bouddhisme de la Terre pure. Ce \textit{sūtra} décrit le paradis occidental \textit{Tây phương Cực lạc} et les mérites du Bouddha \textit{Amitābha}. Il explique que ce monde est un lieu exempt de souffrances, où seul règne le bonheur absolu que dans ce pays, il n’y a aucune souffrance, seulement des joies infinies ; c’est pourquoi il est appelé Terre pure (page 31).

Cet écrit bouddhique décrit la magnificence de ce monde avec sept rangées de balustrades, sept rangées de filets et sept rangées d’arbres, tous composés de joyaux précieux. Il mentionne aussi des bassins d’eau de mérite dont l’eau est toujours pure et rafraîchissante. Nguyễn Kim Muôn insiste sur le fait que le Bouddha \textit{Amitābha} a formulé 48 grands vœux, dont l’un des plus importants est de guider toutes les âmes qui récitent son nom vers la Terre pure après leur mort.

Il souligne donc ainsi l’importance de la récitation du nom du Bouddha \textit{Amitābha}, et affirme que c'est le moyen le plus simple de se connecter à lui et d’obtenir sa bénédiction. Il affirme qu'il suffit d’une sincérité absolue pour être assuré de renaître en Terre pure (page 22).
Nguyễn Kim Muôn encourage les pratiquants à réciter le \textit{Sūtra} d’\textit{Amitābha} régulièrement, surtout lors des jours de pleine lune, du premier jour de chaque mois ou des fêtes bouddhiques, afin d’accumuler des mérites et de faciliter la renaissance en Terre pure.

\subsection{\textit{Phật giáo khuyến tu} ; Morale bouddhique [traduction du dépôt légal] ; Bouddhisme de l’encouragement à la pratique [notre traduction] - 1928}

\textit{Phật giáo khuyến tu} \footnote{Nguyễn Kim Muôn, \textit{Phật giáo khuyên tu} [Bouddhisme de l’encouragement à la pratique], Xưa Nay, Saigon, 1928} est un livre de 32 pages, imprimé à 3500 exemplaires pour sa première édition par l'imprimerie Xưa nay à Saïgon en 1928 et financé par Lê Thị Phó. Il est republié cette même année par la même imprimerie, puis une troisième fois le 26 septembre 1928 pour 1000 exemplaire. Le 5 février février 1929; il est réédité pour une quatrième fois à 1000 tirages à Sa Ðéc par l'imprimerie Ho van. Une cinquième édition de 5000 exemplaires est publiée le 8 juillet 1929; puis une sixième fois le 4 juin 1929, édité par la femme de Nguyễn Kim Muôn, à 1000 exemplaires. Un septième tirage est effectué par l'imprimerie Bùi Văn Nhân à Bến Tre, qui édite 1000 exemplaire supplémentaires le 16 janvier 1932. Cette même année à Saïgon, on édite pour la huitième fois l'ouvrage avec 3000 exemplaires supplémentaires chez l'imprimerie Huỳnh Kim Danh, le 9 décembre 1932. Le livre sera réédité une nouvdlle fois en 1932 par l'imprimerie L'Ouest à Cần Thơ.

\begin{figure}[H]
    \centering
    \includegraphics[width=0.4\textwidth]
    {images/phatgiaokhuyentu.png}
    \caption{Couverture du livre Bouddhisme de l’encouragement à la pratique - source : Gallica}
\end{figure}

Nguyễn Kim Muôn présente une prescription pour Nourrir la Voie [notre traduction] \textit{Toa thuốc Bổ Đạo}\footcite{phatgiaokhuyentu} comme une méthode pour nourrir le cœur et le corps, basée sur le fondement des Cinq Éléments \textit{Ngũ hành} :
\begin{itemize}
    \item Le Métal \textit{Kim}
    \item Le Bois \textit{Mộc}
    \item L'Eau \textit{Thủy}
    \item Le Feu \textit{Hỏa}
    \item La Terre \textit{Thổ}
\end{itemize}

Cette prescription n'est pas un médicament matériel, mais plutôt un ajustement des habitudes et de l'état d'esprit pour atteindre l'équilibre dans la vie et la pratique spirituelle. Il associe chaque élément à un organe du corps : 
\begin{itemize}
    \item Le Métal aux poumons
    \item Le Bois au foie
    \item L'Eau aux reins
    \item Le Feu au cœur
    \item La Terre à l'estomac
\end{itemize}
Chaque élément est régulé par des habitudes saines, comme (page 4) :
\begin{itemize}
    \item Parler peu pour nourrir les poumons
    \item Éviter la colère pour maintenir un foie sain
    \item S'abstenir de désirs pour protéger les reins
    \item Abandonner l'inquiétude pour nourrir le cœur
    \item Manger avec modération pour aider l'estomac à digérer
\end{itemize}
Il affirme que la pratique correcte de la prescription pour Nourrir la Voie aidera à nourrir les Trois Joyaux \textit{Tam Bửu} (Essence, Énergie, Esprit), apportant santé et bien-être. En même temps, il met en garde contre certaines choses à éviter, comme le tabagisme pour les hommes et la consommation de bétel pour les femmes (page 4). Si elle est pratiquée correctement, cette méthode crée un équilibre entre le corps et l'esprit, apportant non seulement une longue vie en bonne santé, mais aussi un soutien pour que le chemin de la pratique spirituelle devienne plus facile et plus efficace.

Nguyễn Kim Muôn souligne que la pratique spirituelle est de la responsabilité de tous, sans distinction de classe ou de circonstances. Il affirme que celui ou celle qui pratique atteint, tandis que celles ou ceux qui ne pratiquent pas n'atteignent pas (page 5), soulignant ainsi que seule la pratique personnelle peut aider les gens à se libérer de la souffrance et à atteindre le bien-être. Selon lui, la vie mondaine n'est qu'un séjour temporaire, où toutes les richesses et la gloire sont dénuées de sens à la mort. L'âme est alors soumise au jugement dans le Pays des Ombres \textit{Quê âm cảnh} et peut même faire face au Roi des Enfers \textit{Diêm vương} si elle porte des péchés de vies antérieures. Il suggère que, sachant cela, dans cette vie éphémère, il est préférable de cultiver la vertu et d'accumuler les mérites, afin que lorsque l'âme retourne au Pays des Ombres, elle soit pure (page 7). La réincarnation et le karma sont les lois qui régissent tous les êtres. Les riches de cette vie le sont souvent grâce aux mérites cachés \textit{âm đức} de leurs ancêtres ou aux mérites accumulés dans des vies antérieures (page 8). À l'inverse, ceux qui sont pauvres et souffrent le sont à cause du mauvais karma qu'ils ont créé auparavant, il affirme que la pauvreté et la souffrance sont le remboursement des mauvaises actions passées (page 10). Nguyễn Kim Muôn encourage donc les riches à ne pas se reposer sur leur fortune actuelle, mais à continuer d'accumuler des mérites, tandis qu'il encourage les pauvres à pratiquer pour améliorer leur destin dans leurs vies futures.

Ainsi, il énumère différentes méthodologies de pratiques spirituelles :
\begin{itemize}
    \item La Culture du Cœur \textit{Tu tâm}
    \item La Culture de l'Immortalité \textit{Tu tiên}
    \item La Culture de la Méditation \textit{Tu thiền}
    \item La Culture du Tantra \textit{Tu mật tông}
    \item La Culture de la Terre pure \textit{Tu tịnh độ}
\end{itemize}
Parmi celles-ci, il considère la Culture de la Terre pure comme la voie la plus facile et la plus efficace, car, comme nous l'avions dit plus tôt, elle convient à toutes les classes sociales et ne nécessite pas de rituels complexes. Il compare donc cette méthode à un raccourci. Selon lui,  pratiquer selon la Terre pure, c'est comme prendre un raccourci, moins fatigant et plus rapide (page 15). En effet, ceux qui pratiquent la Terre pure n'ont qu'à installer un autel au Bouddha \textit{Amitābha} (\textit{Phật A Di Đà}) chez eux, orienté vers l'Ouest, pratiquer le Rituel Quotidien (\textit{Nhựt Khóa}) matin et soir en brûlant de l'encens, en offrant de l'eau fraîche et en récitant le nom de Bouddha \textit{Amitābha} (\textit{Nam mô a di đà phật}). Il souligne que réciter ces six mots jusqu'à 300 000 fois, c'est être reconnu par le Bouddha, ce qui signifie que l'on lui confie sa propre vie (page 19). De plus, il encourage les pratiquants à adopter un régime végétarien, allant du végétarisme périodique (2, 6 ou 10 jours) au végétarisme permanent, et à observer les cinq préceptes fondamentaux :
\begin{itemize}
    \item Ne pas tuer
    \item Ne pas voler
    \item Ne pas commettre d'adultère
    \item Ne pas mentir
    \item Ne pas boire d'alcool
\end{itemize}
Enfin, Nguyễn Kim Muôn adresse des conseils à chaque groupe de personnes. Les personnes âgées doivent pratiquer dès maintenant, car si l'on ne connaît pas le chemin du Paradis pendant sa vie, il est difficile d'échapper aux portes de l'Enfer après la mort (page 24). Les jeunes sont quant à eux encouragés à étudier et pratiquer simultanément pour vivre une vie morale et réussir. Les pauvres doivent pratiquer pour accumuler des mérites cachés et améliorer leur karma, tandis que les riches doivent continuer à cultiver la vertu et à accumuler des mérites pour éviter de renaître dans la souffrance. Il conclut que la pratique de la Terre pure n'apporte pas seulement des bénéfices dans la vie présente, mais que c'est aussi le chemin sûr pour échapper au cycle des réincarnations et atteindre la libération finale (page 26).

Nguyễn Kim Muôn consacre également une section à l'encouragement de la pratique spirituelle pour les femmes (\textit{Phụ-nữ khuyến-tu}) pour souligner le rôle important des femmes dans la pratique spirituelle. Il encourage ces dernières à se consacrer à la pratique, car elles contribuent non seulement à créer une famille heureuse, mais aussi à semer les graines de la moralité chez leurs enfants. Il affirme que la femme est la source de la famille, et que si la famille est corrompue, la société est en désordre (page 27). Aussi, il conseille aux femmes de ne pas s'attacher aux plaisirs du monde, mais plutôt de se concentrer sur le maintien de la moralité, la récitation du nom de Bouddha et les bonnes actions. Pour lui, la Culture de la Terre pure est la voie la plus simple et la plus appropriée, car si les femmes savaient pratiquer la Terre pure, il leur serait facile d'atteindre la Terre pure de l'Ouest \textit{Cực lạc} et d'éviter la souffrance du cycle des réincarnations (page 28). Il encourage particulièrement les femmes à pratiquer le végétarisme, car ce n'est pas seulement un acte de purification du corps, mais aussi une aide pour nourrir la compassion.

Dans la section Les mérites de la récitation du nom de Bouddha (\textit{Công đức niệm phật}), Nguyễn Kim Muôn explique pourquoi la récitation du nom de Bouddha n'apporte pas seulement des mérites à soi-même, mais crée aussi des bénédictions pour la famille et la société. Il précise que celui qui récite le nom de Bouddha est comme celui qui plante des lotus jour après jour, que les fleurs s'épanouissent et répandent son parfum au loin (page 30). La récitation du nom de Bouddha apaise l'esprit, élimine les afflictions et augmente les mérites. Il explique que chaque récitation de \textit{Nam mô a di đà phật} est une graine semée dans le cœur, aidant le pratiquant à surmonter progressivement la souffrance et à se diriger vers la libération.

Dans la section Méthode pour enseigner aux enfants la récitation du nom de Bouddha (\textit{Phương pháp dạy trẻ niệm phật}), Nguyễn Kim Muôn souligne que guider les jeunes enfants à réciter le nom de Bouddha et à vivre une vie morale est la responsabilité des parents. Il suggère que les enfants sont comme une feuille de papier vierge, qu'ils deviennent bons ou mauvais en fonction de l'éducation des adultes (page 33). Il propose une méthode simple consistant à apprendre aux enfants à réciter le nom de Bouddha selon les Six Syllabes (\textit{Lục tự}) et les encourage à faire de bonnes actions dès leur plus jeune âge pour établir une base solide pour leur âme.

Dans la dernière partie de l'ouvrage, Nguyễn Kim Muôn donne des conseils généraux pour la pratique spirituelle (\textit{khuyến tu chung}) et souligne que cette dernière n'apporte pas seulement des bénéfices dans cette vie, mais qu'elle est aussi la meilleure préparation pour la vie future. Il conseille à tous de ne pas attendre d'être vieux pour pratiquer, car à ce moment-là, la santé n'est pas suffisante pour pratiquer la Voie (page 36). Il encourage également chacun à commencer par de petites choses comme le végétarisme, l'observation des préceptes et la récitation régulière du nom de Bouddha chaque jour. En particulier, il rappelle que la foi en Bouddha et la sincérité sont des facteurs décisifs pour la réussite de la pratique spirituelle. L'ouvrage termine en affirmant que quiconque pratiquant selon la vraie méthode de la Terre pure, avec peu ou beaucoup de mérites, atteindra la Terre pure de l'Ouest (Cực lạc) (page 40).

\subsection{\textit{Thờ trời tu phật} ; Pratiquer le bouddhisme [traduit du dépôt légal] ; Vénérer le ciel et pratiquer le bouddhisme [notre traduction] - 1929}

\textit{Thờ trời tu phật}\footnote{\textit{Thờ trời tu Phật} [Vénérer le ciel et pratiquer le bouddhisme], Xưa Nay, Saigon, 1929} est un livre de 35 pages, imprimé à 15000 exemplaires pour sa première édition par l'imprimerie Xưa nay à Saïgon le 17 janvier 1929, puis à 1000 exemplaires pour sa quatrième édition le 4 décembre 1929. Des informations manquent pour estimer la quantité totale d'exemplaires édités à ce jour.
\textit{« Thờ trời tu phật »}, de Nguyễn Kim Muôn fournit un guide complet de la pratique bouddhiste, en mettant l'accent à la fois sur les connaissances théoriques et les applications pratiques. 
\begin{figure}[H]
    \centering
    \includegraphics[width=0.4\textwidth]
    {images/Thotroituphat.png}
    \caption{Couverture du livre Vénérer le ciel et pratiquer le bouddhisme - source : Gallica }
\end{figure}

Nguyễn Kim Muôn commence par souligner l'importance de respecter les trois grandes fêtes que la fête de Dieu (\textit{Đức chúa trời}) en janvier, la fête du Bouddha \textit{Amitābha} (\textit{Phật a di đà}) en novembre et la fête du nirvana (\textit{Niết bàn}) en décembre. Il explique que le respect de ces trois fêtes témoigne de la piété envers le ciel et le Bouddha, aidant les gens à vivre en accord avec la morale. Nguyễn Kim Muôn affirme que ce livre n'est ni une écriture canonique ni une traduction, mais plutôt un partage de ses expériences et de sa volonté personnelle d'aider les autres à atteindre l'éveil.

Aussi, dans la section suivante, Nguyễn Kim Muôn présente les trois niveaux de pratiquant spirituel que le laïc (\textit{Cư sĩ}) qui pratique chez lui, le pratiquant (\textit{Đạo nhân}) qui renonce aux désirs et se consacre à la pratique, et le maître spirituel (\textit{Đạo sư}) qui renonce au monde, coupe tous les liens mondains pour se consacrer uniquement à la pratique et atteindre un éveil clair. Il partage sa propre histoire, lorsqu'il a rencontré son véritable bonze \textit{Chơn sư}, qui lui avait conseillé de ne pas quitter sa famille car sa dette envers le monde était encore lourde, et qu'il devait accomplir sa mission d'aider les êtres à s'éveiller. Ce conseil lui a fait prendre conscience de sa grande responsabilité dans la diffusion de deux enseignements principaux que la Terre pure (\textit{Tịnh độ}), avec ses aspects manifesté (\textit{Hữu vi}) et non-manifesté (\textit{Vô vi}), et les quatre bonheurs (\textit{Bốn hạnh}) du ciel et du Bouddha. Il décrit cela comme un lourd fardeau qu'il doit porter, semblable à un palanquier portant deux charges à chaque extrémité.

Nguyễn Kim Muôn raconte sa rencontre avec un ami pratiquant de la Société théosophique \textit{Hội thông thiên học}, où tous deux ont trouvé des similitudes dans les idéaux et les principes du bouddhisme, ainsi que dans la voie du ciel (\textit{Đạo trời}). Il a constaté que ces deux religions visent une vérité commune que le salut des êtres. Il a donc commencé à écrire des articles et des essais sur la revitalisation du bouddhisme, en collaboration avec la grande assemblée du \textit{Đạo} (\textit{Đại đạo hội}) pour aider les gens à retrouver leurs racines. Ce faisant, il encourage également les lecteurs à se rappeler qu'en tant qu'êtres humains, ils doivent vénérer le ciel et pratiquer le bouddhisme, car c'est essentiel pour maintenir l'équilibre et éviter la souffrance du cycle des réincarnations.

Dans la dernière partie de l'extrait, Nguyễn Kim Muôn aborde la pratique de la Terre pure et explique que les pratiquants doivent se concentrer sur la récitation du nom du Bouddha \textit{Amitābha} 300 000 fois pour obtenir le sceau et la reconnaissance du Bouddha. Il conseille également de ne pas trop attendre de signes surnaturels comme des auras ou des rêves de Bouddha, mais de se concentrer sur la purification de l'esprit et la pratique régulière. Il écrit que le Bouddha n'est pas quelque part au loin, il est en nous, si le cœur est pur, le Bouddha apparaîtra (page 10). Le livre souligne que le bouddhisme n'est pas seulement un chemin vers l'éveil, mais aussi un moyen d'améliorer sa vie présente et d'accumuler des mérites pour les vies futures.

Nguyễn Kim Muôn débute cette section en exprimant son respect et sa commémoration envers Monsieur Trần Phong Sắc, qu'il considère comme une figure importante dans le développement et la diffusion de la pratique de la Terre pure. Il mentionne le rôle de pionnier de Trần Phong Sắc dans la rédaction d'œuvres classiques telles que \textit{Tây qui trực chỉ} (Guide direct vers l'Ouest) et \textit{Lão nhơn đắc ngộ} (L'éveil du vieil homme), qui ont aidé des millions de fidèles à comprendre les valeurs fondamentales du bouddhisme. Nguyễn Kim Muôn décrit Trần Phong Sắc comme un homme vertueux, toujours dévoué à guider les gens dans la pratique correcte, en particulier dans la récitation du nom de Bouddha et l'observance des préceptes.

Nguyễn Kim Muôn relate ses souvenirs et ses discussions avec Trần Phong Sắc, où tous deux ont débattu du sens de la pratique spirituelle et de la propagation du bouddhisme. Il souligne que le décès de Trần Phong Sắc n'est pas seulement une grande perte pour la communauté bouddhiste, mais aussi un rappel que chacun doit poursuivre sa mission dans le développement de la pratique de la Terre pure. Nguyễn Kim Muôn affirme que la volonté du maître est claire, les générations futures doivent la suivre et ne pas négliger ce qu'il a éclairé (page 12).

Après la commémoration, Nguyễn Kim Muôn explique en détail les principes de pratique spirituelle promus par Trần Phong Sắc. Il souligne que le bouddhisme est la voie de la libération, axée sur le maintien d'un esprit pur et tourné vers le Bouddha \textit{Amitābha} par la récitation de son nom. Nguyễn Kim Muôn conseille de nouveau aux pratiquants d'observer les préceptes, de ne pas tuer, de ne pas commettre d'adultère et de pratiquer le végétarisme comme moyen de cultiver le corps et l'esprit. Il écrit que la sincérité dans la récitation du nom de Bouddha et le maintien de la moralité sont le fondement pour atteindre la renaissance (page 14).

Nguyễn Kim Muôn souligne également que la pratique spirituelle n'est pas seulement une responsabilité individuelle, mais aussi un moyen de contribuer à la société. Trần Phong Sắc a laissé un exemple brillant par sa vie dévouée à la voie et à la communauté. Il écrit que celui qui pratique véritablement ne cherche pas seulement le bien-être pour lui-même, mais apporte également la lumière de la morale aux autres (page 15).

Cette section se termine par un rappel de Nguyễn Kim Muôn que ceux qui lui succèdent, comme lui, doivent continuer à développer l'héritage laissé par Trần Phong Sắc. Nguyễn Kim Muôn encourage donc chacun à apprendre continuellement, à étudier les écritures classiques et à les appliquer dans la pratique pour apporter des bénéfices à soi-même et à la communauté. Il conclut que vivre comme Trần Phong Sắc, c'est vivre une vie qui n'est pas gaspillée, laissant une marque indélébile dans le cœur des gens (page 16).

Nguyễn Kim Muôn commence cette section par une présentation générale de l'une des écoles importantes du bouddhisme — l'école de la Terre pure (\textit{Tịnh độ tông}). Il répète à nouveau qu'il s'agit d'une école mettant particulièrement l'accent sur la récitation du nom du Bouddha \textit{Amitābha}, afin de prier pour la renaissance dans la Terre pure de l'Ouest (\textit{Cực lạc}) ; que l'école de la Terre pure convient à toutes les classes sociales, des personnes les plus érudites à celles qui ont peu d'instruction, car sa méthode de pratique reste simple mais très efficace. Il écrit : La Terre pure est un chemin large et facile à parcourir, adapté aux personnes occupées comme à celles qui ont beaucoup de temps pour pratiquer la voie (page 17).

Nguyễn Kim Muôn expose clairement les points essentiels de cette pratique, notamment la récitation du nom de Bouddha, le vœu de renaissance, la pratique de bonnes actions et l'accumulation de mérites. Parmi ceux-ci, la récitation du nom \textit{Nam mô a di đà phật} avec sincérité est l'élément le plus important. Il souligne que la récitation du nom de Bouddha est le moyen d'ouvrir la porte de la Terre pure de l'Ouest, aidant l'esprit à se libérer des afflictions et à mener les gens vers le bien-être (page 18).

Nguyễn Kim Muôn compare également l'école de la Terre pure à d'autres écoles bouddhistes, comme l'école zen ou l'école tantrique. Il soutient que ces dernières nécessitent une maîtrise des écritures ou de rituels complexes, tandis que l'école de la Terre pure se concentre sur la simplicité et la sincérité. Il écrit que  le bouddhisme ne demande pas aux pratiquants de s'adonner à l'ascèse ou à des rêveries lointaines. Il suffit de réciter le nom de Bouddha, de faire le vœu, pour atteindre la renaissance (page 19).

Nguyễn Kim Muôn poursuit donc en expliquant que la Terre pure de l'Ouest n'est pas seulement un lieu idéal, mais aussi un objectif réalisable pour les pratiquants. Il décrit la Terre pure comme un lieu sans souffrance, sans cycle des réincarnations, où règnent uniquement le bien-être et la joie éternelle. Il souligne que tous les êtres peuvent atteindre la Terre pure s'ils persévèrent dans la pratique de cette école. Cela fait de la Terre pure la voie la plus accessible et la plus facile pour surmonter les souffrances de la vie mondaine.

Nguyễn Kim Muôn conclut cette section en encourageant les pratiquants à commencer dès maintenant. Il écrit que la Terre pure de l'Ouest attend toujours ceux qui ont un cœur sincère. N'attendez pas d'être vieux pour pratiquer, car à ce moment-là, le temps et l'énergie ne seront plus suffisants (page 20). Il conseille à chacun de réciter le nom de Bouddha quotidiennement, de garder un cœur sincère et de faire de nombreuses bonnes actions pour accumuler des mérites.

Nguyễn Kim Muôn, dans la section \textit{Nói về đạo trời} (Parler de la voie du ciel), souligne que la voie du ciel et le bouddhisme sont étroitement liés, tous deux visant à sauver les êtres et à aider les gens à vivre en accord avec la morale. Il explique que la voie du ciel est la morale fondamentale, axée sur le maintien de la vertu, l'accomplissement des responsabilités envers la famille et la société, et le maintien d'une relation harmonieuse entre le ciel, la terre et l'homme. Nguyễn Kim Muôn écrit que le ciel est le père de toutes choses, la voie du ciel est la base de toutes les bonnes actions (page 21).

La voie du ciel met l'accent sur trois grands principes, appelés \textit{Tam cang} (trois liens) que la loyauté envers le souverain, la piété filiale envers les parents et la justice envers les amis. En même temps, les gens doivent pratiquer \textit{Ngũ thường} (cinq constantes), qui incluent la bienveillance, la justice, la bienséance, la sagesse et la confiance. Nguyễn Kim Muôn explique qu'en respectant ces principes, les gens atteindront l'harmonie avec le ciel et la terre, créant ainsi des bénédictions pour eux-mêmes et les générations futures.

Nguyễn Kim Muôn explique également la différence entre la voie du ciel et le bouddhisme. Si le bouddhisme met l'accent sur la libération de la souffrance et l'atteinte du nirvana ou de la Terre pure, la voie du ciel se concentre sur le maintien de l'ordre et de la moralité dans la société humaine. Il affirme que les deux voies se complètent que la voie du ciel est le fondement de la vie mondaine, tandis que le bouddhisme est le chemin qui nous aide à échapper au cycle des réincarnations (page 22).

Nguyễn Kim Muôn poursuit en expliquant que pour vivre en accord avec la voie du ciel, les gens doivent savoir vénérer le ciel và obéir à la volonté céleste. Il écrit vivre sous le ciel sans le vénérer, c'est comme un enfant ingrat envers ses parents (page 23). Vénérer le ciel ne se limite pas à offrir de l'encens ou à prier, c'est aussi vivre en accord avec la morale, maintenir un esprit pur et faire le bien. Il souligne que ceux qui vivent selon la voie du ciel recevront des bénédictions du ciel, tandis que ceux qui agissent contre la morale connaîtront le malheur et l'infortune.

En conclusion, Nguyễn Kim Muôn encourage chacun à combiner la voie du ciel et le bouddhisme dans sa vie. Il souligne que vénérer le ciel et pratiquer le bouddhisme ne sont pas contradictoires mais complémentaires, aidant ainsi les gens à atteindre l'harmonie à la fois dans la vie mondaine et spirituelle. Il écrit : Vénérer le ciel et pratiquer le bouddhisme sont deux façons de nourrir le corps et l'esprit, tous deux guidant les gens vers le bien (page 24).

Nguyễn Kim Muôn continue de souligner l'importance de la combinaison de la voie du ciel et du bouddhisme dans la vie quotidienne. Il explique que chacun doit vénérer le ciel pour honorer le créateur et pratiquer le bouddhisme pour cultiver un esprit pur. Nguyễn Kim Muôn écrit que vénérer le ciel pour maintenir la moralité humaine, pratiquer le bouddhisme pour se libérer de la souffrance du cycle des réincarnations (page 24). Il considère que ce n'est pas seulement une façon de bien vivre dans la vie présente, mais aussi le chemin vers la paix et le bonheur durables.

Dans cette section, Nguyễn Kim Muôn consacre un passage à l'importance de la pratique de la récitation du nom de Bouddha dans la vie quotidienne. Il souligne que la récitation du nom de Bouddha n'est pas seulement un acte spirituel, mais aussi un moyen de cultiver l'esprit, d'aider les gens à mieux vivre et à éviter les mauvaises actions. Il écrit ceux qui récitent le nom de Bouddha non seulement maintiennent un esprit pur, mais sont également bénis par le ciel et le Bouddha, réduisant ainsi la souffrance et rencontrant plus de chance dans la vie (page 25).

Nguyễn Kim Muôn donne également des conseils spécifiques sur la façon d'installer un autel et de pratiquer le culte à la maison. Il insiste sur le fait que l'autel doit être placé dans un endroit solennel et propre, orienté vers l'ouest en signe de respect envers le Bouddha \textit{Amitābha}. Offrir de l'eau fraîche, brûler de l'encens et réciter le nom \textit{Nam mô a di đà phật} chaque jour sont des moyens de maintenir un lien spirituel avec le Bouddha. Il encourage chacun à le faire avec sincérité, car le Bouddha ne regarde pas les offrandes matérielles, mais le cœur sincère de celui qui l'adore (page 26).

Dans la section suivante, Nguyễn Kim Muôn met l'accent sur l'importance de faire le bien et d'accumuler des mérites. Il affirme que faire le bien n'aide pas seulement les autres, mais est aussi un moyen d'accumuler des mérites, créant des bénédictions pour soi-même et sa famille. Il écrit cue celui qui fait le bien est comme celui qui sème de bonnes graines, qui récoltera des fruits sucrés le moment venu (page 27). Nguyễn Kim Muôn note également que, lorsqu'on fait le bien, il ne faut pas s'attendre à recevoir quelque chose en retour, mais agir par compassion et sincérité.

Enfin, il conclut que les gens doivent accomplir deux tâches simultanément : vénérer le ciel et pratiquer le bouddhisme. Vénérer le ciel aide à maintenir la moralité et l'ordre dans la vie mondaine, tandis que pratiquer le bouddhisme est le chemin vers la libération et la renaissance dans la Terre pure de l'Ouest. Il interpelle le lecteur : Vivez de manière à accomplir votre devoir d'être humain et à atteindre la sérénité dans votre âme, c'est la façon la plus complète de vivre (page 28).

Nguyễn Kim Muôn débute la section \textit{Chấn hưng phật giáo} (\textit{Revitalisation du bouddhisme}) en affirmant que cette tâche est non seulement importante mais aussi monumentale. Il écrit : La revitalisation du bouddhisme n'est pas une mince affaire, je ne suis pas assez talentueux pour la réaliser seul. Il faut au moins que plusieurs personnes se répartissent dans tous les lieux pour prêcher les écrits bouddhiques, ou pour les expliquer (page 39). Cependant, il insiste sur le fait que cela doit être fait en transmettant la vérité, sans recourir à la magie noire ou à la superstition pour attirer les gens. Il s'oppose fermement aux pratiques telles que dessiner des talismans, réciter des incantations ou effectuer des rituels non orthodoxes, car cela nuit à la nature noble du bouddhisme.

Nguyễn Kim Muôn souligne que les écritures jouent un rôle crucial dans la revitalisation du bouddhisme. Il écrit : Les écritures sont quelque chose qui peut, à l'avenir, éveiller tout le monde, enseigner aux gens la voie, enseigner à faire le bien et à éviter le mal (page 40). Il encourage la traduction et l'impression des écrits en \textit{quốc ngữ} (écriture nationale) afin de les diffuser largement à toutes les classes sociales, au lieu d'utiliser uniquement les caractères chinois que la plupart des gens ne peuvent pas lire. Il considère cela comme une approche appropriée pour que les gens comprennent les valeurs morales fondamentales, même s'ils ne pratiquent pas la voie en profondeur.

Dans cette section, Nguyễn Kim Muôn exprime également sa déception face à l'état de nombreux temples à cette époque. Il critique certains moines et nonnes qui ne respectent pas les préceptes et ne vivent pas en accord avec l'esprit du bouddhisme. Il écrit : << Ces moines qui mangent de la viande, violent les préceptes, ont femme et enfants... ternissent la noblesse du bouddhisme >> (page 38, notre traduction). Nguyễn Kim Muôn appelle à la rigueur de la part des pratiquants, tout en soulignant leur rôle en tant que modèles pour la communauté.

Nguyễn Kim Muôn explique que la Terre pure manifestée (\textit{Tịnh độ hữu vi}) est la première étape, la base nécessaire pour atteindre la Terre pure non-manifestée (\textit{Tịnh độ vô vi}), tout comme un arbre a besoin de racines pour avoir des branches. La pratique de la Terre pure manifestée est résumée en six mots : \textit{Nam mô a di đà phật}, ce qui signifie confier son corps au Bouddha \textit{Amitābha} dans la Terre pure de l'Ouest. Il souligne que les gens sont liés par les « cinq impuretés du monde » et les « six désirs », mais les six syllabes d’\textit{Amitābha} sont le remède aux six sens (yeux, oreilles, nez, langue, corps, esprit). Plus précisément, \textit{Nam} purifie les yeux, \textit{mô} les oreilles, \textit{a} le nez, \textit{di} la langue, \textit{đà} le corps et \textit{phật} l'esprit (pages 47-48).

Ceux qui pratiquent la Terre pure manifestée doivent pratiquer le rituel de repentance (\textit{Sám hối}) pour éliminer le karma négatif et purifier leur âme. La repentance est comparée à un acte de lavage des fautes, aidant les pratiquants à poursuivre leur chemin spirituel dans la pureté. Il écrit : La pratique de la Terre pure a pour base le rituel de repentance, comme un baptême, pour se purifier avant de pratiquer (page 48). Il conseille également aux pratiquants d'effectuer la repentance à des jours fixes du mois, face à un autel bouddhiste installé avec soin et respect.

Nguyễn Kim Muôn insiste sur le fait que la pratique du rituel quotidien (\textit{Nhựt khóa}) — le matin et le soir — est essentielle pour ceux qui pratiquent la Terre pure. Ceux qui savent lire doivent apprendre et réciter les écrits bouddhiques, tandis que ceux qui savent peu ou pas lire peuvent réciter le nom du Bouddha \textit{Amitābha}. Il souligne que chaque rituel quotidien ne prend qu'un quart d'heure et que tous ceux qui ont du temps libre peuvent le pratiquer (page 49).

Nguyễn Kim Muôn explique également que, lorsqu'on installe un autel bouddhiste, il n'est pas nécessaire de procéder à un rituel d'ouverture des yeux, mais qu'il suffit de suspendre une image du Bouddha avec sincérité. Il écrit : Le pouvoir spirituel réside en moi, pas dans l'image. Suspendez-la et inclinez-vous avec sincérité, et elle deviendra naturellement sacrée (pages 49-50). Il rejette ainsi l'idée de vénérer uniquement dans son cœur, mais montre que la sincérité des actes de respect permet de protéger sa famille du malheur au travers d'une image qui rappelle à chacun qu'il est capital de se tourner vers le bien.

Les pratiquants de la Terre pure manifestée sont encouragés à réciter 300 000 fois le nom \textit{Nam mô a di đà phật}. Il affirme que lorsqu'ils atteignent ce nombre, le Bouddha \textit{Amitābha} les reconnaîtra et les aidera à renaître dans la Terre pure de l'Ouest. Il écrit pour ceux qui pratiquent la Terre pure, le plus important est de maintenir un esprit pur, sinon tous les efforts de pratique seront vains (pages 50-51).

À la fin de cette section, Nguyễn Kim Muôn raconte des histoires sur les réponses miraculeuses de la pratique de la Terre pure, như le fait que lui et d'autres ont reçu le sceau du Bouddha. Ces histoires servent de preuves pour affirmer l'efficacité et la réalité de la Terre pure manifestée. Il conclut qu'on veut voir des réponses miraculeuses, essayez de pratiquer la Terre pure, car vouloir les voir sans pratiquer est une illusion (pages 54-55).

Nguyễn Kim Muôn commence la section Terre pure non-manifestée (\textit{Tịnh độ vô vi}) en la reliant à la philosophie de la vie, en particulier au cycle de naissance, vieillesse, maladie et mort (\textit{Sanh, lão, bệnh, tử}), qu'il considère comme les souffrances fondamentales de l'être humain. Il explique que la cause profonde de la souffrance est l'ignorance (\textit{Mê muội}) - la racine des émotions, des désirs et du cycle des réincarnations. Selon lui, pour se libérer de la souffrance, l'homme doit se détacher de l'ignorance, progresser vers l'émotion (\textit{Cảm động}), la connaissance (\textit{Tri biết}), la forme (\textit{Sắc tướng}), les six sens (\textit{Lục căn}), et finalement éliminer complètement les désirs et le cycle des réincarnations (pages 71-72). Il cite le livre occidental \textit{La Vie du Bouddha} pour expliquer que l'ignorance est la source de la souffrance, pour éliminer la souffrance, il faut mettre fin au cycle des réincarnations en renonçant au désir sexuel et à la création (page 72). C'est le chemin vers la libération et la purification de l'âme. 

Nguyễn Kim Muôn poursuit en expliquant le concept de non-agir (\textit{Vô vi}), qu'il considère comme l'état de « faire sans faire », atteignant le vide absolu dans l'esprit. Il cite le \textit{Kinh kim cang} (sūtra du diamant) : Si un bodhisattva a la pensée du soi, la pensée de la personne, la pensée des êtres vivants, la pensée de la vie, alors ce n'est pas un bodhisattva (page 73). Il souligne que la pratique de la Terre pure non-manifestée ne se limite pas à l'essor de l'esprit, mais exige également que l'on nourrisse \textit{Tam bửu} (trois joyaux : essence, énergie, esprit). Ces trois éléments sont la base du développement spirituel et physique. Il écrit que le pratiquant doit savoir comment nourrir, purifier et éduquer sa propre âme (page 74). Cette pratique vise à atteindre un état de transcendance du cycle des réincarnations, où l'on n'est plus prisonnier de la souffrance (pages 74-75).

Nguyễn Kim Muôn partage également que, dans le cheminement vers la Terre pure non-manifestée, il faut prendre soin de son corps physique comme d'un « réceptacle de trésors ». Il écrit si le corps s'affaiblit et disparaît, il n'y aura plus rien pour contenir l'âme et ces trois joyaux (page 75). Par conséquent, il propose la prescription « nourrir la voie » (\textit{Bổ đạo}) pour nourrir le corps, composée de cinq ingrédients représentant les cinq éléments : le métal (poumons), le bois (foie), l'eau (reins), le feu (cœur) et la terre (estomac). Il explique que le développement du corps aide les pratiquants à atteindre la santé, la longévité et à maintenir les trois joyaux (page 75).

Enfin, il souligne que ceux qui pratiquent le non-agir doivent atteindre un état de maîtrise totale, transcendant la vie et la mort. Ce faisant, ils échapperont au cycle des réincarnations et atteindront le royaume de la Terre pure de l'Ouest.

Enfin, Nguyễn Kim Muôn encourage tous les pratiquants, des moines aux bouddhistes laïcs, à s'unir pour revitaliser le bouddhisme. Il estime que la revitalisation n'est pas seulement la responsabilité des moines, mais nécessite également la contribution de toute la société. Il conclut ceux qui suivent le bouddhisme, remplissez votre devoir, faites le bien et évitez le mal, récitez le nom de Bouddha chaque jour pour maintenir un esprit paisible et construire une société meilleure (page 47).

Dans la dernière partie, Nguyễn Kim Muôn aborde les rites de la vie des pratiquants de la Terre pure, notamment ce qui concerne le travail et le foyer (\textit{Quan}), le mariage (\textit{Hôn}), les funérailles (\textit{Tang}) et les offrandes (\textit{Tế}), en insistant sur la simplicité et la conformité aux enseignements de la Terre pure. Il conseille aux fidèles d'unifier les pratiques, de créer une cohérence au sein de la communauté et de vivre en accord avec l'esprit du bouddhisme.

Concernant le travail et le foyer : lors de la construction d'une maison, de l'organisation d'une fête ou de la prise de fonction, les pratiquants de la Terre pure doivent conserver un esprit de simplicité, organiser des repas végétariens et inviter parents, amis et coreligionnaires à y participer. Ce n'est pas seulement une occasion de partager la joie, mais aussi de communiquer et de discuter de la voie, contribuant ainsi à la diffusion des enseignements bouddhistes. Il souligne que cela ne doit pas être fait par ostentation, mais dans le but de maintenir de bonnes relations avec la communauté et la famille.

Concernant le mariage, en ce qui concerne le mariage, Nguyễn Kim Muôn conseille que les cérémonies de mariage soient célébrées avec simplicité et solennité. Le banquet de mariage doit être végétarien, en évitant le gaspillage. Il souligne également que le couple doit se présenter devant l'autel bouddhiste, prier le ciel et le Bouddha pour leur bénédiction, puis s'incliner devant l'autel des ancêtres en signe de piété filiale.

Concernant les funérailles, Nguyễn Kim Muôn consacre de nombreux conseils aux rites funéraires, car il s'agit d'une question importante. Il divise les funérailles en deux catégories : les funérailles d'un pratiquant de la Terre pure et les funérailles d'un membre de la famille d'un pratiquant de la Terre pure. S'il s'agit des funérailles d'un membre de la famille, les rites peuvent être pratiqués selon les coutumes, mais il convient d'éviter les récitations d'écrits bouddhiques ou les offrandes trop compliquées. Dans le cas où la personne sur le point de décéder est un pratiquant de la Terre pure, les coreligionnaires doivent se relayer pour réciter le nom de Bouddha et prier, afin d'aider cette personne à maintenir un esprit pur et à se concentrer sur le ciel et le Bouddha dans ses derniers instants. Si la personne décédée a atteint la renaissance, les rites doivent être simplifiés, sans offrandes ni cérémonies hebdomadaires, mais il faut noter la date de la renaissance dans l'histoire de la Terre pure pour servir d'exemple aux autres.

Concernant les offrandes, il insiste sur le fait que tous les rites d'offrandes et les commémorations des ancêtres doivent être effectués avec des aliments végétariens, sans utiliser de viande. Lors des commémorations des ancêtres, les pratiquants de la Terre pure doivent inviter des coreligionnaires à réciter des écrits bouddhiques et le nom de Bouddha, à la fois pour prier pour le défunt et pour maintenir les échanges au sein de la communauté de pratiquants. Il conseille à chaque village pratiquant la Terre pure de choisir une personne honnête et digne de confiance pour guider l'organisation des rites.

Nguyễn Kim Muôn conclut cette section en conseillant que, même si les rites peuvent varier en fonction des circonstances, la dévotion au ciel et la pratique du bouddhisme doivent toujours être maintenues. Il se propose également d'aider toute personne de la communauté de pratiquants, proche ou lointaine, afin de garantir que les rites soient effectués conformément aux enseignements de la Terre pure.

\subsection{\textit{Đạo phật thích ca} ; Bouddhisme de Shakyamuni [notre traduction] - 1929}

\textit{Đạo phật thích ca}\footnote{Nguyễn Kim Muôn, \textit{Đạo Phật Thích Ca}[Bouddhisme de Shakyamuni], Đức Lưu Phương, Saigon, 1929} est un livre de 30 pages, imprimé à 1000 exemplaires par l'imprimerie Đức Lưu Phương à Saïgon le 5 août 1929.

\begin{figure}[H]
    \centering
    \includegraphics[width=0.4\textwidth]
    {images/daophatthichca1929.JPEG}
    \caption{Couverture du livre Bouddhisme de Shakyamuni - source : Gallica }
\end{figure}

\textit{Đạo phật thích ca} de Nguyễn Kim Muôn est un essai sur le bouddhisme, expliquant son origine, sa pratique et ses différents niveaux de réalisation spirituelle. D'emblée, Nguyễn Kim Muôn affirme que le bouddhisme de Shakyamuni est une voie de compassion, enseignant aux êtres à atteindre l'éveil de leur véritable nature et à se libérer des désirs afin d'atteindre un état de conscience élevé. Selon Nguyễn Kim Muôn, il existe deux types d'enseignements: 
\begin{itemize}
    \item L'enseignement orthodoxe (confucianisme - bouddhisme - taoïsme)
    \item L'enseignement hétérodoxe (3600 voies extérieures)
\end{itemize}
L'enseignement orthodoxe guide les êtres vers la cultivation de l'esprit et le développement des qualités vertueuses.  Nguyễn Kim Muôn souligne le rôle crucial de l'Esprit dans la pratique spirituelle, affirmant que tout provient de l'Esprit et que la pratique vise à retourner à l'Esprit.

Nguyễn Kim Muôn évoque également les difficultés de la pratique spirituelle, affirmant que les êtres sont facilement égarés et peuvent tomber dans des voies erronées.  Il insiste sur l'importance de cultiver à la fois l'âme et le corps, c'est-à-dire de combiner la pratique de l'esprit avec le développement physique. 

Nguyễn Kim Muôn partage son expérience sur son propre cheminement spirituel, passant d'un manque d'intérêt initial pour le bouddhisme à l'éveil et à la détermination de pratiquer. Il considère que la pratique spirituelle de chacun est liée au karma et qu'à un moment propice, l'éveil se manifeste naturellement.  Il explique également les méthodes de pratique, distinguant deux niveaux principaux: 

\begin{itemize}
    \item Le pratiquant laïc
    \item Le moine
\end{itemize}

Le pratiquant laïc peut s'engager sur la voie bouddhiste tout en restant chez lui, avec sa famille, son travail et ses biens. Il pratique le Dharma en récitant des écrits bouddhiques, en psalmodiant le nom du Bouddha, en suivant un régime végétarien et en maintenant un esprit pur. Nguyễn Kim Muôn affirme que cette voie permet d'atteindre des niveaux élevés de réalisation, de se libérer du cycle des réincarnations et d'accéder aux 36 niveaux du ciel.  Il recommande également la lecture de certains écrits bouddhiques pour soutenir la pratique, tels que: \textit{Phật giáo khuyến tu, Tịnh độ tông, Thờ trời tu phật,etc...}

Le moine, quant à lui, suit une voie plus exigeante, nécessitant sacrifice et renoncement.  Il doit quitter sa famille, se raser la tête, vivre au temple et observer des préceptes stricts. Nguyễn Kim Muôn compare la pratique spirituelle à l'ascension d'une échelle, où il faut changer de corps et de rôle pour atteindre des niveaux de conscience supérieurs.  Il souligne que la voie monastique n'est pas accessible à tous, mais requiert des conditions karmiques favorables et une détermination inébranlable. Nguyễn Kim Muôn aborde également la pratique visant à atteindre la renaissance dans la Terre pure de l'Ouest, c'est-à-dire la renaissance dans le monde de la béatitude occidentale du Bouddha Amitābha. Il considère cet état comme supérieur aux 36 cieux, nécessitant une pratique assidue, l'observance des préceptes, l'abandon des désirs et la mise en œuvre du \textit{Tâm pháp tam muội} Le Dharma du Cœur du Samadhi [notre traduction]. Il présente également son propre ouvrage, \textit{Tâm pháp chỉ ngay} La méthode du Cœur qui montre directement [notre traduction], qui fournit des instructions détaillées sur la pratique pour atteindre cet état. Enfin, Nguyễn Kim Muôn appelle tous les êtres, en particulier les personnes âgées, à pratiquer avec diligence et à s'éveiller à la vérité. Il affirme que \textit{<< Sớm mơi nghe đạo, chiều chết củng được siêu >>} (page 36), ce qui signifie que l'éveil à la vérité conduit à la libération, même face à la mort [notre traduction].

\subsection{\textit{Phật giáo vệ sanh} ; L'hygiène bouddhiste [notre traduction] - 1929}

\textit{Phật giáo vệ sanh}\footnote{ Nguyễn Kim Muôn, \textit{Phật giáo vệ sinh} [Prescriptions bouddhiques], Đức Lưu Phương, Saigon, 1929} est un livre de 22 pages, imprimé à 1000 exemplaires par l'imprimerie Đức Lưu Phương à Saïgon le 19 août 1929.

\begin{figure}[H]
    \centering
    \includegraphics[width=0.4\textwidth]
    {images/phatgiaovesinh.JPEG}
    \caption{Couverture du livre L'Hygiène Bouddhiste - source : Gallica}
\end{figure}

L'Hygiène Bouddhiste \textit{Phật giáo vệ sinh} de Nguyễn Kim Muôn vise à aider les pratiquants spirituels à mieux comprendre l'hygiène physique et mentale dans leur cheminement. Nguyễn Kim Muôn souligne que de nombreuses personnes qui adoptent un régime végétarien tombent souvent dans un état de faiblesse et sont sujettes aux maladies, ce qui entraîne une perte de confiance dans la pratique spirituelle (page 5). Les trois trésors, Essence \textit{Tinh}, Énergie \textit{Khí} et Esprit \textit{Thần}, sont considérés comme le fondement du maintien de la vie. L'Essence représente l'énergie fondamentale; l'Énergie est le souffle et le flux de la vie; l'Esprit est la lumière de l'intellect et des sensations. Nguyễn Kim Muôn affirme qu'une pratique spirituelle correcte nécessite de protéger ces trois éléments (pages 7-12).

En matière de nutrition, Nguyễn Kim Muôn propose des principes alimentaires adaptés aux pratiquants, tels que ne pas trop manger, bien mastiquer pour favoriser la digestion, éviter les aliments gras, en conserve ou transformés industriellement. Au lieu de cela, les aliments frais et naturels sont encouragés, car ils améliorent la santé sans diminuer la vigilance nécessaire à la pratique spirituelle (pages 13-17). En ce qui concerne l'hygiène personnelle, Nguyễn Kim Muôn souligne l'importance de maintenir la propreté du corps, comme se brosser les dents, se rincer la bouche, éviter les habitudes nocives telles que la consommation de bétel ou le tabagisme, afin de préserver la santé et un état d'esprit pur (pages 9-10).

Les méthodes de pratique spirituelle efficaces présentées comprennent le maintien d'une respiration régulière et profonde pour éliminer les impuretés énergétiques, le choix d'un lieu de vie propre et aéré, ainsi que la limitation de l'avidité, de la colère et des pensées errantes par le contrôle des six sens (yeux, oreilles, nez, langue, corps et esprit) (pages 18-19). Nguyễn Kim Muôn aborde également l'alimentation et le sommeil. Les pratiquants doivent adopter un régime végétarien strict, réduire progressivement les aliments inutiles, privilégier les repas légers, manger en fonction du climat et des besoins du corps, en évitant de manger trop ou trop peu (pages 21-23). L'eau potable doit être propre, en quantité suffisante et naturelle, sans impuretés, afin de maintenir la santé des reins (page 24). Le sommeil est également mis en avant comme un élément essentiel pour maintenir la santé mentale, avec le conseil de dormir suffisamment et d'éviter de veiller tard inutilement (pages 25-26).

En conclusion, Nguyễn Kim Muôn affirme que la pratique spirituelle ne se limite pas à la récitation de écrits bouddhiques ou du nom de Bouddha, mais nécessite également de prendre soin du corps et de l'esprit de manière holistique. Il appelle chacun à apprendre, à étudier et à pratiquer correctement pour atteindre un état de vigilance, de paix et de proximité avec la vérité (page 26).\\
\noindent\rule{0.35\linewidth}{0.6pt}

\clearpage
\subsection{\textit{Chấn hưng phật giáo} ; Propagation bouddhique [traduit par dépôt légal] ; Rénovation du bouddhisme [notre traduction] - 1929}

\textit{Chấn hưng phật giáo}\footnote{Nguyễn Kim Muôn, \textit{Chấn hưng Phật giáo} [Rénovation du bouddhisme], Đức Lưu Phương, Saigon, 1929} est un livre de 25 pages, imprimé à 1000 exemplaires par l'imprimerie Đức Lưu Phương à Saïgon le 7 septembre 1929. Nguyễn Kim Muôn aborde la nécessité de revitaliser les enseignements bouddhistes pour qu'ils restent pertinents dans le monde moderne.

\begin{figure}[H]
    \centering
    \includegraphics[width=0.4\textwidth]
    {images/chanhungphatgiao.png}
    \caption{Couverture du livre Rénovation du bouddhisme - source : Gallica }
\end{figure}

Nguyễn Kim Muôn discute de la revitalisation du bouddhisme, une entreprise immense et pleine de défis. Il affirme qu'il est facile de parler de revitalisation, mais plus difficile de la mettre en œuvre. La difficulté ne réside pas dans la discussion, mais dans l'action, la mise en pratique de ce qui a été dit et écrit.

Il souligne que le bouddhisme, bien qu'étant une grande religion, a traversé des décennies d'oubli et de désintérêt. Les moines ont préservé la voie, mais la société en général semble considérer le bouddhisme comme une simple forme de rituel, invoqué lors de fêtes ou d'événements importants, puis oublié. Cependant, récemment, un mouvement de renouveau spirituel a émergé, avec de nombreuses personnes vertueuses et de grande envergure qui commencent à reconsidérer la valeur du bouddhisme (page 8).

Nguyễn Kim Muôn compare les nouveaux pratiquants à des personnes à qui l'on ouvre un trésor qu'ils découvrent de nombreuses choses précieuses qu'ils ignoraient auparavant. Cependant, il critique également ceux qui, nouvellement convertis, se tournent vers la critique des moines — ceux-là mêmes qui ont préservé le bouddhisme pendant de nombreuses années. Il insiste que connaître les affaires des autres et les critiquer est une erreur, alors que connaître pour améliorer sa propre maison est sage (page 8).

Revitaliser le bouddhisme, selon lui, c'est restaurer le bouddhisme selon les écritures et les enseignements du Bouddha, en évitant l'hypocrisie. Il compare le bouddhisme à une grande porte ouverte à tous, mais qui, aujourd'hui, est rétrécie et gardée par ceux qui s'autoproclament chefs, forçant ainsi les pratiquants à se bousculer et à payer pour entrer. Il écrit qu'aujourd'hui, le bouddhisme semble avoir perdu sa vraie voie. Un enfant sait maintenant ce qu'est la voie, sait comment pratiquer, mais il ne peut pas marcher librement car la porte de la voie est étroite et exige de l'argent (page 9).

Il appelle à éliminer ces obstacles, à élargir la porte de la voie pour que tous puissent y entrer naturellement, sans être limités par l'argent ou les formalités. Selon lui, c'est là le véritable sens de la revitalisation du bouddhisme que restaurer le bouddhisme dans sa pureté et sa naturalité originelles (page 9).

Nguyễn Kim Muôn poursuit en analysant les différents moyens de revitaliser le bouddhisme, insistant sur la nécessité d'agir concrètement. Il souligne qu'aujourd'hui, certains maîtres zen ont pris l'initiative d'organiser des activités telles que l'ouverture d'écoles bouddhistes, l'enseignement des écrits bouddhiques et des préceptes, l'enseignement du chinois classique et l'organisation de retraites de méditation dans les temples. Cependant, il considère que ces efforts ne sont que des solutions temporaires, comme un feu de paille, qui ne peuvent pas durer (page 9).

Il énumère quatre limites principales à se reposer uniquement sur la création d'écoles bouddhistes :

Difficulté à se défaire des habitudes que les bonzes sont habitués à fonder des temples, à vénérer le Bouddha et à effectuer des rituels, il leur est donc difficile de changer de perspective pour mettre en œuvre de nouvelles méthodes de revitalisation du bouddhisme (page 10).

Lenteur à agir : si les écoles bouddhistes avaient été ouvertes il y a de nombreuses années, il n'y aurait pas besoin de revitalisation aujourd'hui. Attendre que le bouddhisme soit sur le point de disparaître pour commencer est trop tard (page 10).

Difficultés liées aux conditions sociales que les travailleurs sont occupés par leur vie quotidienne et ont du mal à trouver le temps d'étudier et de pratiquer au temple. Cela empêche les pauvres d'accéder au bouddhisme (pages 10-11).

Obstacles juridiques qu'à cette époque, le gouvernement interdisait les grands rassemblements, ce qui entravait l'organisation de cours et d'activités bouddhistes à grande échelle (page 11).

Nguyễn Kim Muôn estime qu'au lieu de fonder des écoles ou des associations, la revitalisation du bouddhisme devrait se concentrer sur la traduction, l'étude et la diffusion des écritures. Il insiste que la bouche parle, la main écrit, la chair agit, ce qui signifie qu'il faut allier la parole et l'action de manière cohérente (page 12). Selon lui, la meilleure façon est de traduire les écrits bouddhiques en langue vernaculaire afin que toutes les classes sociales, des personnes âgées aux jeunes, des riches aux pauvres, puissent y accéder et pratiquer chez elles (page 12).

Les instructions pour la pratique à domicile qu'il propose sont les suivantes :

Installer un autel bouddhiste à la maison, vénérer le Bouddha et réciter des écrits bouddhiques selon les manuels.
La pratique spirituelle ne nécessite pas de renoncer à sa famille, à son travail ou à ses biens, mais peut être intégrée à la vie quotidienne. Par exemple, adopter progressivement un régime végétarien (du végétarisme occasionnel au végétarisme permanent) pour réduire la violence et accumuler des mérites (page 13).
Il souligne également que la pratique à domicile présente de nombreux avantages, tels qu'aider les parents à être des modèles pour leurs enfants, améliorer le caractère et apporter la paix à la famille. En même temps, il critique ceux qui considèrent la pratique spirituelle comme une simple formalité, allant au temple et faisant des offrandes uniquement pour demander de la chance, sans réellement cultiver leur esprit et leur caractère (page 14).

Enfin, Nguyễn Kim Muôn insiste sur le fait que pour véritablement revitaliser le bouddhisme, le leader doit être une personne vertueuse, ayant une pratique spirituelle solide, une bonne éducation et une profonde compréhension de la voie (page 15).

Nguyễn Kim Muôn poursuit son analyse approfondie de la nature de la revitalisation du bouddhisme, en se concentrant sur les questions pratiques et philosophiques liées à la pratique spirituelle.

Le fardeau de la revitalisation du bouddhisme :
Il compare la revitalisation du bouddhisme à un fardeau de millions de kilos, que seule une personne ayant les capacités, la vertu et le dévouement nécessaires peut assumer. Ce fardeau doit être porté par compassion et par vœu, et non par la recherche de gloire ou de profit personnel (page 16).
Il critique ceux qui s'autoproclament maîtres spirituels mais qui, en réalité, utilisent la religion pour rechercher la gloire et des avantages personnels. Il insiste sur le fait que le vrai bouddhisme n'a pas besoin de preuves, mais repose sur la connaissance de soi et l'introspection (page 17).

\textbf{La vraie compréhension de la voie :
}Il souligne que la vérité réside en soi, et non dans les preuves ou le mysticisme. Le bouddhisme enseigne qu'il faut éviter les sons et les formes, et se concentrer sur son monde intérieur pour atteindre l'éveil (page 17).
Il critique ceux qui pratiquent en se basant sur des visions extérieures, rêvant souvent de scènes telles que le Bouddha \textit{Amitābha} ou des divinités, mais qui ne sont en réalité que des produits de l'imagination et n'apportent aucun bénéfice réel (page 18).

\textbf{L'esprit et les méthodes de pratique spirituelle :
}Il rappelle aux pratiquants de toujours se souvenir du Bodhisattva Avalokiteshvara (l'observation de la liberté de l'esprit) pour ne pas se laisser distraire pendant la pratique.
Pendant la méditation ou la pratique, il faut se concentrer entièrement sur l'esprit, le maintenir clair et ne pas se laisser perturber par les formes ou les sensations extérieures (pages 18-19). La méthode d'introspection (\textit{Hồi quan phản chiếu}) est mise en avant  lorsqu'une image ou une sensation apparaît, il faut se tourner vers l'intérieur et examiner son esprit, le maintenir éveillé et ne pas le laisser se tourner vers l'extérieur (page 20).

\textbf{Critique du mysticisme :
}Il s'oppose fermement aux phénomènes mystiques dans la pratique spirituelle, tels que voir des auras, des fleurs de lotus ou des apparitions du Bouddha, car ce sont des signes d'un esprit encore agité et non purifié (page 20).
Au lieu de se tourner vers l'extérieur, il insiste sur le fait que toutes les pratiques et tous les enseignements doivent revenir à l'esprit, comme l'enseigne le Bouddha : tous les phénomènes retournent à l'esprit (\textit{Vạn pháp quy tâm}).
Nguyễn Kim Muôn poursuit sa discussion sur les aspects philosophiques, les méthodes et la pratique de la revitalisation du bouddhisme, en mettant l'accent sur l'autonomie et les pratiques essentielles. L'autonomie comme fondement de la pratique spirituelle :
Il souligne que la pratique spirituelle et l'éveil dépendent de chaque individu, personne ne peut les réaliser à la place d'un autre. La phrase Bodhisattva Avalokiteshvara signifie que l'on doit soi-même prendre conscience et agir pour atteindre l'éveil (page 21).
Il critique la confiance aveugle en des formes extérieures ou la dépendance à autrui, considérant que c'est la raison pour laquelle de nombreuses personnes pratiquent sans obtenir de résultats (page 21).

\textbf{La pratique spirituelle est pratiquer la perfection de la sagesse (\textit{Bát Nhã Ba La Mật}) :}
Il explique que la perfection de la sagesse est un mystère profond, exigeant une persévérance et une pratique approfondie pour atteindre la vision constante (\textit{Đa thời chiếu kiến}) — un état d'observation continue de son monde intérieur.
Lorsque les cinq sens et les cinq organes (cœur, foie, rate, poumons, reins) atteignent l'harmonie, le pratiquant peut surmonter toutes les difficultés et atteindre l'éveil (page 22).

\textbf{La pratique à domicile :}
Il encourage la pratique à domicile, l'installation d'un autel bouddhiste chez soi, le maintien de la sincérité, la récitation du nom de Bouddha et le respect des principes moraux dans la vie quotidienne. Cette méthode convient à toutes les classes sociales, des riches aux pauvres, des personnes âgées aux jeunes (pages 23-24). Il souligne que la pratique à domicile ne fait pas perdre de temps et permet de maintenir ses responsabilités familiales et professionnelles, facilitant ainsi la participation de nombreuses personnes (page 23).

\textbf{La pratique et les obstacles au développement de l'esprit :}
Il décrit les difficultés rencontrées dans la pratique, telles que la confrontation aux pensées errantes, aux émotions perturbatrices, et la nécessité de persévérer pour maintenir un esprit calme. La persévérance et la concentration sur le monde intérieur (la méthode d'introspection) aideront le pratiquant à atteindre l'état d'esprit unique et imperturbable (\textit{Nhất tâm bất loạn}) (page 24). Atteindre l'esprit unique et imperturbable pendant un jour ou sept jours est considéré comme la clé de l'éveil, ce qui nécessite une pratique continue et approfondie (pages 24-25).

\textbf{La signification des écrits bouddhiques importants :}
Le \textit{sūtra} d'\textit{Amitābha} : met l'accent sur le vœu et la confiance en \textit{Amitābha}, considérés comme un soutien dans le cheminement spirituel.
L'écrit bouddique du diamant\textit{ Kinh kim cang} : considéré comme le \textit{sūtra} le plus important, avec ses enseignements profonds sur l'impermanence, le détachement des formes et la concentration sur l'esprit (page 27).
La collection des sources du \textit{dharma} (\textit{Bộ Qui Ngươn}) : il s'agit d'une collection d'écrits bouddhiques rassemblant l'essence d'autres textes tels que le \textit{sūtra} du lotus, le \textit{sūtra} de \textit{Shurangama}, le \textit{sūtra} de la lumière infinie, etc., pour aider les pratiquants à comprendre et appliquer rapidement les enseignements (page 25).

Nguyễn Kim Muôn affirme que la revitalisation du bouddhisme ne se limite pas à la diffusion des écritures ou à la construction de temples, mais consiste à aider chacun à prendre conscience de l'importance de la pratique spirituelle autonome et à agir immédiatement. Il encourage la pratique à domicile, met l'accent sur le développement intérieur et présente les écrits bouddhiques comme des guides importants sur le chemin de l'éveil.

\subsection{\textit{Đạo có một} ; Il n'y a qu'un seul chemin [notre traduction] - 1929}
\textit{Đạo có một}\footnote{Nguyễn Kim Muôn, \textit{Đạo có một} [Bouddhisme], Đức Lưu Phương, Saigon, 1929} est un livre de 35 pages, imprimé à 1000 exemplaires par l'imprimerie Đức Lưu Phương à Saïgon le 21 octobre 1929.

Il n'y a qu'un seul chemin est une œuvre qui transmet un message sur l’unification des religions et des pratiques spirituelles, en insistant sur une vérité ultime–le \textit{Chánh giáo}. 1000 examplaires

 \begin{figure}[H]
    \centering
    \includegraphics[width=0.4\textwidth]
    {images/daocomot.png}
    \caption{Couverture du livre Il n'y a qu'un seul chemin - source : Gallica}
\end{figure}
Nguyễn Kim Muôn souligne que tous les chemins spirituels, quels que soient leur origine religieuse ou philosophique, convergent vers un but commun : libérer les humains du cycle des réincarnations et atteindre l’illumination. L’ouvrage affirme : << Đạo có một, là một chánh giáo dạy người được siêu phàm nhập thánh, được thân ngoại hữu thân, được liễu đạo, được vãng sanh, nói tóm là được thoát kiếp luân hồi. >> (La voie est une : un enseignement authentique qui guide l’homme à dépasser le commun pour entrer dans la sainteté, à obtenir un corps au-delà du corps, à réaliser la voie, à renaître dans la Terre pure, en bref à se libérer du cycle des réincarnations) [notre traduction] (page 7).

Toutes les religions et pratiques spirituelles existantes sont perçues comme des portes différentes menant à un même objectif, celui de l’harmonie entre l’humanité et la vérité ultime :\\
<< Những gì đạo đã ra đời, bất kỳ là tông giáo nào, bất kỳ là pháp môn nào, sẽ một ngày rồi qui về một mối, là một chánh giáo thật. >> (Toutes les voies apparues, quelle que soit la religion ou la méthode, reviendront un jour à une seule source, qui est un enseignement véritable) [notre traduction] (page 7).

L’ouvrage divise le cheminement spirituel en trois niveaux distincts : le \textit{Cư sĩ}, le \textit{Đạo nhân} et le \textit{Đạo sư}. Le \textit{Cư sĩ} est celui qui pratique chez lui, vivant toujours dans un cadre familial, en respectant les principes éthiques et en accomplissant des actes vertueux tels que la récitation de prières, la méditation ou l’observance d’un régime végétarien. Le \textit{Đạo nhân} représente un niveau supérieur, où l’individu renonce aux désirs mondains, adopte une vie austère et se consacre à la méditation profonde. Enfin, le \textit{Đạo sư} est le niveau le plus élevé, celui où une personne abandonne complètement le monde matériel pour se concentrer sur sa pratique spirituelle et guider les autres vers l’illumination. L’auteur décrit cette progression comme une évolution spirituelle nécessaire :
<< Đạo sư, ly gia cắc ái, không ở trần gian, công phu luyện kỷ, ấn chứng rõ ràng. >> (Le maître spirituel, détaché de la famille et de l’affection, ne vit pas dans le monde profane ; il cultive assidûment sa personne et reçoit des preuves claires de réalisation [notre traduction] (page 3).

Dans son cheminement spirituel, l’auteur raconte sa rencontre avec un maître spirituel qu’il appelle \textit{Chơn Sư}. Celui-ci lui confie une mission importante qu’il décrit comme le lourd fardeau à deux extrémités : promouvoir les deux pratiques spirituelles basées sur \textit{Trời} (le Ciel) et \textit{Phật} (le Bouddha), tout en unissant les philosophies orientales et occidentales à travers les valeurs culturelles et religieuses. Comme le \textit{Chơn Sư} le lui a dit :
<< Kìa một đám dân lành rải rác chia đều cả bốn phương trời đang ngóng đạo, ta nay giao cho ngươi một cái gánh, nặng cả đôi quang, ngươi khá tận tâm gánh lấy, chừng nào cất được gánh đầy nhẹ rồi, thì ngươi sẽ có sẵn chỗ xuất gia. >> (Regarde ces bonnes gens dispersés aux quatre coins du ciel, attendant la voie. Je te confie aujourd’hui une charge, lourde aux deux extrémités ; porte-la avec tout ton cœur, et lorsque tu l’auras soulevée pleinement et allégée, tu auras alors ta place pour entrer dans la vie monastique) [notre traduction] (page 7).

Un des thèmes centraux de l’ouvrage est le lien étroit entre la vie humaine et le cycle des réincarnations. L’auteur affirme que chaque situation présente résulte des actions passées. Cette conception de la causalité montre que le karma gouverne chaque individu, sans distinction de richesse, de classe sociale ou de caractère moral. Il écrit :
<< Khen cho kiếp trước khéo tu, nay mới được vồng đủ nghìn ngàn. >> (Félicitations à celui qui, dans une vie antérieure, a bien pratiqué, car aujourd’hui il reçoit en abondance mille bienfaits) [notre traduction] (page 8).

Avec cette conviction, il encourage les gens à mener une vie vertueuse pour améliorer leur prochaine existence et échapper au cycle des réincarnations. L’ouvrage met également en garde contre les conséquences graves auxquelles devront faire face ceux qui ne pratiquent pas ou vivent à l’encontre des principes du \textit{Đạo} :
<< Con người ở đời, ai có thân cũng lo, có nợ thì phải sắm, có tạo thì có lập, ấy đã đành với cái kiếp này rồi. Rồi còn cái kiếp sau, bồ sao không nghĩ thử? >> (Dans la vie, quiconque a un corps doit en prendre soin, s’il a des dettes, il doit les rembourser, s’il a créé, il doit établir ; cela vaut pour cette vie. Et pour la vie suivante, pourquoi ne pas y penser) [notre traduction] ?

L’ouvrage décrit également des phénomènes mystiques, tels que l’utilisation par le \textit{Phật thầy} de la main d’une femme nommée Diệu Văn pour écrire des poèmes mystérieux. Ces messages sont perçus comme des avertissements du monde spirituel, exhortant les humains à pratiquer sans tarder. L’un des poèmes souligne :
\textit{<< Nam mô chi chí chi trì trị - Đạo đã ra đời rồi, cái thời kỳ cũng đã đến, chẳng nên trì huỡn. >>} (Hommage au très haut, au maintien et au soin ; la voie est déjà apparue, le moment est venu, il ne faut pas tarder) [notre traduction] (page 19).

De plus, l’ouvrage donne une description détaillée du concept de \textit{Cực lạc} (la Terre pure) – un royaume en dehors du monde terrestre, où le cycle des réincarnations n’existe plus. C’est un lieu réservé à ceux qui ont atteint l’illumination. L’auteur écrit :
\textit{<< Cực lạc là riêng một cảnh ở ngoại càng khôn, vô sắc giới, là nước vô sanh bất diệt, là nước Phật vậy, người được về cảnh ấy là Phật vậy. >>} (La Terre pure est un royaume distinct, au-delà de l’univers, dans le monde sans forme ; c’est un pays sans naissance ni extinction, c’est le pays du Bouddha ; celui qui y parvient est un Bouddha) [notre traduction] (page 22).

Enfin, l’ouvrage se termine par des conseils spirituels, exhortant chacun à saisir les opportunités de la vie pour pratiquer. L’auteur rappelle que chaque action, bonne ou mauvaise, a des conséquences, et que ceux qui vivent en contradiction avec les principes du \textit{Đạo} devront affronter des punitions sévères. Il écrit :
\textit{<< Hễ ăn ở hiền lành chừng nào thì linh hồn trong sạch mạnh mẽ chừng nấy. Đến khi chết rồi, tách cái linh hồn ra bay theo không khí, hễ nhẹ thì bay lên cao, còn nặng thì trì xuống thấp. >>} (Plus on vit avec bonté, plus l’âme est pure et forte. À la mort, l’âme se détache et s’élève dans l’air ; si elle est légère, elle monte vers le haut, si elle est lourde, elle descend vers le bas) [notre traduction] (page 23).

En conclusion, il insiste sur l’importance de se préparer dans cette vie pour atteindre la paix dans l’au-delà :
\textit{<< Ai mà ăn ở trong đời, trọn câu nhân đạo, Phật Trời chứng minh. >>} (Quiconque vit dans ce monde en accomplissant pleinement l’humanité, Bouddha et Ciel en sont témoins) [notre traduction] (page 8).

Nguyễn Kim Muôn souligne que tous les chemins spirituels, quels que soient leur origine religieuse ou philosophique, convergent vers un but commun : libérer les humains du cycle des réincarnations et atteindre l’illumination. L’ouvrage affirme :
\textit{<< Đạo có một, là một chánh giáo dạy người được siêu phàm nhập thánh, được thân ngoại hữu thân, được liễu đạo, được vãng sanh, nói tóm là được thoát kiếp luân hồi. >>} (La voie est une : un enseignement authentique qui guide l’homme à dépasser le commun pour entrer dans la sainteté, à obtenir un corps au-delà du corps, à réaliser la voie, à renaître dans la Terre pure, en bref à se libérer du cycle des réincarnations)[notre traduction] (page 7).

Toutes les religions et pratiques spirituelles existantes sont perçues comme des portes différentes menant à un même objectif, celui de l’harmonie entre l’humanité et la vérité ultime :
<< Những gì đạo đã ra đời, bất kỳ là tông giáo nào, bất kỳ là pháp môn nào, sẽ một ngày rồi qui về một mối, là một chánh giáo thật. >> (Toutes les voies apparues, quelle que soit la religion ou la méthode, reviendront un jour à une seule source, qui est un enseignement véritable) [notre traduction] (page 7).

L’ouvrage divise le cheminement spirituel en trois niveaux distincts : le \textit{cư sĩ}, le \textit{đạo nhân} et le \textit{đạo sư}. Le \textit{cư sĩ} est celui qui pratique chez lui, vivant toujours dans un cadre familial, en respectant les principes éthiques et en accomplissant des actes vertueux tels que la récitation de prières, la méditation ou l’observance d’un régime végétarien. Le \textit{đạo nhân} représente un niveau supérieur, où l’individu renonce aux désirs mondains, adopte une vie austère et se consacre à la méditation profonde. Enfin, le \textit{đạo sư} est le niveau le plus élevé, celui où une personne abandonne complètement le monde matériel pour se concentrer sur sa pratique spirituelle et guider les autres vers l’illumination. L’auteur décrit cette progression comme une évolution spirituelle nécessaire :
<< Đạo sư, ly gia cắc ái, không ở trần gian, công phu luyện kỷ, ấn chứng rõ ràng. >> (Le maître spirituel, détaché de la famille et de l’affection, ne vit pas dans le monde profane ; il cultive assidûment sa personne et reçoit des preuves claires de réalisation) [notre traduction] (page 3).

Dans son cheminement spirituel, l’auteur raconte sa rencontre avec un maître spirituel qu’il appelle \textit{chơn sư}. Celui-ci lui confie une mission importante qu’il décrit comme le lourd fardeau à deux extrémités : promouvoir les deux pratiques spirituelles basées sur \textit{trời} (le ciel) et \textit{phật} (le bouddha), tout en unissant les philosophies orientales et occidentales à travers les valeurs culturelles et religieuses. Comme le \textit{chơn sư} le lui a dit :
<< Kìa một đám dân lành rải rác chia đều cả bốn phương trời đang ngóng đạo, ta nay giao cho ngươi một cái gánh, nặng cả đôi quang, ngươi khá tận tâm gánh lấy, chừng nào cất được gánh đầy nhẹ rồi, thì ngươi sẽ có sẵn chỗ xuất gia. >> (Regarde ces bonnes gens dispersés aux quatre coins du ciel, attendant la voie. Je te confie aujourd’hui une charge, lourde aux deux extrémités ; porte-la avec tout ton cœur, et lorsque tu l’auras soulevée pleinement et allégée, tu auras alors ta place pour entrer dans la vie monastique) [notre traduction] (page 7).

Un des thèmes centraux de l’ouvrage est le lien étroit entre la vie humaine et le cycle des réincarnations. L’auteur affirme que chaque situation présente résulte des actions passées. Cette conception de la causalité montre que le karma gouverne chaque individu, sans distinction de richesse, de classe sociale ou de caractère moral. Il écrit :
<< Khen cho kiếp trước khéo tu, nay mới được vồng đủ nghìn ngàn. >> (Félicitations à celui qui, dans une vie antérieure, a bien pratiqué, car aujourd’hui il reçoit en abondance mille bienfaits) [notre traduction] (page 8).

Avec cette conviction, il encourage les gens à mener une vie vertueuse pour améliorer leur prochaine existence et échapper au cycle des réincarnations. L’ouvrage met également en garde contre les conséquences graves auxquelles devront faire face ceux qui ne pratiquent pas ou vivent à l’encontre des principes du \textit{đạo} :
<< Con người ở đời, ai có thân cũng lo, có nợ thì phải sắm, có tạo thì có lập, ấy đã đành với cái kiếp này rồi. Rồi còn cái kiếp sau, bồ sao không nghĩ thử? >> (Dans la vie, quiconque a un corps doit en prendre soin, s’il a des dettes, il doit les rembourser, s’il a créé, il doit établir ; cela vaut pour cette vie. Et pour la vie suivante, pourquoi ne pas y penser) [notre traduction] ?

L’ouvrage décrit également des phénomènes mystiques, tels que l’utilisation par le \textit{phật thầy} de la main d’une femme nommée Diệu Văn pour écrire des poèmes mystérieux. Ces messages sont perçus comme des avertissements du monde spirituel, exhortant les humains à pratiquer sans tarder. L’un des poèmes souligne :
<< Nam mô chi chí chi trì trị – đạo đã ra đời rồi, cái thời kỳ cũng đã đến, chẳng nên trì huỡn. >> (Hommage au très haut, au maintien et au soin ; la voie est déjà apparue, le moment est venu, il ne faut pas tarder) [notre traduction] (page 19).

De plus, l’ouvrage donne une description détaillée du concept de \textit{Cực lạc} (la Terre pure) – un royaume en dehors du monde terrestre, où le cycle des réincarnations n’existe plus. C’est un lieu réservé à ceux qui ont atteint l’illumination. L’auteur écrit :
<< Cực lạc là riêng một cảnh ở ngoại càng khôn, vô sắc giới, là nước vô sanh bất diệt, là nước phật vậy, người được về cảnh ấy là phật vậy. >> (La Terre pure est un royaume distinct, au-delà de l’univers, dans le monde sans forme ; c’est un pays sans naissance ni extinction, c’est le pays du bouddha ; celui qui y parvient est un bouddha.) [notre traduction] (page 22).

Enfin, l’ouvrage se termine par des conseils spirituels, exhortant chacun à saisir les opportunités de la vie pour pratiquer. L’auteur rappelle que chaque action, bonne ou mauvaise, a des conséquences, et que ceux qui vivent en contradiction avec les principes du \textit{đạo} devront affronter des punitions sévères. Il écrit :
<< Hễ ăn ở hiền lành chừng nào thì linh hồn trong sạch mạnh mẽ chừng nấy. Đến khi chết rồi, tách cái linh hồn ra bay theo không khí, hễ nhẹ thì bay lên cao, còn nặng thì trì xuống thấp. >> (Plus on vit avec bonté, plus l’âme est pure et forte. À la mort, l’âme se détache et s’élève dans l’air ; si elle est légère, elle monte vers le haut, si elle est lourde, elle descend vers le bas.) [notre traduction] (page 23)

Il écrit une phrase de bohème sur l’importance de se préparer dans cette vie pour atteindre la paix dans l’au-delà :
<< Ai mà ăn ở trong đời\\ Trọn câu nhân đạo, phật trời chứng minh. >> (Qui conque vit dans ce monde en accomplissant pleinement l’humanité, bouddha et ciel en sont témoins.) [notre traduction].
\\
\noindent\rule{0.35\linewidth}{0.6pt}
\clearpage
\subsection{\textit{Huệ cảnh tây phang} ; Le paradis de l'Ouest [traduction selon Nguyễn Kim Muôn] - 1930}

\textit{Huệ cảnh tây phang}\footnote{Nguyễn Kim Muôn, \textit{Huệ cảnh Tây phang} [Le paradis de l'Ouest], Đức Lưu Phương, Saigon, 1930} est un livre de 36 pages, imprimé à 3000 exemplaires par l'imprimerie Đức Lưu Phương à Saïgon le 17 janvier 1930.

\begin{figure}[H]
    \centering
    \includegraphics[width=0.4\textwidth]
    {images/huecanhtayphang.png}
    \caption{Couverture du livre Le paradis de l'Ouest - source : Gallica}
\end{figure}

\textbf{Partie 1 : Introduction au bouddhisme : (pages 1-8)}

L'écrit commence par une introduction au bouddhisme, mettant l'accent sur la pensée du non-agir \textit{vô vi} et la nature du \textit{Đạo}. Le \textit{Đạo} n'a ni commencement ni fin, ni intérieur ni extérieur, mais est toujours présent. Ce point de vue est comparé aux ondes radio, capables de se propager partout, mais que l'homme ne peut pas facilement percevoir (page 5). Le pratiquant spirituel doit maintenir son corps pur, comme une station réceptrice d'ondes radio, afin d'absorber l'énergie du \textit{Đạo}.

Nguyễn Kim Muôn évoque également les enseignements de Lao Tseu et du bouddhisme, affirmant que la pratique spirituelle ne consiste pas à réciter des sūtras ou à faire des offrandes, mais à se cultiver soi-même. La question se pose alors: qu'est-ce que la pratique spirituelle? Selon Nguyễn Kim Muôn, il ne s'agit pas de construire des temples ou de réciter des sūtras, mais de pointer directement vers le mystère profond \textit{Trực chỉ diệu huyền}, c'est-à-dire de réaliser directement la vraie nature du \textit{Đạo} sans avoir besoin d'écrits (page 7).

\textbf{Partie 2 : Le Patriarche Bodhidharma transmet les enseignements (pages 9-14)}

Le Patriarche Bodhidharma est le fils du roi du Sud de l'Inde \textit{Nam thiên trúc}, qui a renoncé à la gloire et à la fortune pour devenir moine. Il transmet le Dharma en Chine, mais sans utiliser de livres, enseignant directement par transmission orale (page 9). À son arrivée, Bodhidharma rencontre l'empereur Liang Wudi, qui l'interroge sur les mérites qu'il avait accumulés en construisant des temples et en imprimant des écrits bouddhiques. Bodhidharma rétorque que ceux-ci ne sont pas de véritables mérites, ce qui met en colère l'empereur qui le fait expulser sur-le-champ (pages 10-11).

Plus tard, Bodhidharma rencontre Shen Guang (qui deviendra plus tard Huike), qui s'efforce d'apprendre le \textit{Đạo} auprès de lui. Shen Guang se coupe alors le bras pour prouver sa sincérité, et Bodhidharma l'accepte comme disciple. Il lui transmet le Dharma du \textit{sūtra} sans mots \textit{Vô tự chân kinh} (page 14).

\textbf{Partie 3 : Explication du \textit{Đạo} et de la nature de Bouddha (pages 15-22)}

Bodhidharma explique la nature du \textit{Đạo}, soulignant que tout réside dans l'Esprit \textit{Tâm}. L'Esprit est Bouddha, mais à cause de leurs pensées errantes, les êtres ne le réalisent pas. La pratique spirituelle doit se concentrer sur l'Observation de la Liberté \textit{Quan tự tại}, c'est-à-dire l'introspection (page 16).

La nature humaine comprend trois éléments principaux: l'Essence \textit{Tinh}, l'Énergie \textit{Khí} et l'Esprit \textit{Thần}. Le pratiquant doit harmoniser ces trois éléments pour atteindre l'éveil (page 18). Le \textit{Đạo} ne se limite pas à l'apprentissage, il faut aussi le mettre en pratique, tout comme un épéiste ne peut se contenter de la théorie sans s'entraîner au combat.

\textbf{Partie 4 : Le mystère profond du \textit{Đạo} et le vrai écrit bouddhique (pages 23-29)}

Nguyễn Kim Muôn met l'accent sur le concept de sutra sans mots \textit{Vô tự chân kinh} [notre traduction]. La signification de cet écrit bouddhique est que tous les enseignements sont déjà présents dans le cœur de chacun. Étudier les écritures sans les mettre en pratique, c'est comme lire un livre sur la natation sans jamais entrer dans l'eau (page 27).

Le bouddhisme ne se résume pas aux écrits bouddhiques, mais à la réalisation de la vraie nature de soi. Celui qui comprend le mot vide atteindra l'état d'éveil.

Nguyễn Kim Muôn affirme que le pratiquant spirituel n'a pas besoin de chercher à l'extérieur, mais doit se tourner vers lui-même.
\textit{<< Đạo tại thân trung, thân ngoại vô đạo >>} (Le \textit{Đạo} est en nous, il n'y a pas de \textit{Đạo} à l'extérieur.) [notre traduction] (page 30)

Enfin, Nguyễn Kim Muôn conseille à chacun de se concentrer sur sa propre pratique spirituelle et de ne pas se perdre dans la recherche d'écrits bouddhiques ou d'enseignements extérieurs.
\\
\noindent\rule{0.35\linewidth}{0.6pt}
\clearpage
\subsection{\textit{Thích giáo chơn truyền} ; Transmission authentique de l’enseignement bouddhique [notre traduction] - 1930}

\textit{Thích giáo chơn truyền}\footnote{Nguyễn Kim Muôn,\textit{Thích giáo chơn truyền} [Transmission authentique de l’enseignement bouddhique], Đức Lưu Phương, Saigon, 1930 } est un livre de 46 pages, imprimé à 1000 exemplaires par l'imprimerie Đức Lưu Phương à Saïgon le 22 janvier 1930.

\begin{figure}[H]
    \centering
    \includegraphics[width=0.4\textwidth]
    {images/thichgiaochontruyen.JPEG}
    \caption{Couverture du livre Transmission authentique de l’enseignement bouddhique - source : Gallica}
\end{figure}

Cet écrit de Nguyễn Kim Muôn se concentre sur le rôle central de l'esprit dans la pratique spirituelle. Nguyễn Kim Muôn explique que la pratique ne réside pas dans les formes, les rituels ou la dépendance à des éléments matériels, mais dans la purification et la maîtrise de son propre esprit. Le concept de l'observation de la liberté \textit{Quan tự tại} est mis en avant pour souligner que chacun doit prendre le contrôle de soi-même, s'observer soi-même pour atteindre l'éveil, sans dépendre entièrement d'un maître ou de rituels (pages 6-8, 12-13). Les concepts clés tels que les trois refuges \textit{Tam quy}, les cinq préceptes \textit{Ngũ giới}, les quatre nobles vérités \textit{Tứ khổ} et le noble octuple sentier \textit{Bát chính đạo} sont expliqués clairement, en insistant sur le fait que n'importe qui, quel que soit son statut social, peut les pratiquer avec détermination (pages 12-15, 18-19).

Nguyễn Kim Muôn encourage la pratique spirituelle à domicile, affirmant qu'il n'est pas nécessaire de renoncer à sa famille ou à son travail pour aller au temple, mais que l'on peut pratiquer dans la vie quotidienne. Il encourage l'installation d'un autel bouddhiste chez soi, la récitation d'écrits bouddhiques, la récitation du nom de Bouddha et le maintien d'un esprit pur, permettant à tous, y compris les personnes pauvres ou occupées, de s'engager sur la voie de la pratique spirituelle (pages 15-19). En même temps, Nguyễn Kim Muôn critique fermement les formes erronées de pratique spirituelle, telles que l'accent mis sur les rituels, la commercialisation du bouddhisme ou l'exploitation d'éléments mystiques pour attirer les fidèles, ce qui conduit les pratiquants à s'égarer et à ne pas atteindre l'objectif de l'éveil (pages 7-8, 20-21).

L'ouvrage appelle également les fidèles à faire des vœux sincères et à les mettre en pratique dans leur vie. Nguyễn Kim Muôn souligne que la revitalisation du bouddhisme doit commencer par le changement de chaque individu. Il encourage la traduction et la diffusion des écritures dans différentes langues afin que chacun puisse accéder aux enseignements bouddhistes et les comprendre (pages 4-6, 15-16). Lorsque la pratique est correcte, le pratiquant atteint la paix intérieure, se libère des afflictions de la vie et se rapproche de l'éveil. La cultivation de l'esprit est décrite comme un processus de purification pour revenir à l'esprit originel \textit{Chơn tâm} — la partie la plus pure de chaque être humain (pages 19-22).

Nguyễn Kim Muôn propose également des méthodes de pratique spécifiques telles que la méditation, la récitation du nom de Bouddha et l'introspection. Il encourage les pratiquants à persévérer chaque jour pour progresser. Ces méthodes visent à maintenir l'esprit stable et concentré sur soi-même (pages 20-28). Dans l'ouvrage, Nguyễn Kim Muôn appelle à la revitalisation du bouddhisme en restaurant ses valeurs originales et en diffusant ses enseignements à toutes les couches de la population. Il insiste sur le fait que la revitalisation ne réside pas dans la construction de temples ou l'organisation de cérémonies grandioses, mais dans l'intégration des enseignements bouddhistes dans la vie quotidienne, afin de faciliter la pratique pour tous (pages 6-10, 26-28).
\\
\noindent\rule{0.35\linewidth}{0.6pt}

\clearpage
\subsection{\textit{Đeo theo chưng phật} ; Sur le trace de Bouddha [traduction selon Nguyễn Kim Muôn] - 1932}

\textit{Đeo theo chưng phật}\footnote{Nguyễn Kim Muôn, \textit{Đeo theo chưng Phật} [Sur les traces de Bouddha], Đức Lưu Phương, Saigon, 1932} est un livre de 92 pages, imprimé à 1000 exemplaires par l'imprimerie Đức Lưu Phương à Saïgon le 6 février 1932.

\begin{figure}[H]
    \centering
    \includegraphics[width=0.4\textwidth]
    {images/daotheochungphat.png}
    \caption{Couverture du l'écrit Sur le trace de Bouddha - source : Gallica }
\end{figure}


\textit{Đeo theo chưng phật} de Nguyễn Kim Muôn est un traité sur la voie de la pratique bouddhiste, soulignant les défis et les éléments clés pour atteindre l'éveil et la libération. D'emblée, Nguyễn Kim Muôn affirme que la pratique spirituelle est un chemin exigeant une diligence, une étude approfondie des enseignements et une mise en pratique constante. L'apprentissage du Dharma ne se limite pas à la lecture des écrits bouddhiques et à la récitation, mais nécessite une compréhension profonde et une application concrète. Le pratiquant sincère doit renoncer aux désirs matériels, à la recherche de la gloire et du profit, pour se concentrer sur la cultivation de l'esprit et le développement des qualités vertueuses.
\textit{<< Việc gì cũng vậy, nói miệng tai thì không thấy gì, phải có ngồi môi có thấy, chớ cái thứ ở ngoài, chen lấn với việc trần, đầu bác lắm đi mấy cũng không thấu nổi chỗ tiên phật. >>} (Ce qui est valable pour toute chose : parler seulement avec la bouche et entendre avec les oreilles ne mène à rien, il faut pratiquer concrètement; les choses extérieures, mêlées aux affaires mondaines, même si l'on s'y adonne beaucoup, ne permettent pas d'atteindre le lieu des sages et des bouddhas.) [notre traduction] (page 5).

Nguyễn Kim Muôn aborde également l'importance de l'écoute des enseignements, comparant la différence entre les anciens et les modernes dans la compréhension du Dharma. Si, dans le passé, l'éveil était facilement atteint par la simple écoute des enseignements, de nos jours, l'écoute doit être associée à la pratique pour porter ses fruits.
<< Như thị ngã văn, nhứt thời phật tại xá vệ quốc... là ta nghe ông phật nói như vậy. >> (Ainsi ai-je entendu, en ce temps-là le Bouddha se trouvait au royaume de Xá Vệ... c'est-à-dire que j'ai entendu le Bouddha parler ainsi.) [notre traduction] (page 6). Pour atteindre l'éveil, Nguyễn Kim Muôn identifie quatre conditions essentielles que le pratiquant doit remplir: la discrimination, la libération, la conduite vertueuse et la sincérité du cœur. La discrimination est la capacité de discerner le bien du mal, le vrai du faux, le juste de l'injuste. La libération est l'affranchissement des liens du monde, la non-attachement aux sept émotions et aux six désirs. La conduite vertueuse consiste à pratiquer les bonnes qualités telles que la patience, la diligence, la compassion, la joie altruiste. La sincérité du cœur est la confiance absolue dans le Dharma et dans son maître spirituel.
\textit{<< Muốn ngộ đạo, là được chỗ chơn truyền điểu pháp thì phải giữ mình (riêng lòng) trong bốn điều: 1: Sự phân biệt (tri kiến). 2: Sự giải thoát (mình biết mình). 3: Sự công hạnh (giữ luật). 4: Sự thành tâm (bền lòng). >>} (Pour s'éveiller et recevoir la transmission authentique du Dharma, il faut se maintenir dans quatre aspects: 1: la discrimination (vision juste); 2: la libération (se connaître soi-même); 3: la conduite vertueuse (respect des préceptes); 4: la sincérité du cœur (persévérance).) [notre traduction] (page 11)

La discrimination est mise en avant comme la première étape, cruciale. Le pratiquant doit posséder une sagesse claire pour reconnaître les fausses voies et le Dharma authentique. Nguyễn Kim Muôn souligne les erreurs courantes dans le discernement du bien et du mal, mettant en garde le lecteur contre les risques d'égarement.

Ensuite, la libération est une condition essentielle pour progresser sur la voie de l'éveil. Se libérer du monde, ne plus être lié par les sept émotions et les six désirs est l'objectif que le pratiquant vise. Nguyễn Kim Muôn analyse en détail les « démons » intérieurs, à savoir l'âme consciente et l'âme inconsciente, qui empêchent l'être humain de se défaire de l'illusion et de l'ignorance.

La conduite vertueuse est la troisième condition, englobant la pratique des bonnes qualités comme la patience, la diligence, la compassion, la joie altruiste. Nguyễn Kim Muôn analyse minutieusement chaque vertu, indiquant les méthodes pour les cultiver et atteindre la perfection. Il insiste sur l'importance de dompter les démons pour en faire des protecteurs du Dharma, c'est-à-dire transformer les énergies négatives en énergies positives pour soutenir la pratique spirituelle.

Enfin, la sincérité du cœur est l'élément clé pour obtenir des résultats. La confiance absolue dans le Dharma et dans son maître permet au pratiquant de rester ferme sur le chemin de l'éveil. Nguyễn Kim Muôn analyse également le mot foi dans les trois joyaux (foi - nature - vœu), affirmant que la foi doit provenir de la nature du ciel et non de l'âme inconsciente ou de l'âme consciente.

En outre, Nguyễn Kim Muôn souligne le rôle du maître spirituel dans le guidage et l'enseignement du disciple. Un bon maître aidera le disciple à s'éveiller et à éviter les erreurs sur la voie de la pratique. \textit{<< Bỏ cha bỏ mẹ bỏ làng theo thầy học đạo nghỉnh ngang sợ gì? >>} (Abandonner père, mère et village pour suivre un maître et étudier la voie, de quoi aurait-on peur?) [notre traduction] (page 43)

Pour conclure, Nguyễn Kim Muôn réaffirme la valeur du bouddhisme, le considérant comme la voie juste pour atteindre l'éveil et la libération. Il appelle chacun à suivre les enseignements du Bouddha, à mettre en pratique le Dharma pour atteindre le bonheur et la paix. \textit{<< Rốt cuộc, đạo Phật cho là một đạo từ bi và quán tử gồm cả nhơn luân và chánh trị. >>} (En fin de compte, le bouddhisme est considéré comme une voie de compassion et de contemplation, englobant à la fois l'éthique humaine et la gouvernance juste.) [notre traduction] (page 89)

\subsection{\textit{Danh truyền đạo tập} ; La Pratique de la religion [traduction du dépôt légal] ; Recueil de la transmission de la voie [notre traduction] - 1932}

\textit{Danh truyền đạo tập}\footnote{Nguyễn Kim Muôn, Danh truyền đao tập [Recueil de la transmission de la voie], Saigon, Bảo Tồn,1932} est un livre imprimé par l'imprimerie Bảo Tồn à Saïgon en 1932. Cet ouvrage de 64 pages ne mentionne ni la date de publication exacte, ni le nombre d'exemplaires tirés. Il s'agit d'un essai sur le bouddhisme qui se concentre sur la cultivation de la Nature et de la Vie pour atteindre l'immortalité et la libération.

\begin{figure}[H]
    \centering
    \includegraphics[width=0.4\textwidth]
    {images/danhtruyendaotap.JPEG}
    \caption{Couverture du livre \textit{Danh truyền đạo tập} ; Recueil de la transmission de la voie [notre traduction] - source : Gallica}
\end{figure}

Nguyễn Kim Muôn commence par souligner la valeur de la nature et de la vie humaines, encourageant le lecteur à les chérir et à s'efforcer de les préserver par la pratique spirituelle. Il affirme que les gens se soucient généralement de choses futiles et inutiles, oubliant que la mort est inévitable.
<< Con người sanh ra, cha mẹ chịu đau đớn banh đa xẻ thị, rồi đức Ngọc Đế còn bang cho một linh hồn, mới phân ra làm hai gọi là tánh mạng. >> (L'homme naît, ses parents souffrent de douleurs déchirantes, puis l'empereur de jade lui accorde une âme spirituelle, qui se divise en deux, appelées nature et vie) [notre traduction] (page 19).
Nguyễn Kim Muôn affirme:
\textit{<< Thì tánh mạng của con người mình qui báu lắm chớ, nào phải như bùng đất mà hủy hoại. >>} (Alors la nature et la vie de l'homme sont précieuses, il ne faut pas les détruire comme de la poussière)[notre traduction] (page 19).

Nguyễn Kim Muôn définit la cultivation comme le fait de se perfectionner, de se corriger et de faire de bonnes actions pour s'améliorer. Il soutient que les gens se préoccupent généralement de manger và de se vêtir sans se soucier de la pratique spirituelle, et que lorsqu'ils approchent de la fin de leur vie, il est trop tard pour s'inquiéter.
<< Hồi nào đến giờ nào có ai lo thử mà coi đâu, nên nói sao phải chết. >> (Jusqu'à présent, personne ne s'en est soucié, alors comment se fait-il qu'on doive mourir ? ) [notre traduction] (page 20).
Il conseille aux gens de se soucier de la mort avant de se soucier de la vie, c'est-à-dire de penser à la mort et de trouver un moyen de ne pas mourir pour atteindre l'immortalité. << Đức Thế Tôn nói: nếu chủng sanh biết có chết thì thì trong chỗ chết hãy ráng tìm, ắc được cái sống trong cái sống. >> (Le Bouddha a dit: si les êtres vivants savent qu'ils vont mourir, alors dans la mort, ils doivent s'efforcer de trouver, ils trouveront certainement la vie dans la vie) [notre traduction] (page 21).
Nguyễn Kim Muôn critique également les pratiques spirituelles erronées qui se concentrent uniquement sur les aspects extérieurs tels que la récitation d'écrits bouddhiques, la psalmodie du nom du Bouddha et les offrandes, sans comprendre la véritable nature du bouddhisme. << Thuở nay ai nghe nói tu, thì chỉnh gồ mỏ tụng nh, công phu bái sám, cất chùa lập miều, trì trai giải sát, vân vân... đó gọi là tu. >> (De nos jours, quand les gens entendent parler de cultivation, ils se contentent de réciter des écrits bouddhiques, de faire des offrandes, de construire des temples et des sanctuaires, de faire des retraites végétariennes, etc., c'est ce qu'ils appellent la cultivation) [notre traduction] (page 21). Il affirme que l'homme est un microcosme, contenant tous les éléments comme le ciel et la terre, et qu'il faut savoir comment les harmoniser pour atteindre l'équilibre et ainsi l'immortalité. << Trời hiệp âm dương hằng còn hoài. Người hiệp tánh mạng cũng sống hoài >> (Le ciel unit le yin et le yang pour toujours. L'homme unit la nature et la vie pour vivre éternellement) [notre traduction] (page 22).

Selon Nguyễn Kim Muôn, l'homme meurt parce qu'il lui manque un méridien par rapport au ciel, le méridien de la longévité. << Vả chăng con người có 27 mạch, trời có 28 vì sao, thiếu hơn trời một, nên mới chết.>> (De plus, l'homme a 27 méridiens, le ciel en a 28, c'est pourquoi il lui en manque un, et c'est pour cela qu'il meurt) [notre traduction] (page 23).
Il indique que la cultivation vise à trouver ce méridien et à nourrir l'essence et le sang pour que le corps ne se détériore pas. << Ấy là cái mạch trường sanh bất tử. >> (C'est le méridien de l'immortalité) [notre traduction] (page 23).
Il critique également ceux qui prétendent que la cultivation les empêchera de mourir, alors qu'en réalité ils n'ont jamais pratiqué. << Kế thế thật chưa biết chữ tu là gì; nên mới nói như thế; chở bắt ngay ra; tuy xuất gia, tuy ở thất, mà chưa có tu được một chút xiếu nào. >> (Les gens ordinaires ne savent pas vraiment ce qu'est la cultivation, c'est pourquoi ils disent cela, mais en réalité, même s'ils sont moines et vivent dans un temple, ils n'ont pas encore cultivé le moins du monde) [notre traduction] (page 24).

Nguyễn Kim Muôn redéfinit le mot cultiver comme développer, nourrir et faire le bien. Il insiste sur le fait que la cultivation doit commencer par colmater les fuites du corps, c'est-à-dire contrôler les sens et ne pas les laisser être affectés par le monde extérieur. << Nên nói tu là: nhắm mắt, ngậm miệng, nút tai, chợn không đi, tay không rờ, mũi chẳng thở, chẳng đại, chẳng tiểu. >> (Donc, on dit que la cultivation consiste à fermer les yeux, fermer la bouche, boucher les oreilles, ne pas marcher, ne pas toucher avec les mains, ne pas respirer par le nez, ne pas déféquer, ne pas uriner) [notre traduction] (page 26).
Il soutient que la pratique doit être persévérante et patiente, sans se décourager ni abandonner à mi-chemin. << Cái đạo hay nơi ngồi bền (bền xả chở sự kết quả thì nội nháy mắt). >> (La voie est bonne pour ceux qui restent assis longtemps, si on persévère sans se soucier du résultat, on l'atteindra en un clin d'œil) [notre traduction] (page 58).

Nguyễn Kim Muôn souligne également trois interdits dans la cultivation: la luxure, les paroles inutiles et la gourmandise. Il affirme que ces choses épuisent l'essence et le souffle vitaux, empêchant la cultivation d'aboutir. << Đại đạo vả chăng kị ba đều: 1: một là dâm dục (lọt tỉnh). 2: Hai là nói (hay nói, nói xàm làm cho hao khí). 3: Ba là ăn (ăn nhiều, vô độ, hay thèm, làm cho mê mụi cái thần). >> (La grande voie interdit trois choses: 1: la luxure (perte d'essence); 2: les paroles (bavardages inutiles qui épuisent le souffle); 3: la nourriture (manger excessivement, sans modération, avoir des envies, ce qui obscurcit l'esprit)) [notre traduction] (page 59).
Il conseille aux pratiquants de nourrir l'essence originelle et de limiter la nourriture afin de préserver leur énergie vitale et d'améliorer l'efficacité de leur pratique.

Nguyễn Kim Muôn explique également comment cultiver l'esprit et le souffle. Il affirme que l'esprit est le cœur, le souffle est les reins, et qu'il faut unir l'eau et le feu pour créer le remède. Il insiste sur le fait que la pratique doit se concentrer sur le regard, c'est-à-dire l'introspection, afin d'harmoniser le souffle et le sang, de dégager les méridiens et d'atteindre l'état de retour de l'esprit et du souffle. << Thần đã nói là tâm, tâm thuộc hỏa, ở nơi ngoài là song mâu, gọi là thần quang, chỗ nói; vận thần quang nội chiếu trong cung, là ngô trân nơi khí huyệt. >> (L'esprit est dit être le cœur, le cœur appartient au feu, à l'extérieur se trouvent les deux yeux, appelés lumière spirituelle; la lumière spirituelle brille à l'intérieur du palais, c'est regarder fixement le point vital) [notre traduction] (page 77).

Enfin, Nguyễn Kim Muôn affirme que la cultivation n'est pas difficile, la difficulté réside dans l'abandon des mauvaises habitudes et des désirs ordinaires. << Vậy, kết cuộc, th đạo là đề, đâu có khó; khó là khó cái tánh của con người, bổ sao cho được tánh cũ, ra sao cho khỏi dục tâm (hất tình lục dục) thì đạo ở một bên, lo gì thành cùng chẳng thành. >> (Donc, en fin de compte, la voie est facile, elle n'est pas difficile; la difficulté réside dans la nature humaine, comment se débarrasser de la vieille nature, comment se libérer du désir — des sept émotions et des six désirs — alors la voie est là, à côté, pourquoi s'inquiéter de réussir ou d'échouer.) [notre traduction] (page 82). Il souligne également que la cultivation de la nature est la première étape et la plus importante sur la voie spirituelle.

\subsection{\textit{Đoạn dâm căng} ; Destruction de la source du désir [traduction selon Nguyễn Kim Muôn] - 1932}

\textit{Đoạn dâm căng}\footnote{Nguyễn Kim Muôn, \textit{Đoạn dâm căng} [ Destruction de la source du désir], Đức Lưu Phương, Saigon, 1932} est un livre de 48 pages, imprimé à 2000 exemplaires par l'imprimerie Đức Lưu Phương à Saïgon le 28 novembre 1932.
\begin{figure}[H]
    \centering
    \includegraphics[width=0.4\textwidth]
    {images/doandamcang.JPEG}
    \caption{Couverture du livre Destruction de la source du désir - source : Gallica}
\end{figure}


Il écrit ce livre sur l'île de Phú Quốc entre octobre 1931 et mars 1932, l'ouvrage a été imprimé à Saïgon. En ce moment Nguyễn Kim Muôn se présente comme un moine diplômé qui a fondé sa propre lignée et a construit des temples et des retraites sur l'île. Il affirme avoir réalisé ce projet seul, sans aide financière extérieure.

\textbf{La philosophie de la luxure comme source de souffrance}

Selon Nguyễn Kim Muôn, la phrase du Bouddha << Traverser toutes les souffrances >>, se réfère directement à la luxure \textit{dâm dục}. Il affirme que la luxure détruit le corps, le rendant faible, vieux et finalement mortel. La longévité et la vitalité humaines proviennent de l'essence et du sang \textit{ Tinh huyết,} que les hommes et les femmes doivent préserver. Nguyễn Kim Muôn critique les pratiquants qui sont végétariens mais ne renoncent pas à la luxure, affirmant qu'ils épuisent leur essence vitale. Pour lui, la luxure est la source de mille maux \textit{đầu giọc của muôn sự ác}.

Le texte insiste sur la nécessité de pratiquer le végétarisme à long terme et le célibat absolu. Rompre avec la luxure est une discipline difficile car elle ne se limite pas aux actes, mais inclut également la maîtrise des désirs intérieurs et des sens.

\textbf{Les pratiques fondamentales : l'alchimie interne}

Nguyễn Kim Muôn décrit deux méthodes secrètes essentielles transmises par les maîtres de la voie : le \textit{Đoạn dâm căng} (la rupture avec la racine de la luxure) et le \textit{Hườn tinh bồ não} (le retour de l'essence pour revigorer le cerveau). Un véritable maître doit enseigner ces méthodes, et quiconque en enseigne d'autres est considéré comme un faux maître.

\begin{itemize}
    \item L'alchimie de l'essence vitale : Nguyễn Kim Muôn explique que l'essence \textit{tinh} est la racine de l'immortalité. La luxure fait s'échapper cette essence, ce qui affaiblit l'individu. Le but est de raffiner l'essence pour la transformer en énergie vitale \textit{khí}. Ce processus est une alchimie intérieure, comparée à faire bouillir de l'eau (l'essence) pour en faire de la vapeur (l'énergie).
    \item Le \textit{Trúc cơ} (Construction des fondations) est une étape cruciale qui consiste à utiliser la volonté \textit{ý} et l'esprit \textit{thần} pour contrôler le moteur de la luxure \textit{máy dâm}. Ce processus est décrit comme un feu ardent et un vent rapide. Pour les hommes, ce processus est appelé \textit{Sát bạch hổ} tuer le tigre blanc, et pour les femmes, \textit{Trảm xích long} (abattre le dragon rouge). L'objectif est de raffiner l'essence de manière si complète que même un acte sexuel physique ne la ferait plus s'échapper.
    \item Le\textit{ {tâm lặng thần trong}} (Esprit calme, esprit clair) est le fondement de la grande voie. Nguyễn Kim Muôn décrit cela comme la maîtrise des six sens : les yeux ne regardent pas, les oreilles n'écoutent pas, la bouche ne parle pas, les mains ne touchent pas, les pieds ne marchent pas, et l'esprit n'a pas de pensées de désir.
    \item Le \textit{Hồi quang phản chiếu} (Retour de la lumière pour éclairer l'intérieur) est la technique clé pour concentrer l'esprit. Elle consiste à sceller les portes des sens et à diriger l'attention vers l'intérieur pour unifier l'esprit, l'essence et l'énergie en un seul point.
\end{itemize}

\textbf{La voie du pratiquant}

La voie du pratiquant est un processus de transformation complète. Nguyễn Kim Muôn utilise l'analogie d'une maison pour décrire le corps : les yeux, les oreilles, la bouche et le nez sont les quatre  portes que l'on doit fermer, et l'esprit \textit{thần} est le maître de la maison qui doit y rester. Il compare la pratique à une poule qui couve son œuf ou à un dragon qui garde sa perle. Il met en garde contre la pratique intermittente, affirmant que le succès vient d'un effort constant et ininterrompu.

Nguyễn Kim Muôn insiste sur la nécessité de se retirer de la société \textit{ly gia các ái}. Il explique que les pratiquants qui ont atteint un certain niveau ne peuvent plus supporter l'atmosphère de la société. Il raconte que les grands maîtres d'autrefois, même en rendant visite à leur famille, évitaient tout contact avec leur femme pour ne pas réactiver les désirs.

\textbf{Le témoignage de Nguyễn Kim Muôn et l'appel à la pratique}

Nguyễn Kim Muôn termine son œuvre en partageant son propre parcours. Il a passé six années à revitaliser le bouddhisme \textit{chấn hưng Đạo Phật} et a construit ses temples et ses ermitages seul, sans l'aide de personne. Il est revenu de son isolement pour fonder une imprimerie et un Institut bouddhiste  \textit{Phật viện}, dans le but de publier et de diffuser ses écrits.

Contrairement à la pratique courante consistant à distribuer les textes religieux gratuitement, Nguyễn Kim Muôn soutient qu'il est préférable de les vendre à bas prix. Il affirme que cela garantit que les livres seront achetés par des personnes réellement motivées par la pratique et qu'elles ne les gaspilleront pas.

\subsection{\textit{Dục Tâm} ; Le désir [traduction selon Nguyễn Kim Muôn] - 1932}

\textit{Dục Tâm}\footnote{Nguyễn Kim Muôn, \textit{Dục tâm. Tâm hư tắc thân ngưng} [Le désir], Đức Lưu Phương, Saigon, 1932} est un livre de 28 pages, imprimé à 1000 exemplaires par l'imprimerie Đức Lưu Phương à Saïgon le 5 décembre 1932.

Le désir de Nguyễn Kim Muôn est une œuvre qui traite du chemin de la cultivation, de l’illumination et de la libération de la souffrance humaine. L'œuvre met l'accent sur la soumission de l'esprit du désir pour atteindre la pureté, revenir à sa nature innée et ainsi atteindre l'illumination et la libération.

\begin{figure}[H]
    \centering
    \includegraphics[width=0.4\textwidth]
    {images/ductam.png}
    \caption{Couverture du livre Le désir \textit{Dục Tâm} - source : Gallica}
\end{figure}

Nguyễn Kim Muôn commence par affirmer que le désir est la naissance du cœur, qui est l’origine de toutes choses dans le monde. Les gens sont guidés par la luxure, ce qui provoque d’innombrables souffrances et problèmes. Bouddha a expliqué que, dans le monde, les êtres vivants sont confus et souffrent sans fin, mais ne peuvent pas comprendre la source de la souffrance, car ils ne peuvent pas vaincre l'esprit du désir (page 5 - 6).

Ce n’est que lorsque l’esprit meurt, c’est-à-dire qu’il n’est plus contrôlé par le désir, que les gens cessent de souffrir. Le désir est le fusible du Postnatal ; si le désir bloque le fusible, le Postnatal sera bloqué. Il vaut mieux rétablir rapidement le Prénatal pour que ce soit plus facile. Pour soumettre l’inné, il est important de couper la luxure. Même en cultivant pendant des milliers d’années, si l’on ne peut pas abandonner la luxure, on ne pourra jamais retrouver son état inné.

Pour éliminer la luxure, Nguyễn Kim Muôn propose la méthode appelée L'esprit meurt, l'esprit vit. Un esprit mort est un état d’esprit qui n’est plus contrôlé par des désirs ou des illusions, comme le sommeil, où seul le souffle reste et rien d’autre n’est connu. À ce moment-là, l’Esprit — ou l’âme — est libéré du corps. Selon lui, s’entraîner pour atteindre cet état revient à utiliser le sommeil comme moyen d’échange contre la pratique de la cultivation. Pour y parvenir, il propose la méthode de l'oubli. Oublier ici ne signifie pas tout effacer, mais laisser de côté les désirs et les convoitises du Postnatal afin de se souvenir de ce qui appartient au Prénatal. Autrement dit, oublier le Souffle Postnatal, se souvenir du Souffle Prénatal.

Le cœur humain, encore rempli de désirs, ressemble au sac d’un mendiant : on y met ce qui nous plaît, on y laisse ce qui nous pèse. Il faut vider ce sac et éliminer toutes les envies afin d’atteindre la pureté et l’illumination (page 7).

Nguyễn Kim Muôn analyse ensuite les Cinq Agrégats - Forme, Sentiment, Perception, Formation Mentale, Conscience - pour que les pratiquants puissent clairement identifier les obstacles sur le chemin de l’illumination:

\begin{itemize}
    \item Forme : il faut l’examiner jusqu’à ce qu’elle soit vide, c’est-à-dire ne plus être aveuglé par la luxure. Ne pas considérer signifie ne pas désirer, et ne pas désirer vient du fait qu’on s’en est déjà détaché.
    \item Sentiment : il résulte de l’attachement à la Forme. Il faut rompre la dépendance et ne pas laisser l’âme de l’Illusion ou l’âme de l’Illumination nous entraîner.
    \item Pensée : c’est l’illusion mentale. Il faut arrêter de penser inutilement et ne pas laisser les idées aléatoires dominer.
    \item Action : il faut éviter les mauvaises actions et les mauvaises voies.
    \item Conscience : c’est la connaissance juste, qui ne se laisse pas séduire par de mauvaises choses. Il faut réfléchir avec discernement et examiner attentivement avant d’agir (page 10-18).
\end{itemize}

L’auteur insiste sur le rôle de la persévérance et de la patience dans la pratique. Il compare cela au fait de vouloir devenir riche : il faut travailler dur, gratter les cendres, tamiser la paille, persévérer jusqu’au bout pour y parvenir. De même, devenir une fée ou un Bouddha exige encore plus de détermination et de sacrifice. Nguyễn Kim Muôn raconte aussi son expérience de moine sur l'île de Phú Quốc, où il a dû endurer la faim et les difficultés, tout en restant persévérant et patient, jusqu’à atteindre l'illumination (page 19 - 30).

En conclusion, l'auteur souligne que le véritable chemin spirituel est avant tout une quête intérieure, qui ne tolère aucune forme de dépendance. Il insiste sur le fait que la libération passe par l'élimination du désir, même dans ses manifestations les plus subtiles. Pour lui, les pratiques extérieures comme le végétarisme, le respect des préceptes ou le renoncement à la vie familiale ne sont pas suffisantes ; elles ne sont que des étapes si elles ne mènent pas à l'objectif ultime de se libérer de tout attachement. C'est pourquoi il exhorte le lecteur à s'engager dans un processus d'auto-cultivation et d'auto-correction, en toute indépendance, car le salut réside en soi et non dans des rites ou dans la guidance d'autrui.

\subsection{\textit{Phật Đạo. Giải về hai chữ đạo đức} ; Le bouddhisme. Les Explications de deux mots : religion et vertu. [traduction selon Nguyễn Kim Muôn] - 1932}

\textit{Phật Đạo.Giải về hai chữ đạo đức}\footnote{Nguyễn Kim Muôn, \textit{Phật đạo. Giải về hai chữ Đạo Đức} [La doctrine du bouddhisme], Xưa Nay, Saigon, 1932} est un livre de 40 pages, imprimé à 1000 exemplaires par l'imprimerie Bảo Tồn à Saïgon le 13 juin 1932.

Le bouddhisme est un sermon sur le bouddhisme et explique les deux mots \textit{đạo} (chemin) et \textit{đức} (vertu) de Nguyễn Kim Muôn, axé sur l'explication du sens profond des deux mots. Nguyễn Kim Muôn met l'accent sur la pratique de la moralité, en vivant en accord avec l'esprit du bouddhisme pour atteindre la paix et le bonheur pour soi-même et pour la société.
\begin{figure}[H]
    \centering
    \includegraphics[width=0.4\textwidth]
    {images/phatdao.JPEG}
    \caption{Couverture du livre \textit{Phật Đạo} - source : Gallica}
\end{figure}
\clearpage
\textbf{Partie 1 : Affirmer la valeur du bouddhisme (Pages 6-12)}

 Nguyễn Kim Muôn commence par souhaiter la bienvenue aux invités et exprimer sa gratitude envers les fidèles. Il considère le bouddhisme comme une religion noble et compatissante et souhaite que chacun atteigne rapidement l’illumination et suive la voie de la moralité.

 Il compare le bouddhisme à d’autres religions et affirme qu’il s’agit de la religion originelle, le fondement des autres traditions. Il critique la croyance aveugle dans les superstitions et encourage à suivre le bouddhisme d’une manière authentique et éclairée.

\textbf{Partie 2 : Signification du mot Đạo (Pages 13-18)}

 Nguyễn Kim Muôn analyse le mot \textit{Đạo} d’après la structure des caractères chinois, en soulignant sa profondeur. Pour lui, le \textit{Đạo} est le chemin vers l’illumination, la libération, et aussi la voie de la pratique menant à l’immortalité.

 Il explique que le \textit{Đạo} consiste à s’approprier le mécanisme du ciel et la puissance de la création, c’est-à-dire que chacun doit pratiquer par lui-même pour atteindre l’illumination et la libération, sans s’en remettre aux dieux ni à d’autres forces surnaturelles.

\textbf{Partie 3 : Signification du mot \textit{Đức} (Pages 19-27)}

 Il définit le mot \textit{Đức} (vertu) sous deux angles : les bénédictions présentes et la vertu éclairée, cette dernière représentant l’illumination et la compréhension profonde de la moralité. Pour lui, la vertu ordinaire est secondaire, tandis que la vertu claire est essentielle.

 Nguyễn Kim Muôn considère la vertu comme le mandat céleste yin-vertu, une moralité conférée par le ciel, qui correspond à l’illumination et à la sagesse sur la moralité. Il précise que \textit{Đức} signifie également \textit{Huệ}, c’est-à-dire la sagesse et la compréhension véritable de la moralité.

\textbf{Partie 4 : L’accent sur la pratique morale (Pages 28-31)}

 L’auteur affirme que le bouddhisme est une religion de pratique et non seulement de théorie. Il encourage chacun à suivre la Voie, à vivre selon l’esprit du bouddhisme, à observer les préceptes, à accomplir de bonnes actions et à aider autrui.

 Il critique les moines qui se focalisent sur les apparences extérieures sans cultiver la moralité, estimant que non seulement ces personnes n’obtiennent aucun résultat, mais qu’elles nuisent aussi à la réputation du bouddhisme. Pour lui, le bouddhisme est la voie vers la paix et le bonheur pour soi-même et pour la société, et il invite à imiter l’exemple du Bouddha pour parvenir à l’illumination et à la libération.

\textbf{Partie 5 : Le bouddhisme, chemin de l’immortalité (Pages 32-34)}

 Nguyễn Kim Muôn poursuit en présentant le bouddhisme comme le Dharma suprême, profond et merveilleux, ainsi que comme la voie de la non-naissance et de la non-mort. Cette voie, découverte par le Bouddha, aide les êtres sensibles à échapper aux quatre grandes souffrances : la naissance, le vieillissement, la maladie et la mort.

 Pratiquer le bouddhisme signifie ne pas mourir, non pas au sens physique, mais mourir au désir, à l’illusion, aux sept émotions et aux six désirs. Ce qui reste alors est l’âme, partageant la même énergie que le vide, soit le corps du Dharma et le corps physique réunis.

 Il cite des passages des écritures pour appuyer sa vision de l’absence de mort et souligne que si la mort physique est inévitable, on peut atteindre l’immortalité spirituelle en trouvant la vérité du Dharma, en abandonnant désirs et illusions et en éveillant l’esprit véritable.

\textbf{Partie 6 : Histoire du bouddhisme et de son déclin (Pages 35-37)}

 Nguyễn Kim Muôn retrace l’histoire du bouddhisme depuis sa création, en affirmant qu’il n’a prospéré pleinement que durant les 500 premières années, avant de se fragmenter en de nombreuses sectes, ce qui a entraîné son déclin.

 Il réaffirme toutefois que le bouddhisme reste la religion originelle, source des autres traditions, et appelle chacun à renforcer ses racines plutôt qu’à courir après les branches et finir par chuter. Il critique ceux qui délaissent le bouddhisme sous prétexte qu’il est décadent, rappelant que, décadent ou non, il nous a été transmis par nos ancêtres et ne doit pas être abandonné.

\textbf{Partie 7 : Appel à l’impression des écritures bouddhistes (Pages 38-44)}

 L’auteur appelle à contribuer financièrement à l’impression des écritures bouddhistes pour en diffuser l’accès au plus grand nombre. Il considère cet acte comme vertueux, aidant les gens à s’éveiller et à se détourner du mal. Il recommande d’imprimer des textes précieux et utiles à la vie, et d’éviter les écrits superstitieux.

\textbf{Partie 8 : Critique du gaspillage d’argent par superstition (Pages 41-43)}

 Nguyễn Kim Muôn déplore que beaucoup dépensent de l’argent pour des cultes ou des pèlerinages, comme à la montagne To, tout en négligeant l’impression et la diffusion des écritures bouddhistes. Il voit là une superstition inutile, qui n’apporte aucun bénéfice ni pour soi-même ni pour la société.

\clearpage
\subsection{\textit{Tại sao tôi tu Phật?} ; Pourquoi je suis boudhiste ? [traduction selon Nguyễn Kim Muôn] - 1932}

\textit{Tại sao tôi tu Phật?}\footnote{Nguyễn Kim Muôn, \textit{Tại sao tôi tu Phật} [Pourquoi je suis bouddhiste ?], Đức Lưu Phương, Saigon, 1932} est un livre de 42 pages, imprimé à 1000 exemplaires par l'imprimerie Đức Lưu Phương à Saïgon en 1932, le document n'indique pas de date exacte lisible.

Ce livre de Nguyễn Kim Muôn est un essai expliquant pourquoi Nguyễn Kim Muôn a choisi de suivre le bouddhisme. L’ouvrage est profondément philosophique, se plongeant dans l’analyse de la nature du bouddhisme, tout en critiquant les idées fausses sur la pratique religieuse.
\begin{figure}[H]
    \centering
    \includegraphics[width=0.4\textwidth]
    {images/taisaotoituphat.JPEG}
    \caption{Couverture du livre Pourquoi je suis boudhiste? - \textit{Tại sao tôi tu Phật?} - source : Gallica}
\end{figure}

\textbf{Partie 1 : Pourquoi je pratique le bouddhisme (Pages 7-8)}

Nguyễn Kim Muôn commence en affirmant qu’il pratique le bouddhisme parce qu’il voit la Vérité. Pour lui, la Vérité est le chemin de la vertu, menant les pratiquants à la Terre pure. Depuis la naissance du Bouddha et pendant les cinq siècles suivants, il n’existait qu’un seul chemin de Vérité, grâce auquel d’innombrables personnes ont atteint l’illumination et la réalisation.

\textbf{Partie 2 : La division des chemins (Pages 8-10)}

 Après cinq siècles, de nombreux chemins ramifiés sont apparus, correspondant à l’émergence de différentes religions et enseignements. Cette diversité a semé la confusion chez les pratiquants, les égarant et les menant à la souffrance. Nguyễn Kim Muôn critique ceux qui pratiquent de manière dépendante, en attendant des bénédictions des dieux ou des bouddhas. Selon lui, la souffrance vient du fait de ne pas comprendre la Vérité, de ne pas pratiquer par soi-même et de s’en remettre à des forces surnaturelles. Ainsi, l’homme crée ses propres malheurs, meurt et renaît sans fin, et se noie dans la mer de la souffrance.

\textbf{Partie 3 : La Voie dans le corps (Pages 11-12)}

 Nguyễn Kim Muôn affirme que la Voie se trouve au milieu du corps, c’est-à-dire dans les profondeurs de chaque être humain. L’homme possède trois âmes : l’âme de la conscience, l’âme de l’illumination et l’âme de l’esprit. L’âme est la nature de Bouddha, la vraie nature. Cultiver le Bouddha signifie se cultiver soi-même, éliminer les mauvaises habitudes et purifier les six sens, afin d’atteindre l’illumination et la libération.

\textbf{Partie 4 : Critique des pratiques erronées (Pages 12-16)}

 Il critique les pratiques spirituelles dévoyées telles que les planches Ouija, la voyance, la recherche incessante d’enseignants ou l’étude religieuse superficielle. Pour lui, ces activités ne servent qu’au divertissement et n’apportent aucun bénéfice à la cultivation. Il dénonce également ceux qui se font appeler maîtres éclairés mais trompent les autres. Selon lui, chacun possède la Voie et n’a pas besoin de prendre refuge ou de chercher le Dharma auprès de qui que ce soit.

\textbf{Partie 5 : La pratique comme don (Pages 13-16)}

 Nguyễn Kim Muôn explique que pratiquer, c’est donner, et se pratiquer soi-même, c’est se donner à soi-même. Donner signifie ici retirer ou abandonner. La première étape consiste à renoncer à l’apparence, en éliminant les mauvaises habitudes comme fumer, boire ou jouer. Ensuite, il faut renoncer à l’intérieur, c’est-à-dire abandonner les sept émotions et les six désirs.

\textbf{Partie 6 : Contre l’hypocrisie dans la vie monastique (Pages 13-16)}

 Il critique ceux qui se disent moines mais continuent à se comporter de manière indécente, à fumer, à boire, ou qui, malgré le végétarisme, tiennent des propos malveillants et mentent dans leur cœur. Pour lui, ces personnes ne pratiquent réellement nulle part.

\textbf{Partie 7 : Comprendre la colère (Pages 17-19)}

 Nguyễn Kim Muôn analyse l’origine de la colère, qu’il attribue au sang chaud et non à un cœur brûlant ou à un tempérament fougueux. Il encourage à rechercher la cause profonde pour pouvoir l’éliminer.

\textbf{Partie 8 : Les cinq organes et les cinq éléments (Pages 19-20)}

 Il explique la relation entre les cinq \textit{Zang} (cinq organes internes) et les cinq éléments, montrant la correspondance entre l’homme, le ciel et la terre. Selon lui, le ciel et l’homme ont la même origine.

\textbf{Partie 9 : La Terre pure à l’intérieur (Pages 21-22)}

 Nguyễn Kim Muôn évoque le Troisième Lieu, la Terre pure ou Palais Céleste Occidental, lieu de la Lumière Mystérieuse à Une Ouverture. Il affirme que la Terre pure n’est pas lointaine mais se trouve à l’intérieur de chaque personne.

\textbf{Partie 10 : Atteindre la Voie \textit{Đạo} (Pages 23-24)}

 Il précise qu’atteindre le \textit{Đạo} ne consiste pas à obtenir un objet, mais à parvenir à l’illumination et à comprendre la nature du Tao. C’est se rendre compte que la Voie est en soi, comme une lampe du cœur qui s’allume soudainement dans le vide. Il reconnaît avoir cherché la Voie à l’extérieur sans savoir qu’il la portait déjà en lui. Il cite les paroles du Bouddha affirmant que chaque maison a un Bouddha, ainsi que celles de Bodhidharma, pour appuyer son point de vue.

\textbf{Partie 11 : Les trois âmes (Pages 24-26)}

 Nguyễn Kim Muôn décrit les trois âmes humaines : l’âme de Bouddha, qui est la vraie nature, la meilleure part de chaque être ; l’âme humaine, tournée vers la gloire, la fortune et les biens matériels, source de troubles et de souffrances ; et l’âme animale, encline à la colère et à la luxure. Les âmes humaine et animale voilent souvent l’âme de Bouddha, empêchant de reconnaître son vrai cœur. Pratiquer consiste à retirer ces deux âmes pour laisser se manifester l’âme de Bouddha.

\textbf{Partie 12 : Conclusion (Pages 27-30)}

 Il conclut que la Voie est proche et que Bouddha est dans l’esprit, ce qui signifie que la Voie et Bouddha sont présents en chaque personne. Chacun doit se sauver soi-même, pratiquer et éveiller son véritable esprit. Il recommande de chercher de vrais moines auprès desquels apprendre et d’éviter les faux enseignants. La cultivation doit se faire naturellement, sans dépendre d’autrui. Il critique également ceux qui se concentrent sur les apparences extérieures, considérant que des gestes comme adorer, offrir, s’incliner, chanter des écritures bouddhiques ou réciter le nom de Bouddha ne sont que superficiels et n’apportent aucun bénéfice réel à la cultivation. Enfin, il appelle à revenir à la Vérité, à pratiquer la moralité et à vivre selon l’esprit du bouddhisme pour trouver la paix et le bonheur pour soi-même et pour la société.

 
\clearpage
\subsection{ \textit{Ai muốn tu?} ; Qui veut être bonze (bouddhisme) ? [traduction du dépôt légal] ; Qui veux être bouddhiste ? [notre traduction] - 1933}

\textit{Ai muốn tu?}\footnote{Nguyễn Kim Muôn, \textit{Ai muốn tu? Phật giáo vấn đáp} [Qui veut être bonze ?], Đức Lưu Phương, Saigon, 1933} est un livre de 24 pages, imprimé à 1000 exemplaires par l'imprimerie Đức Lưu Phương à Saïgon le 29 mars 1933. Ce livre est un discours sur l'ouvrage Questions et réponses sur le bouddhisme de Nguyễn Kim Muôn écrit sous la forme d'un dialogue entre deux personnages moi \textit{anh} et toi \textit{em}, dans le but de répondre à des questions sur le bouddhisme et le sens de la pratique du bouddhisme et l'illumination. L’ouvrage est éclairant, aidant les lecteurs à avoir une vision plus correcte du chemin de la pratique et de la libération.
\begin{figure}[H]
    \centering
    \includegraphics[width=0.4\textwidth]
    {images/aimuontu.JPEG}
    \caption{Couverture du livre Qui veut être bonze ? \textit{Ai muốn tu?} - source : Gallica}
\end{figure}
\textbf{Partie 1 : Pourquoi pratiquer ? (Page 5-8)}

L'œuvre s'ouvre sur la question  : Pourquoi pratiquons-nous ? \textit{Anh }affirme que les humains sont nés dans ce monde pour garder les biens \textit{giữ của} et garder les zombies \textit{giữ thay ma}. Garder les biens matériels signifie préserver ce que vous avez créé au cours de votre vie antérieure, et garder les zombies consiste à préserver ce corps pour qu'il revienne à son origine.
Le personnage \textit{em} se demande alors à quel endroit on peut retourner à son origine. Le personnage \textit{anh} explique que les humains n'appartiennent pas à ce monde, mais viennent d'un autre royaume, appelé la lumière divine de l'empereur de jade, et que leur objectif et de retourner à cette ancienne patrie, là où ont eu lieu leurs vies antérieures. 
\clearpage
\textbf{Partie 2 : Pourquoi les gens se réincarnent-ils ? (Pages 6-8)}

 \textit{Anh} explique que les gens se réincarnent dans ce monde parce qu’ils veulent quelque chose, comme voyager dans un autre pays. Cependant, après être descendus ici, les êtres humains sont liés par le sexe, c’est-à-dire les sept émotions et les six désirs, et ils ne peuvent donc pas revenir.

 Le sexe est une force invisible qui s’accroche à l’âme, alourdit l’être et le tire vers le bas, autrement dit vers les enfers. Les humains doivent y rester jusqu’à ce que leur énergie sexuelle soit épuisée avant de pouvoir se réincarner.
 
\textbf{Partie 3 : L’univers est le mortier de l’enfer (Pages 8-9)}

 \textit{Anh} considère que cet univers est le moulin de l’enfer, un lieu où les humains peuvent se confronter et s’exercer. Si l’on réussit, on retourne au royaume supérieur ; si l’on échoue, on continue à se réincarner.

 L’univers est un espace rempli de souffrance, c’est pourquoi les gens doivent pratiquer pour s’en libérer. Cette souffrance est créée par les humains eux-mêmes et non par une force extérieure.

 \textbf{Partie 4 : Comment se souvenir de ses racines ? (Pages 10-13)}

 \textit{Em} se demande comment se souvenir de ses racines, et \textit{Anh} recommande de lire des livres taoïstes ainsi que des écritures bouddhistes. Cependant, il précise qu’il ne faut pas être esclave des livres, mais réfléchir et contempler par soi-même.
 Grâce à cette pratique, les êtres humains parviennent à harmoniser les deux âmes, humaine et animale, pour révéler l’âme du Bouddha. À ce moment-là, ils savent distinguer le bien du mal, se détachent des choses du monde et ne sont plus tentés par celui-ci.

\textbf{Partie 5 : La signification du végétarisme et la lecture des écrits bouddhiques (Pages 13-15)}

 \textit{Anh} explique que le végétarisme consiste à corriger sa nature en éliminant les désirs et les convoitises. Les aliments carnés nourrissent les âmes humaine et animale, tandis que les aliments végétariens nourrissent l’âme divine.

 Lire les écrits bouddhiques, c’est chercher la raison, comprendre la vérité et éviter la superstition. Il faut persévérer dans la lecture et la contemplation des écritures bouddhiques jusqu’à en saisir clairement le sens.

\textbf{Partie 6 : Qu’est-ce que la persistance ? (Pages 15-16)}

 \textit{Anh} définit la persévérance comme une constance et une stabilité intérieure, ce qui vient avant restant toujours identique. Un pratiquant doit posséder cette persévérance pour obtenir des résultats.

\textbf{Partie 7 : Comment pratiquer (Pages 16-20)}

\textit{Anh} donne ici des méthodes précises de pratique :
\begin{itemize}
    \item Abandonner les mauvaises habitudes comme mâcher du bétel, boire de l’alcool, fumer, jouer et adopter une mauvaise conduite sexuelle.
    \item Garder les six sens purs : les yeux ne voient que soi-même, les oreilles n’entendent que soi-même, le nez ne sent pas les mauvaises choses, la langue ne bavarde pas, le corps ne s’agite pas, l’esprit reste calme et pur.
    \item Pratiquer le végétarisme, de manière permanente ou temporaire, mais avec un estomac et une bouche réellement végétariens.
    \item Réciter le nom de Bouddha avec l’esprit, non avec la bouche, en gardant un esprit pur.
    \item Chanter en comprenant les principes, plutôt que de simplement finir la lecture ; il faut lire les livres religieux des vrais pratiquants, au lieu de réciter des écritures difficiles à comprendre.
    \item Faire des offrandes avec le cœur, et non seulement avec des biens matériels.
    \item S’incliner avec l’esprit est suffisant, il n’est pas nécessaire de le faire avec le corps.
\end{itemize}
 
\textbf{Partie 8 : Critique du refuge dans la secte (Pages 20-23)}

 Dans cette partie, Nguyễn Kim Muôn critique la pratique consistant à se réfugier dans une secte, qu’il considère comme une coutume superstitieuse n’apportant aucun bénéfice réel à la cultivation spirituelle. Il conseille à chacun de brûler les feuilles ou certificats de la secte et de se réfugier directement en Bouddha, sans passer par l’intermédiaire d’une autre personne.

\textbf{Partie 9 : La culture de soi et l’illumination (Pages 23-27)}

 Nguyễn Kim Muôn réaffirme que chacun porte le Bouddha en soi et qu’il n’est pas nécessaire de chercher un maître pour apprendre le Dharma. Il suffit de pratiquer par soi-même et d’éveiller son véritable cœur. Il recommande cependant de se tourner vers de vrais pratiquants pour apprendre et de se tenir à l’écart des maîtres trompeurs.

 Il souligne que l’esprit est Esprit, l’esprit est Nature, et la Nature est Vide, ce qui signifie que l’esprit humain est fondamentalement vide et pur. Revenir à la source, éliminer les illusions et les attachements, permet d’atteindre l’illumination et la libération.

\textbf{Partie 10 : Conclusion et recommandations (Pages 27-30)}

 Nguyễn Kim Muôn conclut en affirmant que si la Voie est difficile, on ne peut l’apprendre, invitant chacun à avoir confiance en sa propre capacité à parvenir à l’illumination. Il mentionne aussi certaines de ses autres œuvres afin de fournir aux lecteurs davantage de références utiles sur le chemin de la pratique et de la libération.
\\
\noindent\rule{0.35\linewidth}{0.6pt}
\clearpage

\subsection{\textit{Gương huệ} ; La prière \textit{huệ cảnh} traduite et expliquée [traduction du dépôt légal] - Miroir de la sagesse [notre traduction] - 1933}

\textit{Gương huệ}\footnote{Nguyễn Kim Muôn, \textit{Gương Huệ} [Miroir de la sagesse], Bảo Tồn, Saigon, 1933} est un livre de 40 pages, publié à Saïgon le 5 septembre 1933 par l'imprimerie Bảo Tồn. L'ouvrage est tiré à 1000 exemplaires.

Ce livre explique la signification de deux écritures bouddhistes importantes, \textit{huệ cảnh} vision de sagesse [notre traduction] et \textit{lục tự} six mots [notre traduction]. Nguyễn Kim Muôn met l’accent sur la pratique des enseignements des Écritures, tout en critiquant les idées fausses sur la pratique spirituelle.

\textbf{Partie 1 : Introduction aux deux écrits bouddhiques \textit{huệ cảnh} et \textit{lục tự} (Pages 5-6)}

\begin{figure}[H]
    \centering
    \includegraphics[width=0.4\textwidth]
    {images/guonghue.png}
    \caption{Couverture du livre La prière \textit{huệ cảnh} traduit et expliqué - Miroir de la sagesse - \textit{Gương huệ} - source : Gallica}
\end{figure}

Nguyễn Kim Muôn commence par affirmer la valeur des deux concepts la sagesse \textit{huệ cảnh} et six mots \textit{lục tự}, affirmant qu'ils expriment les principes taoïstes corrects et peuvent aider les gens à devenir des bouddhas. Nguyễn Kim Muôn affirme qu'il a déjà expliqué le livre Six personnages dans les 6 livres précédents et qu'il explique maintenant davantage le livre \textit{Hui Jing}.

\textbf{Partie 2 : Critique de la distorsion (pages 6-7)}

Nguyễn Kim Muôn critique ceux qui se contentent de lire et relire les écritures bouddhistes, d'utiliser leurs connaissances pour discuter mais ne pratiquent pas selon les enseignements des écritures. Il estime que ces personnes sont déformées, c’est-à-dire qu’elles ont une mauvaise compréhension du bouddhisme.
Selon l’auteur, les humains ont deux natures : la nature mortelle et la nature de Bouddha. Si vous ne suivez que le monde terrestre, vous serez toujours un mortel. Mais si vous savez utiliser la nature de Bouddha pour lire les écritures bouddhistes, vous serez en mesure de comprendre les principes.

\textbf{Partie 3 : Expliquer le \textit{Tao} (pages 9-10)}

Nguyễn Kim Muôn explique le \textit{Tao} en affirmant que le \textit{Tao} n'a ni début ni fin, ce qui signifie que le \textit{Tao} existe depuis des temps immémoriaux, et qu'il est impossible de savoir quand il a commencé et quand il finira. Le \textit{Tao} est partout, du Ciel et de la Terre, du Yin et du Yang, des cinq éléments aux êtres humains.
C'est pourquoi, selon lui, les dictons : Chercher la Voie ou Apprendre la Voie, sont tous des mensonges, car la Voie n'est pas quelque part au loin, mais bien au plus profond de chacun.

\textbf{Partie 4 : Explication de la pureté (pages 10-11)}

Nguyễn Kim Muôn explique la Terre pure expliquant que c'est un état d'esprit qui n'est plus contrôlé par les illusions et les soucis. Selon lui, les gens sont souvent tellement occupés à manquer de souffle qu’ils ne savent pas que le \textit{Đạo} est dans ce Souffle.
Purifier le Souffle est la méthode du Wu Wei, c'est-à-dire entraîner le souffle à devenir léger et harmonieux, savoir  changer le Post-Ciel en Pré-Ciel alors on atteindra le Juste Chemin.

\textbf{Partie 5 : Pratiquez le oui comme non (pages 11 à 13)}

Nguyễn Kim Muôn affirme que le bouddhisme est silencieux, sans prétention ni ostentation, ce qui signifie qu'il n'est pas nécessaire de pratiquer quoi que ce soit. Cependant, les humains naissent en étant rien, c'est-à-dire non, puis les influences extérieures apportent le oui, c'est-à-dire ici les sept émotions, les six désirs, les six sens. Pratiquer c'est donc s'efforcer de retourner à l'existence du non, c'est-à-dire éliminer les choses qui appartiennent à l'existence du oui pour revenir à sa vraie nature.

\textbf{Partie 6 : Garder les six sens purs \textit{lục căn} (pages 13-14)}

Nguyễn Kim Muôn insiste sur la nécessité de garder les six sens (yeux, oreilles, nez, langue, corps, esprit) purs, car les six sens sont la source de toutes choses. Les écritures bouddhistes disent que la pratique du Bodhi possède de bonnes racines, ce qui signifie que pratiquer revient à faire en sorte que les six sens deviennent bons.
Si vous pouvez garder les six sens purs, il n'est pas nécessaire de chanter des écritures bouddhiques et de réciter le nom de Bouddha, car cela reste, selon lui, très superficiel.

\textbf{Partie 7 : Trois choses stupides (pages 14-16)}

Nguyễn Kim Muôn mentionne \textit{tam ngu} (trois choses stupides [notre traduction]) : les yeux ne voient pas, les oreilles n'entendent pas, la bouche ne parle pas, affirmant que c'est une manière d'entraîner le Souffle à être régulier. Ici, le souffle signifie \textit{Qi} véritable, \textit{Qi} inné .

\textbf{Partie 8 : Sept émotions \textit{thất tình}, Six désirs \textit{lục dục,} Six racines \textit{lục căn} (pages 16 à 19)}

Nguyễn Kim Muôn analyse les sept émotions, les six désirs, les six racines, affirmant qu'ils appartiennent tous à la nature, c'est-à-dire à l'inaction, qui sont des choses invisibles. Ainsi, seuls les Saints et les Gentilshommes peuvent pratiquer leur Nature.
Les gens ordinaires qui veulent la pratiquer doivent d’abord pratiquer leur vie, c’est-à-dire abandonner les mauvaises habitudes et rendre leur corps léger et pur.

\textbf{Partie 9 : Les trois joyaux \textit{Tam bảo} (Pages 19-21)}

Nguyễn Kim Muôn mentionne les Trois joyaux  (Bouddha, Dharma, Sangha), affirmant que : les yeux ne voient pas, les oreilles n'entendent pas, la bouche ne parle pas. Ainsi, pratiquer le bouddhisme signifie pratiquer directement dans les trois joyaux, c’est-à-dire pratiquer directement en soi-même.

\textbf{Partie 10 : La transmission orale (pages 21-24)}

Nguyễn Kim Muôn définit la \textit{khẩu khuyết} comme une transmission orale, un enseignement profond gravé dans les tendons et les os, qui ne peut être pleinement exprimé par des mots. Il précise que le défaut de la bouche est à l’intérieur, ce qui signifie que chacun porte déjà cet enseignement en soi et qu’il est inutile de le chercher à l’extérieur.

\textbf{Partie 11 : Le Corps physique et le Corps du Dharma (pages 24-27)}

Pour Nguyễn Kim Muôn, le corps physique est illusoire tandis que le corps du Dharma, donc la nature de Bouddha ou l’âme intérieure, est réel. Parvenir à l’illumination implique de se détacher du corps physique et d’écouter la nature de Bouddha. L’abandon évoqué ici correspond à un état de repos profond, semblable au sommeil, permettant de calmer l’esprit et de méditer afin de percevoir la vérité cachée.

\textbf{Partie 11 bis : Les défauts merveilleux (pages 30-31)}

Nguyễn Kim Muôn commente un quatrain issu de huệ cảnh, en expliquant que les défauts merveilleux sont en réalité des enseignements oraux impossibles à formuler en mots. Le chiffre cinq mille symbolise le nombre 5, en lien avec \textit{tam ngũ nhất độ tam ca tự}, et trois cent mille véritables mystères désigne la pratique de la Terre pure avec trois cent mille récitations, associée aux Trois joyaux (oreilles, yeux, bouche). L’absence de mots dans le cœur signifie un esprit libéré d’illusions ; ne pas être avide est déjà une forme de pratique menant à l’illumination sans effort supplémentaire.

\textbf{Partie 12 : Enseignements de pratiques \textbf{(pages 27-30)}}

Nguyễn Kim Muôn affirme que la Voie est en nous, que chacun possède la nature de Bouddha en soi et qu'il n’a pas besoin de la chercher ailleurs. Il encourage la pratique personnelle et l’éveil de la vraie nature, sans dépendre d’autrui. Il souligne que la culture est la racine de toutes les bonnes choses et recommande de demander des Écritures à lire, à l’image d’une offrande faite à un temple. Il invite aussi à adopter le végétarisme, à créer des groupes de production de tofu et de sauce soja pour s’entraider et diffuser le bouddhisme au sein du peuple.

\textbf{Partie 12 bis : Devenir un Bouddha sans pratiquer (pages 31-32)}

Il soutient que pratiquer empêche de réussir, tandis que ne pas pratiquer conduit à devenir Bouddha. Cette idée ne signifie pas l’inaction, mais le refus des méthodes figées pour revenir spontanément à sa vraie nature. Ne pas pratiquer signifie ici ne pas se limiter à des techniques fixes, mais suivre le cours naturel des choses en fonction du destin, des occasions et des circonstances.

\textbf{Partie 13 : Importer et exporter (pages 31-32)}

Nguyễn Kim Muôn critique les moines qui savent entrer en méditation mais ne savent pas en sortir. Pour lui, calmer l’esprit ne suffit pas, il faut aussi savoir appliquer la méditation dans la vie quotidienne. Selon lui, le dharma bouddhique n’est ni mouvement ni immobilité, ni entrée ni sortie.

\textbf{Partie 14 : L’histoire de Bodhi dharma (pages 32-34)}

Pour illustrer que la pratique ne doit pas être figée, il raconte l’histoire de Bodhi dharma, 28\textsuperscript{ème} patriarche du Zen, qui se rendit au pays de l’empereur Liang Wu en un seul pas, car même si la distance était longue, comme sa volonté était assez forte, le voyage se fit en un instant.

\textbf{Partie 15 : Éclairer l’esprit et voir la nature (pages 34-37)}

Nguyễn Kim Muôn explique \textit{Minh tâm kiến tánh} comme l’art de pénétrer le mystère, rester immobile dans le vide, examiner la lumière de l’illumination et dissiper l’obscurité. Voir sa vraie nature suppose d’éclairer son cœur, et pour cela, il faut briller de l’intérieur. Regarder à l’intérieur signifie utiliser ses yeux pour s’examiner soi-même, et non pas se contenter de fixer son for intérieur au sens littéral.

\textbf{Partie 16 : Réflexion (pages 37-38)}

La rétrospection consiste à regarder dans le point Qi divin, sans pour autant utiliser physiquement les yeux. C’est une observation intérieure et subtile, qui n’est pas un regard direct.

\textbf{Partie 17 : Conclusion finale (pages 38-40)}

Nguyễn Kim Muôn enseigne que le Gentilhomme a des yeux mais ne voit pas, ne voit pas mais voit ; a des oreilles mais n’entend pas, n’entend pas mais comprend ; la bouche, bien que silencieuse, transmet de bonnes paroles. Cultiver de cette manière consiste d’abord à harmoniser oreilles, yeux et bouche, puis corps et esprit, jusqu’à purifier les six sens pour atteindre la voie sacrée. Il critique ceux qui se limitent à être végétariens, à se raser la tête, à devenir moines ou à se retirer dans la montagne sans pratiquer l’examen intérieur. Pour lui, de telles démarches sans compréhension sont inutiles. Enfin, il rappelle que prendre refuge dans les Trois Joyaux signifie se regarder, s’écouter et garder la bouche close, et conseille de préserver les cinq (éléments, organes internes) afin de vivre en harmonie morale.

\clearpage
\subsection{\textit{Khẩu khuyết} ; L'éducation de respiration [traduction du dépôt légal] ; Précepte oral [notre traduction] - 1933}

 \textit{Khẩu khuyết}\footnote{Nguyễn Kim Muôn, \textit{Khẩu khuyết} [L’éducation de la respiration], Bảo Tồn, Saigon, 1933} est un livre de 28 pages, qui comporte des schémas et une couverture illustrée, il est publié le 27 mars 1933 à Saïgon par l'imprimerie Bảo Tồn. L'ouvrage est tiré à 1000 exemplaires.
\begin{figure}[H]
    \centering
    \includegraphics[width=0.4\textwidth]
    {images/khaukhuyet.JPEG}
    \caption{Couverture du livre - source : Gallica}
\end{figure}
 

\textit{Khẩu khuyết}\footnote{Khẩu khuyết est compris comme instruction orale ou formule secrète} de Nguyễn Kim Muôn se concentre sur la fourniture d'instructions détaillées sur les méthodes de pratique et d'entraînement pour atteindre l'illumination et la libération. L’ouvrage met l’accent sur la pratique et l’expérience, tout en offrant des conseils pratiques aux pratiquants.

Au début, Nguyễn Kim Muôn expose quatre principes généraux servant de base : la loi d’initiation et les interdits \textit{Tam quy, Ngũ giới} Trois refuges, Cinq préceptes [notre traduction]) pour les personnes ordonnées ; les règles concernant la manière de vivre, le régime \textit{ăn chay} régime végétarien [notre traduction], l’hygiène, les vêtements et objets des pratiquants ; le travail sur soi, la maîtrise de la sexualité, le maintien des \textit{lục căn} six sens [notre traduction] dans la pureté ; et la pratique permettant de \textit{minh tâm kiến tánh} éclairer l’esprit et voir la nature [notre traduction], de développer la sagesse, de se connaître et de s’éveiller par soi-même.
\clearpage
Ensuite, l’auteur présente les \textit{Thập giới cấm} (Dix interdits) pour les pratiquants de niveau suprême. Ces dix règles sont : 
\begin{itemize}
    \item Ne pas tuer
    \item Ne pas voler
    \item Observer la chasteté absolue et éviter y compris le \textit{di tinh} émission nocturne [notre traduction]
    \item Toujours dire la vérité et ne pas mentir
    \item Ne pas manger de viande ni consommer d’aliments ou de boissons « chaudes » (café, chocolate, \textit{ngũ vị tân cinq} ou épices piquantes [notre traduction], lait de vache, gâteaux sucrés…)
    \item Ne pas chanter, assister à des spectacles ou participer à des divertissements mondains
    \item Manger avec modération et à heure fixe
    \item ne pas se vanter ni se parer, ne pas utiliser de parfums ou d’objets issus d’animaux 
    \item ne pas dormir sur un lit luxueux mais seulement sur une surface dure 
    \item Vivre dans l’ascèse, vêtu d’un habit suffisant pour couvrir le corps, sans luxe.
\end{itemize}

La partie \textit{Phép công phu} (méthode de pratique [notre traduction]) s’adresse particulièrement aux femmes observant la \textit{trường trai} (abstinence alimentaire prolongée [notre traduction]) et la chasteté, mais vivant encore dans le monde. La pratique quotidienne est divisée en quatre périodes : 
\begin{itemize}
    \item \textit{Tý} L'heure du Rat, minuit [notre traduction]
    \item \textit{Ngọ} L'heure du Cheval, midi [notre traduction]
    \item \textit{Mẹo} L'heure du Lièvre, 5-7h du matin [notre traduction]
    \item \textit{Dậu} L'heure du Coq, fin d’après-midi
\end{itemize}

Le pratiquant doit utiliser un chapelet, porter le signe \textit{chữ Vạn} (svastika), et à chaque période réciter le \textit{Bạch y chú} (Mantra de la Robe Blanche [notre traduction]) ainsi que le \textit{Ngũ Bộ Chú} (Cinq Mantras [notre traduction]). La pratique s’effectue sur le lieu de sommeil, dans un lit avec une moustiquaire propre, sans que personne d’étranger n’y dorme. 

À l’heure \textit{Tý}, avant de se coucher, on pratique environ une demi-heure ; si l’on se réveille la nuit, on pratique immédiatement. 

L’heure \textit{Mẹo} correspond à environ 5 h du matin et dure jusqu’au lever du soleil. 

À l’heure \textit{Ngọ}, on peut pratiquer dans n’importe quelle posture, pourvu que l’esprit reste concentré et la respiration dirigée vers le \textit{đơn điền} (champ d’énergie inférieur [notre traduction]).

La technique centrale consiste à s’asseoir, fermer les yeux et concentrer la pensée sur le \textit{trung khiếu} (point médian du corps, entre le nombril et la colonne vertébrale [notre traduction]) ; respirer du nez vers le ventre, sans gonfler la poitrine, en laissant d’abord le ventre se dilater ; coordonner le \textit{Ngó} (regard mental [notre traduction]) et le \textit{Thở} souffle [notre traduction] pour former la méthode \textit{Thủy hỏa ký tế} (union de l’Eau et du Feu [notre traduction]). 

Il existe aussi des méthodes complémentaires : 
\begin{itemize}
    \item \textit{Trảm xích long} : méthode de concentration et de respiration pratiquée cinq jours avant les règles pour réguler l’énergie, ramenant le souffle de la région génitale vers le \textit{trung khiếu}(point médian du corps [notre traduction]) 
    \item \textit{Khởi hỏa hậu} : allumer le feu interne par respiration profonde continue, conserver le souffle dans le ventre jusqu’à la transpiration, puis expirer fortement pour dissiper le désir sexuel [notre traduction]
    \item \textit{ Lưỡng khước} : double poussée respiratoire [notre traduction] à effectuer avant et après chaque séance, incluant des respirations profondes, la descente du souffle vers les jambes, puis sa remontée par la colonne vertébrale jusqu’au cerveau.
\end{itemize}

La partie \textit{Phép công phu bực tối thượng} (méthode de pratique de niveau suprême [notre traduction]) est destinée à ceux qui ont quitté la famille et les attachements, vivant en abstinence alimentaire prolongée et ayant fait voeu de chasteté. Il recommander de pratiquer une demi-heure au début de la nuit, puis de s'endormir. Au réveil, quelle que soit l’heure, il faut pratiquer à nouveau. 

Avant la séance, il recommande les exercices suivants :
\begin{itemize}
    \item \textit{huyền quang} (Observer la lumière mystique [notre traduction]) jusqu’à voir la couleur jaune
    \item Réciter le mantra
    \item \textit{khử trược} (Purifier les impuretés [notre traduction])
    \item Faire le \textit{lưỡng khước} (double poussée respiratoire [notre traduction])
    \item Respirer 30 fois
    \item Pratiquer le \textit{thủ trung} (garder le centre [notre traduction])
    \item \textit{khởi hỏa hậu} (Allumer le feu interne par respiration profonde continue, conserver le souffle dans le ventre jusqu’à la transpiration, puis expirer fortement pour dissiper le désir sexuel [notre traduction]) une seule fois par nuit, au début de l’heure \textit{Mẹo}
\end{itemize}

L’auteur insiste sur la nécessité de garder un esprit de \textit{từ bi} (compassion) pour éviter la chute spirituelle.

Vient ensuite le \textit{Phép vệ sinh} (règles d’hygiène [notre traduction]) comprenant cinq points : 
\begin{itemize}
    \item Après avoir mangé, il faut se brosser soigneusement les dents, maintenir la langue recourbée pour avaler la salive \textit{Ba la mật} (salive douce précieuse [notre traduction])
    \item Après l’heure Ngọ, ne rien manger ni boire pour éviter la dépendance aux aliments
    \item Tôt le matin, après l’heure Mẹo, il faut se laver, et en hiver allumer le feu interne d’abord
    \item Dormir dans un endroit aéré, non clos
    \item manger peu et pratiquer beaucoup pour préserver le \textit{tiên thiên khí} (souffle inné [notre traduction]), éviter les soucis car l’inquiétude épuise ce souffle.
\end{itemize}

En conclusion, l’auteur aborde la philosophie, affirmant que le \textit{Phật} (bouddhisme), le\textit{ Nho} (confucianisme) et le \textit{Tiên} (taoïsme) portent des noms différents mais se ramènent tous au\textit{tâm – tánh} (esprit – nature). L’esprit est le \textit{Thần} (esprit vital [notre traduction]), la nature est fondamentalement \textit{Hư không} (vide absolu [notre traduction]). Pratiquer consiste à rendre l’esprit vide, sans passion ni désir. \textit{Tinh - khí - thần} (foi – souffle – esprit) aboutissent tous au « Vide », et la Voie se trouve en soi ; le maître ne fait que montrer le fil conducteur au départ, le reste dépend de la pratique personnelle.

\subsection{\textit{Lục tự chơn giải} ; Explication de six mots: \textit{Nam mô a di đà phật} (salut au bouddha Amidah) ; [traduction du dépôt légal] ; Explication authentique des six mots [notre traduction] - 1933}

La série \textit{Lục tự chơn giải}\footnote{Nguyễn Kim Muôn, \textit{Lục tự chơn giải} [Explication des six mots « Nam mô a di đà phật »], Đức Lưu Phương, Saigon, 1933} se compose de 8 volumes, dont les trois premiers sont enregistrés auprès du dépôt légal. Le premier volume, d'une longueur de 40 pages, est publié à Saïgon en 1933 par l'imprimerie Đức Lưu Phương, à 1000 exemplaires. Les deuxième et troisième volumes sont publiés conjointement le 26 août 1933. Imprimés par l'imprimerie Bảo Tồn à Saïgon, le deuxième volume compte 55 pages, tandis que le troisième en compte 42. Ce dernier a été tiré à 1 000 exemplaires.

L'ouvrage \textit{Lục tự chơn giải de Nguyễn} Nguyễn Kim Muôn explique une exploration du mantra bouddhiste \textit{Nam mô a di đà phật}. Nguyễn Kim Muôn explore la signification profonde et l'application pratique de ce mantra, en soulignant son importance en tant qu'outil pour atteindre l'illumination et la libération.

\begin{figure}[H]
    \centering
    \includegraphics[width=0.4\textwidth]
    {images/luctuchongiai.JPEG}
    \caption{Couverture du livre Explication de six mot - \textit{Lục tự chơn giải} - source : Gallica}
\end{figure}

Nguyễn Kim Muôn commence par souligner l'importance de comprendre la véritable signification des enseignements et des pratiques religieuses. Il met en garde contre le fait de suivre aveuglément les rituels et les traditions sans en comprendre leurs principes sous-jacents. Nguyễn Kim Muôn souligne la nécessité de l'auto-réflexion et de l'introspection, exhortant les lecteurs à s'engager activement dans les enseignements et à les appliquer à leur propre vie.

L'objectif central de l'ouvrage est l'analyse des six mots du mantra. Chaque mot est imprégné d'une signification profonde et représente un aspect spécifique du chemin vers l'illumination. Nguyễn Kim Muôn fournit des explications détaillées pour chaque mot, en s'appuyant sur les écritures et les enseignements bouddhistes pour éclairer leur signification.

Nguyễn Kim Muôn souligne l'importance de la pleine conscience et de la culture d'un cœur pur. Il souligne la nécessité de surmonter les émotions et les désirs négatifs, qui entravent le progrès spirituel. Il encourage les lecteurs à pratiquer la méditation et la pleine conscience pour atteindre la paix et la tranquillité intérieures.

Tout au long de l'ouvrage. Nguyễn Kim Muôn souligne l'importance de l'effort personnel et de la persévérance dans la poursuite de l'illumination. Il rappelle aux lecteurs que la véritable libération vient de l'intérieur et ne peut être atteinte par des moyens extérieurs. Il encourage les lecteurs à prendre la responsabilité de leur propre développement spirituel et à ne pas compter sur les autres pour leur salut.
\\
\noindent\rule{0.35\linewidth}{0.6pt}
\clearpage
\subsection{\textit{Một chữ thương} ; La pitié (bouddhisme) [traduction du dépôt légal] ; Un mot : compassion [notre traduction] - 1933} 

\textit{Một chữ thương}\footnote{Nguyễn Kim Muôn, \textit{Một chữ thương} [Un mot : compassion], Bảo Tồn, Saigon, 1933} est un livre de 46 pages, imprimé à 1000 exemplaires par l'imprimerie Bảo Tồn à Saigon le 16 août 1933.

Dans \textit{Một chữ thương}, Nguyễn Kim Muôn explore la nature multiforme de la compassion dans le bouddhisme vietnamien, explorant ses pièges potentiels et son pouvoir transformateur sur le chemin de la libération spirituelle. Il fournit une analyse nuancée de la compassion, soulignant ses complexités et ses défis tout en insistant sur son rôle crucial dans la réalisation de l'illumination.

\begin{figure}[H]
    \centering
    \includegraphics[width=0.4\textwidth]
    {images/motchuthuong.JPEG}
    \caption{Couverture du livre Le Mot : compassion - \textit{Một chữ thương} - source : Gallica}
\end{figure}

Le texte commence par des réflexions personnelles sur la compréhension et l'expérience de la compassion de Nguyễn Kim Muôn. Il partage sa profonde empathie pour ses compagnons de pratique, exprimant un amour qui transcende même les liens familiaux. Il considère ceux qui sont capables de pratiquer de leur vivant comme supérieurs même à ses propres propres parents, qui sont décédés sans en avoir eu l'occasion. Cette perspective unique souligne l'importance profonde de la pratique spirituelle dans sa vision du monde. Nguyễn Kim Muôn réfléchit à l'interdépendance de tous les êtres, en mettant l'accent sur le cheminement commun des âmes en quête de libération de la souffrance. Il reconnaît également les défis et les complexités de la compassion, en particulier dans la gestion des relations interpersonnelles et le maintien de l'équanimité au milieu des inévitables conflits et malentendus qui surgissent dans la vie en communauté. 

Nguyễn Kim Muôn explore la double nature de la compassion, soulignant à la fois son potentiel constructif et destructeur. Il met en garde contre l'attachement excessif et la possessivité, qui peuvent conduire à la souffrance et entraver le progrès spirituel. Il souligne l'importance de cultiver une compassion équilibrée et désintéressée qui transcende les désirs personnels et les motivations égoïstes. Il donne des conseils pratiques sur la façon de gérer les relations difficiles et de maintenir l'équanimité face à l'adversité, en soulignant l'importance de l'introspection, du pardon et de la compréhension. 

Le texte célèbre également le pouvoir transformateur de la compassion, en soulignant son rôle crucial dans l'atteinte de l'illumination. Nguyễn Kim Muôn souligne l'importance de cultiver une véritable empathie et une préoccupation pour les autres, en reconnaissant l'interdépendance de tous les êtres et l'aspiration commune à la libération de la souffrance. Il fournit des exemples inspirants d'individus compatissants qui ont apporté des contributions significatives à la société et au développement spirituel des autres, et encourage aussi les lecteurs à imiter ces modèles, en cultivant un cœur altruiste et compatissant qui s'étend à tous les êtres. 

Nguyễn Kim Muôn souligne l'importance d'étendre la compassion non seulement aux autres mais aussi à soi-même, il encourage les lecteurs à pratiquer l'auto-soin et le pardon, en reconnaissant que la véritable compassion commence par l'acceptation et la compréhension de soi et des autres. 
Il met en avant la compassion comme un puissant facteur de motivation pour la pratique spirituelle, inspirant les individus à rechercher la libération non seulement pour eux-mêmes mais aussi pour tous les êtres.

Nguyễn Kim Muôn souligne le rôle de la compassion dans la promotion de relations harmonieuses et la construction de communautés fortes. Il encourage les lecteurs à cultiver l'empathie et la compréhension, en favorisant la coopération et le soutien mutuel sur le chemin spirituel.

Nguyễn Kim Muôn explique les concepts dans son écrit:
\begin{itemize}
    \item \textit{Thương} (Compassion) : il fournit une analyse détaillée du concept de \textit{Thương}, en faisant la distinction entre la compassion authentique et les expressions superficielles d'empathie. Il souligne l'importance de cultiver une compassion profonde et durable qui motive l'action désintéressée et guide les individus vers l'illumination.
    \item \textit{Thầy} (Enseignant) : il souligne le rôle crucial de l'enseignant spirituel \textit{Thầy} pour guider les pratiquants sur le chemin de la libération. Il souligne l'importance de trouver un enseignant qualifié et digne de confiance, capable de fournir des instructions et un soutien authentiques.
    \item \textit{Đạo} (Chemin) : Il discute du concept de \textit{Đạo} comme chemin spirituel menant à l'illumination. Il souligne l'importance de s'aligner sur le vrai chemin et d'éviter les faux enseignements ou pratiques qui peuvent égarer.
    \item \textit{Tánh} (Nature) : Il explore le concept de \textit{Tánh} comme nature inhérente ou vrai soi. Il encourage les lecteurs à cultiver leur bonté innée et à purifier leur esprit des émotions et des désirs négatifs.
\end{itemize}

\subsection{\textit{Phép công phu} ; L'éducation du corps, précepte à l'usage des bonzes [traduction du dépôt légal] ; Méthode de pratique [notre traduction] - 1933 }

\textit{Phép công phu}\footnote{Nguyễn Kim Muôn, \textit{Phép công phu} [L’éducation du corps], Đức Lưu Phương, Saigon, 1933} est un livre publié en 1933 par l'imprimerie Đức Lưu Phương à Saïgon. Cet ouvrage de 70 pages ne mentionne ni la date de publication exacte, ni le nombre d'exemplaires tirés.

Le livre \textit{Phép công phu} de Nguyễn Kim Muôn est un guide de pratique spirituelle dans le bouddhisme vietnamien, en particulier dans la tradition de Theravāda \textit{Nam tông}, qui met l'accent sur la conscience de soi et la purification pour une cultivation réussie. Le livre fournit un programme détaillé pour les pratiquants, comprenant des postures physiques, des techniques de respiration et des exercices mentaux.

\begin{figure}[H]
    \centering
    \includegraphics[width=0.4\textwidth]
    {images/phepcongphu.JPEG}
    \caption{Couverture du livre Méthode de pratique [notre traduction] - \textit{Phép công phu} - source : Gallica}
\end{figure}

Avant de commencer, les pratiquants doivent se procurer un rosaire de 108 perles et porter un symbole \textit{vạn} sauvastika. Les débutants doivent réciter le \textit{ngũ bộ chú} (cinq mantras) pendant 100 jours au cours des quatre saisons. Ensuite, ils peuvent le réciter chaque fois qu'ils ont du temps libre.

La routine de méditation comporte plusieurs étapes [notre traduction] :
\begin{itemize}
    \item \textit{lưỡng khước} (souffle purifiant)
    \item récitation du \textit{ngũ bộ chú}, \textit{đại công phu} (grande pratique)
    \item \textit{khởi hỏa hậu} (éveil du feu intérieur)
    \item \textit{soi huyền quang} (observation de la lumière profonde)
    \item \textit{đoạn dâm căng} (coupure du feu intérieur)
    \item \textit{huờn tinh bồ não} (essence régénératrice)
    \item \textit{ngũ phép} (pratique du sommeil)
    \item \textit{phục khí tiên thiên} (récupération de l'énergie pré-céleste)
\end{itemize}

Nguyễn Kim Muôn insiste sur la régularité et conseille aux pratiquants de méditer à des heures précises [notre traduction] :

\begin{itemize}
    \item \textit{Heure du Chat} (\textit{Giờ mẹo}, 3h00 à 5h00) : Au réveil, idéalement vers 4h00, méditer une ou plusieurs fois jusqu'à 6h00 ou 7h00.
    \item \textit{Heure du Cheval} (\textit{Giờ ngọ}, 11h00 à 13h00) : Méditer avant le déjeuner pendant environ une demi-heure.
    \item \textit{Heure du Coq} (\textit{Giờ dậu}, 17h00 à 19h00) : Éviter de manger, mais rester hydraté avec de l'eau de coco, du jus d'orange ou de la canne à sucre.
    \item \textit{Heure du Rat} (\textit{Giờ tý}, 23h00 à 1h00) : Après une nuit complète de sommeil, méditer jusqu'à avoir sommeil, puis dormir à nouveau.
\end{itemize}

Il recommande d'éviter de faire la sieste pendant la journée et de limiter la période de sommeil à 6 ou 8 heures la nuit. Il invite à suivre une routine cohérente et à persévérer, même lorsque qu'aucun résultat immédiat n'est visible.

Il liste ensuite différentes méthodes de méditation à pratiquer ainsi que leurs bénéfices :
\begin{itemize}
    \item \textit{Allumer le feu postnatal} [notre traduction] (\textit{Khởi hỏa hậu}) : Cette pratique implique la rétention du souffle et la circulation de l'énergie vitale dans tout le corps.
    \item \textit{Observer la lumière mystique} [notre traduction] (\textit{Soi huyền quang}) : Les pratiquants observent leur lumière intérieure à des heures précises pour évaluer leur progression spirituelle et purifier leur esprit.
    \item \textit{Couper la racine de la luxure} [notre traduction] (\textit{Đoạn dâm căng}) : Cette pratique consiste à rompre l'attachement à la luxure par l'autodiscipline et à cultiver un esprit pur.
    \item \textit{Rendre l'essence pour nourrir le cerveau} [notre traduction] (\textit{Huờn tinh bồ não}) : Cette technique est destinée aux personnes âgées ou faibles, et se concentre sur la reconstitution de l'essence vitale et la récupération de l'énergie perdue.
    \item \textit{Cinq pratiques} [notre traduction] (\textit{Ngũ phép}) et \textit{Restaurer le souffle prénatal} [notre traduction] (\textit{Phục khí tiên thiên}) : Il s'agit de pratiques avancées liées au sommeil et à la culture de l'énergie, qui nécessitent une orientation et une compréhension appropriées.
\end{itemize}

Nguyễn Kim Muôn conclut avec des poèmes et des réflexions, offrant des conseils supplémentaires sur le chemin spirituel. Il met l'accent sur la conduite éthique, le détachement et la persévérance.

\textit{Phép công phu} fournit un cadre pour la pratique spirituelle dans le bouddhisme vietnamien, décrivant un chemin structuré vers l'illumination.

\subsection{\textit{Cao đài chơn giải} ; La doctrine du caodaïsme expliquée et commentée [traduction du dépôt légal] ; Explication authentique du caodaïsme [notre traduction] - 1933}

\textit{Cao đài chơn giải}\footnote{Nguyễn Kim Muôn, \textit{Cao Đài chơn giải} [La doctrine du caodaïsme expliquée et commentée], Thanh Thị Mậu, Saigon, 1933} est un livre publié en 1933 par l'imprimerie Mậu Thị Thạnh à Saïgon. Cet ouvrage de 48 pages ne mentionne ni la date de publication exacte, ni le nombre d'exemplaires tirés.

\begin{figure}[H]
    \centering
    \includegraphics[width=0.4\textwidth]
    {images/caodaichongiai.png}
    \caption{Couverture du livre La doctrine du caodaïsme expliquée et commentée - \textit{Cao đài chơn giải} - source : Gallica}
\end{figure}

\textit{Cao đài chơn giải} de Nguyễn Kim Muôn est une analyse de l'édit sacré de caodaïsme : \textit{<< Nam mô cao đài tiên ông đại bồ tát ma ha tát >>}. Il explique méticuleusement chaque mot, en approfondissant leurs implications philosophiques, éthiques et décrit des exercices pour les pratiquants du caodaïsme.

Caodaïsme signifie le lieu de culte de \textit{đức chí tôn} - l'Être suprême [notre traduction], un sanctuaire où l'on peut trouver la vérité et l'illumination. Il incarne les idéaux de noblesse, de pureté et de détachement des affaires du monde. 

Le cœur de l'ouvrage se concentre sur le but de la pratique religieuse au sein de cette foi. Nguyễn Kim Muôn soutient que le but ultime est d'atteindre l'illumination et la libération, en revenant essentiellement à son vrai soi et en devenant un immortel \textit{tiên} ou Bouddha \textit{Phật}. Il ne s'agit pas de rechercher un sauveur extérieur, mais plutôt d'effectuer un voyage de découverte et de libération intérieure. Nguyên Kim Muôn affirme que le bouddhisme et le caodaïsme ne font qu'un car il pense que toutes les religions proviennent d'un seul et même chemin. Selon son raisonnement, les différences entre les religions ne sont qu'une question de forme extérieure, tandis que leur essence fondamentale reste la même.

Nguyên Kim Muôn est convaincu que le but ultime de toute religion est d'aider les gens à découvrir la \textit{« Chơn lý »} la vérité et le \textit{« Đức chí tôn »} Dieu [notre traduction] à l'intérieur d'eux-mêmes, plutôt que de les chercher à l'extérieur.

Chaque mot de l'édit sacré sert de repère sur ce chemin.
\begin{itemize}
    \item « Nam » symbolise le feu et le cœur humain, exhortant les pratiquants à contrôler et à affiner leurs émotions, empêchant leur feu intérieur de les consumer. 
    \item « mô » représente la vacuité, l'état primordial du cosmos et du soi, encourageant un retour à cet état en renonçant à tous les désirs matériels, attachements et préoccupations mondaines.
    \item « Tiên ông » [la fée masculine âgée] fait référence à ceux qui ont atteint l'illumination et sont devenus immortels, servant de guides aux autres sur le chemin.
    \item « đại bồ tát » désigne les individus dotés d'une immense compassion et d'une immense sagesse, qui se consacrent à aider les autres à atteindre la libération. 
    \item Enfin, « ma ha tát » désigne ceux qui ont atteint le plus haut niveau d'illumination, complètement libérés de tout attachement et capables de guider les autres vers la libération.
\end{itemize}
  
Nguyễn Kim Muôn souligne l'importance de l'action dans la pratique religieuse et affirme que la véritable pratique consiste à faire plutôt qu'à simplement écouter ou lire. Il s'agit de contrôler ses sens et ses désirs, de les empêcher de nous égarer et de revenir finalement à un état de paix intérieure et de pureté.

Nguyễn Kim Muôn affirme avec force l'importance de l'autonomie dans la poursuite de l'illumination. Personne ne peut atteindre la libération à la place d'autrui ; chaque individu doit s'efforcer de se réaliser et ne pas compter sur des forces extérieures. C'est un chemin de découverte de soi, d'autodiscipline et, en fin de compte, d'auto-libération.
\\
\noindent\rule{0.35\linewidth}{0.6pt}

\clearpage
\subsection{\textit{Tu thân} ; Éducation de soi-même. Bouddhisme [traduction du dépôt légal] ; Se cultiver soi-même [notre traduction] - 1933 }

\textit{Tu thân} \footnote{Nguyễn Kim Muôn, \textit{Tu thân} [L’éducation de soi-même], Xưa Nay, Saigon, 1933} est un livre de 80 pages, imprimé à 1000 exemplaires par l'imprimerie Xưa Nay à Saïgon, le 9 février 1933.

 Le livre \textit{Tu thân} de Nguyễn Kim Muôn parle du bouddhisme vietnamien, en mettant l'accent sur les aspects internes de la pratique plutôt que sur les rituels externes. Il critique la pratique courante consistant à se concentrer sur le chant des écritures et à effectuer des cérémonies sans véritable réflexion sur soi ni changement.

\begin{figure}[H]
    \centering
    \includegraphics[width=0.4\textwidth]
    {images/Tuthan.JPEG}
    \caption{Couverture du livre Cultiver sa personne - \textit{Tu thân} - source ! Gallica}
\end{figure}

Nguyễn Kim Muôn soutient que la véritable pratique spirituelle réside dans la compréhension et le raffinement de sa propre nature (\textit{tánh}). Il encourage les lecteurs à examiner leurs pensées, leurs émotions et leurs actions, à identifier et à corriger les schémas négatifs. Nguyễn Kim Muôn critique ceux qui adoptent une pratique superficielle, se concentrant sur les apparences extérieures tout en négligeant le travail intérieur. Il fournit une analyse détaillée du concept de \textit{tu thân}, le décomposant en ses éléments fondamentaux et soulignant l'interdépendance entre la culture de sa nature intérieure (\textit{tánh}) et de son corps physique (\textit{thân}).

Le texte fournit des conseils pratiques sur la façon de cultiver un esprit pur et une conduite vertueuse. Nguyễn Kim Muôn souligne l'importance d'un comportement éthique, notamment l'honnêteté, la compassion et l'autodiscipline. Il souligne également la nécessité de surmonter les émotions négatives telles que la colère, la cupidité et la jalousie. Nguyễn Kim Muôn fournit des instructions spécifiques sur la façon de pratiquer le \textit{chay lòng} ou (purification du cœur [notre traduction]), qui consiste à maintenir un état de pureté mentale et émotionnelle.

Nguyễn Kim Muôn prône une approche équilibrée de la pratique spirituelle, combinant la culture intérieure avec une action consciente dans le monde. Il encourage les lecteurs à s'engager dans une introspection tout en s'acquittant de leurs responsabilités envers la famille et la société. Nguyễn Kim Muôn souligne l'importance de la persévérance et de la détermination dans la poursuite de l'illumination, rappelant aux lecteurs qu'une véritable transformation exige un effort et un engagement soutenus.  Il offre une perspective nuancée sur le rôle du \textit{chay trường} (végétarisme à long terme [notre traduction]) dans la pratique spirituelle, soulignant l'importance de la pureté intérieure par rapport aux restrictions alimentaires.

Le texte comprend également une discussion stimulante sur la commercialisation des enseignements spirituels, remettant en question la pratique courante de fournir des écritures et des enseignements gratuitement. Nguyễn Kim Muôn soutient que le fait de demander un prix raisonnable pour les textes spirituels peut encourager un engagement plus profond de la part des chercheurs et assurer la durabilité de la diffusion de ces enseignements.

Nguyễn Kim Muôn critique ceux qui suivent aveuglément les pratiques religieuses sans en comprendre le sens ou le but, il encourage les lecteurs à remettre en question et à analyser les enseignements plutôt que de les accepter tels quels. Il souligne la nécessité de se faire guider par un enseignant qualifié \textit{Minh sư} qui peut fournir des instructions et un soutien authentiques sur le chemin spirituel.
Nguyễn Kim Muôn rappelle aux lecteurs qu'ils sont en fin de compte responsables de leur propre développement spirituel.  Il encourage l'effort personnel et la persévérance, soulignant que l'illumination ne peut être atteinte par des moyens extérieurs. Il critique les croyances et pratiques superstitieuses, encourageant les lecteurs à se fier à la raison et à la sagesse plutôt qu'à rechercher les conseils de diseurs de bonne aventure ou de médiums.
\\
\noindent\rule{0.35\linewidth}{0.6pt}

\clearpage
\subsection{\textit{Phật giáo} ; Le bouddhisme [traduction selon Nguyễn Kim Muôn] - 1935}

\textit{Phật giáo}\footnote{Nguyễn Kim Muôn, \textit{Phật giáo} [Bouddhisme], Bảo Tồn, Saigon, 1935} est un livre de 24 pages, imprimé à 1000 exemplaires par l'imprimerie Bảo Tồnde à Saigon le 8 août 1935. L'œuvre est le premier volume de la collection Libération de la souffrance [notre traduction] (\textit{Thoát khổ}).

\begin{figure}[H]
    \centering
    \includegraphics[width=0.4\textwidth]
    {images/phatgiao.JPEG}
    \caption{Couverture du livre Le bouddhisme - source : Gallica}
\end{figure}

Nguyễn Kim Muôn commence par affirmer que le bouddhisme contemporain est en déclin et a perdu sa vraie tradition (\textit{thất chơn truyền } (page 5)). Selon lui, le bouddhisme originel n'est qu'une seule voie (\textit{đạo có một}) (page 6), une vérité unique et sans nom spécifique. Il soutient que le Bouddha n'est ni une divinité envoyée du ciel, ni une personne qui est devenue Bouddha en pratiquant (page 6). Au contraire, le terme Bouddha est simplement une appellation désignant quiconque a atteint l'état de parfaite clarté et illumination( \textit{trọn tỉnh, trọn giác}) (page 7), une lumière (\textit{vẻ sáng}) sans forme, sans son, présente partout (page 7).

Il critique vivement la pratique consistant à donner des noms aux Bouddhas tels qu'Amitabha, Shakyamuni, Guanyin et à créer des écoles distinctes comme le Zen, la Terre pure ou le Vajrayana (pages 5, 11). L'auteur prétend que ces actions visent à endoctriner, exploiter, et tromper (\textit{nhồi sọ, lợi dụng, gạt gẫm}) les gens (page 9). Il appelle ces maîtres Bouddhas vivants (\textit{phật sống}) et affirme qu'ils sauvent les gens par la bouche, c'est-à-dire en inventant des histoires pour séduire les fidèles (page 10).

Nguyễn Kim Muôn estime que la pratique religieuse actuelle est mal orientée car les pratiquants sont divisés en deux types (page 11) : 
\begin{itemize}
    \item Les Excessifs \textit{Thái quá} qui font le mal
    \item Les Insuffisants \textit{Bất cập }qui pratiquent par peur 
\end{itemize}
Les deux sont tombés dans le piège de la religion (\textit{ tôn giáo}). L'auteur condamne les rituels tels que l'encens, les offrandes de fleurs et les récitations de \textit{sūtras}, les considérant comme des actes turbulents (\textit{vọng động}) (page 19) qui empêchent le cœur de trouver la paix (page 15) .

Selon lui, le vrai bouddhisme est serein et vide (\textit{ thanh hư vắng lặng}) (page 19), où les oreilles n'entendent pas, les yeux ne voient pas, et le cœur n'a pas de pensées illusoires (page 15). Il insiste sur le fait que le Bouddha ne sauve personne et que chacun doit se sauver lui-même par sa propre force (\textit{Tự lực}) (page 19). Il utilise sa propre expérience pour montrer que toutes les tentatives de trouver le Bouddha à l'extérieur sont vaines, et que la foi, si elle se concentre sur notre Bouddha intérieur, finit par revenir (pages 17, 18).

Nguyễn Kim Muôn conclut en affirmant que le bouddhisme n'est pas une religion, mais une doctrine de non-action et de tranquillité (\textit{vô vi vắng lặng}) (page 23). Il soutient que les Occidentaux pratiquent une forme de bouddhisme plus authentique car leurs sociétés sont  civilisées et progressistes, et que leur pratique est calme et non-turbulente (page 22).

Son appel final aux bouddhistes est d'arrêter les activités bruyantes et organisées. Il affirme qu'ils doivent regarder à l'intérieur, se cultiver et se revitaliser (\textit{chấn hưng mình})(page 25). Il confirme que le problème ne réside pas dans le déclin du bouddhisme, mais dans la  corruption  des pratiquants (page 24), car si les hommes sont corrompus, la Loi; elle, ne l'est pas (\textit{Nhơn hư, Pháp bất hư})(page 16).

En outre, la vraie pratique consiste à sortir de ce qu'il appelle le piège religieux (\textit{trận đồ tôn giáo}) (page 20) et à revenir aux principes fondamentaux [notre traduction] (page 24) :
\begin{itemize}
    \item Faire le bien (\textit{làm lành})
    \item Éviter le mal (\textit{lánh dữ})
    \item Vivre honnêtement et sincèrement  \textit{ăn ngay ở thật}
\end{itemize}
\noindent\rule{0.35\linewidth}{0.6pt}

\clearpage
\subsection{\textit{Công phu} ; La doctrine du cœur [traduction selon Nguyễn Kim Muôn] - 1935}

\textit{Công phu} de Nguyễn Kim Muôn, est un livre de 11 pages, imprimé à 1000 exemplaires par l’Imprimerie Xưa Nay à Sài Gòn le 29 août 1935.

\begin{figure}[H]
    \centering
    \includegraphics[width=0.4\textwidth]
    {images/congphu.png}
    \caption{Couverture du l'écrit Doctrine du cœur - \textit{Công phu} - source : Gallica}
\end{figure}

\textit{Công phu} de Nguyễn Kim Muôn est un essai discutant de la méthode de pratique et d’entraînement pour atteindre l’illumination et la libération. L’ouvrage se concentre sur l’analyse et la clarification des idées fausses sur la pratique spirituelle, tout en fournissant des instructions spécifiques et des exercices aux pratiquants. 

Faisant suite aux affaires de 1935, détaillées dans la partie 1 de ce mémoire, Nguyễn Kim Muôn commence par expliquer pourquoi il n'a pas répondu aux lettres de presse, mais a préféré écrire un livre. Il estime que la presse n'a pas la capacité suffisante pour contenir tous ses propos, et souhaite créer les conditions idéales pour aider le public à comprendre son point de vue immédiatement, sans attendre la validation d'une rédaction journalistique.

Nguyễn Kim Muôn affirme que le bouddhisme est une religion pour toute l'humanité, que l'on soit riche ou pauvre, noble ou humble, que tout le monde peut la pratiquer. Il critique ceux qui comptent sur et attendent Bouddha, pensant qu'il doit aussi libérer et sauver les gens de la souffrance car cela est faux.

Selon lui, pratiquer le bouddhisme signifie être autonome, éveillé, libéré et ne dépendre de personne. Il critique également les gens qui affirment devenir bouddhiste au moment de leur retraite. Selon lui, pratiquer est une chose temporaire, et doit se faire dans le présent, dans chaque action, dans chaque pensée. Nguyễn Kim Muôn estime que le bouddhisme est une religion commune au pays mais aussi au monde, que tout le monde peut la pratiquer, indépendamment du fait d'aller au temple ou de rester à la maison, de manger de la nourriture végétarienne ou non végétarienne, d'être membre du Concile ou moine. Il critique les moines qui ne savent que se raser la tête, aller au temple et passer chaque repas à se soucier de prier et de se repentir. Il affirme qu'ils ne sont que des moines et que ceux qui sont à l'extérieur comme les membres de l'Association ne sont pas assez bien considérés. L’auteur estime que cette façon de penser est très néfaste car elle crée des discriminations et des divisions entre les gens.

Nguyễn Kim Muôn affirme que les Trois Enseignements sont un, mais que les gens les divisent en trois, ce qui signifie que le confucianisme, le bouddhisme et le taoïsme ont tous la même origine et enseignent tous aux gens à faire le bien, éviter le mal et à être honnête et sincère. Ainsi, tout le monde est un pratiquant des trois religions, pas seulement les moines. Selon Nguyễn Kim Muôn, la pratique consiste à libérer l'âme, à échapper au cycle de la réincarnation. Il distingue deux types de pratiques : 
\begin{itemize}
    \item La pratique ascétique
    \item La pratique sereine
\end{itemize}
La pratique ascétique signifie faire souffrir son corps, renoncer à la vie, quitter les affaires du monde, quitter sa famille et tous ses désirs, être végétarien, respecter les préceptes, chanter des écrits bouddhiques et réciter le nom de Bouddha, pratiquer la repentance, aller à la montagne et entrer dans un isolement, être immobile. 

La pratique sereine [notre traduction] \textit{thanh nhàn tu} signifie la pratique par soi-même [notre traduction] \textit{tu theo tự lực}, tout le monde peut pratiquer, faire n'importe quoi, occupé ou libre.

Nguyễn Kim Muôn conclut que la cultivation n'est rien de plus que trois choses : la foi, l'énergie et l'esprit, et en théorie, il n'y a que deux choses, l'esprit et l'énergie. L'Esprit est le Mental, le \textit{Qi} est le \textit{Rein}, et l'Esprit est le Feu, le \textit{Rein} est l'Eau, donc la cultivation consiste à entraîner l'Eau et le Feu à interagir. 

Enfin, Nguyễn Kim Muôn conseille aux pratiquants d'étudier et de suivre ce qui a été enseigné, de ne pas poser trop de questions, car la Voie n'est qu'un petite partie du chemin.
\\
\noindent\rule{0.35\linewidth}{0.6pt}

\clearpage
\subsection{\textit{Đạo khả đạo} ; Le véritable chemin de la religion [traduction du dépôt légal]; La Voie qui peut être nommée [notre traduction] - 1935 }

\textit{Đạo khả đạo}\footnote{Nguyễn Kim Muôn, \textit{Đạo khả đạo} [Le véritable chemin de la religion], Bảo Tồn, Saigon, 1935} de Nguyễn Kim Muôn est un livre de 22 pages, imprimé à 1000 exemplaires par l’Imprimerie Bảo Tồn à Sài Gòn le 12 septembre 1935, au sein de la série La doctrine du non agir.

\textit{Đạo khả đạo} parle du concept de pratique religieuse, du rôle de la religion et de la façon dont les gens abordent la moralité. Le livre fonde son travail sur le concept que la Voie ne peut pas être parlée, et que si elle pouvait l'être, elle ne serait plus la vraie Voie.

\begin{figure}[H]
    \centering
    \includegraphics[width=0.4\textwidth]
    {images/daokhadao.png}
    \caption{Couverture de l'écrit Le \textit{Đạo} pratique éternellement le non-agir \textit{Đạo khả đạo} - source : Gallica}
\end{figure}

Nguyễn Kim Muôn soutient que les humains ont déformé le véritable sens du \textit{Đạo} en essayant d’attacher des mots, des écritures ou des formes de pratique à quelque chose qui est intrinsèquement inexprimable et naturel. Selon lui, la vraie voie ne peut pas être limitée par des livres ou des dogmes, mais doit être ressentie et expérimentée directement.

L’auteur avance l’argument selon lequel le bouddhisme accepté aujourd’hui par les Vietnamiens est essentiellement une version empruntée à la Chine. Il souligne que personne au Vietnam n'est jamais allé en Inde pour étudier les enseignements du bouddhisme primitif, mais les a seulement reçus au travers de la diffusion chinoise.

Par conséquent, l’auteur estime qu'adorer le bouddhisme de manière excessive, sans l’étudier soi-même, et se contenter de répéter les enseignements comme un croyant aveugle, est une honte. Il ironise en affirmant que si l'histoire avait été différente et que les Vietnamiens avaient été influencés par l'Islam plutôt que par le bouddhisme, peut-être qu'aujourd'hui ils adoreraient aussi l'Islam avec la même foi, au lieu de remettre eux-mêmes la vérité en question.

Nguyễn Kim Muôn estime que les Vietnamiens ont adoré passivement le bouddhisme, simplement parce qu'il a été transmis par les générations précédentes, au lieu de l'apprendre et de le vérifier par eux-mêmes. Il a souligné que le respect excessif pour les caractères chinois et les écritures bouddhistes chinoises a conduit de nombreuses personnes à tomber dans la superstition et à être incapables d'accéder à la véritable nature du Tao.

Nguyễn Kim Muôn soutient que si l’on veut vraiment comprendre le Tao, on ne peut pas se fier uniquement aux enseignements ou aux écritures, mais qu'il faut observer, penser et expérimenter directement. Il compare cela au fait de manger un plat, qu'il ne faut pas simplement écouter les autres affirmer qu'il est délicieux, mais plutôt le goûter soi-même pour savoir s'il l'est vraiment ou non.

Nguyễn Kim Muôn critique vivement les mouvements du renouveau bouddhiste, qui créent des associations, écrivent des magazines et construisent des institutions religieuses. Il estime qu'il s'agit essentiellement d'une simple continuation d'un vieux système superstitieux qui n'apporte pas de véritable valeur aux pratiquants.

Nguyễn Kim Muôn a souligné que pratiquer une religion ne consiste pas en la construction d'installations matérielles ou au fait d'organiser des mouvements superficiels, mais bien à éveiller son moi intérieur et à s'entraîner soi-même pour atteindre l'illumination. Ceux qui courent après la forme et oublient l'essence ne font que renforcer les chaînes qui les lient.

Une idée importante dans l’ouvrage est le concept que la religion est un moteur pour stimuler la confiance en soi, car chacun doit trouver la vérité par lui-même. Nguyễn Kim Muôn souligne que personne ne peut sauver personne, et que l'on ne peut pas non plus s'attendre à ce que Bouddha ou toute autre puissance n'accorde sa grâce pour accomplir nos mérites à notre place.

Il rappelle qu’en réalité, chacun doit se libérer par lui-même, car nul ne peut le faire à la place d’un autre. Pour lui, la véritable pratique ne réside pas dans des actions extérieures comme le chant ou l’adoration, mais repose sur la conscience de soi et la libération intérieure. 

Nguyễn Kim Muôn affirme que la vraie religion ne peut être limitée par aucune doctrine, aucune écriture ou aucun système. Il la compare à la lumière du soleil : personne n'a appris au soleil à briller, mais c'est grâce à cette lumière que toutes choses peuvent pousser.

Selon lui, ceux qui comprennent véritablement la Voie n’ont pas besoin de suivre les écritures ou les formulaires, mais doivent expérimenter, observer et pratiquer par eux-mêmes. La véritable cultivation consiste à retourner à son propre moi naturel, sans se laisser gouverner par le dogme ou la tradition.

\clearpage
\subsection{\textit{Đời người giải thoát} ; La vie libérée [traduction selon Nguyễn Kim Muôn] - 1935}

 \textit{Đời người giải thoát} \footnote{Nguyễn Kim Muôn, \textit{Đời người giải thoát} [La vie libérée], Đức Lưu Phương, Saigon, 1935} de Nguyễn Kim Muôn, est un livre de 18 pages, imprimé à 1000 exemplaires par l'imprimerie Đức Lưu Phương à Saïgon, au 7 novembre 1935.
 Il propose une approche unique de la pratique bouddhiste, mettant l'accent sur l'autonomie et la libération des rituels et dépendances externes.  Nguyễn Kim Muôn remet en question les notions conventionnelles de la pratique religieuse, plaidant pour un chemin plus direct et personnel vers l'illumination.

\begin{figure}[H]
    \centering
    \includegraphics[width=0.4\textwidth]
    {images/doinguoigiaithoat.png}
    \caption{Couverture du livre La vie libérée - \textit{Đời người giải thoát} - source : Gallica}
\end{figure}

Nguyễn Kim Muôn critique la pratique courante consistant à s'appuyer sur des autorités et des rituels extérieurs, affirmant que la véritable libération vient de l'intérieur. Il souligne l'importance de la conscience de soi, de la pensée critique et de la responsabilité personnelle dans la navigation du voyage spirituel. Il remet également en question la compréhension traditionnelle des écritures et des enseignements bouddhistes, suggérant qu'ils devraient être considérés comme des guides plutôt que comme des doctrines rigides.  Il pense qu'une dépendance excessive aux écritures et aux rituels peut conduire à la dépendance et entraver la véritable libération. 

Le texte prône une approche pratique et engagée de la pratique bouddhiste, encourageant les lecteurs à appliquer les enseignements à leur vie quotidienne et à contribuer à la société. Nguyễn Kim Muôn souligne l'importance d'une conduite éthique, de la compassion et de la pleine conscience pour cultiver une vie libérée et épanouissante. Il encourage également les lecteurs à remettre en question les croyances et les pratiques conventionnelles, en cherchant une compréhension plus profonde de la vraie nature de la réalité. 

Nguyễn Kim Muôn souligne l'importance de l'autonomie et de la responsabilité personnelle pour atteindre la libération spirituelle.  Il encourage les lecteurs à cultiver leur propre sagesse et leur propre discernement, plutôt que de suivre aveuglément les autorités ou les rituels extérieurs. 
Il encourage les lecteurs à examiner de manière critique les écritures et les enseignements bouddhistes, en cherchant une compréhension plus profonde de leur signification et de leur application. Il remet en question la notion de foi aveugle, en prônant une approche plus raisonnée et analytique de la pratique spirituelle.

Nguyễn Kim Muôn souligne l'importance d'appliquer les enseignements bouddhistes à la vie quotidienne, en encourageant les lecteurs à cultiver une conduite éthique, la compassion et la pleine conscience dans leurs interactions avec le monde. Il prône également la responsabilité sociale et la contribution au bien-être des autres. 
Il remet en question la compréhension conventionnelle de la voie bouddhiste, en suggérant une approche plus directe et personnelle de l'illumination. Il souligne l'importance de la conscience de soi, de la paix intérieure et de vivre dans le moment présent.

Nguyễn Kim Muôn fournit une analyse du concept de libertation (\textit{giải thoát}), en soulignant l'importance de se libérer des attachements, des illusions et des émotions négatives.  Il souligne le pouvoir transformateur de la pratique bouddhiste pour atteindre la véritable libération et vivre une vie épanouissante. Il explore le concept d'autonomie (\textit{tự lực}) comme la capacité de s'appuyer sur sa propre sagesse et sa propre force pour naviguer dans le voyage spirituel. Il encourage les lecteurs à cultiver l'autonomie et à prendre la responsabilité de leur propre développement spirituel. Nguyễn Kim Muôn discute du concept de bouddhisme (\textit{đạo phật}) comme un chemin de découverte de soi et de libération.  Il souligne l'importance de s'aligner sur les véritables enseignements du Bouddha et d'éviter les pratiques fausses ou trompeuses. 

Nguyễn Kim Muôn fournit une analyse critique du rôle de la religion (\textit{tôn giáo}) dans la société, soulignant à la fois ses avantages et ses inconvénients potentiels. Il encourage les lecteurs à aborder la religion avec discernement et esprit critique, en évitant la foi aveugle ou le dogmatisme. Il affirme que la véritable pratique bouddhiste ne consiste pas à rejeter le monde mais à y vivre avec un sentiment de détachement et d'équanimité. Il critique la tendance de certains pratiquants à devenir trop dépendants des formes extérieures de culte, telles que le chant des écritures et l'accomplissement de cérémonies, sans cultiver une véritable transformation intérieure. 

Nguyễn Kim Muôn soutient que la véritable libération vient de la compréhension et de l'application des principes fondamentaux du bouddhisme, tels que la conscience de soi, la conduite éthique et la compassion, à sa vie quotidienne.  Il souligne l'importance de cultiver une vie libérée (\textit{đời người giải thoát}), caractérisée par la paix intérieure, la liberté des attachements et un engagement conscient avec le monde. 

\clearpage
\subsection{\textit{Phép thanh tịnh} ; Bouddhisme. La pureté. [traduction du dépôt légal]; Le calme [traduction selon Nguyễn Kim Muôn] - 1935} 
D'après les informations de la BnF, \textit{Phép thanh tịnh} \footnote{Nguyễn Kim Muôn, \textit{Phép thanh tịnh} [La pureté], Bảo Tồn, Saigon, 1935} fait partie d'une série d'ouvrages : La doctrine du non-agir, écrite par Nguyễn Kim Muôn, et imprimée par la maison d'édition Bảo Tồn à Saïgon le 7 septembre 1935. Cet ouvrage, long de 25 pages, a été tiré à 1000 exemplaires.

\textit{Phép thanh tịnh} de Nguyễn Kim Muôn est un traité sur la voie correcte de la pratique bouddhiste, mettant en avant la nécessité d’éliminer les agitations mentales et les superstitions afin d’atteindre l’état de pureté absolue. L’auteur critique sévèrement les formes de pratiques religieuses qui s’appuient sur les rituels, les textes sacrés et la dépendance aux cérémonies. Pour lui, la véritable pratique repose sur la purification de l’esprit, l’absence de perturbation mentale et la discipline personnelle.
\begin{figure}[H]
    \centering
    \includegraphics[width=0.4\textwidth]
    {images/phepthanhtinh.JPEG}
    \caption{Couverture du livre La Doctrine du non-agir \textit{Phép thanh tịnh} - source : Gallica}
\end{figure}

L’auteur cite plusieurs passages du Sutra du Diamant pour affirmer que le véritable Dharma ne possède aucune forme fixe et ne peut être transmis par l’écrit. Selon lui, les individus qui s’attachent aux textes et aux rituels sans comprendre l’essence de la pratique sombrent dans l’illusion et l’agitation mentale.  

Nguyễn Kim Muôn soutient que l’existence même des doctrines bouddhistes a conduit à la tentation et à l’aveuglement, car les pratiquants se focalisent davantage sur les formes extérieures que sur la transformation intérieure. Par conséquent, la plupart des adeptes restent coincés dans un cycle de pratiques vaines qui ne leur permet pas d’atteindre la véritable tranquillité.  

\textbf{Partie 1. Critique des pratiques superficielles et de la dépendance aux rituels}
 
L’auteur dénonce vivement l’attachement excessif à l’utilisation des cloches rituelles, aux \textit{sûtras}\footnote{\textit{sûtras} : écritures bouddhiques, \textit{kinh Phật}}, aux statues de Bouddha et aux cérémonies religieuses. Il compare cela à des mouches attirées par du miel, signifiant que les gens sont séduits par l’apparence extérieure de la religion et en oublient l’essence véritable.  

Il insiste sur le fait que les prières, les chants et les offrandes ne sont que des moyens, et non une fin en soi. Ceux qui s’accrochent trop à ces pratiques renforcent leur propre illusion et s’éloignent de la discipline personnelle et de l’éveil véritable.  

\textbf{Partie 2. \textit{Phép thanh tịnh} – La voie authentique de la pratique}
 
Nguyễn Kim Muôn affirme que la véritable pratique spirituelle consiste à atteindre un état de pureté absolue, dans lequel l'esprit est totalement libéré des désirs, des souffrances et des distractions.  

Il met en avant deux principes essentiels pour parvenir à cet état :  
\begin{itemize}
    \item \textit{Ngoại bất nhập} Ni sortie intérieure [notre traduction] : Ne pas se laisser influencer par les éléments extérieurs (rituels, doctrines, croyances populaires).  
    \item \textit{Nội bất xuất} Ni entrée extérieure [notre traduction] : Ne pas laisser l’esprit être troublé par des pensées agitées.  
\end{itemize}
 
L’auteur explique que les êtres humains sont piégés dans le cycle de la souffrance parce qu’ils écoutent trop, regardent trop et parlent trop. Ces distractions les empêchent d’atteindre la véritable sérénité. Il préconise donc la pratique du silence et de l’immobilité absolue, en évitant les influences extérieures pour purifier l’esprit.  

\textbf{Partie 3. Réformer le bouddhisme ou se réformer soi-même ?}

Nguyễn Kim Muôn soutient que le bouddhisme n’a pas besoin d’être réformé, mais que ce sont les pratiquants qui doivent changer. Il critique sévèrement les mouvements visant à moderniser ou réformer le bouddhisme en établissant des associations, des revues et des écoles bouddhistes, estimant que cela ne fait qu’alimenter l’agitation mentale et la superficialité.

Selon lui, la seule manière d’améliorer la pratique religieuse est que chaque individu prenne la responsabilité de son propre cheminement, en se concentrant sur l’introspection et la pureté intérieure, plutôt que de chercher à modifier des structures extérieures.  

Nguyễn Kim Muôn ne se contente pas d’exposer ses idées, il critique ouvertement les dérives modernes des pratiques bouddhistes, affirmant que c’est l’attachement aux formes extérieures et à la ritualisation qui empêche les individus d’accéder à la véritable libération.  

\section{Analyse de la doctrine de Nguyễn Kim Muôn à travers ses écrits}
\subsection{L'entrée dans la vie religieuse (1927-1928)}
Au début de sa carrière de propagation du Dharma, Nguyễn Kim Muôn s'est particulièrement concentré sur la méthode de la Terre pure \textit{Tịnh độ tông}, la considérant comme la voie de pratique essentielle, simple et adaptée à toutes les couches sociales. Cette période est clairement illustrée par deux de ses œuvres principales : \textit{Tịnh độ tông}, publiée en 1927, et \textit{Phật giáo khuyến tu}, publiée en 1928. L'objectif fondamental de la pratique à cette époque, selon Nguyễn Kim Muôn, est d'atteindre la renaissance dans la Terre pure \textit{Cực lạc}, le monde occidental du Bouddha \textit{Amitābha}, un lieu sans souffrance, uniquement de joie infinie. Nguyễn Kim Muôn souhaite aider les pratiquants à  ouvrir une voie étroite pour que les êtres puissent marcher sur le grand chemin. Cette méthode est décrite par lui-même comme une méthode rapide, une doctrine simple comme un jeu d'enfant. Il affirme également que, quelle que soit la durée ou l'intensité de la pratique, ou l'accumulation de mérites,  si l'on suit la méthode de la Terre pure, on devient Bouddha.

La méthode de pratique principale recommandée par Nguyễn Kim Muôn est la récitation du nom du Bouddha \textit{Amitābha}. Il souligne la simplicité de cette approche, qui ne nécessite pas de rituels complexes. Nguyễn Kim Muôn indique que la récitation sincère du nom du Bouddha \textit{Amitābha} 300 000 fois est suffisante pour recevoir l'assurance du Bouddha et être guidé. Il affirme avec certitude que si l'on atteint 300 000 récitations, le Bouddha apparaîtra assurément pour donner son assurance. Il affirme qu'il est essentiel de maintenir un cœur pur pour que la pratique ne soit pas vaine.

En plus de la récitation du nom du Bouddha, Nguyễn Kim Muôn propose d'autres éléments de soutien importants pour la pratique. Les pratiquants doivent observer les préceptes, pratiquer le végétarisme, s'abstenir de tuer et maintenir une âme pure. Le végétarisme est perçu comme une action purificatrice du corps et un moyen de cultiver la compassion. Il encourage diverses formes de végétarisme, allant de pratiques intermittentes (2, 6 ou 10 jours par mois) au végétarisme permanent.

Nguyễn Kim Muôn valorise également le culte domestique. Il conseille d'établir un autel bouddhiste à la maison, orienté vers l'ouest, et, quelle que soit sa taille, d'y placer au minimum une image du Bouddha \textit{Amitābha}. Ceux qui en ont les moyens peuvent ajouter les Trois Joyaux (Bouddha \textit{Amitābha}, \textit{Bodhisattva} \textit{Avalokiteshvara}, \textit{Bodhisattva Mahasthamaprapta}). Les offrandes doivent être simples et pures, comme de l'eau claire (appelée \textit{tịnh thủy}, eau pure sans ajout de thé ou d'alcool), des fleurs de lotus ou d'autres fleurs pures (les lys et les soucis étant des fleurs pures appréciées du Bouddha), et des fruits bien lavés. Il insiste sur le fait que le Bouddha n'accepte que la sincérité et la pureté du cœur. Des éléments tels que des lampes en cristal (ou lampes à huile, bougies, appelées \textit{ngọn thái-cực-đăng}) et des brûleurs d'encens sont également nécessaires. Avant d'allumer l'encens, le mantra \textit{Án lam} doit être récité pour la purification et l'expression de la sincérité. Nguyễn Kim Muôn mentionne aussi le culte des divinités protectrices comme \textit{Thổ địa}, \textit{Thổ thần} et \textit{Táo quân} à domicile pour maintenir l'équilibre entre la vie spirituelle et familiale. Il conseille de placer l'autel bouddhiste loin du lit pour éviter toute distraction et préserver sa pureté. L'offrande d'encens symbolise également les Cinq Parfums spirituels : 
\begin{itemize}
    \item Parfums des Préceptes
    \item Parfums de Concentration
    \item Parfums de Sagesse
    \item Parfums de Connaissance
    \item Parfums de Libération
\end{itemize}

En ce qui concerne la pratique quotidienne \textit{nhựt-khóa}, il s'agit d'un rituel effectué deux fois par jour, matin et soir. Il n'y a pas d'heure fixe, la pratique s'adaptant à l'emploi du temps de chacun. Avant de commencer, les pratiquants doivent se laver le visage, les mains, se rincer la bouche, se concentrer et calmer leur esprit. Ceux qui savent lire récitent les écritures bouddhiques du livre de \textit{Nhựt-khóa}, tandis que les autres peuvent réciter \textit{Nam mô a di đà phật} 300 000 fois ou pratiquer la méthode des Dix Récitations. Il encourage l'usage de chapelets noirs pour le décompte des récitations, soulignant qu'avec le temps, il devient sacré et efficace. La régularité de la pratique est essentielle pour stabiliser l'esprit et éviter les distractions. Nguyễn Kim Muôn suggère également de disposer d'un espace dédié \textit{liêu} pour la pratique, séparé du lieu de repos, afin d'éviter la dispersion mentale.

Dans son œuvre \textit{Phật giáo khuyến tu} (1928), Nguyễn Kim Muôn introduit la méthode Prescription pour Nourrir la Voie \textit{Toa thuốc bổ đạo}, basée sur la théorie des Cinq Éléments (Métal, Bois, Eau, Feu, Terre) pour cultiver le corps et l'esprit, visant l'équilibre dans la vie et la pratique. Il formule des conseils spécifiques pour la régulation de la respiration :
\begin{itemize}
    \item Métal - Poumons : Parler peu aide à nourrir les poumons et à maintenir la paix de l'esprit, la maîtrise de la colère 
    \item Bois - Foie : Ne pas laisser la colère endommager le foie, cultiver la compassion, la modération des désirs 
    \item Eau - Reins : La tempérance aide les reins à ne pas perdre d'énergie, l'abandon de l'inquiétude 
    \item Feu - Cœur : Abandonner l'inquiétude et maintenir l'esprit en paix nourrira mieux le cœur, et une alimentation équilibrée 
    \item Terre - Estomac : Manger avec modération aidera l'estomac à mieux digérer, pour préserver les Trois Joyaux (Essence, Énergie, Esprit).
\end{itemize} 

Il insiste également sur l'abstention du tabac pour les hommes et de la chique de bétel pour les femmes.

Nguyễn Kim Muôn affirme que la pratique spirituelle est une responsabilité individuelle, sans distinction de statut social ou de genre. Une section spécifique, Exhortation des femmes à la pratique \textit{Phụ nữ khuyến tu}, souligne d'ailleurs le rôle essentiel des femmes dans la pratique, contribuant non seulement au bonheur familial mais aussi à l'inculcation de valeurs morales aux enfants. Il écrit que la femme est la racine de la famille, que si la famille est corrompue, alors la société est désordonnée. Il encourage donc les femmes à adopter le végétarisme, y voyant un acte de purification corporelle et de cultivation de la compassion. Il considère la Terre pure comme la voie la plus facile et la plus efficace pour se libérer de la souffrance et du cycle des renaissances. Nguyễn Kim Muôn explique d'ailleurs les principes du karma et du cycle des renaissances, posant que la condition matérielle (richesse ou pauvreté) résulte des mérites ou des mauvaises actions des vies antérieures. Aussi, il encourage chacun à accumuler des mérites par des actions vertueuses pour améliorer son destin. Il conseille également d'initier les enfants dès le plus jeune âge à la récitation du nom du Bouddha et à une vie éthique, afin de jeter des bases solides pour leur âme.

\subsection{Phase d'élargissement, de synthèse et de critique initiale (environ 1929-1932)}

La période de 1929 à 1932 est caractérisée par une évolution significative dans la pensée de Nguyễn Kim Muôn, s'écartant du focus initial. Les œuvres de cette période, telles que \textit{Đạo phật thích ca} (1929), \textit{Thờ trời tu phật} (1929), \textit{Chấn hưng phật giáo} (1929), \textit{Đạo có một} (1929), \textit{Phật giáo vệ sinh} (1929), \textit{Huệ cảnh tây phang} (1930), \textit{Đeo theo chưng Phật} (1932), \textit{Đoạn dâm căng} (1932), \textit{Dục Tâm} (1932), et \textit{Phật Đạo} (1932), démontrent une approche plus synthétique et pratique, intégrant des analyses critiques de la situation religieuse contemporaine.

Durant cette phase, Nguyễn Kim Muôn approfondit la doctrine de Il n'y a qu'une seule Voie \textit{Đạo có một}, affirmant que les diverses traditions religieuses, incluant le Confucianisme, le Bouddhisme et le Taoïsme, convergent vers une vérité et un objectif universels : la libération du cycle des renaissances et l'atteinte de l'illumination. Il soutient que les Trois Doctrines enseignent toutes les principes de la bienveillance, de l'évitement du mal et de la sincérité. Cette perspective révèle une recherche d'unité qui transcende les systèmes de croyances. Nguyễn Kim Muôn aborde également l'interconnexion entre les philosophies orientales et la Théosophie, suggérant une ouverture à l'intégration de divers courants de pensée. Il considère cette harmonie comme essentielle pour élargir la voie et soutenir tous les êtres.

La pensée de Nguyễn Kim Muôn s'oriente également vers une emphase sur l'autonomie et l'auto-réalisation. Il insiste sur l'idée que la voie est en chacun de nous et que le Bouddha est dans le cœur, que la vérité réside en chaque individu. Il estime que chaque personne doit s'examiner et s'observer pour atteindre l'illumination, sans dépendre des divinités, des maîtres ou des rituels extérieurs. Il évoque le concept de Sutra Sans Mots \textit{Vô tự chân kinh}, impliquant que la vérité ne peut être pleinement exprimée par le langage, mais nécessite une réalisation directe par l'expérience personnelle.

Nguyễn Kim Muôn commence par critiquer ouvertement les problématiques du bouddhisme de son temps et appelle à une réforme pragmatique. Il analyse l'oubli de la doctrine bouddhiste, le non-respect des préceptes par certains membres du clergé, et mets l'accent sur les formes extérieures de pratique au détriment de la cultivation morale et intérieure. Il critique les pratiques superstitieuses, telles que la fabrication d'amulettes ou les rituels non canoniques, les considérant comme des facteurs d'illusion. Au lieu de solutions superficielles comme la construction de pagodes ou l'organisation d'associations (qu'il juge limitées en efficacité et durabilité), Nguyễn Kim Muôn propose des solutions plus fondamentales : l'impression et la diffusion des écrits bouddhiques en écriture vietnamienne romanisée \textit{Quốc ngữ} pour faciliter l'accès à la compréhension des principes bouddhistes, et ainsi éveiller et inciter à faire le bien et éviter le mal. Il insiste sur la congruence entre la parole et l'action \textit{miệng nói, tay viết, thịt làm}. Il encourage particulièrement la pratique à domicile, jugée adaptée à la vie quotidienne et ne nécessitant pas de renoncer à la vie laïque.

Nguyễn Kim Muôn porte également un intérêt croissant au lien entre la santé physique et mentale. Dans son œuvre \textit{Phật giáo vệ sinh} (1929), il propose des principes d'hygiène et de nutrition comme éléments intégrés à la pratique spirituelle. Il observe que certains végétariens rencontrent des problèmes de santé, ce qui affaiblit leur confiance dans la pratique. Il formule donc les conseils suivants : 
\begin{itemize}
    \item Modération alimentaire
    \item Mastication approfondie
    \item Évitement des graisses et des aliments transformés
    \item Privilégier les aliments frais pour maintenir la santé et la clarté mentale. 
\end{itemize}

Il souligne également l'importance de l'hygiène personnelle (dentaire, buccale, abstinence de tabac et de bétel), d'une respiration profonde et régulière pour purifier l'énergie, et du choix d'un environnement propre. L'objectif est de protéger et de nourrir les Trois Joyaux (Essence, Énergie, Esprit). 

Les œuvres \textit{Đoạn dâm căng} (1932) et \textit{Dục tâm} (1932) sont spécifiquement consacrées à l'identification et à la régulation des désirs, en particulier du désir sexuel, considéré comme un obstacle majeur. Il propose des méthodes d'introspection telles que Refléter la lumière vers sois [notre traduction] \textit{Hồi quang Phản chiếu}, un état de concentration mentale pour maîtriser les pensées de désir, et L'esprit meurt, l'âme vit [notre traduction] \textit{Tâm chết, thần sống}, pour pratiquer un état de détachement où l'esprit n'est plus dominé par le désir ou les illusions.

La période 1929-1932 représente une transition importante pour Nguyễn Kim Muôn, qui passe d'un statut de propagateur de la Terre pure vers celle d'un penseur religieux à la vision élargie, synthétique et réformatrice, jetant les bases de concepts plus profonds sur l'autonomie et la sagesse.\\
\noindent\rule{0.35\linewidth}{0.6pt}
\clearpage

\subsection{Phase de synthèse finale (environ 1933-1935)}

La période d'environ 1933 à 1935 marque une phase de synthèse, d'autonomie et d'approfondissement de la sagesse dans la pensée de Nguyễn Kim Muôn. Durant cette période, il consolide ses vues, adoptant une approche plus internalisée, tout en analysant clairement les formes externes de pratique et la dépendance aux forces extérieures.

Dans cette phase, il mets l'accent sur l'idée que Bouddha est dans le cœur et que la Voie est en nous. Il affirme que la vérité et le Bouddha ne se trouvent pas au loin, mais en chaque individu. Il encourage chacun à s'éveiller par soi-même, à se cultiver et à se corriger, sans attendre d'aide de quiconque ou de forces extérieures. L'œuvre \textit{Ai muốn tu?} (1933) illustre cette perspective, où il répond aux questions sur le karma et la renaissance, soulignant que le corps est un trésor à préserver pour la pratique, et encourageant l'auto refuge en Bouddha, sans dépendre d'autrui. Il soutient également que la Voie ne faisant pas de distinction entre riches et pauvres, tout le monde peut la pratiquer.

Nguyễn Kim Muôn clarifie la notion de Vrai Maître, non pas comme une personne spécifique, mais comme une illumination interne ou un guide spirituel. Dans \textit{Đạo khả đạo} (1935), il approfondit l'idée que la Voie qui peut être exprimée n'est pas la Voie éternelle \textit{Đạo khả đạo phi thường đạo}, signifiant que la vérité ne peut être pleinement rendue par les mots ou les textes, mais requiert une expérience directe. Il critique aussi l'attachement aux textes et la dépendance excessive aux écrits bouddhiques en chinois, arguant que cela empêche de nombreuses personnes d'accéder à la véritable vérité. Il analyse également les mouvements de renaissance bouddhique qu'il considère comme superficiels, car axés sur la construction d'infrastructures ou l'organisation d'associations, sans se concentrer sur la pratique intérieure, et établit ces activités comme une continuation d'un ancien système de superstitions sans valeur réelle.

Dans son œuvre \textit{Một chữ thương} (1933), Nguyễn Kim Muôn explique en profondeur la compassion. Il soutient que la véritable compassion doit transcender les liens familiaux pour viser la libération de tous les êtres, sans être limitée par l'attachement ou la possession personnelle. Il souligne également l'importance de prendre soin de soi et de pardonner, car la véritable compassion commence par l'acceptation et la compréhension de soi et des autres.

Dans \textit{Phép công phu} (1933), Nguyễn Kim Muôn fournit des instructions détaillées sur le Qi gong\footnote{Qi gong : Gymnastique traditionnelle chinoise, fondée sur la connaissance et la maîtrise de l’énergie vitale, et associant mouvements lents, exercices respiratoires et concentration.} et la méditation. Il décrit les postures, les techniques respiratoires (respirer par le nez jusqu'à l'abdomen, sans monter à la poitrine, en pratiquant jusqu'à ce que l'on respire à un seul endroit, et regarde à un seul endroit) et les moments spécifiques pour pratiquer au cours de la journée:
\begin{itemize}
    \item heure du chat \textit{mão} 3h00 à 5h00
    \item heure du cheval \textit{ngọ} 11h00 à 13h00
    \item heure du coq \textit{dậu} 17h00 à 19h00
    \item heure du rat \textit{tý} 23h00 à 01h00
\end{itemize}

Il établit également des exigences éthiques strictes pour les pratiquants, notamment : \begin{itemize}
    \item ne pas tuer
    \item ne pas voler
    \item ne pas commettre d'adultère
    \item ne pas mentir
    \item ne pas consommer d'alcool
    \item ne pas chanter ni danser
    \item manger avec modération
    \item maintenir une bonne hygiène
    \item ne pas dormir sur un lit luxueux
    \item mener une vie simple
\end{itemize}

Il met particulièrement l'accent sur la cultivation de l'Essence, de l'Énergie et de l'Esprit \textit{tinh, khí, thần} par des méthodes telles que la transformation de l'Essence en Énergie (par l'abstinence sexuelle et le jeûne), la transformation de l'Énergie en Esprit (par la respiration profonde et la régulation du souffle), et la transformation de l'Esprit en Vide (par l'abandon de toutes les pensées et agitations pour atteindre un état de vacuité et de pureté).\\
Il détaille les étapes de la pratique selon la méthode du Non-Agir \textit{vô vi} :

\begin{itemize}
    \item Retraite (7 jours)
    \item Construction de la Fondation (100 jours)
    \item Grossesse de Dix Mois (10 mois)
    \item Marche de Trois Ans (3 ans)
    \item Méditation Murale de Neuf Ans (9 ans)
\end{itemize}

Dans \textit{Cao đài chơn giải} (1933), Nguyễn Kim Muôn explique chaque mot de la formule sacrée \textit{Cao đài} : \textit{<< Nam mô cao đài tiên ông đại bồ tát ma ha tát >>}.\\
Cependant, bien qu'il explique de caodaïsme, il maintient sa position d'autonomie, affirmant que le but final est l'auto-illumination, devenir Immortel/Bouddha, sans chercher un sauveur extérieur.

Dans \textit{Đời người giải thoát} (1935), il présente une approche de la libération, axée sur l'affranchissement des rituels et des dépendances externes pour mener une vie paisible, sans attachement, même dans la vie quotidienne.

\textit{Phép thanh tịnh} (1935) se concentre sur l'élimination de toutes les distractions et superstitions pour atteindre un état de pureté absolue. Il énonce trois principes fondamentaux : 
\begin{itemize}
    \item \textit{Ngoại bất nhập} (ne pas être influencé par les éléments extérieurs)
    \item \textit{Nội bất xuất} (l'esprit n'est pas perturbé par des pensées chaotiques)
    \item \textit{Huờn bổn nguyên} (retour à la nature originelle)
\end{itemize}
Il affirme qu'il est inutile de réformer le bouddhisme, mais qu'il faut se réformer soi-même, soulignant que le bouddhisme n'a pas besoin de changer, mais que c'est le pratiquant lui-même qui doit se transformer, en se concentrant sur l'introspection et la pureté intérieure.

Dans cette phase, Nguyễn Kim Muôn a également systématisé les rituels de la vie du pratiquant de la Terre pure, incluant les cérémonies officielles \textit{quan}, les mariages \textit{hôn}, les funérailles \textit{tang}, et les commémorations \textit{tế}. Le principe général étant la simplicité, le pragmatisme, l'absence d'ostentation, et la conformité à l'éthique et au but de la pratique.

Après l'analyse des pensées et des œuvres de Nguyễn Kim Muôn dans la Partie II, cette thèse a mis en évidence sa méthode de pratique spirituelle autonome et son approche novatrice. Cependant, comme cela a été exposé dans la Partie I, son parcours n'a pas été sans heurts, émaillé de controverses et de critiques.

La Partie III de cette thèse continuera d'étudier l'héritage et la mémoire de Nguyễn Kim Muôn, qui perdurent jusqu'à aujourd'hui. Cette partie examinera comment ses enseignements et ses idées ont été préservés et transmis par ses disciples. De même, elle explorera le rôle des pagodes qu'il a fondées dans la conservation et la perpétuation de son héritage, montrant ainsi comment une figure autrefois controversée est devenue une partie intégrante de l'histoire religieuse du Sud du Viêt Nam.