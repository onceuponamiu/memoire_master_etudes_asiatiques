À partir des années 1920, le \textit{Viêt Nam} colonial entre dans une période de transformation sociale, culturelle et religieuse particulièrement intense. L’historiographie récente s’accorde à qualifier cette période de \textit{« vaste laboratoire de transformations »}\footnote{	Trinh Van Thao, \textit{ Vietnam du confucianisme au communisme : un essai itinéraire intellectuel }, L'Harmattan, Paris, 1995, p. 6.}, où les structures traditionnelles du savoir, du pouvoir et du sacré sont mises à l’épreuve par les logiques coloniales modernes : rationalisme, urbanisation, imprimerie, individualisme. L’élite lettrée, notamment dans les milieux urbains, se montre de plus en plus séduite par les valeurs occidentales, au point de rejeter les fondements jugés obsolètes du confucianisme, du bouddhisme populaire ou des cultes locaux. Des milliers de vietnamiens, surtout dans le sud, adoptent les vêtements occidentaux, insistent pour parler français, et expérimentent de nouvelles formes d’organisation collective, comme les partis politiques ou les associations caritatives\footnote{David G. Marr, \textit{Vietnamese Tradition on Trial, 1920–1945}, University of California Press, 1981 ; Christopher Goscha, \textit{Vietnam: A New History}, Basic Books, 2016.}.

Cette dynamique de modernisation religieuse s’accompagne d’une redéfinition progressive des repères spirituels. Alors que le confucianisme semble perdre de son autorité dans la société coloniale tardive, bien qu’ayant été la référence de l’organisation sociale et morale du pays, d'autres formes de spiritualité trouvent un nouvel écho auprès des populations. C’est le cas du bouddhisme qui, bien qu'affaibli par des querelles internes et des rivalités cléricales sur la moralité du clergé, reste profondément enraciné dans le quotidien des campagnes. Ses expressions populaires, notamment la dévotion au Bouddha \textit{Amitābha} (\textit{A di đà}) et la tradition zen (\textit{thiền}), continuent de structurer les manières de penser et de vivre de nombreux fidèles ruraux\footnote{Jérémy Jammes, \textit{Les oracles du Cao Đài}, les Indes savantes, Paris, 2021.}.

Dans ce contexte de crise, un mouvement de réforme bouddhique, inspiré par le renouveau chinois, émerge dès les années 1920. Ses promoteurs, intellectuels et moines lettrés, plaident pour une meilleure compréhension des textes du Grand Véhicule (\textit{Mahāyāna}) et une revitalisation morale du clergé. Toutefois, cette réforme reste marginale à court terme : selon le moine Thích Mật Thể, \textit{« la grande majorité des moines... rêvassent encore ou dorment, et n'ont encore rien fait pour montrer qu'ils se sont réveillés »}\footnote{Shawn Frederick McHale, \textit{Print and Power: Confucianism, Communism, and Buddhism in the Making of Modern Vietnam}, University of Hawai’i Press, 2004, p. 144.}. En parallèle, de nouveaux mouvements syncrétiques comme le Caodaïsme ou le bouddhisme \textit{Hòa Hảo} proposent des visions alternatives du sacré, à la croisée des traditions asiatiques, du spiritisme occidental et de l'anticléricalisme populaire, apportant ainsi des réponses plus immédiates aux aspirations spirituelles dans une société en pleine recomposition.

L’exemple de Nguyễn Kim Muôn (1892–1946) illustre ce moment de pluralisme religieux actif et de créativité doctrinale. Formé en partie à l’étranger, à la fois prédicateur, auteur, éducateur et fondateur d’un centre de pratique bouddhique à Phú Quốc, Nguyễn Kim Muôn incarne un type nouveau de figure religieuse moderne, qui utilise de nouveaux moyens de communication comme l'imprimerie pour diffuser sa pensée. Il publie plus de trente livrets de prières et de catéchèses bouddhiques, qu’il considère comme une forme d’utilité sociale supérieure à la construction de temples : \textit{« cela vaudrait plus que de bâtir soixante-douze pagodes »}\footnote{Shawn McHale, \textit{Print and Power: Confucianism, Communism, and Buddhism in the Making of Modern Vietnam, University of Hawai’i Press, 2004}, p. 142.}. Il appelle les fidèles à \textit{« économiser le bétel et le tabac pour acheter des textes sacrés à distribuer gratuitement dans toute la Cochinchine »}. Par cette stratégie éditoriale, il contribue à faire circuler une forme de religiosité individuelle et portable, dans un monde encore fortement dominé par la transmission orale.

La période 1920–1945 se distingue non seulement par sa dynamique politique et intellectuelle, mais aussi par la diversification des formes de croyances et la concurrence des discours religieux. Face à la désorientation provoquée par la modernité coloniale, certains acteurs comme Nguyễn Kim Muôn ont tenté d’apporter une réponse religieuse enracinée dans l’histoire vietnamienne, mais rénovée dans ses formes, ses moyens et ses aspirations.

Dans un contexte de modernisation religieuse et de pluralisme spirituel, la figure de Nguyễn Kim Muôn reste peu étudiée dans l’historiographie vietnamienne contemporaine, et il n'y a toujours aucun document officiel sur sa personne. Pourtant, son parcours atypique, à la croisée du bouddhisme, du réformisme moral et de l’action sociale, offre un point d’entrée essentiel pour comprendre les reconfigurations religieuses dans le sud du Viêt Nam durant la période coloniale tardive.

Mais alors, comment retracer de manière rigoureuse la biographie de Nguyễn Kim Muôn à partir de sources fragmentaires, souvent issues de ses propres publications ? Que nous apprennent ses écrits (tracts, livrets de prières, textes de prédication) sur sa conception du bouddhisme, sa vision de la pratique religieuse, du salut et de la réforme morale ? Dans quelle mesure son œuvre témoigne-t-elle d’un projet religieux original, en rupture ou en continuité avec les formes institutionnelles du bouddhisme traditionnel ? Enfin, comment situer sa trajectoire dans l’histoire plus large des mutations religieuses vietnamiennes, marquées par l’imprimerie, la montée de nouveaux mouvements religieux et la quête d’autonomie spirituelle ?

L’objectif de ce mémoire est donc de proposer une étude et une synthèse de la biographie et de la bibliographie de Nguyễn Kim Muôn, afin de mieux cerner ses choix doctrinaux, ses pratiques de diffusion et les formes d’autorité religieuse qu’il incarne. Il s’agira également de réfléchir à la manière dont un acteur religieux local s’inscrit dans une dynamique globale de rénovation religieuse à l’époque coloniale.

Pour répondre à ces questions, ce mémoire est divisé en trois temps et se propose d’étudier de manière approfondie la figure de Nguyễn Kim Muôn, à travers l’analyse croisée de sa biographie, de ses écrits, et de la mémoire religieuse qu’il a suscitée.

La première partie présente les éléments de biographie de Nguyễn Kim Muôn, en retraçant les grandes étapes de sa vie, de sa formation à son activité missionnaire. Nous y explorerons les contextes historiques, familiaux et géographiques dans lesquels s'est construit son itinéraire personnel. Nous aborderons plus particulièrement << l'affaire de 1935 >> qui est un événement marquant de la vie du bonze, alors qu'il se trouve sous le feu des critiques de la presse de l'époque.

La deuxième partie est consacrée à l’étude de ses écrits, en tant que support privilégié de sa pensée religieuse. Nous analyserons ses conceptions du bouddhisme, sa vision du salut, ses méthodes de diffusion des enseignements, ainsi que sa posture morale face à la société coloniale. Cette partie permettra de mieux cerner son projet spirituel et réformateur.

Enfin, la troisième partie s’intéresse à l’héritage et à la mémoire de Nguyễn Kim Muôn. Nous examinerons les lieux de culte associés à sa figure, les pratiques commémoratives, les réseaux de fidèles, ainsi que la circulation plus restreinte de ses textes dans les communautés religieuses du sud.


Le mémoire est le cadre de l’étude des documents originaux constitutifs du dépôt légal de l'Indochine, les principales sources archivistiques exploitées comprennent les collections de la Bibliothèque nationale de France (BnF) ainsi que la plateforme numérique Gallica. Par ailleurs, une enquête d’archives, menée sur une période de trois ans (2021-2024), a nécessité de nombreux déplacements vers des centres d’archives majeurs, notamment les Archives nationales d’outre-mer (ANOM) en France, divers centres d’archives au Vietnam, ainsi que les Archives historiques de Crédit Agricole SA. L’objectif de cette phase était de retrouver des traces documentaires relatives à Nguyễn Kim Muôn et aux personnes qui lui étaient associées, en explorant des dossiers administratifs, des correspondances, des articles de presse et d’autres sources historiques.

En parallèle de l’étude archivistique, une enquête de terrain, d’une durée d’environ trois mois, a été menée à travers de nombreux déplacements entre Saïgon, Phú Quốc et Đồng Nai. Au cours de cette période, nous avons collecté des documents internes et des photographies des temples associés à Nguyễn Kim Muôn. De plus, des entretiens approfondis ont été réalisés avec les moines responsables des temples Long Vân et Hùng Long, permettant de recueillir des témoignages précieux. En complément, des fidèles de ces deux temples ont également été interrogés afin de reconstituer les récits oraux et la mémoire collective autour de Nguyễn Kim Muôn et des pratiques religieuses qui lui sont liées. Nous avons également observé et documenté les cérémonies commémoratives dédiées aux moines décédés, afin de mieux comprendre le rôle de Nguyễn Kim Muôn dans la vie religieuse et la communauté bouddhiste du sud du Vietnam dans les années 1920. L’objectif de cette recherche est de constituer un dossier biographique détaillé retraçant la vie de Nguyễn Kim Muôn, tout en mettant en lumière ses contributions au paysage religieux vietnamien au début du XX\textsuperscript{e} siècle.

Ce mémoire s'appuie sur une méthodologie de recherche enseignée à l'École nationale des Chartes (ENC-PSL), dans le domaine des humanités numériques, formation Master en Technologies numériques appliquées à l'Histoire. Aussi, des outils d'automatisation basés sur l'intelligence artificielle (IA) ont été utilisés pour le traitement et l'analyse des données. Les expérimentations ont été programmées en Python, en intégrant des bibliothèques spécialisées dans la reconnaissance optique de caractères (OCR) afin d’extraire le contenu de documents numérisés issus de Gallica et d’autres fonds d’archives. Afin d'optimiser le traitement des documents disponibles sur Gallica, une méthode de Web Scraping a été mise en place pour télécharger automatiquement toutes les ressources relatives à Nguyễn Kim Muôn avant l’application des techniques de reconnaissance optique de caractères (OCR). Cette approche a permis d'extraire un large corpus de textes numérisés, facilitant ainsi l’analyse et l’exploitation des données sans dépendre d’une consultation en ligne individuelle, ainsi que le traitement d'un volume important de sources, sans nécessiter de saisie manuelle pour chacune d'entre elles.

Par ailleurs, Notta AI, un système avancé de reconnaissance de la parole a été utilisé pour retranscrire plus de 20 heures d’enregistrements issus des entretiens et échanges réalisés lors des enquêtes terrain au Vietnam. L'intégration de ces outils d’IA dans le processus de recherche a permis d’optimiser l’analyse des fichiers audio et d’améliorer la précision des données, en comparaison des méthodes traditionnelles de transcription et de saisie à l'oreille.

L’un des défis majeurs de cette recherche réside dans la barrière linguistique et les difficultés techniques liées au traitement de données hétérogènes. Afin de les surmonter, des outils tels que Google Traduction et Gemini ont été mobilisés, non seulement pour faciliter la compréhension des documents en français, mais aussi pour affiner la traduction, la rédaction et la correction du code et de la syntaxe. De plus, cette étude a bénéficié d’un travail de relecture réalisé par des professeurs, des collègues, et des amis, garantissant ainsi la rigueur scientifique, et la cohérence du mémoire.